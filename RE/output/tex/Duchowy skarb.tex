\documentclass[10pt,a4paper,oneside]{article}
\usepackage[utf8]{inputenc}
\usepackage{polski}
\usepackage[polish]{babel}
\usepackage[margin=0.5in,bottom=0.75in]{geometry}
\begin{document}
\centerline{\textbf{\MakeUppercase{Duchowy skarb}}}
\begin{center}
\textbf{Wersety do studium:} \mbox{(Mt 6, 19-21)}, \mbox{(Łk 6, 45)}, \mbox{(Obj 11, 3)}
\end{center}
\section{Temat główny}
\paragraph{}
\begin{quote}
,,Nie gromadźcie sobie skarbów na ziemi, gdzie je mól i rdza niszczą i gdzie złodzieje podkopują i kradną: Ale gromadźcie sobie skarby w niebie, gdzie ani mól, ani rdza nie niszczą i gdzie złodzieje nie podkopują i nie kradną. Albowiem gdzie jest skarb twój - tam będzie i serce twoje'' \mbox{(Mt 6, 19-21)}
\end{quote}
\paragraph{}
\begin{quote}
,,Człowiek dobry z dobrego skarbca serca wydobywa dobro, a zły ze złego wydobywa zło; albowiem z obfitości serca mówią usta jego.'' \mbox{(Łk 6, 45)}
\end{quote}
\paragraph{}
\begin{quote}
,,I dam dwom moim świadkom moc, i będą, odziani w wory, prorokowali przez tysiąc dwieście sześćdziesiąt dni.'' \mbox{(Obj 11, 3)}
\end{quote}
\end{document}