\documentclass[10pt,a4paper,oneside]{article}
\usepackage[utf8]{inputenc}
\usepackage{polski}
\usepackage[polish]{babel}
\usepackage[margin=0.5in,bottom=0.75in]{geometry}
\begin{document}
\centerline{\textbf{\MakeUppercase{Pożądliwości cielesne}}}
\begin{center}
\textbf{Wersety do studium:} \mbox{(1J 2, 15-17)}, \mbox{(Rz 1, 18-25)}, \mbox{(Rz 6, 11-12)}, \mbox{(Rz 7, 4-6)}, \mbox{(Rz 7, 7-8)}, \mbox{(Rz 7, 14-24)}, \mbox{(Ga 5, 24)}, \mbox{(Jk 1, 13-14)}, \mbox{(Jk 4, 1-3)}, \mbox{(2P 1, 1-9)}, \mbox{(2P 1, 20-21; 2, 1-22)}, \mbox{(2P 3, 3-7)}
\end{center}
\paragraph{}
\begin{quote}
,,Albowiem gniew Boży z nieba objawia się przeciwko wszelkiej bezbożności i nieprawości ludzi, którzy przez nieprawość tłumią prawdę. Ponieważ to, co o Bogu wiedzieć można, jest dla nich jawne, gdyż Bóg im to objawił. Bo niewidzialna jego istota, to jest wiekuista jego moc i bóstwo, mogą być od stworzenia świata oglądane w dziełach i poznane umysłem, tak iż nic nie mają na swoją obronę, Dlatego że poznawszy Boga, nie uwielbili go jako Boga i nie złożyli mu dziękczynienia, lecz znikczemnieli w myślach swoich, a ich nierozumne serce pogrążyło się w ciemności. Mienili się mądrymi, a stali się głupi. I zamienili chwałę nieśmiertelnego Boga na obrazy przedstawiające śmiertelnego człowieka, a nawet ptaki, czworonożne zwierzęta i płazy; Dlatego też wydał ich Bóg na łup pożądliwości ich serc ku nieczystości, aby bezcześcili ciała swoje między sobą, Ponieważ zamienili Boga prawdziwego na fałszywego i oddawali cześć, i służyli stworzeniu zamiast Stwórcy, który jest błogosławiony na wieki. Amen.'' \mbox{(Rz 1, 18-25)}
\end{quote}
\paragraph{}
\begin{quote}
,,Podobnie i wy uważajcie siebie za umarłych dla grzechu, a za żyjących dla Boga w Chrystusie Jezusie, Panu naszym. Niechże więc nie panuje grzech w śmiertelnym ciele waszym, abyście nie byli posłuszni pożądliwościom jego,'' \mbox{(Rz 6, 11-12)}
\end{quote}
\paragraph{}
\begin{quote}
,,Przeto, bracia moi, i wy umarliście dla zakonu przez ciało Chrystusowe, by należeć do innego, do tego, który został wzbudzony z martwych, abyśmy owoc wydawali dla Boga. Albowiem gdy byliśmy w ciele, grzeszne namiętności rozbudzone przez zakon były czynne w członkach naszych, aby rodzić owoce śmierci; Lecz teraz zostaliśmy uwolnieni od zakonu, gdy umarliśmy temu, przez co byliśmy opanowani, tak iż służymy w nowości ducha, a nie według przestarzałej litery.'' \mbox{(Rz 7, 4-6)}
\end{quote}
\paragraph{}
\begin{quote}
,,Cóż więc powiemy? Że zakon to grzech? Przenigdy! Przecież nie poznałbym grzechu, gdyby nie zakon; wszak i o pożądliwości nie wiedziałbym, gdyby zakon nie mówił: Nie pożądaj! Lecz grzech przez przykazanie otrzymał bodziec i wzbudził we mnie wszelką pożądliwość, bo bez zakonu grzech jest martwy.'' \mbox{(Rz 7, 7-8)}
\end{quote}
\paragraph{}
\begin{quote}
,,Wiemy bowiem, że zakon jest duchowy, ja zaś jestem cielesny, zaprzedany grzechowi. Albowiem nie rozeznaję się w tym, co czynię; gdyż nie to czynię, co chcę, ale czego nienawidzę, to czynię. A jeśli to czynię, czego nie chcę, zgadzam się z tym, że zakon jest dobry. Ale wtedy czynię to już nie ja, lecz grzech, który mieszka we mnie. Wiem tedy, że nie mieszka we mnie, to jest w ciele moim, dobro; mam bowiem zawsze dobrą wolę, ale wykonania tego, co dobre, brak; Albowiem nie czynię dobrego, które chcę, tylko złe, którego nie chcę, to czynię. A jeśli czynię to, czego nie chcę, już nie ja to czynię, ale grzech, który mieszka we mnie. Znajduję tedy w sobie zakon, że gdy chcę czynić dobrze, trzyma się mnie złe; Bo według człowieka wewnętrznego mam upodobanie w zakonie Bożym. A w członkach swoich dostrzegam inny zakon, który walczy przeciwko zakonowi, uznanemu przez mój rozum i bierze mnie w niewolę zakonu grzechu, który jest w członkach moich. Nędzny ja człowiek! Któż mnie wybawi z tego ciała śmierci?'' \mbox{(Rz 7, 14-24)}
\end{quote}
\paragraph{}
\begin{quote}
,,A ci, którzy należą do Chrystusa Jezusa, ukrzyżowali ciało swoje wraz z namiętnościami i żądzami.'' \mbox{(Ga 5, 24)}
\end{quote}
\paragraph{}
\begin{quote}
,,Niechaj nikt, gdy wystawiony jest na pokusę, nie mówi: Przez Boga jestem kuszony; Bóg bowiem nie jest podatny na pokusy do złego ani sam nikogo nie kusi. Lecz każdy bywa kuszony przez własne pożądliwości, które go pociągają i nęcą;'' \mbox{(Jk 1, 13-14)}
\end{quote}
\paragraph{}
\begin{quote}
,,Skądże spory i skąd walki między wami? Czy nie pochodzą one z namiętności waszych, które toczą bój w członkach waszych? Pożądacie, a nie macie; zabijacie i zazdrościcie, a nie możecie osiągnąć; walczycie i spory prowadzicie. Nie macie, bo nie prosicie. Prosicie, a nie otrzymujecie, dlatego że źle prosicie, zamyślając to zużyć na zaspokojenie swoich namiętności.'' \mbox{(Jk 4, 1-3)}
\end{quote}
\paragraph{}
\begin{quote}
,,Szymon Piotr, sługa i apostoł Jezusa Chrystusa, do tych, którzy dzięki sprawiedliwości Boga naszego i Zbawiciela Jezusa Chrystusa osiągnęli wiarę równie wartościową co i nasza: Łaska i pokój niech się wam rozmnożą przez poznanie Boga i Pana naszego, Jezusa Chrystusa. Boska jego moc obdarowała nas wszystkim, co jest potrzebne do życia i pobożności, przez poznanie tego, który nas powołał przez własną chwałę i cnotę, Przez które darowane nam zostały drogie i największe obietnice, abyście przez nie stali się uczestnikami boskiej natury, uniknąwszy skażenia, jakie na tym świecie pociąga za sobą pożądliwość. I właśnie dlatego dołóżcie wszelkich starań i uzupełniajcie waszą wiarę cnotą, cnotę poznaniem, Poznanie powściągliwością, powściągliwość wytrwaniem, wytrwanie pobożnością, Pobożność braterstwem, braterstwo miłością. Jeśli je bowiem posiadacie i one się pomnażają, to nie dopuszczą do tego, abyście byli bezczynni i bezużyteczni w poznaniu Pana naszego Jezusa Chrystusa. Kto zaś ich nie ma, ten jest ślepy, krótkowzroczny i zapomniał, że został oczyszczony z dawniejszych swoich grzechów.'' \mbox{(2 P 1, 1-9)}
\end{quote}
\paragraph{}
\begin{quote}
,,Przede wszystkim to wiedzcie, że wszelkie proroctwo Pisma nie podlega dowolnemu wykładowi. Albowiem proroctwo nie przychodziło nigdy z woli ludzkiej, lecz wypowiadali je ludzie Boży, natchnieni Duchem Świętym. (\ldots) Lecz byli też fałszywi prorocy między ludem, jak i wśród was będą fałszywi nauczyciele, którzy wprowadzać będą zgubne nauki i zapierać się Pana, który ich odkupił, sprowadzając na się rychłą zgubę. I wielu pójdzie za ich rozwiązłością, a droga prawdy będzie przez nich pohańbiona. Z chciwości wykorzystywać was będą przez zmyślone opowieści; lecz wyrok potępienia na nich od dawna zapadł i zguba ich nie drzemie. Bóg bowiem nie oszczędził aniołów, którzy zgrzeszyli, lecz strąciwszy do otchłani, umieścił ich w mrocznych lochach, aby byli zachowani na sąd; Również starożytnego świata nie oszczędził, lecz ocalił jedynie ośmioro wraz z Noem, zwiastunem sprawiedliwości, zesławszy potop na świat bezbożnych; Miasta Sodomę i Gomorę spalił do cna i na zagładę skazał jako przykład dla tych, którzy by mieli wieść życie bezbożne, Natomiast wyrwał sprawiedliwego Lota, udręczonego przez rozpustne postępowanie bezbożników, Gdyż sprawiedliwy ten, mieszkając między nimi, widział bezbożne ich uczynki i słyszał o nich, i trapił się tym dzień w dzień w prawej duszy swojej. Umie Pan wyrwać pobożnych z pokuszenia, bezbożnych zaś zachować na dzień sądu celem ukarania, Szczególnie zaś tych, którzy oddają się niecnym pożądliwościom cielesnym, a zwierzchnością pogardzają. Zuchwali, zarozumiali, nie lękają się nawet bluźnić mocom niebieskim. Chociaż aniołowie siłą i mocą są więksi od nich, nie wydają na nich przed Panem wyroku potępienia. Lecz oni, jak nierozumne zwierzęta, które z natury są po to, by je łapano i zabijano, bluźnią temu, czego nie znają; toteż zginą jak one I poniosą karę za nieprawość. Oddawanie się w dzień rozpuście uważają za rozkosz, a gdy współbiesiadują z wami, są zakałą i hańbą, ponieważ nurzają się w swoich pożądliwościach. Oczy ich wypatrują tylko cudzołożnic i nigdy im nie dość grzechu; nęcą dusze słabe, serce mają wyćwiczone w chciwości, synowie przekleństwa. Opuściwszy drogę prostą, zbłądzili i wstąpili na drogę Balaama, syna Beora, który ukochał zapłatę za czyny nieprawe, Lecz został zganiony za swoją nieprawość; nieme bydlę juczne, przemówiwszy głosem ludzkim, zapobiegło nierozumnemu postępkowi proroka. Ludzie ci, to źródła bez wody i obłoki pędzone przez wicher; czeka ich przeznaczony najciemniejszy mrok. Przemawiając bowiem słowami nadętymi a pustymi, nęcą przez żądze cielesne i rozwiązłość tych, którzy dopiero co wyzwolili się od wpływu pogrążonych w błędzie, Obiecując im wolność, chociaż sami są niewolnikami zguby; czemu bowiem ktoś ulega, tego niewolnikiem się staje. Jeśli bowiem przez poznanie Pana i Zbawiciela, Jezusa Chrystusa wyzwolili się od brudów świata, lecz potem znowu w nie uwikłani dają im się opanować, to stan ich ostateczny jest gorszy niż poprzedni. Lepiej bowiem byłoby dla nich nie poznać drogi sprawiedliwości, niż poznawszy ją, odwrócić się od przekazanego im świętego przykazania. Sprawdza się na nich treść owego przysłowia: Wraca pies do wymiocin swoich, oraz: Umyta świnia znów się tarza w błocie.'' \mbox{(2 P 1, 20-21; 2, 1-22)}
\end{quote}
\paragraph{}
\begin{quote}
,,Wiedzcie przede wszystkim to, że w dniach ostatecznych przyjdą szydercy z drwinami, którzy będą postępować według swych własnych pożądliwości I mówić: Gdzież jest przyobiecane przyjście jego? Odkąd bowiem zasnęli ojcowie, wszystko tak trwa, jak było od początku stworzenia. Obstając przy tym, przeoczają, że od dawna były niebiosa i była ziemia, która z wody i przez wodę powstała mocą Słowa Bożego Przez co świat ówczesny, zalany wodą, zginął. Ale teraźniejsze niebo i ziemia mocą tego samego Słowa zachowane są dla ognia i utrzymane na dzień sądu i zagłady bezbożnych ludzi.'' \mbox{(2 P 3, 3-7)}
\end{quote}
\paragraph{}
\begin{quote}
,,Nie miłujcie świata ani tych rzeczy, które są na świecie. Jeśli kto miłuje świat, nie ma w nim miłości Ojca. Bo wszystko, co jest na świecie, pożądliwość ciała i pożądliwość oczu i pycha życia, nie jest z Ojca, ale ze świata. I świat przemija wraz z pożądliwością swoją; ale kto pełni wolę Bożą, trwa na wieki,'' \mbox{(1 J 2, 15-17)}
\end{quote}
\end{document}