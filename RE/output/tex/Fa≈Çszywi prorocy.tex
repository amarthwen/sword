\documentclass[10pt,a4paper,oneside]{article}
\usepackage[utf8]{inputenc}
\usepackage{polski}
\usepackage[polish]{babel}
\usepackage[margin=0.5in,bottom=0.75in]{geometry}
\begin{document}
\centerline{\textbf{\MakeUppercase{Fałszywi prorocy}}}
\begin{center}
\textbf{Wersety do studium:} \mbox{(Mi 3, 5-7)}, \mbox{(Mt 24, 23-28)}
\end{center}
\section{Temat główny}
\paragraph{}
\begin{quote}
,,Tak mówi Pan o prorokach, którzy zwodzą mój lud, gdy zęby ich mają co gryźć, zwiastują pokój, lecz przeciwko temu, który im nic do ust nie da, ogłaszają świętą wojnę. Dlatego zaskoczy was noc i nie będzie widzenia, zapadnie ciemność i nie będzie wieszczby; słońce zajdzie nad prorokami i dzień stanie im się ciemnością. Jasnowidze będą zawstydzeni, a wieszczkowie zawiedzeni; wszyscy zakryją sobie twarze, bo nie będzie odpowiedzi Boga.'' \mbox{(Mi 3, 5-7)}
\end{quote}
\paragraph{}
\begin{quote}
,,Gdyby wam wtedy kto powiedział: Oto tu jest Chrystus albo tam, nie wierzcie. Powstaną bowiem fałszywi prorocy i czynić będą wielkie znaki i cuda, aby, o ile można, zwieść i wybranych. Oto powiedziałem wam. Gdyby więc wam powiedzieli: Oto jest na pustyni - nie wychodźcie; oto jest w kryjówce - nie wierzcie. Gdyż jak błyskawica pojawia się od wschodu i jaśnieje aż na zachód, tak będzie z przyjściem Syna Człowieczego, Bo gdzie jest padlina, tam zlatują się sępy.'' \mbox{(Mt 24, 23-28)}
\end{quote}
\end{document}