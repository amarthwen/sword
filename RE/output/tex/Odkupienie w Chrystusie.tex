\documentclass[10pt,a4paper,oneside]{article}
\usepackage[utf8]{inputenc}
\usepackage{polski}
\usepackage[polish]{babel}
\usepackage[margin=0.5in,bottom=0.75in]{geometry}
\begin{document}
\centerline{\textbf{\MakeUppercase{Odkupienie w Chrystusie}}}
\begin{center}
\textbf{Wersety do studium:} 
\mbox{(Ga 4, 3-7)}, \mbox{(1Kor 6, 19-20)}, \mbox{(Ga 3, 13-14)}, \mbox{(1P 1, 17-19)}
\end{center}
\paragraph{}
\begin{quote}
,,Podobnie i my, gdy byliśmy dziećmi, byliśmy poddani w niewolę żywiołów tego świata; Lecz gdy nadeszło wypełnienie czasu, zesłał Bóg Syna swego, który się narodził z niewiasty i podlegał zakonowi, Aby wykupił tych, którzy byli pod zakonem, abyśmy usynowienia dostąpili. A ponieważ jesteście synami, przeto Bóg zesłał Ducha Syna swego do serc waszych, wołającego: Abba, Ojcze! Tak więc już nie jesteś niewolnikiem, lecz synem a jeśli synem, to i dziedzicem przez Boga.'' \mbox{(Ga 4, 3-7)}
\end{quote}
\paragraph{}
\begin{quote}
,,Albo czy nie wiecie, że ciało wasze jest świątynią Ducha Świętego, który jest w was i którego macie od Boga, i że nie należycie też do siebie samych? Drogoście bowiem kupieni. Wysławiajcie tedy Boga w ciele waszym.'' \mbox{(1 Kor 6, 19-20)}
\end{quote}
\paragraph{}
\begin{quote}
,,Chrystus wykupił nas od przekleństwa zakonu, stawszy się za nas przekleństwem, gdyż napisano: Przeklęty każdy, który zawisł na drzewie, Aby błogosławieństwo Abrahamowe przeszło na pogan w Jezusie Chrystusie, my zaś, abyśmy obiecanego Ducha otrzymali przez wiarę.'' \mbox{(Ga 3, 13-14)}
\end{quote}
\paragraph{}
\begin{quote}
,,A jeśli wzywacie jako Ojca tego, który bez względu na osobę sądzi każdego według uczynków jego, żyjcie w bojaźni przez czas pielgrzymowania waszego, Wiedząc, że nie rzeczami znikomymi, srebrem albo złotem, zostaliście wykupieni z marnego postępowania waszego, przez ojców wam przekazanego, Lecz drogą krwią Chrystusa, jako baranka niewinnego i nieskalanego.'' \mbox{(1 P 1, 17-19)}
\end{quote}
\end{document}