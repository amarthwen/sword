\documentclass[10pt,a4paper,oneside]{article}
\usepackage[utf8]{inputenc}
\usepackage{polski}
\usepackage[polish]{babel}
\usepackage[margin=0.5in,bottom=0.75in]{geometry}
\begin{document}
\centerline{\textbf{\MakeUppercase{Odkupienie w Chrystusie}}}
\begin{center}
\textbf{Wersety do studium:} \mbox{(Rz 3, 21-26)}, \mbox{(Rz 6, 16-23)}, \mbox{(Rz 7, 4-6)}, \mbox{(Rz 7, 21-25)}, \mbox{(Rz 8, 14-17)}, \mbox{(Ga 4, 3-7)}, \mbox{(1Kor 6, 19-20)}, \mbox{(Ga 3, 13-14)}, \mbox{(1P 1, 17-19)}
\end{center}
\section{Temat główny}
\paragraph{}
\begin{quote}
,,Ale teraz niezależnie od zakonu objawiona została sprawiedliwość Boża, o której świadczą zakon i prorocy, I to sprawiedliwość Boża przez wiarę w Jezusa Chrystusa dla wszystkich wierzących. Nie ma bowiem różnicy, Gdyż wszyscy zgrzeszyli i brak im chwały Bożej, I są usprawiedliwieni darmo, z łaski jego, przez odkupienie w Chrystusie Jezusie, Którego Bóg ustanowił jako ofiarę przebłagalną przez krew jego, skuteczną przez wiarę, dla okazania sprawiedliwości swojej przez to, że w cierpliwości Bożej pobłażliwie odniósł się do przedtem popełnionych grzechów, Dla okazania sprawiedliwości swojej w teraźniejszym czasie, aby On sam był sprawiedliwym i usprawiedliwiającym tego, który wierzy w Jezusa.'' \mbox{(Rz 3, 21-26)}
\end{quote}
\paragraph{}
\begin{quote}
,,Czyż nie wiecie, że jeśli się oddajecie jako słudzy w posłuszeństwo, stajecie się sługami tego, komu jesteście posłuszni, czy to grzechu ku śmierci, czy też posłuszeństwa ku sprawiedliwości? Lecz Bogu niech będą dzięki, że wy, którzy byliście sługami grzechu, przyjęliście ze szczerego serca zarys tej nauki, której zostaliście przekazani, A uwolnieni od grzechu, staliście się sługami sprawiedliwości, Po ludzku mówię przez wzgląd na słabość waszego ciała. Jak bowiem oddawaliście członki wasze na służbę nieczystości i nieprawości ku popełnianiu nieprawości, tak teraz oddawajcie członki wasze na służbę sprawiedliwości ku poświęceniu. Gdy bowiem byliście sługami grzechu, byliście dalecy od sprawiedliwości. Jakiż więc mieliście wtedy pożytek? Taki, którego się teraz wstydzicie, a końcem tego jest śmierć. Teraz zaś, wyzwoleni od grzechu, a oddani w służbę Bogu, macie pożytek w poświęceniu, a za cel żywot wieczny. Albowiem zapłatą za grzech jest śmierć, lecz darem łaski Bożej jest żywot wieczny w Chrystusie Jezusie, Panu naszym.'' \mbox{(Rz 6, 16-23)}
\end{quote}
\paragraph{}
\begin{quote}
,,Przeto, bracia moi, i wy umarliście dla zakonu przez ciało Chrystusowe, by należeć do innego, do tego, który został wzbudzony z martwych, abyśmy owoc wydawali dla Boga. Albowiem gdy byliśmy w ciele, grzeszne namiętności rozbudzone przez zakon były czynne w członkach naszych, aby rodzić owoce śmierci; Lecz teraz zostaliśmy uwolnieni od zakonu, gdy umarliśmy temu, przez co byliśmy opanowani, tak iż służymy w nowości ducha, a nie według przestarzałej litery.'' \mbox{(Rz 7, 4-6)}
\end{quote}
\paragraph{}
\begin{quote}
,,Znajduję tedy w sobie zakon, że gdy chcę czynić dobrze, trzyma się mnie złe; Bo według człowieka wewnętrznego mam upodobanie w zakonie Bożym. A w członkach swoich dostrzegam inny zakon, który walczy przeciwko zakonowi, uznanemu przez mój rozum i bierze mnie w niewolę zakonu grzechu, który jest w członkach moich. Nędzny ja człowiek! Któż mnie wybawi z tego ciała śmierci? Bogu niech będą dzięki przez Jezusa Chrystusa, Pana naszego! Tak więc ja sam służę umysłem zakonowi Bożemu, ciałem zaś zakonowi grzechu'' \mbox{(Rz 7, 21-25)}
\end{quote}
\paragraph{}
\begin{quote}
,,Bo ci, których Duch Boży prowadzi, są dziećmi Bożymi. Wszak nie wzięliście ducha niewoli, by znowu ulegać bojaźni, lecz wzięliście ducha synostwa, w którym wołamy: Abba, Ojcze! Ten to Duch świadczy wespół z duchem naszym, że dziećmi Bożymi jesteśmy. A jeśli dziećmi, to i dziedzicami, dziedzicami Bożymi, a współdziedzicami Chrystusa, jeśli tylko razem z nim cierpimy, abyśmy także razem z nim uwielbieni byli.'' \mbox{(Rz 8, 14-17)}
\end{quote}
\paragraph{}
\begin{quote}
,,Podobnie i my, gdy byliśmy dziećmi, byliśmy poddani w niewolę żywiołów tego świata; Lecz gdy nadeszło wypełnienie czasu, zesłał Bóg Syna swego, który się narodził z niewiasty i podlegał zakonowi, Aby wykupił tych, którzy byli pod zakonem, abyśmy usynowienia dostąpili. A ponieważ jesteście synami, przeto Bóg zesłał Ducha Syna swego do serc waszych, wołającego: Abba, Ojcze! Tak więc już nie jesteś niewolnikiem, lecz synem a jeśli synem, to i dziedzicem przez Boga.'' \mbox{(Ga 4, 3-7)}
\end{quote}
\paragraph{}
\begin{quote}
,,Albo czy nie wiecie, że ciało wasze jest świątynią Ducha Świętego, który jest w was i którego macie od Boga, i że nie należycie też do siebie samych? Drogoście bowiem kupieni. Wysławiajcie tedy Boga w ciele waszym.'' \mbox{(1 Kor 6, 19-20)}
\end{quote}
\paragraph{}
\begin{quote}
,,Chrystus wykupił nas od przekleństwa zakonu, stawszy się za nas przekleństwem, gdyż napisano: Przeklęty każdy, który zawisł na drzewie, Aby błogosławieństwo Abrahamowe przeszło na pogan w Jezusie Chrystusie, my zaś, abyśmy obiecanego Ducha otrzymali przez wiarę.'' \mbox{(Ga 3, 13-14)}
\end{quote}
\paragraph{}
\begin{quote}
,,A jeśli wzywacie jako Ojca tego, który bez względu na osobę sądzi każdego według uczynków jego, żyjcie w bojaźni przez czas pielgrzymowania waszego, Wiedząc, że nie rzeczami znikomymi, srebrem albo złotem, zostaliście wykupieni z marnego postępowania waszego, przez ojców wam przekazanego, Lecz drogą krwią Chrystusa, jako baranka niewinnego i nieskalanego.'' \mbox{(1 P 1, 17-19)}
\end{quote}
\end{document}