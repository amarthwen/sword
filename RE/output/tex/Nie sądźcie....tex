\documentclass[10pt,a4paper,oneside]{article}
\usepackage[utf8]{inputenc}
\usepackage{polski}
\usepackage[polish]{babel}
\usepackage[margin=0.5in,bottom=0.75in]{geometry}
\begin{document}
\centerline{\textbf{\MakeUppercase{Nie sądźcie...}}}
\begin{center}
\textbf{Wersety do studium:} \mbox{(Łk 6, 37-45)}, \mbox{(Rz 2, 1-6)}, \mbox{(Jk 2, 1-13)}
\end{center}
\section{Temat główny}
\paragraph{}
\begin{quote}
,,I nie sądźcie, a nie będziecie sądzeni, i nie potępiajcie, a nie będziecie potępieni, odpuszczajcie, a dostąpicie odpuszczenia. Dawajcie, a będzie wam dane; miarę dobrą, natłoczoną, potrzęsioną i przepełnioną dadzą w zanadrze wasze; albowiem jakim sądem sądzicie, takim was osądzą, i jaką miarą mierzycie, taką i wam odmierzą. Opowiedział im też podobieństwo: Czy może ślepy ślepego prowadzić? Czy obaj nie wpadną do dołu? Nie masz ucznia nad mistrza, ale należycie będzie przygotowany każdy, gdy będzie jak jego mistrz. A dlaczego widzisz źdźbło w oku brata swego, a belki w oku własnym nie dostrzegasz? Albo jak powiesz bratu swemu: Pozwól, że wyjmę źdźbło z oka twego, a oto belka jest w oku twoim? Obłudniku, wyjmij najpierw belkę z oka swego, a wtedy przejrzysz, aby wyjąć źdźbło z oka brata swego. Nie ma bowiem drzewa dobrego, które by rodziło owoc zły, ani też drzewa złego, które by rodziło owoc dobry. Każde bowiem drzewo poznaje się po jego owocu, bo nie zbierają z cierni fig ani winogron z głogu. Człowiek dobry z dobrego skarbca serca wydobywa dobro, a zły ze złego wydobywa zło; albowiem z obfitości serca mówią usta jego.'' \mbox{(Łk 6, 37-45)}
\end{quote}
\paragraph{}
\begin{quote}
,,Nie ma przeto usprawiedliwienia dla ciebie, kimkolwiek jesteś, człowiecze, który sądzisz; albowiem, sądząc drugiego, siebie samego potępiasz, ponieważ ty, sędzia, czynisz to samo. Bo wiemy, że sąd Boży słusznie spada na tych, którzy takie rzeczy czynią. Czy mniemasz, człowiecze, który osądzasz tych, co takie rzeczy czynią, a sam je czynisz, że ujdziesz sądu Bożego? Albo może lekceważysz bogactwo jego dobroci i cierpliwości, i pobłażliwości, nie zważając na to, że dobroć Boża do upamiętania cię prowadzi? Ty jednak przez zatwardziałość swoją i nieskruszone serce gromadzisz sobie gniew na dzień gniewu i objawienia sprawiedliwego sądu Boga, Który odda każdemu według uczynków jego:'' \mbox{(Rz 2, 1-6)}
\end{quote}
\paragraph{}
\begin{quote}
,,Bracia moi, nie czyńcie różnicy między osobami przy wyznawaniu wiary w Jezusa Chrystusa, naszego Pana chwały. Bo gdyby na wasze zgromadzenie przyszedł człowiek ze złotymi pierścieniami na palcach i we wspaniałej szacie, a przyszedłby też ubogi w nędznej szacie, A wy zwrócilibyście oczy na tego, który nosi wspaniałą szatę i powiedzielibyście: Ty usiądź tu wygodnie, a ubogiemu powiedzielibyście: Ty stań sobie tam lub usiądź u podnóżka mego, To czyż nie uczyniliście różnicy między sobą i nie staliście się sędziami, którzy fałszywie rozumują? Posłuchajcie, bracia moi umiłowani! Czyż to nie Bóg wybrał ubogich w oczach świata, aby byli bogatymi w wierze i dziedzicami Królestwa, obiecanego tym, którzy go miłują? Wy zaś wzgardziliście ubogim. Czyż nie bogacze ciemiężą was i nie oni ciągną was do sądów? Czy to nie oni zniesławiają zacne dobre imię, które zostało nad wami wezwane? Jeśli jednak wypełniacie zgodnie z Pismem królewskie przykazanie: Będziesz miłował bliźniego swego jak siebie samego, dobrze czynicie. Lecz jeśli czynicie różnicę między osobami, popełniacie grzech i jesteście uznani przez zakon za przestępców. Ktokolwiek bowiem zachowa cały zakon, a uchybi w jednym, stanie się winnym wszystkiego. Bo Ten, który powiedział : Nie cudzołóż, powiedział też: Nie zabijaj; jeżeli więc nie cudzołożysz, ale zabijasz, jesteś przestępcą zakonu. Tak mówcie i czyńcie, jak ci, którzy mają być sądzeni przez zakon wolności. Nad tym, który nie okazał miłosierdzia, odbywa się sąd bez miłosierdzia, miłosierdzie góruje nad sądem.'' \mbox{(Jk 2, 1-13)}
\end{quote}
\end{document}