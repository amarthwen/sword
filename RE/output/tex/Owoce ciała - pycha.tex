\documentclass[10pt,a4paper,oneside]{article}
\usepackage[utf8]{inputenc}
\usepackage{polski}
\usepackage[polish]{babel}
\usepackage[margin=0.5in,bottom=0.75in]{geometry}
\begin{document}
\centerline{\textbf{\MakeUppercase{Owoce ciała - pycha}}}
\begin{center}
\textbf{Wersety do studium:} \mbox{(Iz 2, 11-17)}, \mbox{(Iz 9, 7-12)}, \mbox{(Iz 10, 12-14)}, \mbox{(Iz 14, 11-15)}, \mbox{(Iz 16, 6-7)}, \mbox{(Dn 4, 26-29)}, \mbox{(Ab 1, 2-3)}, \mbox{(So 2, 8-10)}
\end{center}
\section{Temat główny}
\paragraph{}
\begin{quote}
,,Dumne oczy człowieka opuszczą się, a pycha ludzka zniży się, ale jedynie Pan wywyższy się w owym dniu. Bo dla Pana Zastępów nastanie dzień sądu nad wszystkim, co pyszne i wysokie, i nad wszystkim, co wyniosłe, aby było poniżone. Nad wszystkimi wysokimi i sterczącymi cedrami Libanu, i nad wszystkimi dębami Baszanu. Nad wszystkimi wysokimi górami i nad wszystkimi sterczącymi pagórkami. I nad każdą wysoką wieżą, i nad każdym murem obronnym. I nad wszystkimi okrętami tartezyjskimi, i nad wszystkimi kosztownymi statkami. I zniży się pycha człowieka, a wyniosłość ludzka opuści się, lecz jedynie Pan wywyższy się w owym dniu,'' \mbox{(Iz 2, 11-17)}
\end{quote}
\paragraph{}
\begin{quote}
,,Pan zesłał słowo przeciwko Jakubowi, a ono spadło na Izraela. I poczuł je cały lud, Efraim i mieszkańcy Samarii, którzy w pysze i w wyniosłości serca tak mówią: Mury z cegły runęły, wybudujemy je z kamienia ciosanego, sykomory wycięto, zasadzimy cedry. Lecz Pan pobudzi przeciw niemu jego wroga, Resyna, i podburzy jego nieprzyjaciół, Aramejczyków od wschodu, a Filistyńczyków od zachodu. I pożrą Izraela całą paszczą. Mimo to nie ustaje jego gniew, a jego ręka jeszcze jest wyciągnięta. Lecz lud nie nawraca się do tego, który go smaga, i Pana Zastępów nie szuka.'' \mbox{(Iz 9, 7-12)}
\end{quote}
\paragraph{}
\begin{quote}
,,I stanie się, że gdy Pan dokona całego swojego dzieła na górze Syjon i w Jeruzalemie, wtedy ukarze owoc pychy serce króla asyryjskiego i chełpliwą wyniosłość jego oczu. Rzekł bowiem: Uczyniłem to mocą swojej ręki i dzięki swojej mądrości, gdyż jestem rozumny; zniosłem granice ludów i zrabowałem ich skarby, i jak mocarz strąciłem siedzących na tronie. I sięgnęła moja ręka po bogactwo ludów jakby po gniazdo, a jak zagarniają opuszczone jajka, tak ja zagarnąłem całą ziemię, a nie było takiego, kto by poruszył skrzydłem albo otworzył dziób i pisnął.'' \mbox{(Iz 10, 12-14)}
\end{quote}
\paragraph{}
\begin{quote}
,,Twoją pychę i brzęk twoich lutni strącono do krainy umarłych. Twoim posłaniem zgnilizna, a robactwo twoim okryciem. O, jakże spadłeś z nieba, ty, gwiazdo jasna, synu jutrzenki! Powalony jesteś na ziemię, pogromco narodów! A przecież to ty mawiałeś w swoim sercu: Wstąpię na niebiosa, swój tron wyniosę ponad gwiazdy Boże i zasiądę na górze narad, na najdalszej północy. Wstąpię na szczyty obłoków, zrównam się z Najwyższym. A oto strącony jesteś do krainy umarłych, na samo dno przepaści.'' \mbox{(Iz 14, 11-15)}
\end{quote}
\paragraph{}
\begin{quote}
,,Słyszeliśmy o pysze Moabu, że jest bardzo pyszny, o jego wyniosłości, o jego dumie i złości, i że bezpodstawna jest jego chełpliwość. Dlatego niech narzekają Moabici nad Moabem, niech wszyscy narzekają, niech jęczą, z powodu rodzynkowych placków z Kir-Chareset zupełnie przybici.'' \mbox{(Iz 16, 6-7)}
\end{quote}
\paragraph{}
\begin{quote}
,,Gdy bowiem król Nebukadnesar po upływie dwunastu miesięcy przechadzał się po pałacu królewskim w Babilonie, Odezwał się król i rzekł: Czy to nie jest ów wielki Babilon, który zbudowałem na siedzibę króla dzięki potężnej mojej mocy i dla uświetnienia mojej wspaniałości? Gdy słowo to było jeszcze na ustach króla, zagrzmiał głos z nieba: Oznajmia ci się, królu Nebukadnesarze, że władza królewska zostaje ci odjęta. Będziesz wypędzony spośród ludzi, będziesz mieszkał ze zwierzętami polnymi, trawą jak bydło będą cię karmić i siedem wieków przejdzie nad tobą, aż poznasz, że Najwyższy ma władzę nad królestwem ludzkim i że je daje temu, komu chce.'' \mbox{(Dn 4, 26-29)}
\end{quote}
\paragraph{}
\begin{quote}
,,Oto uczyniłem cię małym wśród narodów, wzgardzony jesteś bardzo wśród ludzi. Pycha twojego serca zwiodła cię, który mieszkasz w rozpadlinach skalnych, swoją siedzibę umieściłeś wysoko, myśląc w swoim sercu: Kto mnie sprowadzi na ziemię?'' \mbox{(Ab 1, 2-3)}
\end{quote}
\paragraph{}
\begin{quote}
,,Słyszałem urąganie Moabitów i zniewagi synów Ammonowych, którymi urągali mojemu ludowi i wynosili się nad ich kraj Dlatego jako żyję Ja - mówi Pan Zastępów, Bóg Izraela - Moab stanie się jak Sodoma, a synowie Ammonowi jak Gomora, miejscem bujnych pokrzyw i dołem solnym, i pustynią po wszystkie czasy. Resztki mojego ludu złupią ich, a pozostali z mojego narodu posiądą ich. Spotka ich to za ich pychę, gdyż lżyli i wynosili się ponad lud Pana Zastępów.'' \mbox{(So 2, 8-10)}
\end{quote}
\end{document}