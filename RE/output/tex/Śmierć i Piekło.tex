\documentclass[10pt,a4paper,oneside]{article}
\usepackage[utf8]{inputenc}
\usepackage{polski}
\usepackage[polish]{babel}
\usepackage[margin=0.5in,bottom=0.75in]{geometry}
\begin{document}
\centerline{\textbf{\MakeUppercase{Śmierć i Piekło}}}
\begin{center}
\textbf{Wersety do studium:} 
\mbox{(Obj 1, 18)}, \mbox{(Obj 6, 8)}, \mbox{(Obj 20, 13-14)}, \mbox{(1Kor 15, 54-56)}
\end{center}
\paragraph{}
\begin{quote}
,,I żyjący. Byłem umarły, lecz oto żyję na wieki wieków i mam klucze śmierci i piekła.'' \mbox{(Obj 1, 18)}
\end{quote}
\paragraph{}
\begin{quote}
,,I widziałem, a oto siwy koń, a temu, który na nim siedział, było na imię Śmierć, a piekło szło za nim; i dano im władzę nad czwartą częścią ziemi, by zabijali mieczem i głodem, i morem, i przez dzikie zwierzęta ziemi.'' \mbox{(Obj 6, 8)}
\end{quote}
\paragraph{}
\begin{quote}
,,I wydało morze umarłych, którzy w nim się znajdowali, również śmierć i piekło wydały umarłych, którzy w nich się znajdowali, i byli osądzeni, każdy według uczynków swoich. I śmierć, i piekło zostały wrzucone do jeziora ognistego; owo jezioro ogniste, to druga śmierć.'' \mbox{(Obj 20, 13-14)}
\end{quote}
\paragraph{}
\begin{quote}
,,A gdy to, co skażone, przyoblecze się w to, co nieskażone, i to, co śmiertelne, przyoblecze się w nieśmiertelność, wtedy wypełni się słowo napisane: Pochłonięta jest śmierć w zwycięstwie! Gdzież jest, o śmierci, zwycięstwo twoje? Gdzież jest, o śmierci, żądło twoje? A żądłem śmierci jest grzech, a mocą grzechu jest zakon;'' \mbox{(1 Kor 15, 54-56)}
\end{quote}
\end{document}