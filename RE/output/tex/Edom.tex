\documentclass[10pt,a4paper,oneside]{article}
\usepackage[utf8]{inputenc}
\usepackage{polski}
\usepackage[polish]{babel}
\usepackage[margin=0.5in,bottom=0.75in]{geometry}
\begin{document}
\centerline{\textbf{\MakeUppercase{Edom}}}
\begin{center}
\textbf{Wersety do studium:} \mbox{(Iz 34, 1-10)}, \mbox{(Ml 1, 2-5)}, \mbox{(Ab 1, 5-6)}, \mbox{(Jr 49, 7-13)}, \mbox{(Iz 63, 1-6)}, \mbox{(Ha 3, 3-6)}, \mbox{(Am 9, 11-15)}
\end{center}
\section{Temat główny}
\subsection{Wykonanie wyroku na Edomie}
\paragraph{}
\begin{quote}
,,Zbliżcie się, narody, aby słuchać, i wy, ludy, słuchajcie uważnie! Niech słucha ziemia i to, co ją napełnia, okrąg ziemi i wszystko, co z niej wyrasta! Gdyż rozgniewał się Pan na wszystkie narody, jest oburzony na całe ich wojsko, obłożył je klątwą, przeznaczył je na rzeź. Ich zabici nie będą pochowani, a z trupów będzie się unosił smród; Góry rozmiękną od ich krwi. I wszystkie pagórki rozpłyną się. Niebiosa zwiną się jak zwój księgi i wszystkie ich zastępy opadną, jak opada liść z krzewu winnego i jak opada zwiędły liść z drzewa figowego. Gdyż mój miecz zwisa z nieba, oto spada na Edom i na lud obłożony przeze mnie klątwą. Miecz Pana ocieka krwią, pokryty jest tłuszczem, krwią jagniąt i kozłów, tłuszczem nerek baranów, gdyż krwawą ofiarę urządza Pan w Bosra, wielką rzeź w ziemi edomskiej. I padną wraz z nimi bawoły i cielce z tucznymi wołami; ich ziemia będzie przesiąknięta krwią, a ich proch przesycony tłuszczem. Bo jest to dzień pomsty Pana, rok odwetu za spór z Syjonem. Toteż potoki Edomu zamienią się w smołę, a jego glina w siarkę, a jego ziemia stanie się smołą gorejącą: Ani w nocy, ani w dzień nie zgaśnie, jego dym wznosić się będzie zawsze; z pokolenia w pokolenie będzie pustynią, po wiek wieków nikt nie będzie nią chodził.'' \mbox{(Iz 34, 1-10)}
\end{quote}
\subsection{Spustoszenie Edomu}
\paragraph{}
\begin{quote}
,,Umiłowałem was - mówi Pan - a wy mówicie: W czym okazałeś nam miłość? Czy Ezaw nie był bratem Jakuba? - mówi Pan - a Ja jednak umiłowałem Jakuba, Ezawa zaś znienawidziłem i spustoszyłem jego góry, i jego dziedzictwo wydałem na łup szakalom pustyni. Choćby Edom mówił: Jesteśmy wprawdzie rozbici, lecz znowu odbudujemy ruiny, to Pan Zastępów mówi: Oni odbudują, lecz Ja zburzę; i nazwą ich krajem bezeceństwa i ludem, na który Pan zawsze się gniewa. Wasze oczy będą to oglądać i sami wyznacie: Wielki jest Pan poza granicami Izraela.'' \mbox{(Ml 1, 2-5)}
\end{quote}
\paragraph{}
\begin{quote}
,,Gdy złodzieje wtargną do ciebie albo nocni rabusie jakże będziesz splądrowany! Czy nie będą kradli do woli? Gdy zbieracze winogron przyjdą do ciebie, czy pozostawią choć jedno grono? Jakże ogołocony jest Ezaw; przetrząśnięte są jego skarby ukryte.'' \mbox{(Ab 1, 5-6)}
\end{quote}
\paragraph{}
\begin{quote}
,,O Edomie. Tak mówi Pan Zastępów: Czy już nie ma mądrości w Temanie? Czy zginęła rada roztropnych? Czy wniwecz się obrócił ich dowcip? Uciekajcie, uchodźcie, ukryjcie się dobrze, mieszkańcy Dedanu, gdyż klęskę sprowadzę na Ezawa, czas jego kary! Gdy winogrodnicy przyjdą do ciebie, to nic po sobie nie pozostawią, gdy złodzieje w nocy - to wyrządzą szkodę, jaką tylko zechcą. Gdyż Ja sam obnażę Ezawa, odsłonię jego kryjówki, tak że nie zdoła się ukryć; potomstwo jego będzie wytępione wraz z jego braćmi i sąsiadami i nie będzie go. Zostaw swoje sieroty, Ja je będę utrzymywał, a twoje wdowy mnie niech zaufają! Gdyż tak mówi Pan: Oto ci, którzy nie byli skazani na to, by pić z tego kielicha, muszą pić, a ty chcesz ujść bezkarnie? Nie ujdziesz bezkarnie, lecz musisz pić. Gdyż przysiągłem na siebie samego - mówi Pan - że Bosra stanie się odstraszającym przykładem, hańbą, pustkowiem i przekleństwem, a wszystkie jej miasta ruiną po wszystkie czasy.'' \mbox{(Jr 49, 7-13)}
\end{quote}
\subsection{Powrót Pana Jezusa z Edomu}
\paragraph{}
\begin{quote}
,,Któż to przychodzi z Edomu, z Bosry w czerwonych szatach? Wspaniały On w swoim odzieniu, dumnie kroczy w pełni swojej siły. To Ja, który wyrokuję sprawiedliwie, mam moc wybawić. Skąd ta czerwień twojej szaty? A twoje odzienie jak u tego, który wytłacza wino w tłoczni? Ja sam tłoczyłem do kadzi, bo spośród ludów żadnego nie było ze mną; a tłoczyłem ich w gniewie i deptałem ich w zapalczywości tak, że ich sok pryskał na moją szatę i całe moje odzienie zbryzgałem. Ustanowiłem bowiem dzień pomsty i nadszedł rok mojego odkupienia. Rozglądałem się, lecz nie było pomocnika, zdziwiłem się, że nie było nikogo, kto by mię podparł. Lecz wtedy dopomogło mi moje ramię i wsparła mnie moja zapalczywość. Podeptałem więc ludy w moim gniewie i zdruzgotałem je w mojej zapalczywości, i sprawiłem, że ich sok spłynął na ziemię.'' \mbox{(Iz 63, 1-6)}
\end{quote}
\paragraph{}
\begin{quote}
,,Bóg przychodzi z Temanu, Święty z góry Paran. Sela. Jego wspaniałość okrywa niebiosa, a ziemia jest pełna jego chwały. Pod nim jest blask jak światłość, promienie wychodzą z jego rąk i tam jest ukryta jego moc. Przed nim idzie zaraza, a za nim podąża mór. Trzęsie się ziemia, gdy powstaje, gdy patrzy, drżą narody Pękają odwieczne góry, zapadają się prastare pagórki jego drogi są wieczne.'' \mbox{(Ha 3, 3-6)}
\end{quote}
\subsection{Izrael posiądzie resztki Edomu}
\paragraph{}
\begin{quote}
,,W owym dniu podniosę upadającą chatkę Dawida i zamuruję jej pęknięcia, i podźwignę ją z ruin, i odbuduję ją jak za dawnych dni, Aby posiedli resztki Edomu i wszystkie narody, nad którymi wzywane było moje Imię, mówi Pan, który to czyni. Oto idą dni, mówi Pan, w których oracz będzie przynaglał żniwiarza, a tłoczący winogrona siewcę ziarna; i góry będą ociekały moszczem, a wszystkie pagórki nim opływały. I odmienię los mojego ludu izraelskiego, tak że odbudują spustoszone miasta i osiedlą się w nich. Nasadzą winnice i będą pić ich wino, założą ogrody i będą jeść ich owoce. Zaszczepię ich na ich ziemi; i nikt ich już nie wyrwie z ich ziemi, którą im dałem - mówi Pan, twój Bóg.'' \mbox{(Am 9, 11-15)}
\end{quote}
\end{document}