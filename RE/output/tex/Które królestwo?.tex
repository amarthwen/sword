\documentclass[10pt,a4paper,oneside]{article}
\usepackage[utf8]{inputenc}
\usepackage{polski}
\usepackage[polish]{babel}
\usepackage[margin=0.5in,bottom=0.75in]{geometry}
\begin{document}
\centerline{\textbf{\MakeUppercase{Które królestwo?}}}
\begin{center}
\textbf{Wersety do studium:} \mbox{(Rz 11, 16-24)}, \mbox{(1Kor 6, 15-17)}, \mbox{(1Kor 15, 35-44)}, \mbox{(Ga 5, 16-17)}, \mbox{(1Kor 10, 21)}, \mbox{(Mt 20, 20-23)}, \mbox{(Łk 22, 39-42)}, \mbox{(Jk 3, 7-11)}, \mbox{(Jr 2, 10-13)}, \mbox{(J 4, 7-14)}, \mbox{(Łk 6, 45)}
\end{center}
\section{Temat główny}
\subsection{Który krzew?}
\paragraph{}
\begin{quote}
,,A jeśli zaczyn jest święty, to i ciasto; a jeśli korzeń jest święty, to i gałęzie. Jeśli zaś niektóre z gałęzi zostały odłamane, a ty, będąc gałązką z dzikiego drzewa oliwnego, zostałeś na ich miejsce wszczepiony i stałeś się uczestnikiem korzenia i tłuszczu oliwnego, To nie wynoś się nad gałęzie; a jeśli się chełpisz, to pamiętaj, że nie ty dźwigasz korzeń, lecz korzeń ciebie. Powiesz tedy: Odłamane zostały gałęzie, abym ja był wszczepiony. Słusznie! Odłamane zostały z powodu niewiary, ty zaś trwasz dzięki wierze; wzbijaj się w pychę, ale się strzeż. Jeśli bowiem Bóg nie oszczędził gałęzi naturalnych, nie oszczędził też ciebie. Zważ tedy na dobrotliwość i surowość Bożą surowość dla tych, którzy upadli, a dobrotliwość Bożą względem ciebie, o ile wytrwasz w dobroci, bo inaczej i ty będziesz odcięty. Ale i oni, jeżeli nie będą trwali w niewierze, zostaną wszczepieni, gdyż Bóg ma moc wszczepić ich ponownie. Bo jeżeli ty, odcięty z dzikiego z natury drzewa oliwnego, zostałeś wszczepiony wbrew naturze w szlachetne drzewo oliwne, o ileż pewniej zostaną wszczepieni w swoje drzewo oliwne ci, którzy z natury do niego należą.'' \mbox{(Rz 11, 16-24)}
\end{quote}
\subsection{Które ciało?}
\paragraph{}
\begin{quote}
,,Czy nie wiecie, że ciała wasze są członkami Chrystusowymi? Czy mam tedy wziąć członki Chrystusowe i uczynić je członkami wszetecznicy? Przenigdy! Albo czy nie wiecie, że, kto się łączy z wszetecznicą, jest z nią jednym ciałem? Albowiem, mówi Pismo, ci dwoje będą jednym ciałem. Kto zaś łączy się z Panem, jest z nim jednym duchem.'' \mbox{(1 Kor 6, 15-17)}
\end{quote}
\paragraph{}
\begin{quote}
,,Ale powie ktoś: Jak bywają wzbudzeni umarli? I w jakim ciele przychodzą? Niemądry! To, co siejesz, nie ożywa, jeśli nie umrze. A to, co siejesz, nie jest przecież tym ciałem, które ma powstać, lecz gołym ziarnem, może pszenicznym, a może jakimś innym; Ale Bóg daje mu ciało, jakie chce, a każdemu z nasion właściwe jemu ciało. Nie każde ciało jest jednakowe, bo inne jest ciało ludzkie, a inne zwierząt, jeszcze inne ciało ptaków, a inne ryb. I są ciała niebieskie i ciała ziemskie, lecz inny jest blask niebieskich, a inny ziemskich. Inny blask słońca, a inny blask księżyca, i inny blask gwiazd; bo gwiazda od gwiazdy różni się jasnością. Tak też jest ze zmartwychwstaniem. Co się sieje jako skażone, bywa wzbudzone nieskażone; Sieje się w niesławie, bywa wzbudzone w chwale; sieje się w słabości, bywa wzbudzone w mocy; Sieje się ciało cielesne, bywa wzbudzone ciało duchowe. Jeżeli jest ciało cielesne, to jest także ciało duchowe.'' \mbox{(1 Kor 15, 35-44)}
\end{quote}
\paragraph{}
\begin{quote}
,,Mówię więc: Według Ducha postępujcie, a nie będziecie pobłażali żądzy cielesnej. Gdyż ciało pożąda przeciwko Duchowi, a Duch przeciwko ciału, a te są sobie przeciwne, abyście nie czynili tego, co chcecie.'' \mbox{(Ga 5, 16-17)}
\end{quote}
\subsection{Który stół?}
\paragraph{}
\begin{quote}
,,Nie możecie pić kielicha Pańskiego i kielicha demonów; nie możecie być uczestnikami stołu Pańskiego i stołu demonów.'' \mbox{(1 Kor 10, 21)}
\end{quote}
\subsection{Który kielich?}
\paragraph{}
\begin{quote}
,,Nie możecie pić kielicha Pańskiego i kielicha demonów; nie możecie być uczestnikami stołu Pańskiego i stołu demonów.'' \mbox{(1 Kor 10, 21)}
\end{quote}
\paragraph{}
\begin{quote}
,,Wtedy przystąpiła do niego matka synów Zebedeuszowych z synami swoimi, złożyła mu pokłon i prosiła go o coś. A On jej rzekł: Czego chcesz? Rzecze mu: Powiedz, aby ci dwaj synowie moi zasiedli jeden po prawicy, a drugi po lewicy twojej w Królestwie twoim. A Jezus, odpowiadając, rzekł: Nie wiecie, o co prosicie. Czy możecie pić kielich, który ja pić będę? Mówią mu: Możemy. Mówi im; Kielich mój pić będziecie, ale zasiąść po prawicy mojej czy po lewicy - nie moja to rzecz, lecz Ojca mego, który da to tym, którym zostało przez niego przygotowane.'' \mbox{(Mt 20, 20-23)}
\end{quote}
\paragraph{}
\begin{quote}
,,I wyszedłszy, udał się według zwyczaju na Górę Oliwną; poszli też z nim uczniowie. A gdy przyszedł na miejsce, rzekł do nich: Módlcie się, aby nie popaść w pokuszenie. A sam oddalił się od nich, jakby na rzut kamienia, i padłszy na kolana, modlił się. Mówiąc: Ojcze, jeśli chcesz, oddal ten kielich ode mnie; wszakże nie moja, lecz twoja wola niech się stanie.'' \mbox{(Łk 22, 39-42)}
\end{quote}
\subsection{Które źródło?}
\paragraph{}
\begin{quote}
,,Bo wszelki rodzaj dzikich zwierząt i ptaków, płazów i stworzeń morskich może być ujarzmiony i został ujarzmiony przez rodzaj ludzki. Natomiast nikt z ludzi nie może ujarzmić języka, tego krnąbrnego zła, pełnego śmiercionośnego jadu. Nim wysławiamy Pana i Ojca i nim przeklinamy ludzi, stworzonych na podobieństwo Z tych samych ust wychodzi błogosławieństwo i przekleństwo. Tak, bracia moi, być nie powinno. Czy źródło wydaje z tego samego otworu wodę słodką i gorzką?'' \mbox{(Jk 3, 7-11)}
\end{quote}
\paragraph{}
\begin{quote}
,,Udajcie się więc na wyspy cytyjskie i spójrzcie, poślijcie do Kedareńczyków, dobrze uważajcie i przypatrzcie się, czy stało się coś podobnego; Czy jakiś naród zmienił swoich bogów, chociaż to nie są bogowie? Wszakże mój lud zamienił swoją chwałę na kogoś, kto nie może pomóc. Przeraźcie się, niebiosa, nad tym, zadrżyjcie i zatrwóżcie się bardzo! - mówi Pan - Gdyż mój lud popełnił dwojakie zło: Mnie, źródło wód żywych, opuścili, a wykopali sobie cysterny, cysterny dziurawe, które wody zatrzymać nie mogą.'' \mbox{(Jr 2, 10-13)}
\end{quote}
\paragraph{}
\begin{quote}
,,Wtem przyszła niewiasta samarytańska, aby nabrać wody. Jezus rzekł do niej: Daj mi pić! Uczniowie jego bowiem poszli do miasta, by nakupić żywności. Wtedy niewiasta samarytańska rzekła do niego: Jakże Ty, będąc Żydem, prosisz mnie, Samarytankę, o wodę? (Żydzi bowiem nie obcują z Samarytanami). Odpowiadając jej Jezus, rzekł do niej: Gdybyś znała dar Boży i tego, który mówi do ciebie: Daj mi pić, wtedy sama prosiłabyś go, i dałby ci wody żywej. Mówi do niego: Panie, nie masz nawet czerpaka, a studnia jest głęboka; skądże więc masz tę wodę żywą? Czy może Ty jesteś większy od ojca naszego Jakuba, który dał nam tę studnię i sam z niej pił, i synowie jego, i trzody jego? Odpowiedział jej Jezus, mówiąc: Każdy, kto piję tę wodę, znowu pragnąć będzie; Ale kto napije się wody, którą Ja mu dam, nie będzie pragnął na wieki, lecz woda, którą Ja mu dam, stanie się w nim źródłem wody wytryskującej ku żywotowi wiecznemu.'' \mbox{(J 4, 7-14)}
\end{quote}
\subsection{Który skarb?}
\paragraph{}
\begin{quote}
,,Człowiek dobry z dobrego skarbca serca wydobywa dobro, a zły ze złego wydobywa zło; albowiem z obfitości serca mówią usta jego.'' \mbox{(Łk 6, 45)}
\end{quote}
\end{document}