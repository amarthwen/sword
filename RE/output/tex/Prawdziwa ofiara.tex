\documentclass[10pt,a4paper,oneside]{article}
\usepackage[utf8]{inputenc}
\usepackage{polski}
\usepackage[polish]{babel}
\usepackage[margin=0.5in,bottom=0.75in]{geometry}
\begin{document}
\centerline{\textbf{\MakeUppercase{Prawdziwa ofiara}}}
\begin{center}
\textbf{Wersety do studium:} 
\mbox{(Ps 51, 18-19)}, \mbox{(Ps 50, 14.23)}, \mbox{(Jr 7, 21-27)}, \mbox{(1Sm 15, 22)}, \mbox{(Oz 6, 6)}, \mbox{(Mt 23, 23)}, \mbox{(Prz 21, 3)}, \mbox{(Mt 9, 13)}, \mbox{(J 4, 23-24)}, \mbox{(Am 5, 21-24)}, \mbox{(Iz 1, 14-20)}, \mbox{(Mi 6, 6-8)}, \mbox{(Ez 20, 41-42)}
\end{center}
\section{Temat główny}
\paragraph{}
\begin{quote}
,,Albowiem ofiar nie żądasz, A całopalenia, choćbym ci je dał, nie zechcesz przyjąć. Ofiarą Bogu miłą jest duch skruszony, Sercem skruszonym i zgnębionym nie wzgardzisz, Boże.'' \mbox{(Ps 51, 18-19)}
\end{quote}
\paragraph{}
\begin{quote}
,,Ofiaruj Bogu dziękczynienie I spełnij Najwyższemu śluby swoje! (\ldots) Kto ofiaruje dziękczynienie, czci mnie, A temu, kto nienagannie postępuje, ukażę zbawienie Boże.'' \mbox{(Ps 50, 14.23)}
\end{quote}
\paragraph{}
\begin{quote}
,,Tak mówi Pan Zastępów, Bóg Izraela: Dodajcie swoje całopalenia do swoich krwawych ofiar i jedzcie mięso, Gdyż nic nie powiedziałem waszym ojcom, gdy ich wyprowadzałem z ziemi egipskiej, nic też nie nakazałem im w sprawie całopaleń i ofiar krwawych, Lecz tylko to przykazanie dałem im, mówiąc: Słuchajcie mojego głosu, a Ja będę waszym Bogiem, wy zaś będziecie moim ludem, postępujcie całkowicie tą drogą, którą wam nakazuję, aby się wam dobrze działo! Lecz oni nie usłuchali i nie nakłonili swojego ucha, ale postępowali według skłonności i uporu swojego złego serca i odwrócili się do mnie tyłem, a nie twarzą. Od tego dnia, gdy wasi ojcowie wyszli z ziemi egipskiej, aż do dnia dzisiejszego, posyłałem do was nieprzerwanie i nieustannie swoje sługi, proroków. Lecz oni nie usłuchali mnie ani nie nakłonili swojego ucha, ale usztywnili swój kark i postępowali gorzej niż ich ojcowie. Gdy więc będziesz mówił do nich wszystkie te słowa, oni cię nie usłuchają, gdy będziesz do nich wołał, oni ci nie odpowiedzą.'' \mbox{(Jr 7, 21-27)}
\end{quote}
\paragraph{}
\begin{quote}
,,Samuel odpowiedział: Czy takie ma Pan upodobanie w całopaleniach i w rzeźnych ofiarach, co w posłuszeństwie dla głosu Pana? Oto: Posłuszeństwo lepsze jest niż ofiara, a uważne słuchanie lepsze niż tłuszcz barani.'' \mbox{(1 Sm 15, 22)}
\end{quote}
\paragraph{}
\begin{quote}
,,Gdyż miłości chcę, a nie ofiary, i poznania Boga, nie całopaleń.'' \mbox{(Oz 6, 6)}
\end{quote}
\paragraph{}
\begin{quote}
,,Biada wam, uczeni w Piśmie i faryzeusze, obłudnicy, że dajecie dziesięcinę z mięty i z kopru, i z kminku, a zaniedbaliście tego, co ważniejsze w zakonie: sprawiedliwości, miłosierdzia i wierności; te rzeczy należało czynić, a tamtych nie zaniedbywać.'' \mbox{(Mt 23, 23)}
\end{quote}
\paragraph{}
\begin{quote}
,,Wykonywanie sprawiedliwości i prawa milsze jest Panu niż krwawa ofiara.'' \mbox{(Prz 21, 3)}
\end{quote}
\paragraph{}
\begin{quote}
,,Idźcie i nauczcie się, co to znaczy: Miłosierdzia chcę, a nie ofiary. Nie przyszedłem bowiem wzywać sprawiedliwych, lecz grzeszników.'' \mbox{(Mt 9, 13)}
\end{quote}
\paragraph{}
\begin{quote}
,,Lecz nadchodzi godzina i teraz jest, kiedy prawdziwi czciciele będą oddawali Ojcu cześć w duchu i w prawdzie; bo i Ojciec takich szuka, którzy by mu tak cześć oddawali. Bóg jest duchem, a ci, którzy mu cześć oddają, winni mu ją oddawać w duchu i w prawdzie.'' \mbox{(J 4, 23-24)}
\end{quote}
\paragraph{}
\begin{quote}
,,Nienawidzę waszych świąt, gardzę nimi, i nie podobają mi się wasze uroczystości świąteczne. Nawet gdy mi składacie ofiary całopalne i ofiary z pokarmów, nie mam w nich upodobania, a na ofiary pojednania z tłustych waszych cieląt nie mogę patrzeć. Usuń ode mnie wrzask twoich pieśni! I nie chcę słyszeć brzęku twoich harf. Niech raczej prawo tryska jak woda, a sprawiedliwość jak potok nie wysychający!'' \mbox{(Am 5, 21-24)}
\end{quote}
\paragraph{}
\begin{quote}
,,Waszych nowiów i świąt nienawidzi moja dusza, stały mi się ciężarem, zmęczyłem się znosząc je. A gdy wyciągacie swoje ręce, zakrywam moje oczy przed wami, choćbyście pomnożyli wasze modlitwy, nie wysłucham was, bo na waszych rękach pełno krwi. Obmyjcie się, oczyśćcie się, usuńcie wasze złe uczynki sprzed moich oczu, przestańcie źle czynić! Uczcie się dobrze czynić, przestrzegajcie prawa, brońcie pokrzywdzonego, wymierzajcie sprawiedliwość sierocie, wstawiajcie się za wdową! Chodźcie więc, a będziemy się prawować - mówi Pan! Choć wasze grzechy będą czerwone jak szkarłat, jak śnieg zbieleją; choć będą czerwone jak purpura, staną się białe jak wełna. Jeżeli zechcecie być posłuszni, z dóbr ziemi będziecie spożywać, Lecz jeżeli będziecie się wzbraniać i trwać w uporze, miecz was pożre, bo usta Pana tak powiedziały!'' \mbox{(Iz 1, 14-20)}
\end{quote}
\paragraph{}
\begin{quote}
,,Z czym mam wystąpić przed Panem, pokłonić się Bogu Najwyższemu? Czy mam wystąpić przed nim z całopaleniami, z rocznymi cielętami? Czy Pan ma upodobanie w tysiącach baranów, w dziesiątkach tysięcy strumieni oliwy? Czy mam dać swojego pierworodnego za swoje przestępstwo, własne dziecko na oczyszczenie mojego grzechu? Oznajmiono ci, człowiecze, co jest dobre i czego Pan żąda od ciebie: tylko, abyś wypełniał prawo, okazywał miłość bratnią i w pokorze obcował ze swoim Bogiem.'' \mbox{(Mi 6, 6-8)}
\end{quote}
\paragraph{}
\begin{quote}
,,Przyjmę was łaskawie jako przyjemnie woniejącą ofiarę, gdy was wyprowadzę spośród ludów i zgromadzę was z ziem, w których zostaliście rozproszeni; i dowiodę na was mojej świętości na oczach narodów. Wtedy poznacie, że Ja jestem Pan, gdy zaprowadzę was do ziemi izraelskiej, do ziemi, którą przysiągłem dać waszym ojcom.'' \mbox{(Ez 20, 41-42)}
\end{quote}
\end{document}