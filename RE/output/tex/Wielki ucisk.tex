\documentclass[10pt,a4paper,oneside]{article}
\usepackage[utf8]{inputenc}
\usepackage{polski}
\usepackage[polish]{babel}
\usepackage[margin=0.5in,bottom=0.75in]{geometry}
\begin{document}
\centerline{\textbf{\MakeUppercase{Wielki ucisk}}}
\begin{center}
\textbf{Wersety do studium:} \mbox{(1Moj 4, 6-7)}, \mbox{(Ef 6, 10-12)}, \mbox{(Rz 5, 1-5)}
\end{center}
\section{Temat główny}
\paragraph{}
\begin{quote}
,,I rzekł Pan do Kaina: Czemu się gniewasz i czemu zasępiło się twoje oblicze? Wszak byłoby pogodne, gdybyś czynił dobrze, a jeśli nie będziesz czynił dobrze, u drzwi czyha grzech. Kusi cię, lecz ty masz nad nim panować.'' \mbox{(1 Moj 4, 6-7)}
\end{quote}
\paragraph{}
\begin{quote}
,,W końcu, bracia moi, umacniajcie się w Panu i w potężnej mocy jego. Przywdziejcie całą zbroję Bożą, abyście mogli ostać się przed zasadzkami diabelskimi. Gdyż bój toczymy nie z krwią i z ciałem, lecz z nadziemskimi władzami, ze zwierzchnościami, z władcami tego świata ciemności, ze złymi duchami w okręgach niebieskich.'' \mbox{(Ef 6, 10-12)}
\end{quote}
\paragraph{}
\begin{quote}
,,Usprawiedliwieni tedy z wiary, pokój mamy z Bogiem przez Pana naszego, Jezusa Chrystusa, Dzięki któremu też mamy dostęp przez wiarę do tej łaski, w której stoimy, i chlubimy się nadzieją chwały Bożej. A nie tylko to, chlubimy się też z ucisków, wiedząc, że ucisk wywołuje cierpliwość, A cierpliwość doświadczenie, doświadczenie zaś nadzieję; A nadzieja nie zawodzi, bo miłość Boża rozlana jest w sercach naszych przez Ducha Świętego, który nam jest dany.'' \mbox{(Rz 5, 1-5)}
\end{quote}
\end{document}