\documentclass[10pt,a4paper,oneside]{article}
\usepackage[utf8]{inputenc}
\usepackage{polski}
\usepackage[polish]{babel}
\usepackage[margin=0.5in,bottom=0.75in]{geometry}
\begin{document}
\centerline{\textbf{\MakeUppercase{Cielesny a duchowy żołnierz}}}
\begin{center}
\textbf{Wersety do studium:} \mbox{(Ef 6, 11-12)}, \mbox{(Ef 6, 13-17)}, \mbox{(1Kor 6, 12-13)}, \mbox{(Ga 5, 1.13)}
\end{center}
\section{Temat główny}
\subsection{Cielesny żołnierz}
\subsubsection{Cielesny wróg}
\subsubsection{Cielesna zbroja i oręż}
\subsubsection{Cielesna walka}
\subsubsection{Cielesna niewola}
\subsection{Duchowy żołnierz}
\subsubsection{Duchowy wróg}
\paragraph{}
\begin{quote}
,,Przywdziejcie całą zbroję Bożą, abyście mogli ostać się przed zasadzkami diabelskimi. Gdyż bój toczymy nie z krwią i z ciałem, lecz z nadziemskimi władzami, ze zwierzchnościami, z władcami tego świata ciemności, ze złymi duchami w okręgach niebieskich.'' \mbox{(Ef 6, 11-12)}
\end{quote}
\subsubsection{Duchowa zbroja i oręż}
\paragraph{}
\begin{quote}
,,Dlatego weźcie całą zbroję Bożą, abyście mogli stawić opór w dniu złym i, dokonawszy wszystkiego, ostać się. Stójcie tedy, opasawszy biodra swoje prawdą, przywdziawszy pancerz sprawiedliwości I obuwszy nogi, by być gotowymi do zwiastowania ewangelii pokoju, A przede wszystkim, weźcie tarczę wiary, którą będziecie mogli zgasić wszystkie ogniste pociski złego; Weźcie też przyłbicę zbawienia i miecz Ducha, którym jest Słowo Boże.'' \mbox{(Ef 6, 13-17)}
\end{quote}
\subsubsection{Duchowa walka}
\paragraph{}
\begin{quote}
,,Przywdziejcie całą zbroję Bożą, abyście mogli ostać się przed zasadzkami diabelskimi. Gdyż bój toczymy nie z krwią i z ciałem, lecz z nadziemskimi władzami, ze zwierzchnościami, z władcami tego świata ciemności, ze złymi duchami w okręgach niebieskich.'' \mbox{(Ef 6, 11-12)}
\end{quote}
\subsubsection{Duchowa niewola}
\paragraph{}
\begin{quote}
,,Wszystko mi wolno, ale nie wszystko jest pożyteczne. Wszystko mi wolno, lecz ja nie dam się niczym zniewolić. Pokarm jest dla brzucha, a brzuch jest dla pokarmów; ale Bóg zniweczy jedno i drugie. Ciało zaś jest nie dla wszeteczeństwa, lecz dla Pana, a Pan dla ciała.'' \mbox{(1 Kor 6, 12-13)}
\end{quote}
\paragraph{}
\begin{quote}
,,Chrystus wyzwolił nas, abyśmy w tej wolności żyli. Stójcie więc niezachwianie i nie poddawajcie się znowu pod jarzmo niewoli. (\ldots) Bo wy do wolności powołani zostaliście, bracia; tylko pod pozorem tej wolności nie pobłażajcie ciału, ale służcie jedni drugim w miłości.'' \mbox{(Ga 5, 1.13)}
\end{quote}
\end{document}