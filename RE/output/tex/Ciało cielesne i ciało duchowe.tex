\documentclass[10pt,a4paper,oneside]{article}
\usepackage[utf8]{inputenc}
\usepackage{polski}
\usepackage[polish]{babel}
\usepackage[margin=0.5in,bottom=0.75in]{geometry}
\begin{document}
\centerline{\textbf{\MakeUppercase{Ciało cielesne i ciało duchowe}}}
\begin{center}
\textbf{Wersety do studium:} \mbox{(J 12, 24-25)}, \mbox{(1Kor 15, 35-44)}, \mbox{(Ga 5, 13.16-25; 6, 8)}
\end{center}
\section{Temat główny}
\subsection{Ciało cielesne i ciało duchowe}
\paragraph{}
\begin{quote}
,,Zaprawdę, zaprawdę powiadam wam, jeśli ziarnko pszeniczne, które wpadło do ziemi, nie obumrze, pojedynczym ziarnem zostaje, lecz jeśli obumrze, obfity owoc wydaje. Kto miłuje życie swoje, utraci je, a kto nienawidzi życia swego na tym świecie, zachowa je ku żywotowi wiecznemu.'' \mbox{(J 12, 24-25)}
\end{quote}
\paragraph{}
\begin{quote}
,,Ale powie ktoś: Jak bywają wzbudzeni umarli? I w jakim ciele przychodzą? Niemądry! To, co siejesz, nie ożywa, jeśli nie umrze. A to, co siejesz, nie jest przecież tym ciałem, które ma powstać, lecz gołym ziarnem, może pszenicznym, a może jakimś innym; Ale Bóg daje mu ciało, jakie chce, a każdemu z nasion właściwe jemu ciało. Nie każde ciało jest jednakowe, bo inne jest ciało ludzkie, a inne zwierząt, jeszcze inne ciało ptaków, a inne ryb. I są ciała niebieskie i ciała ziemskie, lecz inny jest blask niebieskich, a inny ziemskich. Inny blask słońca, a inny blask księżyca, i inny blask gwiazd; bo gwiazda od gwiazdy różni się jasnością. Tak też jest ze zmartwychwstaniem. Co się sieje jako skażone, bywa wzbudzone nieskażone; Sieje się w niesławie, bywa wzbudzone w chwale; sieje się w słabości, bywa wzbudzone w mocy; Sieje się ciało cielesne, bywa wzbudzone ciało duchowe. Jeżeli jest ciało cielesne, to jest także ciało duchowe.'' \mbox{(1 Kor 15, 35-44)}
\end{quote}
\subsection{Przeciwieństwo między Duchem a ciałem}
\paragraph{}
\begin{quote}
,,Bo wy do wolności powołani zostaliście, bracia; tylko pod pozorem tej wolności nie pobłażajcie ciału, ale służcie jedni drugim w miłości. (\ldots) Mówię więc: Według Ducha postępujcie, a nie będziecie pobłażali żądzy cielesnej. Gdyż ciało pożąda przeciwko Duchowi, a Duch przeciwko ciału, a te są sobie przeciwne, abyście nie czynili tego, co chcecie. A jeśli Duch was prowadzi, nie jesteście pod zakonem. Jawne zaś są uczynki ciała, mianowicie: wszeteczeństwo, nieczystość, rozpusta, Bałwochwalstwo, czary, wrogość, spór, zazdrość, gniew, knowania, waśnie, odszczepieństwo Zabójstwa, pijaństwo, obżarstwo i tym podobne; o tych zapowiadam wam, jak już przedtem zapowiedziałem, że ci, którzy te rzeczy czynią, Królestwa Bożego nie odziedziczą. Owocem Ducha zaś są: miłość, radość, pokój, cierpliwość, uprzejmość, dobroć, wierność, Łagodność, wstrzemięźliwość. Przeciwko takim nie ma zakonu. A ci, którzy należą do Chrystusa Jezusa, ukrzyżowali ciało swoje wraz z namiętnościami i żądzami. Jeśli według Ducha żyjemy, według Ducha też postępujmy. (\ldots) Bo kto sieje dla ciała swego, z ciała żąć będzie skażenie, a kto sieje dla Ducha, z Ducha żąć będzie żywot wieczny.'' \mbox{(Ga 5, 13.16-25; 6, 8)}
\end{quote}
\end{document}