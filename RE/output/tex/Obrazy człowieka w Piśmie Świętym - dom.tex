\documentclass[10pt,a4paper,oneside]{article}
\usepackage[utf8]{inputenc}
\usepackage{polski}
\usepackage[polish]{babel}
\usepackage[margin=0.5in,bottom=0.75in]{geometry}
\begin{document}
\centerline{\textbf{\MakeUppercase{Obrazy człowieka w Piśmie Świętym - dom}}}
\begin{center}
\textbf{Wersety do studium:} 
\mbox{(Łk 6, 47-49)}, \mbox{(Łk 11, 24-26)}, \mbox{(Obj 3, 20)}, \mbox{(1Moj 4, 3-7)}, \mbox{(1Kor 3, 10-15)}, \mbox{(Jk 4, 5)}, \mbox{(Rz 7, 14-20)}
\end{center}
\section{Temat główny}
\paragraph{}
\begin{quote}
,,Pokażę wam, do kogo jest podobny każdy, kto przychodzi do mnie i słucha słów moich, i czyni je. Podobny jest do człowieka budującego dom, który kopał i dokopał się głęboko, i założył fundament na skale. A gdy przyszła powódź, uderzyły wody o ów dom, ale nie mogły go poruszyć, bo był dobrze zbudowany. Kto zaś słucha, a nie czyni, podobny jest do człowieka, który zbudował dom na ziemi bez fundamentu, i uderzyły weń wody, i wnet runął, a upadek domu owego był zupełny.'' \mbox{(Łk 6, 47-49)}
\end{quote}
\paragraph{}
\begin{quote}
,,Gdy duch nieczysty wyjdzie z człowieka, wędruje po miejscach bezwodnych, szukając ukojenia, a gdy nie znajdzie, mówi: Wrócę do domu swego, skąd wyszedłem. I przyszedłszy, zastaje go wymiecionym i przyozdobionym. Wówczas idzie i zabiera z sobą siedem innych duchów, gorszych niż on, i wchodzą, i mieszkają tam. I bywa końcowy stan człowieka tego, gorszy niż pierwotny.'' \mbox{(Łk 11, 24-26)}
\end{quote}
\paragraph{}
\begin{quote}
,,Oto stoję u drzwi i kołaczę; jeśli ktoś usłyszy głos mój i otworzy drzwi, wstąpię do niego i będę z nim wieczerzał, a on ze mną.'' \mbox{(Obj 3, 20)}
\end{quote}
\paragraph{}
\begin{quote}
,,Po niejakim czasie Kain złożył Panu ofiarę z plonów rolnych; Abel także złożył ofiarę z pierworodnych trzody swojej i z tłuszczu ich. A Pan wejrzał na Abla i na jego ofiarę. Ale na Kaina i na jego ofiarę nie wejrzał; wtedy Kain rozgniewał się bardzo i zasępiło się jego oblicze. I rzekł Pan do Kaina: Czemu się gniewasz i czemu zasępiło się twoje oblicze? Wszak byłoby pogodne, gdybyś czynił dobrze, a jeśli nie będziesz czynił dobrze, u drzwi czyha grzech. Kusi cię, lecz ty masz nad nim panować.'' \mbox{(1 Moj 4, 3-7)}
\end{quote}
\paragraph{}
\begin{quote}
,,Według łaski Bożej, która mi jest dana, jako mądry budowniczy założyłem fundament, a inny na nim buduje. Każdy zaś niechaj baczy, jak na nim buduje. Albowiem fundamentu innego nikt nie może założyć oprócz tego, który jest założony, a którym jest Jezus Chrystus. A czy ktoś na tym fundamencie wznosi budowę ze złota, srebra, drogich kamieni, z drzewa, siana, słomy, To wyjdzie na jaw w jego dziele; dzień sądny bowiem to pokaże, gdyż w ogniu się objawi, a jakie jest dzieło każdego, wypróbuje ogień. Jeśli czyjeś dzieło, zbudowane na tym fundamencie, się ostoi, ten zapłatę odbierze; Jeśli czyjeś dzieło spłonie, ten szkodę poniesie, lecz on sam zbawiony będzie, tak jednak, jak przez ogień.'' \mbox{(1 Kor 3, 10-15)}
\end{quote}
\paragraph{}
\begin{quote}
,,Albo czy sądzicie, że na próżno Pismo mówi: Zazdrośnie chce On mieć tylko dla siebie ducha, któremu dał w nas mieszkanie?'' \mbox{(Jk 4, 5)}
\end{quote}
\paragraph{}
\begin{quote}
,,Wiemy bowiem, że zakon jest duchowy, ja zaś jestem cielesny, zaprzedany grzechowi. Albowiem nie rozeznaję się w tym, co czynię; gdyż nie to czynię, co chcę, ale czego nienawidzę, to czynię. A jeśli to czynię, czego nie chcę, zgadzam się z tym, że zakon jest dobry. Ale wtedy czynię to już nie ja, lecz grzech, który mieszka we mnie. Wiem tedy, że nie mieszka we mnie, to jest w ciele moim, dobro; mam bowiem zawsze dobrą wolę, ale wykonania tego, co dobre, brak; Albowiem nie czynię dobrego, które chcę, tylko złe, którego nie chcę, to czynię. A jeśli czynię to, czego nie chcę, już nie ja to czynię, ale grzech, który mieszka we mnie.'' \mbox{(Rz 7, 14-20)}
\end{quote}
\end{document}