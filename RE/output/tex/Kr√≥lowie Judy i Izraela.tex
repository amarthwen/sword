\documentclass[10pt,a4paper,oneside]{article}
\usepackage[utf8]{inputenc}
\usepackage{polski}
\usepackage[polish]{babel}
\usepackage[margin=0.5in,bottom=0.75in]{geometry}
\begin{document}
\centerline{\textbf{\MakeUppercase{Królowie Judy i Izraela}}}
\begin{center}
\textbf{Wersety do studium:} 
\mbox{(1Sm 13, 1)}, \mbox{(1Krl 2, 11)}, \mbox{(1Krl 11, 42)}, \mbox{(1Krl 14, 21)}, \mbox{(1Krl 15, 1-2)}, \mbox{(1Krl 15, 9-10)}, \mbox{(1Krl 22, 41-42)}, \mbox{(2Krl 8, 16-17)}, \mbox{(2Krl 8, 25-26)}, \mbox{(2Krl 12, 1-2)}, \mbox{(2Krl 14, 1-2)}, \mbox{(2Krl 15, 1-2)}, \mbox{(2Krl 15, 32-33)}, \mbox{(2Krl 16, 1-2)}, \mbox{(2Krl 18, 1-2)}, \mbox{(2Krl 21, 1)}, \mbox{(2Krl 21, 19)}, \mbox{(2Krl 22, 1)}, \mbox{(2Krl 23, 31)}, \mbox{(2Krl 23, 36)}, \mbox{(2Krl 24, 8)}, \mbox{(2Krl 24, 18)}, \mbox{(1Krl 14, 20)}, \mbox{(1Krl 15, 25)}, \mbox{(1Krl 15, 33)}, \mbox{(1Krl 16, 8)}, \mbox{(1Krl 16, 15)}, \mbox{(1Krl 16, 23)}, \mbox{(1Krl 16, 29)}, \mbox{(1Krl 22, 52)}, \mbox{(2Krl 3, 1)}, \mbox{(2Krl 10, 36)}, \mbox{(2Krl 13, 1)}, \mbox{(2Krl 13, 10)}, \mbox{(2Krl 14, 23)}, \mbox{(2Krl 15, 8)}, \mbox{(2Krl 15, 13)}, \mbox{(2Krl 15, 17)}, \mbox{(2Krl 15, 23)}, \mbox{(2Krl 15, 27)}, \mbox{(2Krl 17, 1)}
\end{center}
\section{Królowie przed podziałem Izraela}
\subsection{Saul}
\paragraph{}
\begin{quote}
,,Saul miał pięćdziesiąt lat, gdy został królem, i dwadzieścia dwa lata panował nad Izraelem.'' \mbox{(1 Sm 13, 1)}
\end{quote}
\subsection{Dawid}
\paragraph{}
\begin{quote}
,,Czterdzieści lat panował Dawid nad Izraelem. W Hebronie panował siedem lat, a w Jeruzalemie panował trzydzieści trzy lata.'' \mbox{(1 Krl 2, 11)}
\end{quote}
\subsection{Salomon}
\paragraph{}
\begin{quote}
,,Salomon panował jako król w Jeruzalemie nad całym Izraelem czterdzieści lat.'' \mbox{(1 Krl 11, 42)}
\end{quote}
\section{Królowie po podziale Izraela}
\subsection{Królestwo Judy}
\subsubsection{Rechabeam}
\paragraph{}
\begin{quote}
,,Rechabeam zaś, syn Salomona, panował w Judzie. Czterdzieści jeden lat miał Rechabeam, gdy objął władzę królewską, a siedemnaście lat panował w Jeruzalemie, mieście, które wybrał Pan spośród wszystkich plemion izraelskich, aby tam złożyć swoje imię. Matka zaś jego nazywała się Naama, a była Ammonitką.'' \mbox{(1 Krl 14, 21)}
\end{quote}
\subsubsection{Abijjam}
\paragraph{}
\begin{quote}
,,Abijjam objął władzę królewską nad Judą w osiemnastym roku panowania Jeroboama, syna Nebata, A sprawował ją w Jeruzalemie trzy lata. Matka zaś jego nazywała się Maacha, a była córką Abszaloma.'' \mbox{(1 Krl 15, 1-2)}
\end{quote}
\subsubsection{Asa}
\paragraph{}
\begin{quote}
,,Asa objął władzę królewską nad Judą w dwudziestym roku panowania Jeroboama, króla izraelskiego. A sprawował ją w Jeruzalemie czterdzieści jeden lat; babka zaś jego nazywała się Maacha, a była córką Abszaloma.'' \mbox{(1 Krl 15, 9-10)}
\end{quote}
\subsubsection{Jehoszafat}
\paragraph{}
\begin{quote}
,,A Jehoszafat, syn Asy, objął władzę królewską nad Judą w czwartym roku panowania Achaba, króla izraelskiego. Jehoszafat miał trzydzieści pięć lat, gdy został królem, a w Jeruzalemie panował dwadzieścia pięć lat. Matka jego nazywała się Azuba, a była córką Szilchiego.'' \mbox{(1 Krl 22, 41-42)}
\end{quote}
\subsubsection{Jehoram}
\paragraph{}
\begin{quote}
,,W piątym roku Jorama, syna Achaba, króla izraelskiego, władzę królewską nad Judą objął - choć królem Judy był jeszcze Jehoszafat - Jehoram, syn Jehoszafata. Miał trzydzieści dwa lata, gdy objął władzę królewską, a panował w Jeruzalemie osiem lat.'' \mbox{(2 Krl 8, 16-17)}
\end{quote}
\subsubsection{Achazjasz}
\paragraph{}
\begin{quote}
,,W dwunastym roku Jorama, syna Achaba, króla izraelskiego, objął władzę królewską Achazjasz, syn Jehorama, króla judzkiego. Achazjasz miał dwadzieścia dwa lata, gdy objął władzę królewską, a panował jeden rok w Jeruzalemie; matka jego zaś nazywała się Atalia, a była córką Omriego, króla izraelskiego.'' \mbox{(2 Krl 8, 25-26)}
\end{quote}
\subsubsection{Joasz}
\paragraph{}
\begin{quote}
,,Joasz miał siedem lat, gdy objął władzę królewską. Objął zaś Joasz władzę królewską w siódmym roku panowania Jehu, a panował w Jeruzalemie czterdzieści lat. Matka jego nazywała się Sibia, a pochodziła z Beer-Szeby.'' \mbox{(2 Krl 12, 1-2)}
\end{quote}
\subsubsection{Amasjasz}
\paragraph{}
\begin{quote}
,,W drugim roku panowania Joasza, syna Joachaza, króla izraelskiego, władzę królewską objął Amasjasz, syn Joasza, króla judzkiego. Miał dwadzieścia pięć lat, gdy objął władzę królewską, a panował dwadzieścia dziewięć lat w Jeruzalemie, Matka jego nazywała się Jehoadan, a pochodziła z Jeruzalemu.'' \mbox{(2 Krl 14, 1-2)}
\end{quote}
\subsubsection{Azariasz}
\paragraph{}
\begin{quote}
,,W dwudziestym siódmym roku panowania Jeroboama, króla izraelskiego, objął władzę królewską Azariasz, syn Amasjasza, króla judzkiego. Miał szesnaście lat, gdy został królem, a panował w Jeruzalemie pięćdziesiąt dwa lata. Matka jego nazywała się Jekolia, a pochodziła z Jeruzalemu,'' \mbox{(2 Krl 15, 1-2)}
\end{quote}
\subsubsection{Jotam}
\paragraph{}
\begin{quote}
,,W drugim roku panowania Pekacha, syna Remaliasza, króla izraelskiego, objął władzę królewską Jotam, syn Uzjasza, króla judzkiego. Miał dwadzieścia pięć lat, gdy objął władzę królewską, a panował szesnaście lat w Jeruzalemie. Matka jego nazywała się Jerusza, a była córką Sadoka.'' \mbox{(2 Krl 15, 32-33)}
\end{quote}
\subsubsection{Achaz}
\paragraph{}
\begin{quote}
,,W siedemnastym roku panowania Pekacha, syna Remaliasza, objął władzę królewską Achaz, syn Jotama, króla judzkiego. Achaz miał dwadzieścia lat, gdy objął władzę królewską, a panował szesnaście lat w Jeruzalemie. Lecz nie czynił tego, co prawe w oczach Pana, Boga jego, jak Dawid jego praojciec.'' \mbox{(2 Krl 16, 1-2)}
\end{quote}
\subsubsection{Hiskiasz}
\paragraph{}
\begin{quote}
,,W trzecim roku panowania Ozeasza, syna Eli, króla izraelskiego, objął władzę królewską Hiskiasz, syn Achaza, króla judzkiego. Miał dwadzieścia pięć lat, gdy objął władzę królewską, a dwadzieścia dziewięć lat panował w Jeruzalemie. Matka jego nazywała się Abi, a była córką Zachariasza.'' \mbox{(2 Krl 18, 1-2)}
\end{quote}
\subsubsection{Manasses}
\paragraph{}
\begin{quote}
,,Manasses miał dwanaście lat, gdy objął władzę królewską, a panował pięćdziesiąt pięć lat w Jeruzalemie. Matka jego nazywała się Chefsibach.'' \mbox{(2 Krl 21, 1)}
\end{quote}
\subsubsection{Amon}
\paragraph{}
\begin{quote}
,,Amon miał dwadzieścia dwa lata, gdy objął władzę królewską, a panował dwa lata w Jeruzalemie. Matka jego nazywała się Meszullemet, a była córką Charusa z Jotby.'' \mbox{(2 Krl 21, 19)}
\end{quote}
\subsubsection{Jozjasz}
\paragraph{}
\begin{quote}
,,Jozjasz miał osiem lat, gdy objął władzę królewską, a panował w Jeruzalemie trzydzieści jeden lat. Matka jego nazywała się Jedida, a była córką Adajasza z Boskat.'' \mbox{(2 Krl 22, 1)}
\end{quote}
\subsubsection{Jehoachaz}
\paragraph{}
\begin{quote}
,,Jehoachaz miał dwadzieścia trzy lata, gdy objął władzę królewską, a w Jeruzalemie panował trzy miesiące. Matka jego nazywała się Chamutal, była córką Jeremiasza, a pochodziła z Libny.'' \mbox{(2 Krl 23, 31)}
\end{quote}
\subsubsection{Jehojakim}
\paragraph{}
\begin{quote}
,,Jehojakim miał dwadzieścia pięć lat, gdy objął władzę królewską, a panował jedenaście lat w Jeruzalemie. Matka jego nazywała się Zebudda, była córką Pedajasza z Rumy.'' \mbox{(2 Krl 23, 36)}
\end{quote}
\subsubsection{Jehojachin}
\paragraph{}
\begin{quote}
,,Jehojachin miał osiemnaście lat, gdy objął władzę królewską, a panował w Jeruzalemie trzy miesiące. Matka jego nazywała się Nechuszta, była córką Elnatana z Jeruzalemu.'' \mbox{(2 Krl 24, 8)}
\end{quote}
\subsubsection{Sedekiasz}
\paragraph{}
\begin{quote}
,,Sedekiasz miał dwadzieścia jeden lat, gdy objął władzę królewską, a panował jedenaście lat w Jeruzalemie. Matka jego nazywała się Chamutal, była córką Jeremiasza z Libny.'' \mbox{(2 Krl 24, 18)}
\end{quote}
\subsection{Królestwo Izraela}
\subsubsection{Jeroboam}
\paragraph{}
\begin{quote}
,,Panował zaś Jeroboam dwadzieścia dwa lata, po czym spoczął obok swoich ojców, a władzę królewską po nim objął Nadab, jego syn.'' \mbox{(1 Krl 14, 20)}
\end{quote}
\subsubsection{Nadab}
\paragraph{}
\begin{quote}
,,W drugim roku panowania króla judzkiego Asy objął władzę królewską nad Izraelem Nadab, syn Jeroboama, a panował nad Izraelem dwa lata.'' \mbox{(1 Krl 15, 25)}
\end{quote}
\subsubsection{Baasza}
\paragraph{}
\begin{quote}
,,W trzecim roku panowania Asy, króla judzkiego, objął władzę królewską nas całym Izraelem Baasza, syn Achiasza, a panował w Tirsie dwadzieścia cztery lata.'' \mbox{(1 Krl 15, 33)}
\end{quote}
\subsubsection{Ela}
\paragraph{}
\begin{quote}
,,Ela, syn Baaszy, objął władzę królewską nad Izraelem w dwudziestym szóstym roku panowania Asy, króla judzkiego, a panował Ela, syn Baaszy, w Tirsie dwa lata.'' \mbox{(1 Krl 16, 8)}
\end{quote}
\subsubsection{Zimri}
\paragraph{}
\begin{quote}
,,Zimri objął władzę królewską w dwudziestym siódmym roku panowania Asy, króla judzkiego, ale panował tylko siedem dni w Tirsie, podczas gdy wojownicy oblegali Gibbeton, należące do Filistyńczyków.'' \mbox{(1 Krl 16, 15)}
\end{quote}
\subsubsection{Omri}
\paragraph{}
\begin{quote}
,,Omri objął władzę królewską nad Izraelem w trzydziestym pierwszym roku panowania Asy, króla judzkiego, a panował dwanaście lat, z tego w Tirsie sześć lat.'' \mbox{(1 Krl 16, 23)}
\end{quote}
\subsubsection{Achab}
\paragraph{}
\begin{quote}
,,Achab zaś, syn Omriego, objął władzę królewską nad Izraelem w trzydziestym ósmym roku panowania Asy, króla judzkiego, panował zaś Achab, syn Omriego, nad Izraelem w Samarii dwadzieścia dwa lata.'' \mbox{(1 Krl 16, 29)}
\end{quote}
\subsubsection{Achazjasz}
\paragraph{}
\begin{quote}
,,Achazjasz, syn Achaba, objął władzę królewską nad Izraelem w Samarii w siedemnastym roku panowania Jehoszafata, króla judzkiego, a panował nad Izraelem dwa lata.'' \mbox{(1 Krl 22, 52)}
\end{quote}
\subsubsection{Jehoram}
\paragraph{}
\begin{quote}
,,Jehoram, syn Achaba, objął władzę królewską nad Izraelem w Samarii w osiemnastym roku panowania Jehoszafata, króla judzkiego, i panował dwanaście lat.'' \mbox{(2 Krl 3, 1)}
\end{quote}
\subsubsection{Jehu}
\paragraph{}
\begin{quote}
,,Panował zaś Jehu nad Izraelem w Samarii dwadzieścia osiem lat.'' \mbox{(2 Krl 10, 36)}
\end{quote}
\subsubsection{Jehoachaz}
\paragraph{}
\begin{quote}
,,W dwudziestym trzecim roku panowania Joasza, syna Achazjasza, króla judzkiego, objął władzę królewską nad Izraelem w Samarii Jehoachaz, syn Jehu, a panował siedemnaście lat.'' \mbox{(2 Krl 13, 1)}
\end{quote}
\subsubsection{Jehoasz}
\paragraph{}
\begin{quote}
,,W trzydziestym siódmym roku panowania Joasza, króla judzkiego, objął władzę królewską nad Izraelem w Samarii Jehoasz, syn Jehoachaza, a panował szesnaście lat.'' \mbox{(2 Krl 13, 10)}
\end{quote}
\subsubsection{Jeroboam}
\paragraph{}
\begin{quote}
,,W piętnastym roku panowania Amasjasza, syna Joasza, króla judzkiego, objął władzę królewską w Samarii Jeroboam, syn Joasza, króla izraelskiego, a panował czterdzieści jeden lat.'' \mbox{(2 Krl 14, 23)}
\end{quote}
\subsubsection{Zachariasz}
\paragraph{}
\begin{quote}
,,W trzydziestym ósmym roku panowania Azariasza, króla judzkiego, objął władzę królewską nad Izraelem w Samarii Zachariasz, syn Jeroboama, a panował sześć miesięcy.'' \mbox{(2 Krl 15, 8)}
\end{quote}
\subsubsection{Szallum}
\paragraph{}
\begin{quote}
,,Szallum tedy, syn Jabesza objął władzę królewską w trzydziestym dziewiątym roku panowania Uzjasza, króla judzkiego, a panował w Samarii miesiąc.'' \mbox{(2 Krl 15, 13)}
\end{quote}
\subsubsection{Menachem}
\paragraph{}
\begin{quote}
,,W trzydziestym dziewiątym roku panowania Azariasza, króla judzkiego, objął władzę królewską nad Izraelem Menachem, syn Gadiego, a panował dziesięć lat w Samarii.'' \mbox{(2 Krl 15, 17)}
\end{quote}
\subsubsection{Pekachiasz}
\paragraph{}
\begin{quote}
,,W pięćdziesiątym roku panowania Azariasza, króla judzkiego, objął władzę królewską nad Izraelem Pekachiasz, syn Menachema, a panował w Samarii dwa lata.'' \mbox{(2 Krl 15, 23)}
\end{quote}
\subsubsection{Pekach}
\paragraph{}
\begin{quote}
,,W pięćdziesiątym drugim roku panowania Azariasza, króla judzkiego, objął władzę królewską nad Izraelem Pekach, syn Remaliasza, i panował w Samarii dwadzieścia lat.'' \mbox{(2 Krl 15, 27)}
\end{quote}
\subsubsection{Ozeasz}
\paragraph{}
\begin{quote}
,,W dwudziestym roku panowania Achaza, króla judzkiego, objął władzę królewską nad Izraelem w Samarii Ozeasz, syn Eli, a panował dziewięć lat.'' \mbox{(2 Krl 17, 1)}
\end{quote}
\end{document}