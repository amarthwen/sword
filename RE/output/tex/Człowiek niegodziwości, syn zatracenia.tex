\documentclass[10pt,a4paper,oneside]{article}
\usepackage[utf8]{inputenc}
\usepackage{polski}
\usepackage[polish]{babel}
\usepackage[margin=0.5in,bottom=0.75in]{geometry}
\begin{document}
\centerline{\textbf{\MakeUppercase{Człowiek niegodziwości, syn zatracenia}}}
\begin{center}
\textbf{Wersety do studium:} 
\mbox{(2Tes 2, 1-12)}, \mbox{(Dn 8, 23-25)}, \mbox{(Dn 11, 31-39)}
\end{center}
\section{Temat główny}
\paragraph{}
\begin{quote}
,,Co się zaś tyczy przyjścia Pana naszego Jezusa Chrystusa i spotkania naszego z nim, prosimy was, bracia, Abyście nie tak szybko dali się zbałamucić i nastraszyć, czy to przez jakieś wyrocznie, czy przez mowę, czy przez list, rzekomo przez nas pisany, jakoby już nastał dzień Pański. Niechaj was nikt w żaden sposób nie zwodzi; bo nie nastanie pierwej, zanim nie przyjdzie odstępstwo i nie objawi się człowiek niegodziwości, syn zatracenia, Przeciwnik, który wynosi się ponad wszystko, co się zwie Bogiem lub jest przedmiotem boskiej czci, a nawet zasiądzie w świątyni Bożej, podając się za Boga. Czy nie pamiętacie, że jeszcze będąc u was, o tym wam mówiłem? A wiecie, co go teraz powstrzymuje, tak iż się objawi dopiero we właściwym czasie. Albowiem tajemna moc nieprawości już działa, tajemna dopóty, dopóki ten, który teraz powstrzymuje, nie zejdzie z pola. A wtedy objawi się ów niegodziwiec, którego Pan Jezus zabije tchnieniem ust swoich i zniweczy blaskiem przyjścia swego. A ów niegodziwiec przyjdzie za sprawą szatana z wszelką mocą, wśród znaków i rzekomych cudów, I wśród wszelkich podstępnych oszustw wobec tych, którzy mają zginąć, ponieważ nie przyjęli miłości prawdy, która mogła ich zbawić. I dlatego zsyła Bóg na nich ostry obłęd, tak iż wierzą kłamstwu, Aby zostali osądzeni wszyscy, którzy nie uwierzyli prawdzie, lecz znaleźli upodobanie w nieprawości.'' \mbox{(2 Tes 2, 1-12)}
\end{quote}
\paragraph{}
\begin{quote}
,,Lecz pod koniec ich królowania, gdy zbrodniarze dopełnią swej miary, powstanie król zuchwały i podstępny. Siła jego będzie potężna i spowoduje okropne nieszczęścia; i szczęśliwie mu się powiedzie w działaniu, i zniszczy możnych i lud świętych. Działając podstępnie dzięki mądrości, będzie miał powodzenie; będzie pyszny w sercu i wielu zniszczy niespodzianie. Lecz gdy powstanie przeciwko księciu książąt, zostanie zmiażdżony bez udziału ludzkiej ręki.'' \mbox{(Dn 8, 23-25)}
\end{quote}
\paragraph{}
\begin{quote}
,,A wojska wysłane przez niego wystąpią i zbezczeszczą świątynię i twierdzę, zniosą stałą codzienną ofiarę i postawią obrzydliwość spustoszenia. A tych, którzy bezbożnie będą postępować wbrew przymierzu, zwiedzie pochlebstwami do odstępstwa, lecz lud tych, którzy znają swojego Boga, umocni się i będą działać. A roztropni wśród ludu doprowadzą wielu do właściwego poznania; lecz przez pewien czas padać będą od miecza i ognia, od uprowadzenia i rabunku. A gdy będą padać, doznają małej pomocy; i wielu przyłączy się do nich obłudnie. Nawet niektórzy spośród roztropnych upadną, aby wśród nich nastąpiło wypławienie, oczyszczenie i wybielenie aż do czasu ostatecznego, gdyż to jeszcze potrwa pewien czas. A król zrobi, jak będzie chciał; będzie się wynosił i wywyższał ponad wszelkie bóstwo; i przeciwko Bogu bogów dziwne rzeczy będzie wygadywał, i będzie miał powodzenie, aż dopełni się miara gniewu, bo to, co jest postanowione, wypełni się. Ani o bogów swoich ojców nie będzie się troszczył, nie będzie się troszczył o ulubieńca kobiet ani o żadnego boga, bo wyniesie się ponad wszystkich. Zamiast tego będzie czcił boga warowni; złotem, srebrem, drogimi kamieniami i kosztownościami będzie czcił boga, którego nie znali jego ojcowie. Do warownych grodów wprowadzi lud obcego boga; tych, którzy go znają, obsypie zaszczytami, nada im władzę nad wieloma i w nagrodę obdziela ziemią.'' \mbox{(Dn 11, 31-39)}
\end{quote}
\end{document}