\documentclass[10pt,a4paper,oneside]{article}
\usepackage[utf8]{inputenc}
\usepackage{polski}
\usepackage[polish]{babel}
\usepackage[margin=0.5in,bottom=0.75in]{geometry}
\begin{document}
\centerline{\textbf{\MakeUppercase{Obrazy człowieka w Piśmie Świętym}}}
\begin{center}
\textbf{Wersety do studium:} \mbox{(J 2, 18-21)}, \mbox{(1Kor 3, 16-17)}, \mbox{(1Kor 6, 19)}, \mbox{(2Kor 6, 16)}, \mbox{(J 2, 13-21)}
\end{center}
\section{Temat główny}
\paragraph{}
\begin{quote}
,,Czy nie wiecie, że świątynią Bożą jesteście i że Duch Boży mieszka w was? Jeśli ktoś niszczy świątynię Bożą, tego zniszczy Bóg, albowiem świątynia Boża jest święta, a wy nią jesteście.'' \mbox{(1 Kor 3, 16-17)}
\end{quote}
\paragraph{}
\begin{quote}
,,Albo czy nie wiecie, że ciało wasze jest świątynią Ducha Świętego, który jest w was i którego macie od Boga, i że nie należycie też do siebie samych?'' \mbox{(1 Kor 6, 19)}
\end{quote}
\paragraph{}
\begin{quote}
,,Jakiż układ między świątynią Bożą a bałwanami? Myśmy bowiem świątynią Boga żywego, jak powiedział Bóg: Zamieszkam w nich i będę się przechadzał pośród nich, i będę Bogiem ich, a oni będą ludem moim.'' \mbox{(2 Kor 6, 16)}
\end{quote}
\paragraph{}
\begin{quote}
,,A gdy się zbliżała Pascha żydowska, udał się Jezus do Jerozolimy. I zastał w świątyni sprzedających woły i owce, i gołębie, i siedzących wekslarzy. I skręciwszy bicz z powrózków, wypędził ich wszystkich ze świątyni wraz z owcami i wołami; wekslarzom rozsypał pieniądze i stoły powywracał, A do sprzedawców gołębi rzekł: Zabierzcie to stąd, z domu Ojca mego nie czyńcie targowiska. Wtedy uczniowie jego przypomnieli sobie, że napisano: Żarliwość o dom twój pożera mnie. Wtedy odezwali się Żydzi, mówiąc do niego: Jaki znak pokażesz nam na dowód, że ci to wolno czynić? Jezus, odpowiadając, rzekł im: Zburzcie tę świątynię, a Ja w trzy dni ją odbuduję. na to rzekli Żydzi: Czterdzieści sześć lat budowano tę świątynię, a Ty w trzy dni chcesz ją odbudować? Ale On mówił o świątyni ciała swego.'' \mbox{(J 2, 13-21)}
\end{quote}
\end{document}