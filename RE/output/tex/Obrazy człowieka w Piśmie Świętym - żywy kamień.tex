\documentclass[10pt,a4paper,oneside]{article}
\usepackage[utf8]{inputenc}
\usepackage{polski}
\usepackage[polish]{babel}
\usepackage[margin=0.5in,bottom=0.75in]{geometry}
\begin{document}
\centerline{\textbf{\MakeUppercase{Obrazy człowieka w Piśmie Świętym}}}
\begin{center}
\textbf{Wersety do studium:} \mbox{(1P 2, 4-9)}
\end{center}
\section{Temat główny}
\paragraph{}
\begin{quote}
,,Przystąpcie do niego, do kamienia żywego, przez ludzi wprawdzie odrzuconego, lecz przez Boga wybranego jako kosztowny. I wy sami jako kamienie żywe budujcie się w dom duchowy, w kapłaństwo święte, aby składać duchowe ofiary przyjemne Bogu przez Jezusa Chrystusa. Dlatego to powiedziane jest w Piśmie: Oto kładę na Syjonie kamień węgielny, wybrany, kosztowny, A kto weń wierzy, nie zawiedzie się. Dla was, którzy wierzycie, jest on rzeczą cenną; dla niewierzących zaś kamień ten, którym wzgardzili budowniczowie, pozostał kamieniem węgielnym, Ale też kamieniem, o który się potkną, i skałą zgorszenia; ci, którzy nie wierzą Słowu, potykają się oń, na co zresztą są przeznaczeni. Ale wy jesteście rodem wybranym, królewskim kapłaństwem, narodem świętym, ludem nabytym, abyście rozgłaszali cnoty tego, który was powołał z ciemności do cudownej swojej światłości;'' \mbox{(1 P 2, 4-9)}
\end{quote}
\end{document}