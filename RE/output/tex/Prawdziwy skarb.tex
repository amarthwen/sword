\documentclass[10pt,a4paper,oneside]{article}
\usepackage[utf8]{inputenc}
\usepackage{polski}
\usepackage[polish]{babel}
\usepackage[margin=0.5in,bottom=0.75in]{geometry}
\begin{document}
\centerline{\textbf{\MakeUppercase{Prawdziwy skarb}}}
\begin{center}
\textbf{Wersety do studium:} \mbox{(Ab 1, 5-6)}, \mbox{(Iz 33, 6)}, \mbox{(Mt 25, 14-30)}, \mbox{(Łk 15, 11-13)}, \mbox{(2Kor 4, 7)}, \mbox{(Jr 18, 6)}, \mbox{(Za 14, 20-21)}, \mbox{(Mt 13, 44)}, \mbox{(Obj 3, 14-18)}, \mbox{(Łk 6, 45)}, \mbox{(Mt 12, 35)}, \mbox{(Mk 10, 17-22)}, \mbox{(Łk 18, 22)}, \mbox{(Mt 6, 19-21)}, \mbox{(Łk 12, 32-34)}, \mbox{(1Kor 6, 19-20)}
\end{center}
\section{Temat główny}
\paragraph{}
\begin{quote}
,,Gdy złodzieje wtargną do ciebie albo nocni rabusie jakże będziesz splądrowany! Czy nie będą kradli do woli? Gdy zbieracze winogron przyjdą do ciebie, czy pozostawią choć jedno grono? Jakże ogołocony jest Ezaw; przetrząśnięte są jego skarby ukryte.'' \mbox{(Ab 1, 5-6)}
\end{quote}
\paragraph{}
\begin{quote}
,,Nadejdą dla ciebie czasy bezpieczne. Bogactwem zbawienia to mądrość i poznanie, bojaźń przed Panem to jego skarb.'' \mbox{(Iz 33, 6)}
\end{quote}
\subsection{Przypowieść o talentach}
\paragraph{}
\begin{quote}
,,Będzie bowiem tak jak z człowiekiem, który odjeżdżając, przywołał swoje sługi i przekazał im swój majątek, I dał jednemu pięć talentów, a drugiemu dwa, a trzeciemu jeden, każdemu według zdolności i odjechał. A ten, który wziął pięć talentów, zaraz poszedł, obracał nimi i zyskał dalsze pięć. Podobnie i ten, który wziął dwa, zyskał dalsze dwa. A ten, który wziął jeden, odszedł, wykopał dół w ziemi i ukrył pieniądze pana swego. A po długim czasie powraca pan owych sług i rozlicza się z nimi. I przystąpiwszy ten, który wziął pięć talentów, przyniósł dalsze pięć talentów i rzekł: Panie! Pięć talentów mi powierzyłeś. Oto dalsze pięć talentów zyskałem. Rzekł mu Pan jego: Dobrze, sługo dobry i wierny! Nad tym, co małe, byłeś wierny, wiele ci powierzę; wejdź do radości pana swego. Potem przystąpił ten, który wziął dwa talenty, i rzekł: Panie! Dwa talenty mi powierzyłeś, oto dalsze dwa talenty zyskałem. Rzekł mu pan jego: Dobrze, sługo dobry i wierny! Nad tym, co małe, byłeś wierny, wiele ci powierzę; wejdź do radości pana swego. Wreszcie przystąpił i ten, który wziął jeden talent, i rzekł: Panie! Wiedziałem o tobie, żeś człowiek twardy, że żniesz, gdzieś nie siał, i zbierasz, gdzieś nie rozsypywał. Bojąc się tedy, odszedłem i ukryłem talent twój w ziemi; oto masz, co twoje. A odpowiadając, rzekł mu pan jego: Sługo zły i leniwy! Wiedziałeś, że żnę, gdzie nie siałem, i zbieram, gdzie nie rozsypywałem. Powinieneś był więc dać pieniądze moje bankierom, a ja po powrocie odebrałbym, co moje, z zyskiem. Weźcie przeto od niego ten talent i dajcie temu, który ma dziesięć talentów. Każdemu bowiem, kto ma, będzie dane i obfitować będzie, a temu, kto nie ma, zostanie zabrane i to, co ma. A nieużytecznego sługę wrzućcie w ciemności zewnętrzne; tam będzie płacz i zgrzytanie zębów.'' \mbox{(Mt 25, 14-30)}
\end{quote}
\subsection{Przypowieść o synu marnotrawnym}
\paragraph{}
\begin{quote}
,,Potem rzekł: Pewien człowiek miał dwóch synów. I rzekł młodszy z nich ojcu: Ojcze, daj mi część majętności, która na mnie przypada. Wtedy ten rozdzielił im majętność. A po niewielu dniach młodszy syn zabrał wszystko i odjechał do dalekiego kraju, i tam roztrwonił swój majątek, prowadząc rozwiązłe życie.'' \mbox{(Łk 15, 11-13)}
\end{quote}
\subsection{Skarb w glinianych naczyniach}
\paragraph{}
\begin{quote}
,,Mamy zaś ten skarb w naczyniach glinianych, aby się okazało, że moc, która wszystko przewyższa, jest z Boga, a nie z nas.'' \mbox{(2 Kor 4, 7)}
\end{quote}
\paragraph{}
\begin{quote}
,,Czy nie mogę postąpić z wami, domu Izraela, jak garncarz? - mówi Pan. Oto jak glina w ręku garncarza, tak wy jesteście w moim ręku, domu Izraela!'' \mbox{(Jr 18, 6)}
\end{quote}
\paragraph{}
\begin{quote}
,,W owym dniu będzie na dzwoneczkach koni napis: Poświęcony Panu. A garnków domu Pana będzie jak czasz ofiarnych przed ołtarzem. Każdy garnek w Jeruzalemie i w Judzie będzie poświęcony Panu Zastępów, tak że wszyscy, którzy przyjdą składać ofiary, będą je brali i będą w nich gotowali mięso ofiarne. W dniu owym już nie będzie handlarza w domu Pana Zastępów.'' \mbox{(Za 14, 20-21)}
\end{quote}
\subsection{Podobieństwo o skarbie ukrytym na roli}
\paragraph{}
\begin{quote}
,,Podobne jest Królestwo Niebios do ukrytego w roli skarbu, który człowiek znalazł, ukrył i uradowany odchodzi, i sprzedaje wszystko, co ma, i kupuje oną rolę.'' \mbox{(Mt 13, 44)}
\end{quote}
\subsection{Złoto w ogniu wypróbowane}
\paragraph{}
\begin{quote}
,,A do anioła zboru w Laodycei napisz: To mówi Ten, który jest Amen, świadek wierny i prawdziwy, początek stworzenia Bożego: Znam uczynki twoje, żeś ani zimny, ani gorący. Obyś był zimny albo gorący! A tak, żeś letni, a nie gorący ani zimny, wypluję cię z ust moich. Ponieważ mówisz: Bogaty jestem i wzbogaciłem się, i niczego nie potrzebuję, a nie wiesz, żeś pożałowania godzien nędzarz i biedak, ślepy i goły, Radzę ci, abyś nabył u mnie złota w ogniu wypróbowanego, abyś się wzbogacił i abyś przyodział szaty białe, aby nie wystąpiła na jaw haniebna nagość twoja, oraz maści, by nią namaścić oczy twoje, abyś przejrzał.'' \mbox{(Obj 3, 14-18)}
\end{quote}
\subsection{Dobry skarb w dobrym skarbcu serca}
\paragraph{}
\begin{quote}
,,Człowiek dobry z dobrego skarbca serca wydobywa dobro, a zły ze złego wydobywa zło; albowiem z obfitości serca mówią usta jego.'' \mbox{(Łk 6, 45)}
\end{quote}
\paragraph{}
\begin{quote}
,,Dobry człowiek wydobywa z dobrego skarbca dobre rzeczy, a zły człowiek wydobywa ze złego skarbca złe rzeczy.'' \mbox{(Mt 12, 35)}
\end{quote}
\subsection{Skarb w niebie}
\paragraph{}
\begin{quote}
,,A gdy się wybierał w drogę, przybiegł ktoś, upadł przed nim na kolana i zapytał go: Nauczycielu dobry! Co mam czynić, aby odziedziczyć żywot wieczny? A Jezus odrzekł: Czemu mię nazywasz dobrym? Nikt nie jest dobry, tylko jeden Bóg. Znasz przykazania: Nie zabijaj, nie cudzołóż, nie kradnij, nie mów fałszywego świadectwa, nie oszukuj, czcij ojca swego i matkę. A on mu odpowiedział: Nauczycielu, tego wszystkiego przestrzegałem od młodości mojej. Wtedy Jezus spojrzał nań z miłością i rzekł mu: Jednego ci brak; idź, sprzedaj wszystko, co masz, i rozdaj ubogim, a będziesz miał skarb w niebie, po czym przyjdź i naśladuj mnie. A ten na to słowo sposępniał i odszedł zasmucony, albowiem miał wiele majętności'' \mbox{(Mk 10, 17-22)}
\end{quote}
\paragraph{}
\begin{quote}
,,A gdy to Jezus usłyszał, rzekł do niego: Jeszcze jednego ci brak. Sprzedaj wszystko, co tylko masz, i rozdaj ubogim, a będziesz miał skarb w niebie, a potem przyjdź i naśladuj mnie.'' \mbox{(Łk 18, 22)}
\end{quote}
\paragraph{}
\begin{quote}
,,Nie gromadźcie sobie skarbów na ziemi, gdzie je mól i rdza niszczą i gdzie złodzieje podkopują i kradną: Ale gromadźcie sobie skarby w niebie, gdzie ani mól, ani rdza nie niszczą i gdzie złodzieje nie podkopują i nie kradną. Albowiem gdzie jest skarb twój - tam będzie i serce twoje'' \mbox{(Mt 6, 19-21)}
\end{quote}
\paragraph{}
\begin{quote}
,,Nie bój się, maleńka trzódko! Gdyż upodobało się Ojcu waszemu dać wam Królestwo. Sprzedajcie majętności swoje, a dawajcie jałmużnę. Uczyńcie sobie sakwy, które nie niszczeją, skarb niewyczerpany w niebie, gdzie złodziej nie ma przystępu, ani mól nie niszczy. Albowiem gdzie jest skarb wasz, tam będzie i serce wasze.'' \mbox{(Łk 12, 32-34)}
\end{quote}
\subsection{Skarb, którym zostaliśmy wykupieni z niewoli grzechu}
\paragraph{}
\begin{quote}
,,Albo czy nie wiecie, że ciało wasze jest świątynią Ducha Świętego, który jest w was i którego macie od Boga, i że nie należycie też do siebie samych? Drogoście bowiem kupieni. Wysławiajcie tedy Boga w ciele waszym.'' \mbox{(1 Kor 6, 19-20)}
\end{quote}
\subsection{O bogatym młodzieńcu}
\paragraph{}
\begin{quote}
,,A gdy się wybierał w drogę, przybiegł ktoś, upadł przed nim na kolana i zapytał go: Nauczycielu dobry! Co mam czynić, aby odziedziczyć żywot wieczny? A Jezus odrzekł: Czemu mię nazywasz dobrym? Nikt nie jest dobry, tylko jeden Bóg. Znasz przykazania: Nie zabijaj, nie cudzołóż, nie kradnij, nie mów fałszywego świadectwa, nie oszukuj, czcij ojca swego i matkę. A on mu odpowiedział: Nauczycielu, tego wszystkiego przestrzegałem od młodości mojej. Wtedy Jezus spojrzał nań z miłością i rzekł mu: Jednego ci brak; idź, sprzedaj wszystko, co masz, i rozdaj ubogim, a będziesz miał skarb w niebie, po czym przyjdź i naśladuj mnie. A ten na to słowo sposępniał i odszedł zasmucony, albowiem miał wiele majętności'' \mbox{(Mk 10, 17-22)}
\end{quote}
\end{document}