\documentclass[10pt,a4paper,oneside]{article}
\usepackage[utf8]{inputenc}
\usepackage{polski}
\usepackage[polish]{babel}
\usepackage[margin=0.5in,bottom=0.75in]{geometry}
\begin{document}
\centerline{\textbf{\MakeUppercase{Nie mniemajcie...}}}
\begin{center}
\textbf{Wersety do studium:} \mbox{(Mt 5, 17-20)}, \mbox{(Mt 10, 34-39)}
\end{center}
\section{Temat główny}
\paragraph{}
\begin{quote}
,,Nie mniemajcie, że przyszedłem rozwiązać zakon albo proroków; nie przyszedłem rozwiązać, lecz wypełnić. Bo zaprawdę powiadam wam: Dopóki nie przeminie niebo i ziemia, ani jedna jota, ani jedna kreska nie przeminie z zakonu, aż wszystko to się stanie. Ktokolwiek by tedy rozwiązał jedno z tych przykazań najmniejszych i nauczałby tak ludzi, najmniejszym będzie nazwany w Królestwie Niebios; a ktokolwiek by czynił i nauczał, ten będzie nazwany wielkim w Królestwie Niebios. Albowiem powiadam wam: Jeśli sprawiedliwość wasza nie będzie obfitsza niż sprawiedliwość uczonych w Piśmie i faryzeuszów, nie wejdziecie do Królestwa Niebios'' \mbox{(Mt 5, 17-20)}
\end{quote}
\paragraph{}
\begin{quote}
,,Nie mniemajcie, że przyszedłem przynieść pokój na ziemię; nie przyszedłem przynieść pokój, ale miecz. Bo przyszedłem poróżnić człowieka z jego ojcem i córkę z jego matką, i synową z jej teściową. Tak to staną się wrogami człowieka domownicy jego. Kto miłuje ojca albo matkę bardziej niż mnie, nie jest mnie godzien; i kto miłuje syna albo córkę bardziej niż mnie, nie jest mnie godzien. I kto nie bierze krzyża swego, a idzie za mną, nie jest mnie godzien. Kto stara się zachować życie swoje, straci je, a kto straci życie swoje dla mnie, znajdzie je.'' \mbox{(Mt 10, 34-39)}
\end{quote}
\end{document}