\documentclass[10pt,a4paper,oneside]{article}
\usepackage[utf8]{inputenc}
\usepackage{polski}
\usepackage[polish]{babel}
\usepackage[margin=0.5in,bottom=0.75in]{geometry}
\begin{document}
\centerline{\textbf{\MakeUppercase{Kielich nowego, wiecznego przymierza}}}
\begin{center}
\textbf{Wersety do studium:} 
\mbox{(Hbr 12, 22-24)}, \mbox{(Hbr 13, 20-21)}, \mbox{(Rz 8, 8-11)}, \mbox{(Mt 26, 27-28)}, \mbox{(J 4, 7-15)}, \mbox{(J 7, 37-39)}, \mbox{(Ef 2, 11-13)}
\end{center}
\paragraph{}
\begin{quote}
,,Lecz wy podeszliście do góry Syjon i do miasta Boga żywego, do Jeruzalem niebieskiego i do niezliczonej rzeszy aniołów, do uroczystego zgromadzenia I zebrania pierworodnych, którzy są zapisani w niebie, i do Boga, sędziego wszystkich, i do duchów ludzi sprawiedliwych, którzy osiągnęli doskonałość, I do pośrednika nowego przymierza, Jezusa, i do krwi, którą się kropi, a która przemawia lepiej niż krew Abla.'' \mbox{(Hbr 12, 22-24)}
\end{quote}
\paragraph{}
\begin{quote}
,,A Bóg pokoju, który przez krew przymierza wiecznego wywiódł spośród umarłych wielkiego pasterza owiec, Pana naszego Jezusa, Niech was wyposaży we wszystko dobre, abyście spełnili wolę jego, sprawując w nas to, co miłe jest w oczach jego, przez Jezusa Chrystusa, któremu niech będzie chwała na wieki wieków. Amen.'' \mbox{(Hbr 13, 20-21)}
\end{quote}
\paragraph{}
\begin{quote}
,,Ci zaś, którzy są w ciele, Bogu podobać się nie mogą. Ale wy nie jesteście w ciele, lecz w Duchu, jeśli tylko Duch Boży mieszka w was. Jeśli zaś kto nie ma Ducha Chrystusowego, ten nie jest jego. Jeśli jednak Chrystus jest w was, to chociaż ciało jest martwe z powodu grzechu, jednak duch jest żywy przez usprawiedliwienie. A jeśli Duch tego, który Jezusa wzbudził z martwych, mieszka w was, tedy Ten, który Jezusa Chrystusa z martwych wzbudził, ożywi i wasze śmiertelne ciała przez Ducha swego, który mieszka w was.'' \mbox{(Rz 8, 8-11)}
\end{quote}
\paragraph{}
\begin{quote}
,,Potem wziął kielich i podziękował, dał im, mówiąc: Pijcie z niego wszyscy; Albowiem to jest krew moja nowego przymierza, która się za wielu wylewa na odpuszczenie grzechów.'' \mbox{(Mt 26, 27-28)}
\end{quote}
\paragraph{}
\begin{quote}
,,Wtem przyszła niewiasta samarytańska, aby nabrać wody. Jezus rzekł do niej: Daj mi pić! Uczniowie jego bowiem poszli do miasta, by nakupić żywności. Wtedy niewiasta samarytańska rzekła do niego: Jakże Ty, będąc Żydem, prosisz mnie, Samarytankę, o wodę? (Żydzi bowiem nie obcują z Samarytanami). Odpowiadając jej Jezus, rzekł do niej: Gdybyś znała dar Boży i tego, który mówi do ciebie: Daj mi pić, wtedy sama prosiłabyś go, i dałby ci wody żywej. Mówi do niego: Panie, nie masz nawet czerpaka, a studnia jest głęboka; skądże więc masz tę wodę żywą? Czy może Ty jesteś większy od ojca naszego Jakuba, który dał nam tę studnię i sam z niej pił, i synowie jego, i trzody jego? Odpowiedział jej Jezus, mówiąc: Każdy, kto piję tę wodę, znowu pragnąć będzie; Ale kto napije się wody, którą Ja mu dam, nie będzie pragnął na wieki, lecz woda, którą Ja mu dam, stanie się w nim źródłem wody wytryskującej ku żywotowi wiecznemu. Rzecze do niego niewiasta: Panie, daj mi tej wody, abym nie pragnęła i tu nie przychodziła, by czerpać wodę.'' \mbox{(J 4, 7-15)}
\end{quote}
\paragraph{}
\begin{quote}
,,A w ostatnim, wielkim dniu święta stanął Jezus i głośno zawołał: Jeśli kto pragnie, niech przyjdzie do mnie i pije. Kto wierzy we mnie, jak powiada Pismo, z wnętrza jego popłyną rzeki wody żywej. A to mówił o Duchu, którego mieli otrzymać ci, którzy w niego uwierzyli; albowiem Duch Święty nie był jeszcze dany, gdyż Jezus nie był jeszcze uwielbiony.'' \mbox{(J 7, 37-39)}
\end{quote}
\paragraph{}
\begin{quote}
,,Przeto pamiętajcie o tym, że wy, niegdyś poganie w ciele, nazywani nieobrzezanymi przez tych, których nazywają obrzezanymi na skutek obrzezki, dokonanej ręką na ciele, Byliście w tym czasie bez Chrystusa, dalecy od społeczności izraelskiej i obcy przymierzom, zawierającym obietnicę, nie mający nadziei i bez Boga na świecie. Ale teraz wy, którzy niegdyś byliście dalecy, staliście się w Chrystusie Jezusie bliscy przez krew Chrystusową.'' \mbox{(Ef 2, 11-13)}
\end{quote}
\end{document}