\documentclass[10pt,a4paper,oneside]{article}
\usepackage[utf8]{inputenc}
\usepackage{polski}
\usepackage[polish]{babel}
\usepackage[margin=0.5in,bottom=0.75in]{geometry}
\begin{document}
\centerline{\textbf{\MakeUppercase{Wylanie Ducha Świętego}}}
\begin{center}
\textbf{Wersety do studium:} 
\mbox{(Iz 32, 9-18)}, \mbox{(Jl 3, 1-5)}, \mbox{(Za 12, 10)}, \mbox{(Iz 44, 1-5)}, \mbox{(Ez 39, 25-29)}
\end{center}
\paragraph{}
\begin{quote}
,,Wy, beztroskie kobiety, powstańcie, słuchajcie mojego głosu, wy, córki, pewne siebie, nadstawcie uszu na moje słowo! Nie dłużej niż za rok będziecie drżeć, wy pewne siebie! Gdyż winobranie się skończyło, a owocobrania już nie będzie. Drżyjcie, wy beztroskie, przeraźcie się, wy zadufane! Ściągnijcie szaty i obnażcie się, a przepaszcie biodra. Uderzcie się w piersi, narzekając nad rozkosznymi polami, nad urodzajnymi winnicami, Nad rolą mojego ludu, która porasta cierniem i ostem, nad wszystkimi rozkosznymi domami, nad miastem wesołym! Bo pałace są opuszczone, ustała wrzawa miasta, zamek i baszta strażnicza stały się na zawsze jaskiniami, miejscem uciechy dzikich osłów, pastwiskiem trzód, Aż będzie wylany na nas Duch z wysokości. Wtedy pustynia stanie się urodzajnym polem, a urodzajne pole będzie uważane za las. I zamieszka na pustyni prawo, a sprawiedliwość osiądzie na urodzajnym polu. I pokój stanie się dziełem sprawiedliwości, a niezakłócone bezpieczeństwo owocem sprawiedliwości po wszystkie czasy. I zamieszka mój lud w siedzibie pokoju, w bezpiecznych mieszkaniach i w miejscach spokojnego odpoczynku.'' \mbox{(Iz 32, 9-18)}
\end{quote}
\paragraph{}
\begin{quote}
,,A potem wyleję mojego Ducha na wszelkie ciało, i wasi synowie i wasze córki prorokować będą, wasi starcy będą śnili, a wasi młodzieńcy będą mieli widzenia. Także na sługi i służebnice wyleję w owych dniach mojego Ducha. I ukażę znaki na niebie i na ziemi, krew, ogień i słupy dymu. Słońce przemieni się w ciemność, a księżyc w krew, zanim przyjdzie ów wielki, straszny dzień Pana. Wszakże każdy, kto będzie wzywał imienia Pana, będzie wybawiony; gdyż na górze Syjon i w Jeruzalemie będzie wybawienie jak rzekł Pan a wśród pozostałych przy życiu będą ci, których powoła Pan.'' \mbox{(Jl 3, 1-5)}
\end{quote}
\paragraph{}
\begin{quote}
,,Lecz na dom Dawida i na mieszkańców Jeruzalemu wyleję ducha łaski i błagania. Wtedy spojrzą na mnie, na tego, którego przebodli, i będą go opłakiwać, jak opłakuje się jedynaka, i będą gorzko biadać nad nim, jak gorzko biadają nad pierworodnym.'' \mbox{(Za 12, 10)}
\end{quote}
\paragraph{}
\begin{quote}
,,Lecz teraz słuchaj, Jakubie, mój sługo, i ty, Izraelu, którego wybrałem. Tak mówi Pan, który cię uczynił i ukształtował w łonie matki, który cię wspomaga. Nie bój się, mój sługo, Jakubie, i Jeszurunie, którego wybrałem! Gdyż wyleję wody na spieczoną ziemię i strumienie na suchy ląd; wyleję mojego Ducha na twoje potomstwo i moje błogosławieństwo na twoje latorośle, Aby się rozkrzewiły jak trawa między wodami, jak topole nad ruczajami. Jeden powie wówczas: Ja należę do Pana! Drugi nazwie się imieniem Jakuba, a inny wypisze na swojej ręce: "Własność Pana" - i otrzyma zaszczytne imię "Izrael".'' \mbox{(Iz 44, 1-5)}
\end{quote}
\paragraph{}
\begin{quote}
,,Dlatego tak mówi Wszechmocny Pan: Teraz odmienię los Jakuba, zmiłuję się nad całym domem izraelskim i zadbam gorliwie o święte moje imię. I zapomną o swojej hańbie i wszelkim swoim odstępstwie, którego się dopuścili wobec mnie, gdy bezpiecznie mieszkać będą na swojej ziemi i nikt ich nie będzie straszył. Gdy sprowadzę ich spośród narodów i zgromadzę ich z ziem ich wrogów, wtedy okażę się na nich świętym na oczach wielu narodów. I poznają, że Ja, Pan, jestem ich Bogiem! Gdyż kazałem im iść do niewoli wśród narodów, ale teraz zgromadzę ich na ich ziemi i nie pozostawię tam nikogo z nich. Nie zakryję przed nimi mojego oblicza, ponieważ wylałem mojego ducha na dom izraelski - mówi Wszechmocny Pan.'' \mbox{(Ez 39, 25-29)}
\end{quote}
\end{document}