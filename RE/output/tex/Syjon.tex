\documentclass[10pt,a4paper,oneside]{article}
\usepackage[utf8]{inputenc}
\usepackage{polski}
\usepackage[polish]{babel}
\usepackage[margin=0.5in,bottom=0.75in]{geometry}
\begin{document}
\centerline{\textbf{\MakeUppercase{Syjon}}}
\begin{center}
\textbf{Wersety do studium:} \mbox{(Iz 1, 27-28)}, \mbox{(Iz 2, 2-3)}, \mbox{(Iz 4, 3-6)}, \mbox{(Iz 8, 18)}, \mbox{(Iz 10, 12-14)}, \mbox{(Iz 12, 1-6)}, \mbox{(Iz 51, 2-3)}, \mbox{(Iz 52, 1-2)}, \mbox{(Iz 52, 7-10)}, \mbox{(Ab 1, 16-17)}
\end{center}
\section{Temat główny}
\paragraph{}
\begin{quote}
,,Syjon będzie odkupiony przez sąd, a ci, którzy się w nim nawrócą, przez sprawiedliwość. Odstępców zaś oraz grzeszników spotka zagłada, a ci, którzy opuszczają Pana, zginą.'' \mbox{(Iz 1, 27-28)}
\end{quote}
\paragraph{}
\begin{quote}
,,I stanie się w dniach ostatecznych, że góra ze świątynią Pana będzie stać mocno jako najwyższa z gór i będzie wyniesiona ponad pagórki, a tłumnie będą do niej zdążać wszystkie narody. I pójdzie wiele ludów, mówiąc: Pójdźmy w pielgrzymce na górę Pana, do świątyni Boga Jakuba, i będzie nas uczył dróg swoich, abyśmy mogli chodzić jego ścieżkami, gdyż z Syjonu wyjdzie zakon, a słowo Pana z Jeruzalemu.'' \mbox{(Iz 2, 2-3)}
\end{quote}
\paragraph{}
\begin{quote}
,,I będzie tak, że kto pozostanie na Syjonie i ostoi się w Jeruzalemie, będzie nazwany świętym; każdy, kto jest zapisany wśród żywych w Jeruzalemie. Gdy Pan zmyje brud córek syjońskich i usunie plamy krwi z Jeruzalemu tchnieniem sądu i tchnieniem zniszczenia, Wówczas stworzy Pan nad całym obszarem góry Syjon i nad jej zgromadzeniami obłok w dzień, a dym i blask płomieni ognia w nocy, gdyż nad wszystkim rozciągać się będzie chwała niby osłona I namiot, aby w dzień dać cień przed skwarem oraz ostoję i schronienie przed burzą i deszczem.'' \mbox{(Iz 4, 3-6)}
\end{quote}
\paragraph{}
\begin{quote}
,,Oto ja i dzieci, które mi dał Pan, jesteśmy znakami i przepowiedniami w Izraelu od Pana Zastępów, który mieszka na górze Syjon.'' \mbox{(Iz 8, 18)}
\end{quote}
\paragraph{}
\begin{quote}
,,I stanie się, że gdy Pan dokona całego swojego dzieła na górze Syjon i w Jeruzalemie, wtedy ukarze owoc pychy serce króla asyryjskiego i chełpliwą wyniosłość jego oczu. Rzekł bowiem: Uczyniłem to mocą swojej ręki i dzięki swojej mądrości, gdyż jestem rozumny; zniosłem granice ludów i zrabowałem ich skarby, i jak mocarz strąciłem siedzących na tronie. I sięgnęła moja ręka po bogactwo ludów jakby po gniazdo, a jak zagarniają opuszczone jajka, tak ja zagarnąłem całą ziemię, a nie było takiego, kto by poruszył skrzydłem albo otworzył dziób i pisnął.'' \mbox{(Iz 10, 12-14)}
\end{quote}
\paragraph{}
\begin{quote}
,,I powiesz w owym dniu: Dziękuję ci, Panie, gdyż gniewałeś się wprawdzie na mnie, lecz twój gniew ustał i pocieszyłeś mnie. Oto, Bóg zbawieniem moim! Zaufam i nie będę się lękał: gdyż Pan jest mocą moją i pieśnią moją, i zbawieniem moim. I będziecie czerpać z radością ze zdrojów zbawienia. I będziecie mówić w owym dniu: Dziękujcie Panu, wzywajcie jego imienia, opowiadajcie wśród ludów jego sprawy, wspominajcie, że jego imię jest wspaniałe. Grajcie Panu, bo wielkich dzieł dokonał, niech to będzie wiadome na całej ziemi. Wykrzykuj i śpiewaj radośnie, mieszkanko Syjonu, bo wielki jest pośród ciebie Święty Izraelski.'' \mbox{(Iz 12, 1-6)}
\end{quote}
\paragraph{}
\begin{quote}
,,Spójrzcie na Abrahama, waszego ojca, i na Sarę, waszą rodzicielkę, gdyż jego jednego powołałem, lecz pobłogosławiłem go i rozmnożyłem. Gdyż Pan pocieszy Syjon, pocieszy wszystkie jego rozpadliny. Uczyni z jego pustkowia Eden, a z jego pustyni ogród Pana, radość i wesele zapanują w nim, pieśń dziękczynna i dźwięk pieśni.'' \mbox{(Iz 51, 2-3)}
\end{quote}
\paragraph{}
\begin{quote}
,,Obudź się, obudź się, oblecz się w swoją siłę, Syjonie! Oblecz się w swoją odświętną szatę, Jeruzalem, ty święty grodzie, gdyż już nigdy nie wejdzie do ciebie nieobrzezany ani nieczysty! Strząśnij z siebie proch, powstań branko jeruzalemska, zdejmij z twojej szyi okowy, wzięta do niewoli córko syjońska!'' \mbox{(Iz 52, 1-2)}
\end{quote}
\paragraph{}
\begin{quote}
,,Jak miłe są na górach nogi tego, który zwiastuje radosną wieść, który ogłasza pokój, który zwiastuje dobro, który ogłasza zbawienie, który mówi do Syjonu: Twój Bóg jest królem. Słuchaj! Twoi strażnicy podnoszą głos, razem radośnie wykrzykują, bo na własne oczy oglądają, jak Pan wraca na Syjon. Wykrzykujcie! Śpiewajcie radośnie razem, gruzy Jeruzalemu, gdyż Pan pociesza swój lud, wykupuje Jeruzalem! Pan obnażył swoje święte ramię na oczach wszystkich narodów i oglądają wszystkie krańce ziemi zbawienie naszego Boga.'' \mbox{(Iz 52, 7-10)}
\end{quote}
\paragraph{}
\begin{quote}
,,Bo jak piliście na mojej świętej górze, tak pić będą stale wszystkie narody; pić będą i chłeptać, i będzie z nimi tak, jak gdyby ich nigdy nie było. Lecz na górze Syjon będzie wybawienie, jest ona przecież święta, i dom Jakuba przejmie w posiadanie tych, którzy go posiadłości pozbawili.'' \mbox{(Ab 1, 16-17)}
\end{quote}
\end{document}