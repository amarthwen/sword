\documentclass[10pt,a4paper,oneside]{article}
\usepackage[utf8]{inputenc}
\usepackage{polski}
\usepackage[polish]{babel}
\usepackage[margin=0.5in,bottom=0.75in]{geometry}
\begin{document}
\centerline{\textbf{\MakeUppercase{Duchy nieczyste}}}
\begin{center}
\textbf{Wersety do studium:} \mbox{(Obj 16, 13-14)}, \mbox{(Łk 11, 24-26)}, \mbox{(1Krl 22, 18-22)}, \mbox{(Mi 2, 11)}, \mbox{(Iz 19, 13-14)}, \mbox{(Oz 4, 12; 5, 4)}, \mbox{(Mk 9, 25-27)}, \mbox{(Dz 16, 16-18)}
\end{center}
\section{Temat główny}
\paragraph{}
\begin{quote}
,,Gdy duch nieczysty wyjdzie z człowieka, wędruje po miejscach bezwodnych, szukając ukojenia, a gdy nie znajdzie, mówi: Wrócę do domu swego, skąd wyszedłem. I przyszedłszy, zastaje go wymiecionym i przyozdobionym. Wówczas idzie i zabiera z sobą siedem innych duchów, gorszych niż on, i wchodzą, i mieszkają tam. I bywa końcowy stan człowieka tego, gorszy niż pierwotny.'' \mbox{(Łk 11, 24-26)}
\end{quote}
\subsection{Różne rodzaje duchów}
\subsubsection{Duch kłamstwa}
\paragraph{}
\begin{quote}
,,Wtedy król izraelski rzekł do Jehoszafata: Czy nie mówiłem ci, że ten nie zwiastuje mi nic dobrego, a tylko złe? On zaś rzekł: Słuchaj przeto słowa Pańskiego: Widziałem Pana siedzącego na swoim tronie, a cały zastęp niebieski stał przy nim, po jego prawicy i po lewicy. A Pan rzekł: Kto zwiedzie Achaba, aby wyruszył i poległ w Ramot Gileadzkim? I jeden mówił to, a drugi owo. Wtedy wystąpił Duch i stanął przed Panem, i rzekł: Ja go zwiodę. A Pan rzekł do niego: W jaki sposób? A on odpowiedział: Wyjdę i stanę się duchem kłamliwym w ustach wszystkich jego proroków. Wtedy On rzekł: Tak, ty go zwiedziesz, ty to potrafisz. Idź więc i uczyń tak!'' \mbox{(1 Krl 22, 18-22)}
\end{quote}
\paragraph{}
\begin{quote}
,,Gdyby ktoś przyszedł i w duchu kłamstwa i fałszu mówił: Będę wam prorokował o winie i trunku - to byłby kaznodzieja dla tego ludu!'' \mbox{(Mi 2, 11)}
\end{quote}
\subsubsection{Duch obłędu}
\paragraph{}
\begin{quote}
,,Zgłupieli książęta Soanu, obałamuceni są książęta Memfisu, na manowce sprowadzili Egipt naczelnicy jego okręgów. Pan wylał wśród nich ducha obłędu, i oni na manowce sprowadzili Egipt we wszelkim jego działaniu, że jak pijany tarza się w swoich wymiocinach.'' \mbox{(Iz 19, 13-14)}
\end{quote}
\subsubsection{Duch wszeteczeństwa}
\paragraph{}
\begin{quote}
,,Mój lud radzi się swojego drewna, a jego kij daje mu wyrocznię, gdyż duch wszeteczeństwa ich omamił, a cudzołożąc odstąpili od swojego Boga. (\ldots) Ich uczynki nie pozwalają im zawrócić do swojego Boga, gdyż duch wszeteczeństwa jest w ich sercu, tak że nie znają Pana.'' \mbox{(Oz 4, 12; 5, 4)}
\end{quote}
\subsubsection{Duch głuchy i niemy}
\paragraph{}
\begin{quote}
,,A Jezus, widząc, że tłum się zbiega, zgromił ducha nieczystego i rzekł mu: Duchu niemy i głuchy! Nakazuję ci: Wyjdź z niego i już nigdy do niego nie wracaj. I krzyknął, i szarpnął nim gwałtownie, po czym wyszedł; a chłopiec wyglądał jak martwy, tak iż wielu mówiło, że umarł. A Jezus ujął go za rękę i podniósł go, a on powstał.'' \mbox{(Mk 9, 25-27)}
\end{quote}
\subsubsection{Duch wieszczy}
\paragraph{}
\begin{quote}
,,A gdyśmy przyszli na modlitwę, zdarzyło się, że spotkała nas pewna dziewczyna, która miała ducha wieszczego, a która przez swoje wróżby przynosiła wielki zysk panom swoim. Ta, idąc za Pawłem i za nami, wołała mówiąc: Ci ludzie są sługami Boga Najwyższego i zwiastują wam drogę zbawienia. A to czyniła przez wiele dni. Wreszcie Paweł znękany, zwrócił się do ducha i rzekł: Rozkazuję ci w imieniu Jezusa Chrystusa, żebyś z niej wyszedł. I w tej chwili wyszedł.'' \mbox{(Dz 16, 16-18)}
\end{quote}
\end{document}