\documentclass[10pt,a4paper,oneside]{article}
\usepackage[utf8]{inputenc}
\usepackage{polski}
\usepackage[polish]{babel}
\usepackage[margin=0.5in,bottom=0.75in]{geometry}
\begin{document}
\centerline{\textbf{\MakeUppercase{Życie według Ducha}}}
\begin{center}
\textbf{Wersety do studium:} 
\mbox{(Ga 5, 13.16-25; 6, 8)}, \mbox{(Rz 8, 14-17)}, \mbox{(J 3, 1-8)}, \mbox{(Rz 8, 1-9)}
\end{center}
\section{Temat główny}
\paragraph{}
\begin{quote}
,,Bo wy do wolności powołani zostaliście, bracia; tylko pod pozorem tej wolności nie pobłażajcie ciału, ale służcie jedni drugim w miłości. (\ldots) Mówię więc: Według Ducha postępujcie, a nie będziecie pobłażali żądzy cielesnej. Gdyż ciało pożąda przeciwko Duchowi, a Duch przeciwko ciału, a te są sobie przeciwne, abyście nie czynili tego, co chcecie. A jeśli Duch was prowadzi, nie jesteście pod zakonem. Jawne zaś są uczynki ciała, mianowicie: wszeteczeństwo, nieczystość, rozpusta, Bałwochwalstwo, czary, wrogość, spór, zazdrość, gniew, knowania, waśnie, odszczepieństwo Zabójstwa, pijaństwo, obżarstwo i tym podobne; o tych zapowiadam wam, jak już przedtem zapowiedziałem, że ci, którzy te rzeczy czynią, Królestwa Bożego nie odziedziczą. Owocem Ducha zaś są: miłość, radość, pokój, cierpliwość, uprzejmość, dobroć, wierność, Łagodność, wstrzemięźliwość. Przeciwko takim nie ma zakonu. A ci, którzy należą do Chrystusa Jezusa, ukrzyżowali ciało swoje wraz z namiętnościami i żądzami. Jeśli według Ducha żyjemy, według Ducha też postępujmy. (\ldots) Bo kto sieje dla ciała swego, z ciała żąć będzie skażenie, a kto sieje dla Ducha, z Ducha żąć będzie żywot wieczny.'' \mbox{(Ga 5, 13.16-25; 6, 8)}
\end{quote}
\paragraph{}
\begin{quote}
,,Bo ci, których Duch Boży prowadzi, są dziećmi Bożymi. Wszak nie wzięliście ducha niewoli, by znowu ulegać bojaźni, lecz wzięliście ducha synostwa, w którym wołamy: Abba, Ojcze! Ten to Duch świadczy wespół z duchem naszym, że dziećmi Bożymi jesteśmy. A jeśli dziećmi, to i dziedzicami, dziedzicami Bożymi, a współdziedzicami Chrystusa, jeśli tylko razem z nim cierpimy, abyśmy także razem z nim uwielbieni byli.'' \mbox{(Rz 8, 14-17)}
\end{quote}
\paragraph{}
\begin{quote}
,,A był człowiek z faryzeuszów imieniem Nikodem, dostojnik żydowski. Ten przyszedł do Jezusa w nocy i rzekł mu: Mistrzu! Wiemy, że przyszedłeś od Boga jako nauczyciel; nikt bowiem takich cudów czynić by nie mógł, jakie Ty czynisz, jeśliby Bóg z nim nie był. Odpowiadając Jezus, rzekł mu: Zaprawdę, zaprawdę, powiadam ci, jeśli się kto nie narodzi na nowo, nie może ujrzeć Królestwa Bożego. Rzekł mu Nikodem: Jakże się może człowiek narodzić, gdy jest stary? Czyż może powtórnie wejść do łona matki swojej i urodzić się? Odpowiedział Jezus: Zaprawdę, zaprawdę, powiadam ci, jeśli się kto nie narodzi z wody i z Ducha, nie może wejść do Królestwa Bożego. Co się narodziło z ciała, ciałem jest, a co się narodziło z Ducha, duchem jest. Nie dziw się, że ci powiedziałem: Musicie się na nowo narodzić. Wiatr wieje, dokąd chce, i szum jego słyszysz, ale nie wiesz, skąd przychodzi i dokąd idzie; tak jest z każdym, kto się narodził z Ducha.'' \mbox{(J 3, 1-8)}
\end{quote}
\paragraph{}
\begin{quote}
,,Przeto teraz nie ma żadnego potępienia dla tych, którzy są w Chrystusie Jezusie. Bo zakon Ducha, który daje życie w Chrystusie Jezusie, uwolnił cię od zakonu grzechu i śmierci. Albowiem czego zakon nie mógł dokonać, w czym był słaby z powodu ciała, tego dokonał Bóg: przez zesłanie Syna swego w postaci grzesznego ciała, ofiarując je za grzech, potępił grzech w ciele, Aby słuszne żądania zakonu wykonały się na nas, którzy nie według ciała postępujemy, lecz według Ducha. Bo ci, którzy żyją według ciała, myślą o tym, co cielesne; ci zaś, którzy żyją według Ducha, o tym, co duchowe. Albowiem zamysł ciała, to śmierć, a zamysł Ducha, to życie i pokój. Dlatego zamysł ciała jest wrogi Bogu; nie poddaje się bowiem zakonowi Bożemu, bo też nie może. Ci zaś, którzy są w ciele, Bogu podobać się nie mogą. Ale wy nie jesteście w ciele, lecz w Duchu, jeśli tylko Duch Boży mieszka w was. Jeśli zaś kto nie ma Ducha Chrystusowego, ten nie jest jego.'' \mbox{(Rz 8, 1-9)}
\end{quote}
\end{document}