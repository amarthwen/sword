\documentclass[10pt,a4paper,oneside]{article}
\usepackage[utf8]{inputenc}
\usepackage{polski}
\usepackage[polish]{babel}
\usepackage[margin=0.5in,bottom=0.75in]{geometry}
\begin{document}
\centerline{\textbf{\MakeUppercase{Chleb żywy, który zstąpił z nieba}}}
\begin{center}
\textbf{Wersety do studium:} \mbox{(Mt 26, 26)}, \mbox{(2Moj 16, 14-15)}, \mbox{(1Kor 10, 1-4)}, \mbox{(J 6, 32-58)}, \mbox{(Łk 22, 19)}, \mbox{(Mt 4, 1-4)}, \mbox{(J 4, 31-34)}, \mbox{(J 1, 1-5.14)}
\end{center}
\section{Temat główny}
\paragraph{}
\begin{quote}
,,A gdy oni jedli, wziął Jezus chleb i pobłogosławił, łamał i dawał uczniom, i rzekł: Bierzcie, jedzcie, to jest ciało moje.'' \mbox{(Mt 26, 26)}
\end{quote}
\paragraph{}
\begin{quote}
,,A gdy warstwa rosy się podniosła, oto na powierzchni pustyni było coś drobnego, ziarnistego, drobnego niby szron na ziemi. Gdy to ujrzeli synowie izraelscy, mówili jeden do drugiego: Co to jest? - bo nie wiedzieli, co to było. A Mojżesz rzekł do nich: To jest chleb, który Pan dał wam do jedzenia.'' \mbox{(2 Moj 16, 14-15)}
\end{quote}
\paragraph{}
\begin{quote}
,,A chcę, bracia, abyście dobrze wiedzieli, że ojcowie nasi wszyscy byli pod obłokiem i wszyscy przez morze przeszli, I wszyscy w Mojżesza ochrzczeni zostali w obłoku i w morzu, I wszyscy ten sam pokarm duchowy jedli, I wszyscy ten sam napój duchowy pili; pili bowiem z duchowej skały, która im towarzyszyła, a skałą tą był Chrystus.'' \mbox{(1 Kor 10, 1-4)}
\end{quote}
\paragraph{}
\begin{quote}
,,Wtedy rzekł im Jezus: Zaprawdę, zaprawdę, powiadam wam, nie Mojżesz dał wam chleb z nieba, ale Ojciec mój daje wam prawdziwy chleb z nieba. Albowiem chleb Boży to ten, który z nieba zstępuje i daje światu żywot. Wtedy rzekli do niego: Panie! Dawaj nam zawsze tego chleba! Odpowiedział im Jezus: Ja jestem chlebem żywota; kto do mnie przychodzi, nigdy łaknąć nie będzie, a kto wierzy we mnie, nigdy pragnąć nie będzie. Lecz powiedziałem wam: Nie wierzycie, chociaż widzieliście mnie. Wszystko, co mi daje Ojciec, przyjdzie do mnie, a tego, który do mnie przychodzi, nie wyrzucę precz; Zstąpiłem bowiem z nieba, nie aby wypełniać wolę swoją, lecz wolę tego, który mnie posłał. A to jest wola tego, który mnie posłał, abym z tego wszystkiego, co mi dał, nic nie stracił, lecz wskrzesił to w dniu ostatecznym. A to jest wola Ojca mego, aby każdy, kto widzi Syna i wierzy w niego, miał żywot wieczny, a Ja go wzbudzę w dniu ostatecznym. Wtedy Żydzi szemrali przeciwko niemu, iż powiedział: Ja jestem chlebem, który zstąpił z nieba. I mówili: Czy to nie jest Jezus, syn Józefa, którego ojca i matkę znamy? Jakże więc teraz może mówić: Z nieba zstąpiłem? Wtedy Jezus odpowiedział i rzekł im: Nie szemrajcie między sobą! Nikt nie może przyjść do mnie, jeżeli go nie pociągnie Ojciec, który mnie posłał, a Ja go wskrzeszę w dniu ostatecznym. Napisano bowiem u proroków: I będą wszyscy pouczeni przez Boga. Każdy, kto słyszał od Ojca i jest pouczony, przychodzi do mnie. Nie jakoby ktoś widział Ojca: Ojca widział tylko Ten, który jest od Boga. Zaprawdę, zaprawdę powiadam wam, kto wierzy we mnie, ma żywot wieczny. Ja jestem chlebem żywota. Ojcowie wasi jedli mannę na pustyni i poumierali; Tu natomiast jest chleb, który zstępuje z nieba, aby nie umarł ten, kto go spożywa. Ja jestem chlebem żywota, który z nieba zstąpił; jeśli kto spożywać będzie ten chleb, który Ja dam, to ciało moje, które Ja oddam za żywot świata. Wtedy sprzeczali się Żydzi między sobą, mówiąc: Jakże Ten może dać nam swoje ciało do jedzenia? Na to rzekł im Jezus: Zaprawdę, zaprawdę powiadam wam, jeśli nie będziecie jedli ciała Syna Człowieczego i pili krwi jego, nie będziecie mieli żywota w sobie. Kto spożywa ciało moje i pije krew moją, ten ma żywot wieczny, a Ja go wskrzeszę w dniu ostatecznym. Albowiem ciało moje jest prawdziwym pokarmem, a krew moja jest prawdziwym napojem. Kto spożywa ciało moje i pije krew moją, we mnie mieszka, a Ja w nim. Jak mię posłał Ojciec, który żyje, a Ja przez Ojca żyję, tak i ten, kto mnie spożywa, żyć będzie przeze mnie. Taki jest chleb, który z nieba zstąpił, nie taki, jaki jedli ojcowie i poumierali; kto spożywa ten chleb, żyć będzie na wieki.'' \mbox{(J 6, 32-58)}
\end{quote}
\paragraph{}
\begin{quote}
,,I wziąwszy chleb, i podziękowawszy, łamał i dawał im, mówiąc: To jest ciało moje, które się za was daje; to czyńcie na pamiątkę moją.'' \mbox{(Łk 22, 19)}
\end{quote}
\paragraph{}
\begin{quote}
,,Wtedy Duch zaprowadził Jezusa na pustynię, aby go kusił diabeł. A gdy pościł czterdzieści dni i czterdzieści nocy, wówczas łaknął. I przystąpił do niego kusiciel, i rzekł mu: Jeżeli jesteś Synem Bożym, powiedz, aby te kamienie stały się chlebem. A On odpowiadając, rzekł: Napisano: Nie samym chlebem żyje człowiek, ale każdym słowem, które pochodzi z ust Bożych.'' \mbox{(Mt 4, 1-4)}
\end{quote}
\paragraph{}
\begin{quote}
,,Tymczasem jego uczniowie, prosili go, mówiąc: Mistrzu, jedz! Ale On rzekł do nich: Ja mam pokarm do jedzenia, o którym wy nie wiecie. Wtedy uczniowie mówili między sobą: Czy kto przyniósł mu jeść? Jezus rzekł do nich: Moim pokarmem jest pełnić wolę tego, który mnie posłał, i dokonać jego dzieła.'' \mbox{(J 4, 31-34)}
\end{quote}
\paragraph{}
\begin{quote}
,,Na początku było Słowo, a Słowo było u Boga, a Bogiem było Słowo. Ono było na początku u Boga. Wszystko przez nie powstało, a bez niego nic nie powstało, co powstało. W nim było życie, a życie było światłością ludzi. A światłość świeci w ciemności, lecz ciemność jej nie przemogła. (\ldots) A Słowo ciałem się stało i zamieszkało wśród nas, i ujrzeliśmy chwałę jego, chwałę, jaką ma jedyny Syn od Ojca, pełne łaski i prawdy.'' \mbox{(J 1, 1-5.14)}
\end{quote}
\end{document}