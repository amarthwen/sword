\documentclass[10pt,a4paper,oneside]{article}
\usepackage[utf8]{inputenc}
\usepackage{polski}
\usepackage[polish]{babel}
\usepackage[margin=0.5in,bottom=0.75in]{geometry}
\begin{document}
\centerline{\textbf{\MakeUppercase{Wolność w Chrystusie}}}
\begin{center}
\textbf{Wersety do studium:} \mbox{(Ga 2, 3-5)}, \mbox{(Ga 4, 6-7)}, \mbox{(2Kor 3, 17)}, \mbox{(Ga 4, 29-31)}
\end{center}
\section{Temat główny}
\paragraph{}
\begin{quote}
,,Ale nawet Tytusa, który był ze mną, chociaż był Grekiem, nie zmuszono do obrzezania, Nie bacząc na fałszywych braci, którzy po kryjomu zostali wprowadzeni i potajemnie weszli, aby wyszpiegować naszą wolność, którą mamy w Chrystusie, żeby nas podbić w niewolę, A którym ani na chwilę nie ustąpiliśmy, ani się nie poddaliśmy, aby prawda ewangelii u was się utrzymała.'' \mbox{(Ga 2, 3-5)}
\end{quote}
\paragraph{}
\begin{quote}
,,A ponieważ jesteście synami, przeto Bóg zesłał Ducha Syna swego do serc waszych, wołającego: Abba, Ojcze! Tak więc już nie jesteś niewolnikiem, lecz synem a jeśli synem, to i dziedzicem przez Boga.'' \mbox{(Ga 4, 6-7)}
\end{quote}
\paragraph{}
\begin{quote}
,,A Pan jest Duchem; gdzie zaś Duch Pański, tam wolność.'' \mbox{(2 Kor 3, 17)}
\end{quote}
\paragraph{}
\begin{quote}
,,Lecz jak ongiś ten, który się według ciała narodził, prześladował urodzonego według Ducha, tak i teraz. Lecz co mówi Pismo? Wypędź niewolnicę i syna jej; nie będzie bowiem dziedziczył syn niewolnicy z synem wolnej. Przeto, bracia, nie jesteśmy dziećmi niewolnicy, lecz wolnej!'' \mbox{(Ga 4, 29-31)}
\end{quote}
\end{document}