\documentclass[10pt,a4paper,oneside]{article}
\usepackage[utf8]{inputenc}
\usepackage{polski}
\usepackage[polish]{babel}
\usepackage[margin=0.5in,bottom=0.75in]{geometry}
\begin{document}
\centerline{\textbf{\MakeUppercase{Kielich wina gniewu Bożego}}}
\begin{center}
\textbf{Wersety do studium:} 
\mbox{(Obj 14, 9-10)}, \mbox{(Obj 19, 13-15)}, \mbox{(Obj 14, 18-20)}, \mbox{(Jr 25, 15-16)}, \mbox{(Ab 1, 5)}, \mbox{(Jl 3, 18-20)}, \mbox{(Ha 2, 16)}
\end{center}
\section{Temat główny}
\paragraph{}
\begin{quote}
,,A trzeci anioł szedł za nimi, mówiąc donośnym głosem: Jeżeli ktoś odda pokłon zwierzęciu i jego posągowi i przyjmie znamię na swoje czoło lub na swoją rękę, To i on pić będzie samo czyste wino gniewu Bożego z kielicha jego gniewu i będzie męczony w ogniu i w siarce wobec świętych aniołów i wobec Baranka.'' \mbox{(Obj 14, 9-10)}
\end{quote}
\paragraph{}
\begin{quote}
,,A przyodziany był w szatę zmoczoną we krwi, imię zaś jego brzmi: Słowo Boże. I szły za nim wojska niebieskie na białych koniach, przyobleczone w czysty, biały bisior. A z ust jego wychodzi ostry miecz, którym miał pobić narody, i będzie nimi rządził laską żelazną, On sam też tłoczy kadź wina zapalczywego gniewu Boga, Wszechmogącego.'' \mbox{(Obj 19, 13-15)}
\end{quote}
\paragraph{}
\begin{quote}
,,I jeszcze inny anioł wyszedł z ołtarza, a ten miał władzę nad ogniem; i zawołał donośnie na tego, który miał ostry sierp, mówiąc: Zapuść swój ostry sierp i obetnij kiście winogron z winorośli ziemi, gdyż dojrzały jej grona. I zapuścił anioł sierp swój na ziemi, i poobcinał grona winne na ziemi, i wrzucił je do wielkiej tłoczni gniewu Bożego. I deptano tłocznię poza miastem, i popłynęła z tłoczni krew, aż dosięgła wędzideł końskich na przestrzeni tysiąca sześciuset stadiów.'' \mbox{(Obj 14, 18-20)}
\end{quote}
\paragraph{}
\begin{quote}
,,Tak bowiem rzekł do mnie Pan, Bóg Izraela: Weź z mojej ręki ten kielich wina gniewu i daj z niego pić wszystkim narodom, do których cię wysyłam, Aby piły i zataczały się, i szalały przed mieczem, który między nie posyłam.'' \mbox{(Jr 25, 15-16)}
\end{quote}
\paragraph{}
\begin{quote}
,,Gdy złodzieje wtargną do ciebie albo nocni rabusie jakże będziesz splądrowany! Czy nie będą kradli do woli? Gdy zbieracze winogron przyjdą do ciebie, czy pozostawią choć jedno grono?'' \mbox{(Ab 1, 5)}
\end{quote}
\paragraph{}
\begin{quote}
,,Zapuśćcie sierp, gdyż nadeszło żniwo! Przyjdźcie, zstąpcie, bo pełna jest prasa; pełne są kadzie, gdyż złość ich jest wielka! Tłumy i tłumy zebrane w Dolinie Sądu, bo bliski jest dzień Pana w Dolinie Sądu. Słońce i księżyc będą zaćmione, a gwiazdy stracą swój blask.'' \mbox{(Jl 3, 18-20)}
\end{quote}
\paragraph{}
\begin{quote}
,,Nasyciłeś się więcej hańbą niż chwałą, pij więc i ty i zataczaj się! Przyjdzie i do ciebie kielich z prawicy Pana, i będziesz syt hańby zamiast chwały.'' \mbox{(Ha 2, 16)}
\end{quote}
\end{document}