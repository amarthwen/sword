\documentclass[10pt,a4paper,oneside]{article}
\usepackage[utf8]{inputenc}
\usepackage{polski}
\usepackage[polish]{babel}
\usepackage[margin=0.5in,bottom=0.75in]{geometry}
\begin{document}
\centerline{\textbf{\MakeUppercase{Obrazy Ludu Bożego w Piśmie Świętym}}}
\begin{center}
\textbf{Wersety do studium:} \mbox{(Iz 5, 1-7)}, \mbox{(Jr 12, 10-12)}, \mbox{(Mt 21, 33-46)}, \mbox{(Jr 2, 21)}, \mbox{(J 15, 1-6)}
\end{center}
\section{Temat główny}
\paragraph{}
\begin{quote}
,,Zaśpiewam mojemu ulubieńcowi ulubioną jego pieśń o jego winnicy. Ulubieniec mój miał winnicę na pagórku urodzajnym. Przekopał ją i oczyścił z kamieni, i zasadził w niej szlachetne szczepy. Zbudował w niej wieżę i wykuł w niej prasę, oczekiwał, że wyda szlachetne grona, lecz ona wydała złe owoce. Teraz więc, obywatele jeruzalemscy i mężowie judzcy, rozsądźcie między mną i między moją winnicą! Cóż jeszcze należało uczynić mojej winnicy, czego ja jej nie uczyniłem? Dlaczego oczekiwałem, że wyda szlachetne grona, a ona wydała złe owoce? Otóż teraz chcę wam ogłosić, co uczynię z moją winnicą: Rozbiorę jej płot, aby ją spasiono, rozwalę jej mur, aby ją zdeptano. Zniszczę ją doszczętnie: Nie będzie przycinana ani okopywana, ale porośnie cierniem i ostem, nadto nakażę obłokom, by na nią nie spuszczały deszczu. Zaiste, winnicą Pana Zastępów jest dom izraelski, a mężowie judzcy ulubioną jego latoroślą. Oczekiwał prawa, a oto - bezprawie; sprawiedliwości, a oto - krzyk.'' \mbox{(Iz 5, 1-7)}
\end{quote}
\paragraph{}
\begin{quote}
,,Liczni pasterze zniszczyli moją winnicę, podeptali mój dział, mój dział rozkoszny zamienili w głuchą pustynię. Obrócono go w pustynię, w żałosną dla mnie pustynię. Spustoszony cały kraj, nikt nie wziął tego do serca. Na wszystkie gołe wzniesienia w pustyni wtargnęli rabusie, gdyż miecz Pana pożera od jednego krańca ziemi do drugiego krańca ziemi. Żadne ciało nie ma spokoju.'' \mbox{(Jr 12, 10-12)}
\end{quote}
\paragraph{}
\begin{quote}
,,Innego podobieństwa wysłuchajcie: Był pewien gospodarz, który zasadził winnicę, ogrodził ją płotem, wkopał w nią prasę i zbudował wieżę, i wydzierżawił ją wieśniakom, i odjechał. A gdy nastał czas winobrania, posłał sługi swoje do wieśniaków, aby odebrali jego owoce. Ale wieśniacy pojmali sługi jego; jednego zbili, drugiego zabili, a trzeciego ukamienowali. Znowu posłał inne sługi w większej liczbie niż za pierwszym razem, ale im uczynili to samo. A w końcu posłał do nich syna swego, mówiąc: Uszanują syna mego. Ale gdy wieśniacy ujrzeli syna, mówili między sobą: To jest dziedzic; nuże, zabijmy go, a posiądziemy dziedzictwo jego. I pochwycili go, wyrzucili poza winnicę i zabili. Gdy więc przyjdzie pan winnicy, co uczyni owym wieśniakom? Mówią mu: Wytraci sromotnie tych złoczyńców, a winnicę wydzierżawi innym wieśniakom, którzy mu we właściwym czasie będą oddawać owoce. Rzecze im Jezus: Czy nie czytaliście nigdy w Pismach: Kamień, który odrzucili budowniczowie, stał się kamieniem węgielnym; Pan to sprawił i to jest cudowne w oczach naszych? Dlatego powiadam wam, że Królestwo Boże zostanie wam zabrane, a dane narodowi, który będzie wydawał jego owoce. I kto by upadł na ten kamień, roztrzaska się, a na kogo by on upadł, zmiażdży go. A gdy arcykapłani i faryzeusze wysłuchali jego podobieństw, zrozumieli, że o nich mówi. I usiłowali go pojmać, ale bali się ludu, gdyż miał go za proroka.'' \mbox{(Mt 21, 33-46)}
\end{quote}
\paragraph{}
\begin{quote}
,,A przecież to Ja zasadziłem cię jako szlachetną winorośl, cały szczep prawdziwy, a jakże mi się zmieniłaś na krzew zwyrodniały, na winorośl dziką!'' \mbox{(Jr 2, 21)}
\end{quote}
\paragraph{}
\begin{quote}
,,Ja jestem prawdziwym krzewem winnym, a Ojciec mój jest winogrodnikiem. Każdą latorośl, która we mnie nie wydaje owocu, odcina, a każdą, która wydaje owoc, oczyszcza, aby wydawała obfitszy owoc. Wy jesteście już czyści dla słowa, które wam głosiłem; Trwajcie we mnie, a Ja w was. Jak latorośl sama z siebie nie może wydawać owocu, jeśli nie trwa w krzewie winnym, tak i wy, jeśli we mnie trwać nie będziecie. Ja jestem krzewem winnym, wy jesteście latoroślami. Kto trwa we mnie, a Ja w nim, ten wydaje wiele owocu; bo beze mnie nic uczynić nie możecie. Kto nie trwa we mnie, ten zostaje wyrzucony precz jak zeschnięta latorośl; takie zbierają i wrzucają w ogień, gdzie spłoną.'' \mbox{(J 15, 1-6)}
\end{quote}
\end{document}