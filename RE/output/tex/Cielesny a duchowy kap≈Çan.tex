\documentclass[10pt,a4paper,oneside]{article}
\usepackage[utf8]{inputenc}
\usepackage{polski}
\usepackage[polish]{babel}
\usepackage[margin=0.5in,bottom=0.75in]{geometry}
\begin{document}
\centerline{\textbf{\MakeUppercase{Cielesny a duchowy kapłan}}}
\begin{center}
\textbf{Wersety do studium:} \mbox{(1P 2, 4-5)}, \mbox{(Hbr 8, 1-2)}, \mbox{(Hbr 9, 6-12)}, \mbox{(Hbr 9, 23-28)}, \mbox{(Hbr 12, 22-24)}, \mbox{(Hbr 12, 18-24)}
\end{center}
\section{Temat główny}
\subsection{Cielesny kapłan}
\subsubsection{Ziemska Świątynia}
\subsubsection{Składanie cielesnych ofiar}
\subsubsection{Oczyszczanie ziemskiej Świątyni}
\subsubsection{Pokropienie cielesną krwią}
\subsubsection{Ziemski Syjon}
\subsection{Duchowy kapłan}
\subsubsection{Niebiańska Świątynia}
\paragraph{}
\begin{quote}
,,Główną zaś rzeczą w tym, co mówimy, jest to, że mamy takiego arcykapłana, który usiadł po prawicy tronu Majestatu w niebie, Jako sługa świątyni i prawdziwego przybytku, który zbudował Pan, a nie człowiek.'' \mbox{(Hbr 8, 1-2)}
\end{quote}
\paragraph{}
\begin{quote}
,,A skoro tak te rzeczy zostały urządzone, kapłani sprawujący służbę Bożą wchodzą stale do pierwszej części przybytku, Do drugiej zaś raz w roku sam tylko arcykapłan, i to nie bez krwi, którą ofiaruje za siebie samego i za uchybienia ludu. Przez to Duch Święty wskazuje wyraźnie, że droga do świątyni nie została jeszcze objawiona, dopóki stoi pierwszy przybytek; Ma to znaczenie obrazowe, odnoszące się do teraźniejszego czasu, kiedy to składane bywają dary i ofiary, które nie mogą doprowadzić do wewnętrznej doskonałości tego, kto pełni służbę Bożą; Są to tylko przepisy zewnętrzne, dotyczące pokarmów i napojów, i różnych obmywań, nałożone do czasu zaprowadzenia nowego porządku. Lecz Chrystus, który się zjawił jako arcykapłan dóbr przyszłych, wszedł przez większy i doskonalszy przybytek, nie ręką zbudowany, to jest nie z tego stworzonego świata pochodzący, Wszedł raz na zawsze do świątyni nie z krwią kozłów i cielców, ale z własną krwią swoją, dokonawszy wiecznego odkupienia.'' \mbox{(Hbr 9, 6-12)}
\end{quote}
\subsubsection{Składanie duchowych ofiar}
\subsubsection{Oczyszczanie niebiańskiej Świątyni}
\paragraph{}
\begin{quote}
,,Jest więc rzeczą konieczną, aby odbicia rzeczy niebieskich były oczyszczane tymi sposobami, same zaś rzeczy niebieskie lepszymi ofiarami aniżeli te. Albowiem Chrystus nie wszedł do świątyni zbudowanej rękami, która jest odbiciem prawdziwej, ale do samego nieba, aby się stawiać teraz za nami przed obliczem Boga; I nie dlatego, żeby wielekroć ofiarować samego siebie, podobnie jak arcykapłan wchodzi do świątyni co roku z cudzą krwią, Gdyż w takim razie musiałby cierpieć wiele razy od początku świata; ale obecnie objawił się On jeden raz u schyłku wieków dla zgładzenia grzechu przez ofiarowanie samego siebie. A jak postanowione jest ludziom raz umrzeć, a potem sąd, Tak i Chrystus, raz ofiarowany, aby zgładzić grzechy wielu, drugi raz ukaże się nie z powodu grzechu, lecz ku zbawieniu tym, którzy go oczekują.'' \mbox{(Hbr 9, 23-28)}
\end{quote}
\subsubsection{Pokropienie duchową krwią}
\paragraph{}
\begin{quote}
,,Lecz wy podeszliście do góry Syjon i do miasta Boga żywego, do Jeruzalem niebieskiego i do niezliczonej rzeszy aniołów, do uroczystego zgromadzenia I zebrania pierworodnych, którzy są zapisani w niebie, i do Boga, sędziego wszystkich, i do duchów ludzi sprawiedliwych, którzy osiągnęli doskonałość, I do pośrednika nowego przymierza, Jezusa, i do krwi, którą się kropi, a która przemawia lepiej niż krew Abla.'' \mbox{(Hbr 12, 22-24)}
\end{quote}
\subsubsection{Niebiański Syjon}
\paragraph{}
\begin{quote}
,,Wy nie podeszliście bowiem do góry, której można dotknąć, do płonącego ognia, mroku, ciemności i burzy Ani do dźwięku trąby i głośnych słów, na których odgłos ci, którzy je słyszeli, prosili, aby już do nich nie przemawiano; Nie mogli bowiem znieść nakazu: Gdyby nawet zwierzę dotknęło się góry, ukamienowane będzie. A tak straszne było to zjawisko, iż Mojżesz powiedział: Jestem przerażony i drżący. Lecz wy podeszliście do góry Syjon i do miasta Boga żywego, do Jeruzalem niebieskiego i do niezliczonej rzeszy aniołów, do uroczystego zgromadzenia I zebrania pierworodnych, którzy są zapisani w niebie, i do Boga, sędziego wszystkich, i do duchów ludzi sprawiedliwych, którzy osiągnęli doskonałość, I do pośrednika nowego przymierza, Jezusa, i do krwi, którą się kropi, a która przemawia lepiej niż krew Abla.'' \mbox{(Hbr 12, 18-24)}
\end{quote}
\end{document}