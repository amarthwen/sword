\documentclass[10pt,a4paper,oneside]{article}
\usepackage[utf8]{inputenc}
\usepackage{polski}
\usepackage[polish]{babel}
\usepackage[margin=0.5in,bottom=0.75in]{geometry}
\begin{document}
\centerline{\textbf{\MakeUppercase{Prawdziwa szata}}}
\begin{center}
\textbf{Wersety do studium:} \mbox{(Obj 3, 17-18)}, \mbox{(Za 3, 1-5)}, \mbox{(Obj 19, 6-8)}
\end{center}
\section{Temat główny}
\paragraph{}
\begin{quote}
,,Ponieważ mówisz: Bogaty jestem i wzbogaciłem się, i niczego nie potrzebuję, a nie wiesz, żeś pożałowania godzien nędzarz i biedak, ślepy i goły, Radzę ci, abyś nabył u mnie złota w ogniu wypróbowanego, abyś się wzbogacił i abyś przyodział szaty białe, aby nie wystąpiła na jaw haniebna nagość twoja, oraz maści, by nią namaścić oczy twoje, abyś przejrzał.'' \mbox{(Obj 3, 17-18)}
\end{quote}
\paragraph{}
\begin{quote}
,,Potem ukazał mi Jozuego, arcykapłana, stojącego przed aniołem Pana i szatana stojącego po jego prawicy, aby go oskarżać. Wtedy anioł Pana rzekł do szatana; Niech cię zgromi Pan, szatanie, niech cię zgromi Pan, który obrał Jeruzalem! Czyż nie jest ono głownią wyrwaną z ognia? A Jozue był ubrany w szatę brudną i tak stał przed aniołem, A ten tak odezwał się i rzekł do sług, którzy stali przed nim: Zdejmijcie z niego brudną szatę! Do niego zaś rzekł: Oto ja zdjąłem z ciebie twoją winę i każę cię przyoblec w szaty odświętne. Potem rzekł: Włóżcie mu na głowę czysty zawój! I włożyli mu na głowę czysty zawój, i przyoblekli go w szaty. A anioł Pana stał przy tym.'' \mbox{(Za 3, 1-5)}
\end{quote}
\paragraph{}
\begin{quote}
,,I usłyszałem jakby głos licznego tłumu i jakby szum wielu wód, i jakby huk potężnych grzmotów, które mówiły: Alleluja! Oto Pan, Bóg nasz, Wszechmogący, objął panowanie. Weselmy się i radujmy się, i oddajmy mu chwałę, gdyż nastało wesele Baranka, i oblubienica jego przygotowała się; I dano jej przyoblec się w czysty, lśniący bisior, a bisior oznacza sprawiedliwe uczynki świętych.'' \mbox{(Obj 19, 6-8)}
\end{quote}
\end{document}