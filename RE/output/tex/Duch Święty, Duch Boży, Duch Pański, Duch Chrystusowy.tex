\documentclass[10pt,a4paper,oneside]{article}
\usepackage[utf8]{inputenc}
\usepackage{polski}
\usepackage[polish]{babel}
\usepackage[margin=0.5in,bottom=0.75in]{geometry}
\begin{document}
\centerline{\textbf{\MakeUppercase{Duch Święty, Duch Boży, Duch Pański, Duch Chrystusowy}}}
\begin{center}
\textbf{Wersety do studium:} 
\mbox{(Dz 16, 6-7)}, \mbox{(Rz 5, 5)}, \mbox{(Rz 8, 8-9)}, \mbox{(Rz 8, 14-15)}, \mbox{(Rz 10, 9-10)}, \mbox{(Rz 10, 17)}, \mbox{(Rz 14, 17)}, \mbox{(1Kor 2, 11)}, \mbox{(1Kor 3, 16)}, \mbox{(1Kor 6, 9-10)}, \mbox{(1Kor 6, 19-20)}, \mbox{(1Kor 12, 3)}, \mbox{(2Kor 3, 17)}, \mbox{(2Kor 6, 16)}, \mbox{(Ga 2, 20)}, \mbox{(Ga 3, 5)}, \mbox{(Ga 4, 4-7)}, \mbox{(Ga 5, 19-23)}, \mbox{(Ga 5, 16-18)}, \mbox{(Ef 5, 5)}, \mbox{(Flp 1, 19)}
\end{center}
\paragraph{}
\begin{quote}
,,I przeszli przez frygijską i galacką krainę, ponieważ Duch Święty przeszkodził w głoszeniu Słowa Bożego w Azji. A gdy przyszli ku Mizji, chcieli pójść do Bitynii, lecz Duch Jezusa nie pozwolił im;'' \mbox{(Dz 16, 6-7)}
\end{quote}
\paragraph{}
\begin{quote}
,,A nadzieja nie zawodzi, bo miłość Boża rozlana jest w sercach naszych przez Ducha Świętego, który nam jest dany.'' \mbox{(Rz 5, 5)}
\end{quote}
\paragraph{}
\begin{quote}
,,Ci zaś, którzy są w ciele, Bogu podobać się nie mogą. Ale wy nie jesteście w ciele, lecz w Duchu, jeśli tylko Duch Boży mieszka w was. Jeśli zaś kto nie ma Ducha Chrystusowego, ten nie jest jego.'' \mbox{(Rz 8, 8-9)}
\end{quote}
\paragraph{}
\begin{quote}
,,Bo ci, których Duch Boży prowadzi, są dziećmi Bożymi. Wszak nie wzięliście ducha niewoli, by znowu ulegać bojaźni, lecz wzięliście ducha synostwa, w którym wołamy: Abba, Ojcze!'' \mbox{(Rz 8, 14-15)}
\end{quote}
\paragraph{}
\begin{quote}
,,Bo jeśli ustami swoimi wyznasz, że Jezus jest Panem, i uwierzysz w sercu swoim, że Bóg wzbudził go z martwych, zbawiony będziesz. Albowiem sercem wierzy się ku usprawiedliwieniu, a ustami wyznaje się ku zbawieniu.'' \mbox{(Rz 10, 9-10)}
\end{quote}
\paragraph{}
\begin{quote}
,,Wiara tedy jest ze słuchania, a słuchanie przez Słowo Chrystusowe.'' \mbox{(Rz 10, 17)}
\end{quote}
\paragraph{}
\begin{quote}
,,Albowiem Królestwo Boże, to nie pokarm i napój, lecz sprawiedliwość i pokój, i radość w Duchu Świętym.'' \mbox{(Rz 14, 17)}
\end{quote}
\paragraph{}
\begin{quote}
,,Bo któż z ludzi wie, kim jest człowiek, prócz ducha ludzkiego, który w nim jest? Tak samo kim jest Bóg, nikt nie poznał, tylko Duch Boży.'' \mbox{(1 Kor 2, 11)}
\end{quote}
\paragraph{}
\begin{quote}
,,Czy nie wiecie, że świątynią Bożą jesteście i że Duch Boży mieszka w was?'' \mbox{(1 Kor 3, 16)}
\end{quote}
\paragraph{}
\begin{quote}
,,Albo czy nie wiecie, że niesprawiedliwi Królestwa Bożego nie odziedziczą? Nie łudźcie się! Ani wszetecznicy, ani bałwochwalcy, ani cudzołożnicy, ani rozpustnicy, ani mężołożnicy, Ani złodzieje, ani chciwcy, ani pijacy, ani oszczercy, ani zdziercy Królestwa Bożego nie odziedziczą.'' \mbox{(1 Kor 6, 9-10)}
\end{quote}
\paragraph{}
\begin{quote}
,,Albo czy nie wiecie, że ciało wasze jest świątynią Ducha Świętego, który jest w was i którego macie od Boga, i że nie należycie też do siebie samych? Drogoście bowiem kupieni. Wysławiajcie tedy Boga w ciele waszym.'' \mbox{(1 Kor 6, 19-20)}
\end{quote}
\paragraph{}
\begin{quote}
,,Dlatego oznajmiam wam, że nikt, przemawiając w Duchu Bożym, nie powie: Niech Jezus będzie przeklęty! I nikt nie może rzec: Jezus jest Panem, chyba tylko w Duchu Świętym.'' \mbox{(1 Kor 12, 3)}
\end{quote}
\paragraph{}
\begin{quote}
,,A Pan jest Duchem; gdzie zaś Duch Pański, tam wolność.'' \mbox{(2 Kor 3, 17)}
\end{quote}
\paragraph{}
\begin{quote}
,,Jakiż układ między świątynią Bożą a bałwanami? Myśmy bowiem świątynią Boga żywego, jak powiedział Bóg: Zamieszkam w nich i będę się przechadzał pośród nich, i będę Bogiem ich, a oni będą ludem moim.'' \mbox{(2 Kor 6, 16)}
\end{quote}
\paragraph{}
\begin{quote}
,,Z Chrystusem jestem ukrzyżowany; żyję więc już nie ja, ale żyje we mnie Chrystus; a obecne życie moje w ciele jest życiem w wierze w Syna Bożego, który mnie umiłował i wydał samego siebie za mnie.'' \mbox{(Ga 2, 20)}
\end{quote}
\paragraph{}
\begin{quote}
,,Czy ten, który daje wam Ducha i dokonuje wśród was cudów, czyni to na podstawie uczynków zakonu, czy na podstawie słuchania z wiarą?'' \mbox{(Ga 3, 5)}
\end{quote}
\paragraph{}
\begin{quote}
,,Lecz gdy nadeszło wypełnienie czasu, zesłał Bóg Syna swego, który się narodził z niewiasty i podlegał zakonowi, Aby wykupił tych, którzy byli pod zakonem, abyśmy usynowienia dostąpili. A ponieważ jesteście synami, przeto Bóg zesłał Ducha Syna swego do serc waszych, wołającego: Abba, Ojcze! Tak więc już nie jesteś niewolnikiem, lecz synem a jeśli synem, to i dziedzicem przez Boga.'' \mbox{(Ga 4, 4-7)}
\end{quote}
\paragraph{}
\begin{quote}
,,Jawne zaś są uczynki ciała, mianowicie: wszeteczeństwo, nieczystość, rozpusta, Bałwochwalstwo, czary, wrogość, spór, zazdrość, gniew, knowania, waśnie, odszczepieństwo Zabójstwa, pijaństwo, obżarstwo i tym podobne; o tych zapowiadam wam, jak już przedtem zapowiedziałem, że ci, którzy te rzeczy czynią, Królestwa Bożego nie odziedziczą. Owocem Ducha zaś są: miłość, radość, pokój, cierpliwość, uprzejmość, dobroć, wierność, Łagodność, wstrzemięźliwość. Przeciwko takim nie ma zakonu.'' \mbox{(Ga 5, 19-23)}
\end{quote}
\paragraph{}
\begin{quote}
,,Mówię więc: Według Ducha postępujcie, a nie będziecie pobłażali żądzy cielesnej. Gdyż ciało pożąda przeciwko Duchowi, a Duch przeciwko ciału, a te są sobie przeciwne, abyście nie czynili tego, co chcecie. A jeśli Duch was prowadzi, nie jesteście pod zakonem.'' \mbox{(Ga 5, 16-18)}
\end{quote}
\paragraph{}
\begin{quote}
,,Gdyż to wiedzcie na pewno, iż żaden rozpustnik albo nieczysty, lub chciwiec, to znaczy bałwochwalca, nie ma udziału w Królestwie Chrystusowym i Bożym.'' \mbox{(Ef 5, 5)}
\end{quote}
\paragraph{}
\begin{quote}
,,Wiem bowiem, że przez modlitwę waszą i pomoc Ducha Jezusa Chrystusa wyjdzie mi to ku wybawieniu,'' \mbox{(Flp 1, 19)}
\end{quote}
\end{document}