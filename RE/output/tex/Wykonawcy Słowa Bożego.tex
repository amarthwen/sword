\documentclass[10pt,a4paper,oneside]{article}
\usepackage[utf8]{inputenc}
\usepackage{polski}
\usepackage[polish]{babel}
\usepackage[margin=0.5in,bottom=0.75in]{geometry}
\begin{document}
\centerline{\textbf{\MakeUppercase{Wykonawcy Słowa Bożego}}}
\begin{center}
\textbf{Wersety do studium:} \mbox{(Jk 1, 22-25)}, \mbox{(Łk 6, 46-49)}, \mbox{(Mt 12, 46-50)}, \mbox{(Łk 11, 27-28)}
\end{center}
\section{Temat główny}
\paragraph{}
\begin{quote}
,,A bądźcie wykonawcami Słowa, a nie tylko słuchaczami, oszukującymi samych siebie. Bo jeśli ktoś jest słuchaczem Słowa, a nie wykonawcą, to podobny jest do człowieka, który w zwierciadle przygląda się swemu naturalnemu obliczu; Bo przypatrzył się sobie i odszedł, i zaraz zapomniał, jakim jest. Ale kto wejrzał w doskonały zakon wolności i trwa w nim, nie jest słuchaczem, który zapomina, lecz wykonawcą; ten będzie błogosławiony w swoim działaniu.'' \mbox{(Jk 1, 22-25)}
\end{quote}
\paragraph{}
\begin{quote}
,,Dlaczego mówicie do mnie: Panie, Panie, a nie czynicie tego, co mówię? Pokażę wam, do kogo jest podobny każdy, kto przychodzi do mnie i słucha słów moich, i czyni je. Podobny jest do człowieka budującego dom, który kopał i dokopał się głęboko, i założył fundament na skale. A gdy przyszła powódź, uderzyły wody o ów dom, ale nie mogły go poruszyć, bo był dobrze zbudowany. Kto zaś słucha, a nie czyni, podobny jest do człowieka, który zbudował dom na ziemi bez fundamentu, i uderzyły weń wody, i wnet runął, a upadek domu owego był zupełny.'' \mbox{(Łk 6, 46-49)}
\end{quote}
\paragraph{}
\begin{quote}
,,A gdy On jeszcze mówił do tłumów, oto matka i bracia jego stanęli na dworze, chcąc z nim mówić. I rzekł mu ktoś: Oto matka twoja i bracia twoi stoją na dworze i chcą z tobą mówić. A On, odpowiadając, rzekł temu, co mu to powiedział: Któż jest moją matką? I kto to bracia moi? I wyciągnąwszy rękę ku uczniom swoim, rzekł: Oto matka moja i bracia moi! Albowiem ktokolwiek czyni wolę Ojca mojego, który jest w niebie, ten jest moim bratem i siostrą, i matką.'' \mbox{(Mt 12, 46-50)}
\end{quote}
\paragraph{}
\begin{quote}
,,A gdy On to mówił, pewna niewiasta z tłumu, podniósłszy swój głos, rzekła do niego: Błogosławione łono, które cię nosiło, i piersi, które ssałeś. On zaś rzekł: Błogosławieni są raczej ci, którzy słuchają Słowa Bożego i strzegą go.'' \mbox{(Łk 11, 27-28)}
\end{quote}
\end{document}