\documentclass[10pt,a4paper,oneside]{article}
\usepackage[utf8]{inputenc}
\usepackage{polski}
\usepackage[polish]{babel}
\usepackage[margin=0.5in,bottom=0.75in]{geometry}
\begin{document}
\centerline{\textbf{\MakeUppercase{Wykonawcy Słowa Bożego}}}
\begin{center}
\textbf{Wersety do studium:} \mbox{(Jk 1, 22-25)}, \mbox{(Łk 6, 46-49)}, \mbox{(Mt 12, 46-50)}, \mbox{(Łk 11, 27-28)}, \mbox{(J 15, 1-16)}, \mbox{(Mt 23, 1-3)}, \mbox{(J 14, 22-24)}, \mbox{(Mt 7, 24-27)}, \mbox{(Rz 2, 10-13)}, \mbox{(1J 2, 3-6)}, \mbox{(1J 2, 15-17)}
\end{center}
\section{Temat główny}
\paragraph{}
\begin{quote}
,,A bądźcie wykonawcami Słowa, a nie tylko słuchaczami, oszukującymi samych siebie. Bo jeśli ktoś jest słuchaczem Słowa, a nie wykonawcą, to podobny jest do człowieka, który w zwierciadle przygląda się swemu naturalnemu obliczu; Bo przypatrzył się sobie i odszedł, i zaraz zapomniał, jakim jest. Ale kto wejrzał w doskonały zakon wolności i trwa w nim, nie jest słuchaczem, który zapomina, lecz wykonawcą; ten będzie błogosławiony w swoim działaniu.'' \mbox{(Jk 1, 22-25)}
\end{quote}
\paragraph{}
\begin{quote}
,,Dlaczego mówicie do mnie: Panie, Panie, a nie czynicie tego, co mówię? Pokażę wam, do kogo jest podobny każdy, kto przychodzi do mnie i słucha słów moich, i czyni je. Podobny jest do człowieka budującego dom, który kopał i dokopał się głęboko, i założył fundament na skale. A gdy przyszła powódź, uderzyły wody o ów dom, ale nie mogły go poruszyć, bo był dobrze zbudowany. Kto zaś słucha, a nie czyni, podobny jest do człowieka, który zbudował dom na ziemi bez fundamentu, i uderzyły weń wody, i wnet runął, a upadek domu owego był zupełny.'' \mbox{(Łk 6, 46-49)}
\end{quote}
\paragraph{}
\begin{quote}
,,A gdy On jeszcze mówił do tłumów, oto matka i bracia jego stanęli na dworze, chcąc z nim mówić. I rzekł mu ktoś: Oto matka twoja i bracia twoi stoją na dworze i chcą z tobą mówić. A On, odpowiadając, rzekł temu, co mu to powiedział: Któż jest moją matką? I kto to bracia moi? I wyciągnąwszy rękę ku uczniom swoim, rzekł: Oto matka moja i bracia moi! Albowiem ktokolwiek czyni wolę Ojca mojego, który jest w niebie, ten jest moim bratem i siostrą, i matką.'' \mbox{(Mt 12, 46-50)}
\end{quote}
\paragraph{}
\begin{quote}
,,A gdy On to mówił, pewna niewiasta z tłumu, podniósłszy swój głos, rzekła do niego: Błogosławione łono, które cię nosiło, i piersi, które ssałeś. On zaś rzekł: Błogosławieni są raczej ci, którzy słuchają Słowa Bożego i strzegą go.'' \mbox{(Łk 11, 27-28)}
\end{quote}
\paragraph{}
\begin{quote}
,,Ja jestem prawdziwym krzewem winnym, a Ojciec mój jest winogrodnikiem. Każdą latorośl, która we mnie nie wydaje owocu, odcina, a każdą, która wydaje owoc, oczyszcza, aby wydawała obfitszy owoc. Wy jesteście już czyści dla słowa, które wam głosiłem; Trwajcie we mnie, a Ja w was. Jak latorośl sama z siebie nie może wydawać owocu, jeśli nie trwa w krzewie winnym, tak i wy, jeśli we mnie trwać nie będziecie. Ja jestem krzewem winnym, wy jesteście latoroślami. Kto trwa we mnie, a Ja w nim, ten wydaje wiele owocu; bo beze mnie nic uczynić nie możecie. Kto nie trwa we mnie, ten zostaje wyrzucony precz jak zeschnięta latorośl; takie zbierają i wrzucają w ogień, gdzie spłoną. Jeśli we mnie trwać będziecie i słowa moje w was trwać będą, proście o cokolwiek byście chcieli, stanie się wam. Przez to uwielbiony będzie Ojciec mój, jeśli obfity owoc wydacie i staniecie się uczniami moimi. Jak mnie umiłował Ojciec, tak i Ja was umiłowałem; trwajcie w miłości mojej. Jeśli przykazań moich przestrzegać będziecie, trwać będziecie w miłości mojej, jak i Ja przestrzegałem przykazań Ojca mego i trwam w miłości jego. To wam powiedziałem, aby radość moja była w was i aby radość wasza była zupełna. Takie jest przykazanie moje, abyście się wzajemnie miłowali, jak Ja was umiłowałem. Większej miłości nikt nie ma nad tę, jak gdy kto życie swoje kładzie za przyjaciół swoich. Jesteście przyjaciółmi moimi, jeśli czynić będziecie, co wam przykazuję. Już was nie nazywam sługami, bo sługa nie wie, co czyni pan jego; lecz nazwałem was przyjaciółmi, bo wszystko, co słyszałem od Ojca mojego, oznajmiłem wam. Nie wy mnie wybraliście, ale ja was wybrałem i przeznaczyłem was, abyście szli i owoc wydawali i aby owoc wasz był trwały, by to, o cokolwiek byście prosili Ojca w imieniu moim, dał wam.'' \mbox{(J 15, 1-16)}
\end{quote}
\paragraph{}
\begin{quote}
,,Wtedy Jezus przemówił do ludu i do uczniów swoich tymi słowy: Na mównicy Mojżeszowej zasiedli uczeni w Piśmie i faryzeusze. Wszystko więc, cokolwiek by wam powiedzieli, czyńcie i zachowujcie, ale według uczynków ich nie postępujcie; mówią bowiem, ale nie czynią.'' \mbox{(Mt 23, 1-3)}
\end{quote}
\paragraph{}
\begin{quote}
,,Rzekł mu Judasz, nie Iskariota: Panie, cóż się stało, że masz się nam objawić, a nie światu? Odpowiedział Jezus i rzekł mu: Jeśli kto mnie miłuje, słowa mojego przestrzegać będzie, i Ojciec mój umiłuje go, i do niego przyjdziemy i u niego zamieszkamy. Kto mnie nie miłuje, ten słów moich nie przestrzega, a przecież słowo, które słyszycie, nie jest moim słowem, lecz Ojca, który mnie posłał.'' \mbox{(J 14, 22-24)}
\end{quote}
\paragraph{}
\begin{quote}
,,Każdy więc, kto słucha tych słów moich i wykonuje je, będzie przyrównany do męża mądrego, który zbudował dom swój na opoce. I spadł deszcz ulewny, i wezbrały rzeki, i powiały wiatry, i uderzyły na ów dom, ale on nie runął, gdyż był zbudowany na opoce. A każdy, kto słucha tych słów moich, lecz nie wykonuje ich, przyrównany będzie do męża głupiego, który zbudował swój dom na piasku. I spadł ulewny deszcz, i wezbrały rzeki, i powiały wiatry, i uderzyły na ów dom, i runął, a upadek jego był wielki.'' \mbox{(Mt 7, 24-27)}
\end{quote}
\paragraph{}
\begin{quote}
,,A chwała i cześć, i pokój każdemu, który czyni dobrze, najpierw Żydowi, a potem i Grekowi. Albowiem u Boga nie ma względu na osobę. Bo ci, którzy bez zakonu zgrzeszyli, bez zakonu też poginą; a ci, którzy w zakonie zgrzeszyli, przez zakon sądzeni będą; Gdyż nie ci, którzy zakonu słuchają, są sprawiedliwi u Boga, lecz ci, którzy zakon wypełniają, usprawiedliwieni będą.'' \mbox{(Rz 2, 10-13)}
\end{quote}
\paragraph{}
\begin{quote}
,,A z tego wiemy, że go znamy, jeśli przykazania jego zachowujemy. Kto mówi: Znam go, a przykazań jego nie zachowuje, kłamcą jest i prawdy w nim nie ma. Lecz kto zachowuje Słowo jego, w tym prawdziwie dopełniła się miłość Boża. Po tym poznajemy, że w nim jesteśmy. Kto mówi, że w nim mieszka, powinien sam tak postępować, jak On postępował.'' \mbox{(1 J 2, 3-6)}
\end{quote}
\paragraph{}
\begin{quote}
,,Nie miłujcie świata ani tych rzeczy, które są na świecie. Jeśli kto miłuje świat, nie ma w nim miłości Ojca. Bo wszystko, co jest na świecie, pożądliwość ciała i pożądliwość oczu i pycha życia, nie jest z Ojca, ale ze świata. I świat przemija wraz z pożądliwością swoją; ale kto pełni wolę Bożą, trwa na wieki,'' \mbox{(1 J 2, 15-17)}
\end{quote}
\end{document}