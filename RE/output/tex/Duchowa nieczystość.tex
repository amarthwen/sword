\documentclass[10pt,a4paper,oneside]{article}
\usepackage[utf8]{inputenc}
\usepackage{polski}
\usepackage[polish]{babel}
\usepackage[margin=0.5in,bottom=0.75in]{geometry}
\begin{document}
\centerline{\textbf{\MakeUppercase{Duchowa nieczystość}}}
\begin{center}
\textbf{Wersety do studium:} \mbox{(Łk 11, 24-26)}
\end{center}
\section{Temat główny}
\paragraph{}
\begin{quote}
,,Gdy duch nieczysty wyjdzie z człowieka, wędruje po miejscach bezwodnych, szukając ukojenia, a gdy nie znajdzie, mówi: Wrócę do domu swego, skąd wyszedłem. I przyszedłszy, zastaje go wymiecionym i przyozdobionym. Wówczas idzie i zabiera z sobą siedem innych duchów, gorszych niż on, i wchodzą, i mieszkają tam. I bywa końcowy stan człowieka tego, gorszy niż pierwotny.'' \mbox{(Łk 11, 24-26)}
\end{quote}
\end{document}