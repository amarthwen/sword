\documentclass[10pt,a4paper,oneside]{article}
\usepackage[utf8]{inputenc}
\usepackage{polski}
\usepackage[polish]{babel}
\usepackage[margin=0.5in,bottom=0.75in]{geometry}
\begin{document}
\centerline{\textbf{\MakeUppercase{Bezprawie}}}
\begin{center}
\textbf{Wersety do studium:} \mbox{(Iz 5, 7)}, \mbox{(Iz 9, 15-17)}, \mbox{(Mt 7, 21-23)}
\end{center}
\section{Temat główny}
\paragraph{}
\begin{quote}
,,Zaiste, winnicą Pana Zastępów jest dom izraelski, a mężowie judzcy ulubioną jego latoroślą. Oczekiwał prawa, a oto - bezprawie; sprawiedliwości, a oto - krzyk.'' \mbox{(Iz 5, 7)}
\end{quote}
\paragraph{}
\begin{quote}
,,Zwodzicielami stali się wodzowie tego ludu, a prowadzeni przez nich zostają pochłonięci. Dlatego Pan nie oszczędza młodzieńców jego i nie lituje się nad jego sierotami i wdowami, bo to wszystko niegodziwcy i złoczyńcy, a każde usta mówią niedorzeczności. Mimo to nie ustaje jego gniew, a jego ręka jeszcze jest wyciągnięta. Gdyż bezprawie rozgorzało jak ogień, pożera cierń i oset i zapala gęstwinę leśną, tak że ta się unosi w słupach dymu.'' \mbox{(Iz 9, 15-17)}
\end{quote}
\paragraph{}
\begin{quote}
,,Nie każdy, kto do mnie mówi: Panie, Panie, wejdzie do Królestwa Niebios; lecz tylko ten, kto pełni wolę Ojca mego, który jest w niebie. W owym dniu wielu mi powie: Panie, Panie, czyż nie prorokowaliśmy w imieniu twoim i w imieniu twoim nie wypędzaliśmy demonów, i w imieniu twoim nie czyniliśmy wielu cudów? A wtedy im powiem: Nigdy was nie znałem. Idźcie precz ode mnie wy, którzy czynicie bezprawie.'' \mbox{(Mt 7, 21-23)}
\end{quote}
\end{document}