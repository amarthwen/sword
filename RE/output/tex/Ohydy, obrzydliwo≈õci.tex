\documentclass[10pt,a4paper,oneside]{article}
\usepackage[utf8]{inputenc}
\usepackage{polski}
\usepackage[polish]{babel}
\usepackage[margin=0.5in,bottom=0.75in]{geometry}
\begin{document}
\centerline{\textbf{\MakeUppercase{Ohydy, obrzydliwości}}}
\begin{center}
\textbf{Wersety do studium:} 
\mbox{(Iz 44, 9-20)}, \mbox{(Jr 4, 1-4)}, \mbox{(Ez 11, 17-21)}, \mbox{(Ez 8, 1-18)}, \mbox{(Jr 7, 30)}, \mbox{(Ez 7, 20)}
\end{center}
\paragraph{}
\begin{quote}
,,Wytwórcy bałwanów są w ogóle niczym, a ich ulubione wytwory są bez wartości, ich czciciele są ślepi, a oczywista ich niewiedza naraża ich na hańbę. Któż wytwarza bóstwo i odlewa bałwana, który na nic się nie zda? Oto wszyscy jego towarzysze narażają się na wstyd, a rzemieślnicy wszak to tylko ludzie; niech się zbiorą wszyscy i staną, a przelękną się i wszyscy się zawstydzą. Kowal wytwarza je pracując przy żarze węgla, nadaje mu kształt uderzeniami młota i robi go za pomocą swojego ramienia; gdy jest głodny, traci siłę, gdy nie pije wody, omdlewa. Snycerz rozciąga sznur, kreśli zarysy czerwonym ołówkiem, wycina go dłutem, wymierza go cyrklem i wykonuje go na podobieństwo człowieka, jako piękną postać ludzką, która ma być ustawiona w domu. Narąbie sobie cedrów lub bierze cyprys albo dąb i czeka, aż wyrośnie najsilniejsze pośród drzew leśnych, sadzi jodłę, która rośnie dzięki deszczom. Te służą człowiekowi na opał, bierze je, aby się ogrzać, roznieca także ogień, aby napiec chleba. Nadto robi sobie boga i oddaje mu pokłon, czyni z niego bałwana i pada przed nim na kolana. Połowę jego spala w ogniu, przy drugiej jego połowie spożywa mięso, piecze pieczeń i je do syta, nadto ogrzewa się przy tym i mówi: Ej, rozgrzałem się, poczułem ciepło! A z reszty czyni sobie boga, swojego bałwana, przed którym klęka i któremu oddaje pokłon, i do którego się modli, mówiąc: Ratuj mnie, gdyż jesteś moim bogiem! Nie mają poznania ani rozumu, bo zaślepione są ich oczy, tak że nie widzą, a serca zatwardziałe, tak że nie rozumieją. A nikt tego nie rozważa i nikt nie ma tyle poznania i rozumu, aby rzec: Jedną jego połowę spaliłem w ogniu i na jego węglach napiekłem chleba, upiekłem też mięso i najadłem się, a z reszty zrobię ohydę i będę klękał przed drewnianym klocem? Kto się zadaje z popiołem, tego zwodzi omamione serce, tak że nie uratuje swojej duszy ani też nie powie: Czy to nie złuda, czego się trzymam?'' \mbox{(Iz 44, 9-20)}
\end{quote}
\paragraph{}
\begin{quote}
,,Jeżeli chcesz zawrócić, Izraelu - mówi Pan - to zawróć do mnie; a jeżeli usuniesz swe obrzydliwości sprzed mojego oblicza, nie będziesz się już tułał. I jeżeli będziesz przysięgał: "Żyje Pan" szczerze, uczciwie i sprawiedliwie, wtedy narody będą sobie życzyć błogosławieństwa takiego jak twoje i tobą będą się chlubić. Bo tak mówi Pan do mężów judzkich i jeruzalemskich: Orzcie sobie ugór w nowy zagon, a nie siejcie między ciernie! Obrzeżcie się dla Pana i usuńcie nieobrzezki waszych serc, mężowie judzcy i mieszkańcy Jeruzalemu, aby mój gniew nie wybuchł jak ogień i nie palił, a nikt by go nie ugasił z powodu waszych złych czynów!'' \mbox{(Jr 4, 1-4)}
\end{quote}
\paragraph{}
\begin{quote}
,,Dlatego mów: Tak mówi Wszechmocny Pan: Zbiorę was spośród ludów i zgromadzę was z krajów, po których was rozproszyłem, i dam wam ziemię izraelską. A gdy wrócą tam i usuną z niej wszystkie jej ohydy i wszystkie jej obrzydliwości, Wtedy Ja dam im nowe serce i nowego ducha włożę do ich wnętrza; usunę z ich ciała serce kamienne i dam im serce mięsiste, Aby postępowali według moich przepisów, przestrzegali moich praw i wykonywali je. Wtedy będą mi ludem, a Ja będę im Bogiem. Lecz tym, których serce idzie za ich ohydami i obrzydliwościami, włożę ich postępki na ich głowy - mówi Wszechmocny Pan.'' \mbox{(Ez 11, 17-21)}
\end{quote}
\paragraph{}
\begin{quote}
,,Gdy w szóstym roku, w szóstym miesiącu, piątego dnia tego miesiąca, siedziałem w swoim domu, a starsi judzcy siedzieli przede mną, spoczęła tam na mnie ręka Wszechmocnego Pana, I widziałem, a oto była postać z wyglądu jakby człowiek; w dół od tego, co wyglądało jak jego biodra, był ogień, a w górę od tego, co wyglądało jak jego biodra, był blask, jakby połysk polerowanego kruszcu. I wyciągnął coś w kształcie ręki, i chwycił mnie za włosy na głowie. A duch uniósł mnie w górę między ziemię a niebo, i zaprowadził mnie w widzeniach Bożych do Jeruzalemu, do wejścia bramy wewnętrznej, zwróconej ku północy, gdzie znajdował się znienawidzony posąg, pobudzający do nienawiści. A oto była tam chwała Boga izraelskiego, podobna do tej w widzeniu, które miałem na równinie. I rzekł do mnie: Synu człowieczy! Podnieś swoje oczy w stronę północy! A gdy podniosłem swoje oczy w stronę północy, oto na północ od bramy wschodniej był u wejścia ten znienawidzony posąg. I rzekł do mnie: Synu człowieczy! Czy widzisz, co oni robią - te wielkie obrzydliwości, których dopuszcza się tutaj dom izraelski, aby mnie oddalić od mojej świątyni? Ale ty zobaczysz jeszcze większe obrzydliwości. I zaprowadził mnie do bramy dziedzińca, a gdy spojrzałem, oto w ścianie była dziura. I rzekł do mnie: Synu człowieczy! Przebij ścianę! A gdy przebiłem ścianę, oto było tam przejście. I rzekł do mnie: Wejdź i zobacz te straszne obrzydliwości, które oni tutaj popełniają! A gdy wszedłem i przyjrzałem się, oto tam na ścianie były wszędzie wokoło wyrysowane obrazy wszelkich płazów i bydła, ohydy i wszystkie bałwany domu izraelskiego. A siedemdziesięciu mężów spośród starszych domu izraelskiego z Jaazaniaszem, synem Szafana, stojącym wśród nich, stało przed nimi, a każdy miał w ręku kadzielnicę. Woń dymu kadzidlanego unosiła się w górę. I rzekł do mnie: Synu człowieczy! Czy widziałeś, co robią starsi domu izraelskiego w ciemności, każdy w swoim pokoju z obrazami? Mówią bowiem: Pan nas nie widzi, Pan opuścił kraj. I rzekł do mnie: Zobaczysz jeszcze większe obrzydliwości, które oni popełniają. I zaprowadził mnie do wejścia północnej bramy świątyni Pana; a oto siedziały tam kobiety, które opłakiwały Tammuza. Wtedy rzekł do mnie: Czy widziałeś to, synu człowieczy? Ujrzysz jeszcze większe obrzydliwości niż te. I zaprowadził mnie na wewnętrzny dziedziniec świątyni Pana. A oto u wejścia do przybytku Pana, między przedsionkiem a ołtarzem, było około dwudziestu pięciu mężów; ci, tyłem zwróceni do przybytku Pana, a twarzą ku wschodowi, zwróceni ku wschodowi oddawali pokłon słońcu. Wtedy rzekł do mnie: Synu człowieczy! Czy widziałeś to! Czy to nie dosyć dla domu judzkiego popełniać obrzydliwości, które tu popełniają, napełniając kraj bezprawiem i ustawicznie pobudzając mnie do gniewu? A oto patrz, winną latorośl przykładają do nosa! Dlatego i Ja postąpię z nimi w gniewie: Ani nie drgnie moje oko i nie zlituję się! A gdy będą głośno wołać do moich uszu, nie wysłucham ich.'' \mbox{(Ez 8, 1-18)}
\end{quote}
\paragraph{}
\begin{quote}
,,Bo synowie judzcy czynili to, co złe przed moimi oczyma - mówi Pan - postawili w domu, który jest nazwany moim imieniem, swoje obrzydliwości, aby go bezcześcić.'' \mbox{(Jr 7, 30)}
\end{quote}
\paragraph{}
\begin{quote}
,,Z kosztownej swojej ozdoby uczynili pożywkę swej pychy i nią powodowani robili swoje obrzydliwe posągi, swoje ohydy. Dlatego przemienię je dla nich w śmiecie.'' \mbox{(Ez 7, 20)}
\end{quote}
\end{document}