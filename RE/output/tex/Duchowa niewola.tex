\documentclass[10pt,a4paper,oneside]{article}
\usepackage[utf8]{inputenc}
\usepackage{polski}
\usepackage[polish]{babel}
\usepackage[margin=0.5in,bottom=0.75in]{geometry}
\begin{document}
\centerline{\textbf{\MakeUppercase{Duchowa niewola}}}
\begin{center}
\textbf{Wersety do studium:} \mbox{(1Moj 4, 6-7)}, \mbox{(Rz 6, 16-22)}, \mbox{(Rz 7, 4-6)}, \mbox{(Rz 7, 14-25)}, \mbox{(Rz 8, 14-15)}, \mbox{(Ga 4, 1-7)}, \mbox{(2Tm 2, 24-26)}, \mbox{(2P 2, 9-11.17-19)}
\end{center}
\section{Temat główny}
\paragraph{}
\begin{quote}
,,I rzekł Pan do Kaina: Czemu się gniewasz i czemu zasępiło się twoje oblicze? Wszak byłoby pogodne, gdybyś czynił dobrze, a jeśli nie będziesz czynił dobrze, u drzwi czyha grzech. Kusi cię, lecz ty masz nad nim panować.'' \mbox{(1 Moj 4, 6-7)}
\end{quote}
\paragraph{}
\begin{quote}
,,Czyż nie wiecie, że jeśli się oddajecie jako słudzy w posłuszeństwo, stajecie się sługami tego, komu jesteście posłuszni, czy to grzechu ku śmierci, czy też posłuszeństwa ku sprawiedliwości? Lecz Bogu niech będą dzięki, że wy, którzy byliście sługami grzechu, przyjęliście ze szczerego serca zarys tej nauki, której zostaliście przekazani, A uwolnieni od grzechu, staliście się sługami sprawiedliwości, Po ludzku mówię przez wzgląd na słabość waszego ciała. Jak bowiem oddawaliście członki wasze na służbę nieczystości i nieprawości ku popełnianiu nieprawości, tak teraz oddawajcie członki wasze na służbę sprawiedliwości ku poświęceniu. Gdy bowiem byliście sługami grzechu, byliście dalecy od sprawiedliwości. Jakiż więc mieliście wtedy pożytek? Taki, którego się teraz wstydzicie, a końcem tego jest śmierć. Teraz zaś, wyzwoleni od grzechu, a oddani w służbę Bogu, macie pożytek w poświęceniu, a za cel żywot wieczny.'' \mbox{(Rz 6, 16-22)}
\end{quote}
\paragraph{}
\begin{quote}
,,Przeto, bracia moi, i wy umarliście dla zakonu przez ciało Chrystusowe, by należeć do innego, do tego, który został wzbudzony z martwych, abyśmy owoc wydawali dla Boga. Albowiem gdy byliśmy w ciele, grzeszne namiętności rozbudzone przez zakon były czynne w członkach naszych, aby rodzić owoce śmierci; Lecz teraz zostaliśmy uwolnieni od zakonu, gdy umarliśmy temu, przez co byliśmy opanowani, tak iż służymy w nowości ducha, a nie według przestarzałej litery.'' \mbox{(Rz 7, 4-6)}
\end{quote}
\paragraph{}
\begin{quote}
,,Wiemy bowiem, że zakon jest duchowy, ja zaś jestem cielesny, zaprzedany grzechowi. Albowiem nie rozeznaję się w tym, co czynię; gdyż nie to czynię, co chcę, ale czego nienawidzę, to czynię. A jeśli to czynię, czego nie chcę, zgadzam się z tym, że zakon jest dobry. Ale wtedy czynię to już nie ja, lecz grzech, który mieszka we mnie. Wiem tedy, że nie mieszka we mnie, to jest w ciele moim, dobro; mam bowiem zawsze dobrą wolę, ale wykonania tego, co dobre, brak; Albowiem nie czynię dobrego, które chcę, tylko złe, którego nie chcę, to czynię. A jeśli czynię to, czego nie chcę, już nie ja to czynię, ale grzech, który mieszka we mnie. Znajduję tedy w sobie zakon, że gdy chcę czynić dobrze, trzyma się mnie złe; Bo według człowieka wewnętrznego mam upodobanie w zakonie Bożym. A w członkach swoich dostrzegam inny zakon, który walczy przeciwko zakonowi, uznanemu przez mój rozum i bierze mnie w niewolę zakonu grzechu, który jest w członkach moich. Nędzny ja człowiek! Któż mnie wybawi z tego ciała śmierci? Bogu niech będą dzięki przez Jezusa Chrystusa, Pana naszego! Tak więc ja sam służę umysłem zakonowi Bożemu, ciałem zaś zakonowi grzechu'' \mbox{(Rz 7, 14-25)}
\end{quote}
\paragraph{}
\begin{quote}
,,Bo ci, których Duch Boży prowadzi, są dziećmi Bożymi. Wszak nie wzięliście ducha niewoli, by znowu ulegać bojaźni, lecz wzięliście ducha synostwa, w którym wołamy: Abba, Ojcze!'' \mbox{(Rz 8, 14-15)}
\end{quote}
\paragraph{}
\begin{quote}
,,A mówię: Dopóki dziedzic jest dziecięciem, niczym się nie różni od niewolnika, chociaż jest panem wszystkiego, Ale jest pod nadzorem opiekunów i rządców aż do czasu wyznaczonego przez ojca. Podobnie i my, gdy byliśmy dziećmi, byliśmy poddani w niewolę żywiołów tego świata; Lecz gdy nadeszło wypełnienie czasu, zesłał Bóg Syna swego, który się narodził z niewiasty i podlegał zakonowi, Aby wykupił tych, którzy byli pod zakonem, abyśmy usynowienia dostąpili. A ponieważ jesteście synami, przeto Bóg zesłał Ducha Syna swego do serc waszych, wołającego: Abba, Ojcze! Tak więc już nie jesteś niewolnikiem, lecz synem a jeśli synem, to i dziedzicem przez Boga.'' \mbox{(Ga 4, 1-7)}
\end{quote}
\paragraph{}
\begin{quote}
,,A sługa Pański nie powinien wdawać się w spory, lecz powinien być uprzejmy dla wszystkich, zdolny do nauczania, cierpliwie znoszący przeciwności, Napominający z łagodnością krnąbrnych, w nadziei, że Bóg przywiedzie ich kiedyś do upamiętania i do poznania prawdy I że wyzwolą się z sideł diabła, który ich zmusza do pełnienia swojej woli.'' \mbox{(2 Tm 2, 24-26)}
\end{quote}
\paragraph{}
\begin{quote}
,,Umie Pan wyrwać pobożnych z pokuszenia, bezbożnych zaś zachować na dzień sądu celem ukarania, Szczególnie zaś tych, którzy oddają się niecnym pożądliwościom cielesnym, a zwierzchnością pogardzają. Zuchwali, zarozumiali, nie lękają się nawet bluźnić mocom niebieskim. Chociaż aniołowie siłą i mocą są więksi od nich, nie wydają na nich przed Panem wyroku potępienia. (\ldots) Ludzie ci, to źródła bez wody i obłoki pędzone przez wicher; czeka ich przeznaczony najciemniejszy mrok. Przemawiając bowiem słowami nadętymi a pustymi, nęcą przez żądze cielesne i rozwiązłość tych, którzy dopiero co wyzwolili się od wpływu pogrążonych w błędzie, Obiecując im wolność, chociaż sami są niewolnikami zguby; czemu bowiem ktoś ulega, tego niewolnikiem się staje.'' \mbox{(2 P 2, 9-11.17-19)}
\end{quote}
\end{document}