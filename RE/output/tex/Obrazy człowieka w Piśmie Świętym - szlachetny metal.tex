\documentclass[10pt,a4paper,oneside]{article}
\usepackage[utf8]{inputenc}
\usepackage{polski}
\usepackage[polish]{babel}
\usepackage[margin=0.5in,bottom=0.75in]{geometry}
\begin{document}
\centerline{\textbf{\MakeUppercase{Obrazy człowieka w Piśmie Świętym}}}
\begin{center}
\textbf{Wersety do studium:} \mbox{(Iz 1, 24-25)}, \mbox{(Jr 6, 27-30)}, \mbox{(Ez 22, 17-22)}
\end{center}
\section{Temat główny}
\paragraph{}
\begin{quote}
,,Dlatego tak mówi Pan Zastępów, Mocarz Izraela: Biada! Ulżę sobie na moich nieprzyjaciołach i pomszczę się na moich wrogach! I zwrócę swoją rękę przeciwko tobie, i wytopię w tyglu twój żużel, i usunę wszystkie twoje przymieszki.'' \mbox{(Iz 1, 24-25)}
\end{quote}
\paragraph{}
\begin{quote}
,,Ustanowiłem cię badaczem wśród mojego ludu, abyś poznał i badał ich postępowanie. Arcybuntownikami są wszyscy, chodzą wokoło i oczerniają; twardzi jak miedź i żelazo, wszyscy są zepsuci. Miech sapie, aby ołów się roztopił w ogniu; lecz na próżno się roztapia i roztapia, gdyż złych nie daje się oddzielić. Nazwie ich srebrem odrzuconym, gdyż Pan ich odrzucił.'' \mbox{(Jr 6, 27-30)}
\end{quote}
\paragraph{}
\begin{quote}
,,I doszło mnie słowo Pana tej treści: Synu człowieczy, żużlem stał się dla mnie dom izraelski; wszyscy oni, to tylko brąz i cyna, żelazo i ołów w tyglu, a stali się żużlem. Dlatego mówi Wszechmocny Pan: Ponieważ wy wszyscy staliście się żużlem, oto Ja zgromadzę was w Jeruzalemie. Jak wrzuca się razem srebro i brąz, żelazo, ołów i cynę do tygla i rozpala pod nim ogień, aby je stopić, tak Ja zgromadzę was w moim gniewie i w mojej zapalczywości i wrzucę was razem, i stopię. Zgromadzę was i rozdmucham ogień swojej popędliwości przeciwko wam, abyście byli w nim stopieni. Jak topi się srebro w tyglu, tak wy będziecie w nim roztopieni i poznacie, że Ja wylałem na was swoją zapalczywość.'' \mbox{(Ez 22, 17-22)}
\end{quote}
\end{document}