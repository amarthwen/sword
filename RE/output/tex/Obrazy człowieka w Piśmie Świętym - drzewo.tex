\documentclass[10pt,a4paper,oneside]{article}
\usepackage[utf8]{inputenc}
\usepackage{polski}
\usepackage[polish]{babel}
\usepackage[margin=0.5in,bottom=0.75in]{geometry}
\begin{document}
\centerline{\textbf{\MakeUppercase{Obrazy człowieka w Piśmie Świętym}}}
\begin{center}
\textbf{Wersety do studium:} \mbox{(Ps 1, 1-3)}, \mbox{(Ps 1, 1-6)}, \mbox{(Łk 3, 2-9)}, \mbox{(2Moj 10, 1-20)}, \mbox{(Dn 4, 6-9.16-24)}, \mbox{(Łk 13, 6-9)}, \mbox{(J 15, 1-16)}, \mbox{(Iz 30, 1-5)}, \mbox{(2Moj 10, 12-15)}, \mbox{(Am 4, 9)}, \mbox{(Ml 3, 11)}, \mbox{(Jl 2, 25)}, \mbox{(Am 7, 1-3)}, \mbox{(Jl 1, 4-7)}, \mbox{(Obj 9, 1-11)}, \mbox{(2Krl 19, 20-30)}, \mbox{(Iz 10, 15)}, \mbox{(Jr 17, 7-10)}, \mbox{(Jr 5, 11-14)}, \mbox{(Mt 7, 15-20)}
\end{center}
\section{Wstęp}
\section{Temat główny}
\subsection{Człowiek przyrównany do drzewa}
\subsubsection{Drzewo rodzące owoc we właściwym czasie}
\paragraph{}
\begin{quote}
,,Szczęśliwy mąż, który nie idzie za radą bezbożnych Ani nie stoi na drodze grzeszników, Ani nie zasiada w gronie szyderców, Lecz ma upodobanie w zakonie Pana I zakon jego rozważa dniem i nocą. Będzie on jak drzewo zasadzone nad strumieniami wód, Wydające swój owoc we właściwym czasie, Którego liść nie więdnie, A wszystko, co uczyni, powiedzie się. Nie tak jest z bezbożnymi! Są oni bowiem jak plewa, Którą wiatr roznosi. Przeto nie ostoją się bezbożni na sądzie Ani grzesznicy w zgromadzeniu sprawiedliwych, Gdyż Pan troszczy się o drogę sprawiedliwych, Droga zaś bezbożnych wiedzie do nikąd.'' \mbox{(Ps 1, 1-6)}
\end{quote}
\subsubsection{Ostrzeżenie dla człowieka, który odstąpił od Pana Boga}
\paragraph{}
\begin{quote}
,,Za arcykapłanów Annasza i Kaifasza doszło Słowo Boże Jana, syna Zachariasza, na pustyni. I przeszedł całą krainę nadjordańską, głosząc chrzest upamiętania na odpuszczenie grzechów, Jak było napisane w księdze mów proroka Izajasza: Głos wołającego na pustyni: Gotujcie drogę Pańską, prostujcie ścieżki jego. Każdy padół niech będzie wypełniony, a każda góra i pagórek zniesione, drogi krzywe wyprostowane, a nierówne wygładzone. I ujrzą wszyscy ludzie zbawienie Boże. Mówił więc do tłumów, które przychodziły, aby się dać ochrzcić przez niego: Plemię żmijowe, któż wam poddał myśl, aby uciekać przed przyszłym gniewem? Wydawajcie więc owoce godne upamiętania. A nie próbujcie wmawiać w siebie: Ojca mamy Abrahama; powiadam wam bowiem, że Bóg może z tych kamieni wzbudzić dzieci Abrahamowi. A już i siekiera do korzenia drzew jest przyłożona; wszelkie więc drzewo, które nie wydaje owocu dobrego, zostaje wycięte i w ogień wrzucone.'' \mbox{(Łk 3, 2-9)}
\end{quote}
\subsubsection{Ostrzeżenie dla człowieka, który zatwardza swoje serce}
\paragraph{}
\begin{quote}
,,Potem rzekł Pan do Mojżesza: Idź do faraona, gdyż to Ja sam przywiodłem do zatwardziałości serce jego i serce sług jego, aby czynić te moje znaki wśród nich I abyś ty opowiadał dzieciom swoim i wnukom swoim, jak obszedłem się z Egipcjanami, i jakie znaki czyniłem wśród nich, byście poznali, żem Ja Pan. Poszedł więc Mojżesz z Aaronem do faraona i rzekli do niego: Tak mówi Pan, Bóg Hebrajczyków: Jak długo wzbraniać się będziesz, by się przede mną upokorzyć? Wypuść lud mój, aby mi służył! Bo jeżeli będziesz się wzbraniał wypuścić lud mój, to Ja jutro sprowadzę szarańczę na twój kraj. Pokryje ona całą ziemię, tak że nie będzie można zobaczyć ziemi, i pożre resztę tego, co ocalało, co pozostało po gradzie, i obgryzie wszystkie drzewa, które wam rosną na polu. I napełni domy twoje i domy wszystkich sług twoich, i domy wszystkich Egipcjan, czego nie widzieli ojcowie twoi i ojcowie ojców twoich, odkąd żyją na ziemi aż do dnia dzisiejszego. Potem odwrócił się i wyszedł od faraona. A słudzy faraona rzekli do niego: Jak długo będzie nam ten człowiek przynosił nieszczęście? Wypuść tych ludzi, aby służyli Panu, Bogu swemu. Czy jeszcze nie rozumiesz, że Egipt ginie? Sprowadzono więc z powrotem Mojżesza i Aarona do faraona, a on rzekł do nich: Idźcie, służcie Panu, Bogu waszemu. Lecz którzy to mają iść? Mojżesz odpowiedział: Pójdziemy z naszą młodzieżą i z naszymi starcami; pójdziemy z naszymi synami i naszymi córkami, z naszymi trzodami i z naszym bydłem, gdyż mamy obchodzić święto Pana. Wtedy rzekł do nich: Pan niech będzie z wami, jeżeli ja kiedykolwiek wypuszczę was i dzieci wasze. Patrzcie, jak złe macie zamiary. Nie tak! Idźcie wy, mężczyźni, i służcie Panu, skoro tak tego żądacie! I wypędzono ich od faraona. I rzekł Pan do Mojżesza: Wyciągnij rękę swoją nad ziemią egipską, by przyszła szarańcza. Niech spadnie na ziemię egipską i pożre wszelką roślinność ziemi, wszystko, co pozostawił grad. I wyciągnął Mojżesz laskę swoją nad ziemią egipską. A Pan sprowadził wiatr wschodni na kraj i wiał przez cały dzień i całą noc. A gdy nastał poranek, wiatr wschodni przyniósł szarańczę. I szarańcza przyleciała nad całą ziemię egipską, i osiadła w bardzo wielkiej ilości na całym obszarze Egiptu. Nie było przedtem takiej ilości szarańczy ani już nie będzie. Pokryła ona całą powierzchnię ziemi, tak że ziemia pociemniała. I pożarła całą roślinność ziemi i wszelki owoc drzew, który pozostawił grad. Nie pozostała żadna zieleń na drzewach ani żadna roślinność w całej ziemi egipskiej. Wtedy faraon śpiesznie wezwał Mojżesza i Aarona i rzekł: Zgrzeszyłem przeciwko Panu, Bogu waszemu, i przeciwko wam. Przeto przebaczcie mi jeszcze tym razem mój grzech i wstawcie się do Pana, Boga waszego, by przynajmniej tę zagładę oddalił ode mnie. A gdy wyszedł od faraona, wstawił się do Pana. Wtedy odwrócił Pan wiatr, który zaczął dąć bardzo mocno od zachodu. A ten uniósł szarańczę i wrzucił ją do Morza Czerwonego, tak że na całym obszarze Egiptu nie pozostała ani jedna szarańcza. Lecz Pan przywiódł do zatwardziałości serce faraona, tak iż nie wypuścił synów izraelskich.'' \mbox{(2 Moj 10, 1-20)}
\end{quote}
\subsubsection{Ostrzeżenie dla człowieka, nad którym zaczyna panować grzech}
\paragraph{}
\begin{quote}
,,Baltazarze, przełożony wróżbitów! Wiem, że duch świętych bogów jest w tobie i że żadna tajemnica nie jest dla ciebie trudna. Posłuchaj mojego widzenia sennego, które miałem i podaj mi jego wykład. A oto co widziałem na moim łożu: Widziałem, a oto w środku ziemi było drzewo, a jego wysokość była duża. Drzewo to rosło i było potężne; jego wysokość sięgała nieba, a było widoczne aż po krańce całej ziemi. Liść jego był piękny, owoc jego obfity i był na nim pokarm dla wszystkich. Zwierzęta polne szukały pod nim cienia, a w jego gałęziach gnieździło się ptactwo niebieskie i żywiło się z niego wszelkie ciało. (\ldots) Wtedy Daniel, zwany Baltazarem, przez krótki czas był osłupiały i przerażony w swoich myślach. Król zaś odezwał się i rzekł: Baltazarze, sen i wykład niech cię nie trwożą! Baltazar odpowiedział i rzekł: Panie mój! Ten sen dotyczy twoich nieprzyjaciół, a jego wykład twoich przeciwników. Drzewo, które widziałeś, które rosło i było potężne, którego wysokość sięgała nieba, a było widoczne na całej ziemi, Którego liście były piękne, a owoc obfity, i które miało pokarm dla wszystkich, pod którym mieszkały zwierzęta polne, a w jego gałęziach gnieździły się ptaki niebieskie To jesteś ty, królu: rosłeś i stałeś się potężny, twoja wielkość urosła i sięga nieba, twoja władza rozciąga się aż po krańce ziemi. A to, że król widział anioła, świętego, zstępującego z nieba i mówiącego: Zetnijcie to drzewo i zniszczcie je, lecz jego pień korzenny pozostawcie w ziemi, w obręczy żelaznej i miedzianej, na niwie zielonej, aby rosa niebieska go zraszała i by miał udział ze zwierzętami polnymi, aż przejdzie nad nim siedem wieków! Taki jest wykład, królu, i takie jest rozstrzygnięcie Najwyższego o królu, moim panu: Wypędzą cię spośród ludzi, mieszkać będziesz ze zwierzętami polnymi i jak bydło trawą karmić cię będą; będzie cię zraszać rosa niebieska, siedem wieków przejdzie nad tobą, aż poznasz, że Najwyższy ma władzę nad królestwem ludzkim i że daje je temu, komu chce. A to, że polecono pozostawić pień korzenny drzewa, znaczy: Królestwo twoje będzie ci zachowane, gdy poznasz, że władza należy do niebios. Dlatego, królu, niech ci się spodoba moja rada: Zmaż swoje grzechy sprawiedliwością, a swoje winy miłosierdziem nad ubogimi, może wtedy twoje szczęście będzie trwałe!'' \mbox{(Dn 4, 6-9.16-24)}
\end{quote}
\subsubsection{Trwanie w niewoli grzechu - Pan Jezus jako ogrodnik nawożący drzewo}
\paragraph{}
\begin{quote}
,,I powiedział to podobieństwo: Pewien człowiek miał figowe drzewo zasadzone w winnicy swojej i przyszedł, by szukać na nim owocu, lecz nie znalazł. I rzekł do winogrodnika: Oto od trzech lat przychodzę, by szukać na tym figowym drzewie owocu, a nie znajduję. Wytnij je, po cóż jeszcze ziemię próżno zajmuje? A tamten odpowiadając, rzecze: Panie, pozostaw je jeszcze ten rok, aż je okopię i obłożę nawozem, Może wyda owoc w przyszłości, jeśli zaś nie, wytniesz je.'' \mbox{(Łk 13, 6-9)}
\end{quote}
\subsection{Skąd właściwie biorą się owoce?}
\subsubsection{O wydawaniu dobrego owocu}
\paragraph{}
\begin{quote}
,,Ja jestem prawdziwym krzewem winnym, a Ojciec mój jest winogrodnikiem. Każdą latorośl, która we mnie nie wydaje owocu, odcina, a każdą, która wydaje owoc, oczyszcza, aby wydawała obfitszy owoc. Wy jesteście już czyści dla słowa, które wam głosiłem; Trwajcie we mnie, a Ja w was. Jak latorośl sama z siebie nie może wydawać owocu, jeśli nie trwa w krzewie winnym, tak i wy, jeśli we mnie trwać nie będziecie. Ja jestem krzewem winnym, wy jesteście latoroślami. Kto trwa we mnie, a Ja w nim, ten wydaje wiele owocu; bo beze mnie nic uczynić nie możecie. Kto nie trwa we mnie, ten zostaje wyrzucony precz jak zeschnięta latorośl; takie zbierają i wrzucają w ogień, gdzie spłoną. Jeśli we mnie trwać będziecie i słowa moje w was trwać będą, proście o cokolwiek byście chcieli, stanie się wam. Przez to uwielbiony będzie Ojciec mój, jeśli obfity owoc wydacie i staniecie się uczniami moimi. Jak mnie umiłował Ojciec, tak i Ja was umiłowałem; trwajcie w miłości mojej. Jeśli przykazań moich przestrzegać będziecie, trwać będziecie w miłości mojej, jak i Ja przestrzegałem przykazań Ojca mego i trwam w miłości jego. To wam powiedziałem, aby radość moja była w was i aby radość wasza była zupełna. Takie jest przykazanie moje, abyście się wzajemnie miłowali, jak Ja was umiłowałem. Większej miłości nikt nie ma nad tę, jak gdy kto życie swoje kładzie za przyjaciół swoich. Jesteście przyjaciółmi moimi, jeśli czynić będziecie, co wam przykazuję. Już was nie nazywam sługami, bo sługa nie wie, co czyni pan jego; lecz nazwałem was przyjaciółmi, bo wszystko, co słyszałem od Ojca mojego, oznajmiłem wam. Nie wy mnie wybraliście, ale ja was wybrałem i przeznaczyłem was, abyście szli i owoc wydawali i aby owoc wasz był trwały, by to, o cokolwiek byście prosili Ojca w imieniu moim, dał wam.'' \mbox{(J 15, 1-16)}
\end{quote}
\subsubsection{Dwa scenariusze}
\paragraph{}
\begin{quote}
,,Biada przekornym synom, mówi Pan, którzy wykonują plan, lecz nie mój, i zawierają przymierze, lecz nie w moim duchu, aby dodawać grzech do grzechu. Wyruszają hen do Egiptu, nie radząc się moich ust, aby oddać się pod opiekę faraona i szukać schronienia w cieniu Egiptu. Lecz opieka faraona narazi was na wstyd, a szukanie schronienia w cieniu Egiptu na hańbę. Bo chociaż jego książęta byli w Soanie, a jego posłowie dotarli do Chanes, Wszyscy zawiodą się na ludzie, który jest nieużyteczny, nie udzieli pomocy ani nie da korzyści, przyniesie raczej wstyd, a nawet hańbę.'' \mbox{(Iz 30, 1-5)}
\end{quote}
\section{Szarańcza}
\subsection{Skąd ta plaga?}
\paragraph{}
Dlaczego plaga szarańczy została opisana w Piśmie Świętym?
\subsubsection{Plaga szarańczy w Egipcie}
\paragraph{}
\begin{quote}
,,I rzekł Pan do Mojżesza: Wyciągnij rękę swoją nad ziemią egipską, by przyszła szarańcza. Niech spadnie na ziemię egipską i pożre wszelką roślinność ziemi, wszystko, co pozostawił grad. I wyciągnął Mojżesz laskę swoją nad ziemią egipską. A Pan sprowadził wiatr wschodni na kraj i wiał przez cały dzień i całą noc. A gdy nastał poranek, wiatr wschodni przyniósł szarańczę. I szarańcza przyleciała nad całą ziemię egipską, i osiadła w bardzo wielkiej ilości na całym obszarze Egiptu. Nie było przedtem takiej ilości szarańczy ani już nie będzie. Pokryła ona całą powierzchnię ziemi, tak że ziemia pociemniała. I pożarła całą roślinność ziemi i wszelki owoc drzew, który pozostawił grad. Nie pozostała żadna zieleń na drzewach ani żadna roślinność w całej ziemi egipskiej.'' \mbox{(2 Moj 10, 12-15)}
\end{quote}
\subsubsection{Plaga szarańczy w Izraelu}
\paragraph{}
\begin{quote}
,,Smagałem was posuchą i rdzą, spustoszyłem wasze ogrody i wasze winnice, i szarańcza pożarła wasze drzewa figowe i oliwne, a jednak nie nawróciliście się do mnie - mówi Pan.'' \mbox{(Am 4, 9)}
\end{quote}
\paragraph{}
\begin{quote}
,,I zabronię potem szarańczy pożerać wasze plony rolne, wasz winograd zaś w polu nie będzie bez owocu - mówi Pan Zastępów.'' \mbox{(Ml 3, 11)}
\end{quote}
\paragraph{}
\begin{quote}
,,I wynagrodzę wam szkody lat, których plony pożarła szarańcza, konik polny, larwa i gąsienica, moje wielkie wojsko, które na was wyprawiłem.'' \mbox{(Jl 2, 25)}
\end{quote}
\subsubsection{Plaga szarańczy jako konsekwencja trwania przez dany naród w grzechu}
\paragraph{}
\begin{quote}
,,Takie widzenie dał mi oglądać Wszechmogący Pan: Oto stworzył szarańcze, gdy zaczął odrastać potraw, a był to potraw po kośbie królewskiej. A gdy już dojadała zieleń ziemi, rzekłem: Wszechmogący Panie, przebacz! Jakże ostoi się Jakub, wszak jest tak maleńki? I żałował tego Pan: Nie stanie się - rzekł Pan.'' \mbox{(Am 7, 1-3)}
\end{quote}
\subsection{Czego obrazem w Piśmie Świętym jest szarańcza?}
\paragraph{}
Jeżeli drzewo jest obrazem człowieka to czego obrazem jest szarańcza?
\paragraph{}
\begin{quote}
,,Co zostało po gąsienicy, pożarła szarańcza, a co zostało po szarańczy, pożarł konik polny, co zaś zostało po koniku polnym, pożarła larwa. Ocućcie się, pijani i płaczcie, zawodźcie z powodu moszczu, wy wszyscy, którzy pijecie wino, że odjęty jest od waszych ust. Gdyż moją ziemię naszedł lud, mocny i niezliczony. Jego zęby są jak zęby lwa, jego uzębienie jak uzębienie lwicy. Spustoszył moją winnicę, połamał moje drzewa figowe, odarł je doszczętnie z kory i zostawił tak, że zbielały jego gałęzie.'' \mbox{(Jl 1, 4-7)}
\end{quote}
\paragraph{}
\begin{quote}
,,I zatrąbił piąty anioł; i widziałem gwiazdę, która spadła z nieba na ziemię; i dano jej klucz ud studni otchłani. I otwarła studnię otchłani. I wzbił się ze studni dym jakby dym z wielkiego pieca, a słońce i powietrze zaćmiły się od dymu ze studni. A z tego dymu wyszły na ziemię szarańcze, którym dana została moc jaką jest moc skorpionów na ziemi. I powiedziano im, aby nie wyrządzały szkody trawie, ziemi ani żadnym ziołom, ani żadnemu drzewu, a tylko ludziom, którzy nie mają pieczęci Bożej na czołach. I nakazano im, aby nie zabijały ich, lecz dręczyły przez pięć miesięcy; a ból przez nie wywołany był jak ból od ukłucia skorpiona, gdy ukłuje człowieka. A w owe dni będą ludzie szukać śmierci, lecz jej nie znajdą, i będą chcieli umrzeć, ale śmierć omijać ich będzie Z wyglądu szarańcze te podobne były do koni gotowych do boju, a na głowach ich coś jakby złote korony. a twarze ich jakby twarze ludzkie. A włosy miały jak włosy kobiece, a zęby ich były jak u lwów. Miały też pancerze niby pancerze żelazne, a szum ich skrzydeł jak turkot wozów wojennych i wielu koni pędzących do boju. Miały też ogony podobne do skorpionowych oraz żądła, a w ogonach ich moc wyrządzania ludziom szkody przez pięć miesięcy. I miały nad sobą jako króla anioła otchłani, którego imię brzmi po hebrajsku Abaddon, po grecku zaś imię jego brzmi Apollyon (Niszczyciel).'' \mbox{(Obj 9, 1-11)}
\end{quote}
\section{Podsumowanie}
\section{Do dalszego studium}
\section{Pozostałe wersety}
\paragraph{}
Najazd króla Sancheryba na Judę:
\paragraph{}
\begin{quote}
,,Wtedy posłał Izajasz, syn Amosa, do Hiskiasza taką wiadomość: Tak mówi Pan, Bóg Izraela: Słyszałem, o co się do mnie modliłeś w sprawie Sancheryba, króla asyryjskiego, Takie zaś jest słowo, które Pan wypowiada o nim: Gardzi tobą, szydzi z ciebie panna, córka syjońska, Potrząsa nad tobą głową córka jeruzalemska. Komu urągałeś i bluźniłeś? Przeciw komu podnosiłeś swój głos? I wysoko wznosiłeś swoje oczy? Przeciw Świętemu Izraela! Przez usta swoich posłańców bluźniłeś Panu i mówiłeś: Z mnóstwem moich rydwanów Dotarłem na najwyższe góry, do krańców Libanu I ściąłem rosłe jego cedry, wyborowe jego cyprysy, Wstąpiłem na najwyższy jego szczyt, Do najgęstszego lasu. Ja kazałem kopać i piłem cudze wody, Ja też wysuszę stąpaniem moich stóp wszystkie strumienie Egiptu. Czy nie słyszałeś od dawna, że to Ja uczyniłem, Od pradawnych czasów to ustanowiłem, Co teraz do skutku doprowadziłem, Że ty zamieniasz w kupy gruzów miasta warowne, Mieszkańcy ich zaś, bezsilni, strwożyli się i zmieszali, Stali się jak zioła polne, jak świeża ruń, Jak trawa na dachach, spalona, zanim wyrośnie. Znam wstawanie twoje i siadanie twoje, Wyjście twoje i wejście twoje, Jak również twoją złość na mnie. A ponieważ twoja złość na mnie I twoje zuchwalstwo doszło do moich uszu, Przeto wprawię mój pierścień do twoich nozdrzy, A moje wędzidło do twoich ust I skieruję cię z powrotem na drogę, którą przyszedłeś. A to będzie dla ciebie znakiem: Tego roku wyżywieniem będzie zboże samorodne, w przyszłym roku zboże dziko wyrosłe, a w trzecim roku siejcie i żnijcie, sadźcie winnice i spożywajcie ich owoc. A pozostała przy życiu resztka domu Judy zapuści korzeń w głąb i wyda owoc w górze.'' \mbox{(2 Krl 19, 20-30)}
\end{quote}
\paragraph{}
Inne:
\paragraph{}
\begin{quote}
,,Czy siekiera puszy się przed tym, który nią rąbie? Czy piła wynosi się nad tego, który nią trze? Jak gdyby laska wywijała tym, który ją podnosi, jak gdyby kij podnosił tego, który nie jest z drewna.'' \mbox{(Iz 10, 15)}
\end{quote}
\paragraph{}
\begin{quote}
,,Błogosławiony mąż, który polega na Panu, którego ufnością jest Pan! Jest on jak drzewo zasadzone nad wodą, które nad potok zapuszcza swoje korzenie, nie boi się, gdy upał nadchodzi, lecz jego liść pozostaje zielony, i w roku posuchy się nie frasuje, i nie przestaje wydawać owocu. Podstępne jest serce, bardziej niż wszystko inne, i zepsute, któż może je poznać? Ja, Pan, zgłębiam serce, wystawiam na próbę nerki, aby oddać każdemu według jego postępowania, według owocu jego uczynków.'' \mbox{(Jr 17, 7-10)}
\end{quote}
\paragraph{}
\begin{quote}
,,Gdyż zupełnie mi się sprzeniewierzył dom Izraela i dom Judy - mówi Pan. Zaparli się Pana i rzekli: Nie ma go, nie przyjdzie na nas nic złego, a miecza i głodu nie zobaczymy. A prorocy? Przeminą z wiatrem, gdyż nie ma u nich słowa; tak im się stanie. Dlatego tak mówi Pan, Bóg Zastępów: Ponieważ to mówią, przeto Ja uczynię moje słowa ogniem w twoich ustach, a ten lud drewnem, które pożre ogień.'' \mbox{(Jr 5, 11-14)}
\end{quote}
\paragraph{}
\begin{quote}
,,Strzeżcie się fałszywych proroków, którzy przychodzą do was w odzieniu owczym, wewnątrz zaś są wilkami drapieżnymi! Po ich owocach poznacie ich. Czyż zbierają winogrona z cierni albo z ostu figi? Tak każde dobre drzewo wydaje dobre owoce, ale złe drzewo wydaje złe owoce. Nie może dobre drzewo rodzić złych owoców, ani złe drzewo rodzić dobrych owoców. Każde drzewo, które nie wydaje dobrego owocu, wycina się i rzuca w ogień Tak więc po owocach poznacie ich.'' \mbox{(Mt 7, 15-20)}
\end{quote}
\end{document}