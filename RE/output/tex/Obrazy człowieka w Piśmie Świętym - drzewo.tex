\documentclass[10pt,a4paper,oneside]{article}
\usepackage[utf8]{inputenc}
\usepackage{polski}
\usepackage[polish]{babel}
\usepackage[margin=0.5in,bottom=0.75in]{geometry}
\begin{document}
\centerline{\textbf{\MakeUppercase{Obrazy człowieka w Piśmie Świętym - drzewo}}}
\begin{center}
\textbf{Wersety do studium:} 
\mbox{(Ps 1, 1-3)}, \mbox{(Dn 4, 6-9.16-19)}, \mbox{(Łk 3, 1-3.7-9)}, \mbox{(Łk 6, 43-44)}, \mbox{(Iz 5, 1-7)}, \mbox{(Jl 1, 4-7)}
\end{center}
\section{Wstęp}
\paragraph{}
Podczas tego studium przyjrzymy się bliżej jednemu z obrazów człowieka, ukazanych w Piśmie Świętym, przedstawijącego go jako drzewo.
\section{Człowiek przedstawiony jako drzewo}
\paragraph{}
\begin{quote}
,,Szczęśliwy mąż, który nie idzie za radą bezbożnych Ani nie stoi na drodze grzeszników, Ani nie zasiada w gronie szyderców, Lecz ma upodobanie w zakonie Pana I zakon jego rozważa dniem i nocą. Będzie on jak drzewo zasadzone nad strumieniami wód, Wydające swój owoc we właściwym czasie, Którego liść nie więdnie, A wszystko, co uczyni, powiedzie się.'' \mbox{(Ps 1, 1-3)}
\end{quote}
\paragraph{}
\begin{quote}
,,Baltazarze, przełożony wróżbitów! Wiem, że duch świętych bogów jest w tobie i że żadna tajemnica nie jest dla ciebie trudna. Posłuchaj mojego widzenia sennego, które miałem i podaj mi jego wykład. A oto co widziałem na moim łożu: Widziałem, a oto w środku ziemi było drzewo, a jego wysokość była duża. Drzewo to rosło i było potężne; jego wysokość sięgała nieba, a było widoczne aż po krańce całej ziemi. Liść jego był piękny, owoc jego obfity i był na nim pokarm dla wszystkich. Zwierzęta polne szukały pod nim cienia, a w jego gałęziach gnieździło się ptactwo niebieskie i żywiło się z niego wszelkie ciało. (\ldots) Wtedy Daniel, zwany Baltazarem, przez krótki czas był osłupiały i przerażony w swoich myślach. Król zaś odezwał się i rzekł: Baltazarze, sen i wykład niech cię nie trwożą! Baltazar odpowiedział i rzekł: Panie mój! Ten sen dotyczy twoich nieprzyjaciół, a jego wykład twoich przeciwników. Drzewo, które widziałeś, które rosło i było potężne, którego wysokość sięgała nieba, a było widoczne na całej ziemi, Którego liście były piękne, a owoc obfity, i które miało pokarm dla wszystkich, pod którym mieszkały zwierzęta polne, a w jego gałęziach gnieździły się ptaki niebieskie To jesteś ty, królu: rosłeś i stałeś się potężny, twoja wielkość urosła i sięga nieba, twoja władza rozciąga się aż po krańce ziemi.'' \mbox{(Dn 4, 6-9.16-19)}
\end{quote}
\paragraph{}
\begin{quote}
,,W piętnastym roku panowania cesarza Tyberiusza, gdy namiestnikiem Judei był Poncjusz Piłat, tetrarchą galilejskim Herod, tetrarchą iturejskim i trachonickim Filip, brat jego, a tetrarchą abileńskim Lizaniasz, Za arcykapłanów Annasza i Kaifasza doszło Słowo Boże Jana, syna Zachariasza, na pustyni. I przeszedł całą krainę nadjordańską, głosząc chrzest upamiętania na odpuszczenie grzechów, (\ldots) Mówił więc do tłumów, które przychodziły, aby się dać ochrzcić przez niego: Plemię żmijowe, któż wam poddał myśl, aby uciekać przed przyszłym gniewem? Wydawajcie więc owoce godne upamiętania. A nie próbujcie wmawiać w siebie: Ojca mamy Abrahama; powiadam wam bowiem, że Bóg może z tych kamieni wzbudzić dzieci Abrahamowi. A już i siekiera do korzenia drzew jest przyłożona; wszelkie więc drzewo, które nie wydaje owocu dobrego, zostaje wycięte i w ogień wrzucone.'' \mbox{(Łk 3, 1-3.7-9)}
\end{quote}
\paragraph{}
\begin{quote}
,,Nie ma bowiem drzewa dobrego, które by rodziło owoc zły, ani też drzewa złego, które by rodziło owoc dobry. Każde bowiem drzewo poznaje się po jego owocu, bo nie zbierają z cierni fig ani winogron z głogu.'' \mbox{(Łk 6, 43-44)}
\end{quote}
\section{Winnica Boża}
\paragraph{}
\begin{quote}
,,Zaśpiewam mojemu ulubieńcowi ulubioną jego pieśń o jego winnicy. Ulubieniec mój miał winnicę na pagórku urodzajnym. Przekopał ją i oczyścił z kamieni, i zasadził w niej szlachetne szczepy. Zbudował w niej wieżę i wykuł w niej prasę, oczekiwał, że wyda szlachetne grona, lecz ona wydała złe owoce. Teraz więc, obywatele jeruzalemscy i mężowie judzcy, rozsądźcie między mną i między moją winnicą! Cóż jeszcze należało uczynić mojej winnicy, czego ja jej nie uczyniłem? Dlaczego oczekiwałem, że wyda szlachetne grona, a ona wydała złe owoce? Otóż teraz chcę wam ogłosić, co uczynię z moją winnicą: Rozbiorę jej płot, aby ją spasiono, rozwalę jej mur, aby ją zdeptano. Zniszczę ją doszczętnie: Nie będzie przycinana ani okopywana, ale porośnie cierniem i ostem, nadto nakażę obłokom, by na nią nie spuszczały deszczu. Zaiste, winnicą Pana Zastępów jest dom izraelski, a mężowie judzcy ulubioną jego latoroślą. Oczekiwał prawa, a oto - bezprawie; sprawiedliwości, a oto - krzyk.'' \mbox{(Iz 5, 1-7)}
\end{quote}
\paragraph{}
\begin{quote}
,,Co zostało po gąsienicy, pożarła szarańcza, a co zostało po szarańczy, pożarł konik polny, co zaś zostało po koniku polnym, pożarła larwa. Ocućcie się, pijani i płaczcie, zawodźcie z powodu moszczu, wy wszyscy, którzy pijecie wino, że odjęty jest od waszych ust. Gdyż moją ziemię naszedł lud, mocny i niezliczony. Jego zęby są jak zęby lwa, jego uzębienie jak uzębienie lwicy. Spustoszył moją winnicę, połamał moje drzewa figowe, odarł je doszczętnie z kory i zostawił tak, że zbielały jego gałęzie.'' \mbox{(Jl 1, 4-7)}
\end{quote}
\end{document}