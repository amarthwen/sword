\documentclass[10pt,a4paper,oneside]{article}
\usepackage[utf8]{inputenc}
\usepackage{polski}
\usepackage[polish]{babel}
\usepackage[margin=0.5in,bottom=0.75in]{geometry}
\begin{document}
\centerline{\textbf{\MakeUppercase{Niewola Babilońska}}}
\begin{center}
\textbf{Wersety do studium:} \mbox{(2Krl 25, 1-7)}, \mbox{(2Krl 25, 8-10)}, \mbox{(2Krl 25, 11-21)}, \mbox{(Jr 25, 1-14)}, \mbox{(2Krl 24, 10-16)}, \mbox{(Dn 1, 1-2)}, \mbox{(Jr 28, 1)}, \mbox{(Ez 1, 1)}, \mbox{(Ez 8, 1)}, \mbox{(Ez 20, 1)}, \mbox{(Ez 24, 1)}, \mbox{(Jr 52, 4)}, \mbox{(Jr 39, 1)}, \mbox{(2Krl 25, 1)}, \mbox{(Ez 29, 1)}, \mbox{(Ez 30, 20)}, \mbox{(Ez 31, 1)}, \mbox{(Jr 52, 6)}, \mbox{(Jr 39, 2)}, \mbox{(2Krl 25, 2-4)}, \mbox{(Jr 52, 12)}, \mbox{(Jr 41, 1)}, \mbox{(2Krl 25, 25-26)}, \mbox{(Ez 26, 1)}, \mbox{(Ez 32, 17)}, \mbox{(Ez 33, 21)}, \mbox{(Ez 32, 1)}, \mbox{(Ez 40, 1)}, \mbox{(Jr 52, 31)}, \mbox{(2Krl 25, 27)}
\end{center}
\section{Niewola Babilońska w Starym Testamencie}
\subsection{Niewola Babilońska w Drugiej Księdze Królewskiej}
\subsubsection{Pierwsze oblężenie Jerozolimy}
\subsubsection{Drugie oblężenie oraz zdobycie Jerozolimy}
\paragraph{}
\begin{quote}
,,W dziewiątym roku jego panowania, w dziesiątym miesiącu dziesiątego dnia tegoż miesiąca, nadciągnął Nebukadnesar, król babiloński, wraz z całym swoim wojskiem pod Jeruzalem, obległ je i usypał wokoło niego szańce. I miasto było oblężone aż do jedenastego roku panowania króla Sedekiasza. Dziewiątego dnia tego miesiąca, gdy wzmógł się głód w mieście i nie stało już chleba dla prostego ludu, Zrobiono wyłom w murze miasta i król oraz wszyscy wojownicy uciekli w nocy bramą między dwoma murami obok królewskiego ogrodu, podczas gdy Chaldejczycy jeszcze otaczali miasto, i skierowali się w stronę puszczy. Lecz wojsko chaldejskie puściło się za królem w pogoń i dognali go na stepach jerychońskich, całe zaś jego wojsko opuściwszy go, rozpierzchło się. Pojmali tedy króla i zaprowadzili go do króla babilońskiego do Rybli i tam go osądzili. Synów Sedekiasza na jego oczach zabito, jego samego kazał oślepić, okuć w kajdany i zaprowadzić do Babilonu.'' \mbox{(2 Krl 25, 1-7)}
\end{quote}
\subsubsection{Spalenie Świątyni Pana oraz częściowe zburzenie Jerozolimy}
\paragraph{}
\begin{quote}
,,W piątym miesiącu, siódmego dnia tegoż miesiąca, a był to dziewiętnasty rok panowania króla Nebukadnesara, króla babilońskiego, przybył Nebuzaradan, dowódca gwardii przybocznej, dworzanin króla babilońskiego, do Jeruzalemu. I spalił świątynię Pana, dom królewski i wszystkie domy w Jeruzalemie, wszystkie duże domy spalił ogniem, Wszystkie zaś mury otaczające Jeruzalem zburzyło całe wojsko chaldejskie, które było przy dowódcy gwardii przybocznej,'' \mbox{(2 Krl 25, 8-10)}
\end{quote}
\subsubsection{Deportacja ludności oraz wywiezienie naczyń świątynnych do Babilonu}
\paragraph{}
\begin{quote}
,,Resztę ludu zaś, która jeszcze pozostała w mieście, i tych, którzy zbiegli do króla babilońskiego oraz resztę pospólstwa uprowadził Nebuzaradan, dowódca gwardii przybocznej do niewoli. Niektórych jednak z biedoty wiejskiej pozostawił dowódca gwardii przybocznej jako winogrodników i oraczy. Kolumny zaś spiżowe, które były w świątyni Pana i podwozia i spiżową kadź, która była w świątyni Pana, Chaldejczycy porozbijali i spiż z nich wywieźli do Babilonu. Zabrali także misy, łopatki, szczypce, czasze oraz wszystkie przybory ze spiżu, którymi się posługiwano przy służbie Bożej, Jak również kadzielnice i kropielnice, wszystko, co było ze złota i ze srebra, zabrał dowódca gwardii przybocznej. Trudno podać wagę spiżu wszystkich przyborów: obu tych kolumn, jednej kadzi na wodę i podwozi, które Salomon kazał sporządzić dla świątyni Pana. Wszak osiemnaście łokci wysokości miała jedna kolumna, a nad nią była głowica spiżowa, której wysokość wynosiła trzy łokcie, a wokół głowicy była siatka i jabłuszka granatu, wszystkie ze spiżu; tak samo było na drugiej kolumnie przy siatce. Dowódca gwardii przybocznej zabrał także Serajasza, głównego kapłana, i Sefaniasza, kapłana drugiego z rzędu oraz trzech odźwiernych, Z miasta zaś zabrał jednego eunucha, który był ustanowiony nad wojownikami, oraz pięciu ludzi z najbliższego otoczenia króla, których znaleziono w mieście, i sekretarza dowódcy zastępu, który przeprowadził pobór wojskowy wśród prostego ludu, i sześćdziesięciu ludzi z ludu prostego, których znaleziono w mieście. Tych kazał zabrać Nebuzaradan, dowódca gwardii przybocznej, i zaprowadzić do króla babilońskiego do Ribli. Król babiloński zaś kazał ich pozbawić życia w Ribli, w ziemi Chamat. I tak poszedł Juda z ziemi swojej do niewoli.'' \mbox{(2 Krl 25, 11-21)}
\end{quote}
\subsection{Niewola Babilońska w Drugiej Księdze Kronik}
\subsection{Niewola Babilońska w Księdze Jeremiasza}
\subsection{Niewola Babilońska w Księdze Ezechiela}
\subsection{Niewola Babilońska w Księdze Daniela}
\subsection{Chronologia}
\subsubsection{Zapowiedź niewoli Babilońskiej}
\paragraph{}
\begin{quote}
,,Słowo, które doszło Jeremiasza w czwartym roku Jojakima, syna Jozjasza, króla judzkiego, a był to pierwszy rok Nebukadnesara, króla babilońskiego o całym ludzie judzkim. Wypowiedział je prorok Jeremiasz do całego ludu judzkiego i do wszystkich mieszkańców Jeruzalemu, mówiąc; Od trzynastego roku Jozjasza, syna Amona, króla judzkiego, aż do dnia dzisiejszego, to jest przez dwadzieścia trzy lata, dochodziło mnie słowo Pana, które głosiłem wam nieustannie, ale nie słuchaliście. Posyłał też Pan do was nieustannie i gorliwie wszystkie swoje sługi, proroków, ale nie słuchaliście i nie nakłoniliście swojego ucha, aby słuchać. A mówił; Nawróćcie się, każdy ze swojej złej drogi i swoich złych uczynków, a będziecie mogli mieszkać w ziemi, którą Pan dał wam i waszym ojcom od wieków na wieki. Nie chodźcie za cudzymi bogami, aby im służyć i oddawać im pokłon, i nie pobudzajcie mnie do gniewu dziełem swoich rąk, a nie uczynię wam nic złego. Lecz mnie nie słuchaliście - mówi Pan - pobudzając mnie do gniewu dziełem swoich rąk, ku waszej zgubie. Dlatego tak mówi Pan Zastępów Ponieważ nie słuchaliście moich słów, Oto Ja poślę i zbiorę wszystkie plemiona z północy - mówi Pan - i poślę po Nebukadnesara, króla babilońskiego, mojego sługę, i sprowadzę ich na tę ziemię i na jej mieszkańców, i na wszystkie narody dokoła. I zniszczę je doszczętnie, i uczynię je przedmiotem zgrozy, pośmiewiskiem i hańbą na zawsze. I sprawię, że zamilknie u nich głos radości i głos wesela, głos oblubieńca i głos oblubienicy, ustanie turkot żaren i blask pochodni. I cała ta ziemia stanie się rumowiskiem i pustkowiem, i narody te będą poddane królowi babilońskiemu, siedemdziesiąt lat. A po upływie siedemdziesięciu lat ukarzę króla babilońskiego i ów naród za ich winę - mówi Pan - i kraj Chaldejczyków i obrócę go w wieczną pustynię. I spełnię nad tą ziemią wszystkie moje słowa, które wypowiedziałem o niej, wszystko, co napisano w tej księdze, co prorokował Jeremiasz o wszystkich narodach. Bo i one same będą podbite przez potężne narody i wielkich królów, i odpłacę im według ich uczynków i według dzieła ich rąk.'' \mbox{(Jr 25, 1-14)}
\end{quote}
\subsubsection{Pierwsze oblężenie Jerozolimy. Początek niewoli}
\paragraph{}
\begin{quote}
,,W tym czasie nadciągnęli słudzy Nebukadnesara, króla babilońskiego, pod Jeruzalem i miasto zostało oblężone. Gdy potem sam Nebukadnesar, król babiloński, przybył pod miasto, a słudzy jego je oblegali, Wyszedł Jehojachin, król judzki wraz ze swoją matką i ze swoim dworem, i swymi dostojnikami, i eunuchami do króla babilońskiego. Wtedy król babiloński kazał go pojmać w ósmym roku swojego panowania I kazał wywieźć stamtąd wszystkie skarby świątyni Pana i skarby królewskiego domu i potłuc wszystkie złote naczynia, które sporządził Salomon, król izraelski, dla przybytku Pana, jak to zapowiedział Pan. I zagarnął do niewoli całe Jeruzalem, wszystkich dostojników i całe rycerstwo, dziesięć tysięcy jeńców oraz wszystkich kowali i ślusarzy; nie pozostał nikt oprócz biedoty spośród prostego ludu. Uprowadził do Babilonu Jehojachina, a także królową matkę i żony króla i jego eunuchów, i możnych kraju uprowadził z Jeruzalemu do Babilonu. Nadto wszystkich ludzi znacznych w liczbie siedmiu tysięcy i tysiąc kowali i ślusarzy, całe rycerstwo zdatne do walki, uprowadził król babiloński do niewoli, do Babilonu.'' \mbox{(2 Krl 24, 10-16)}
\end{quote}
\paragraph{}
\begin{quote}
,,Trzeciego roku panowania Jojakima, króla judzkiego, nadciągnął Nebukadnesar, król babiloński, do Jeruzalemu i obległ je. A Pan wydał w jego ręce Jojakima, króla judzkiego, i część naczyń domu Bożego; on zaś sprowadził je do ziemi Szinear, do domu swojego boga, a naczynia wniósł do skarbca swojego boga.'' \mbox{(Dn 1, 1-2)}
\end{quote}
\subsubsection{Czwarty rok niewoli}
\paragraph{}
\begin{quote}
,,W tym samym roku, na początku panowania Sedekiasza, króla judzkiego, w czwartym roku, w piątym miesiącu, rzekł do mnie Chananiasz, syn Azura, prorok z Gibeonu, w domu Pana, wobec kapłanów i całego ludu:'' \mbox{(Jr 28, 1)}
\end{quote}
\subsubsection{Piąty rok niewoli}
\paragraph{}
\begin{quote}
,,W trzydziestym roku, w czwartym miesiącu, piątego dnia tego miesiąca, gdy byłem wśród wygnańców nad rzeką Kebar, otworzyły się niebiosa i miałem widzenie Boże.'' \mbox{(Ez 1, 1)}
\end{quote}
\subsubsection{Szósty rok niewoli}
\paragraph{}
\begin{quote}
,,Gdy w szóstym roku, w szóstym miesiącu, piątego dnia tego miesiąca, siedziałem w swoim domu, a starsi judzcy siedzieli przede mną, spoczęła tam na mnie ręka Wszechmocnego Pana,'' \mbox{(Ez 8, 1)}
\end{quote}
\subsubsection{Siódmy rok niewoli}
\paragraph{}
\begin{quote}
,,W siódmym roku, dziesiątego dnia piątego miesiąca przyszli niektórzy mężowie spośród starszych izraelskich, aby się radzić Pana, i usiedli przede mną.'' \mbox{(Ez 20, 1)}
\end{quote}
\subsubsection{Dziewiąty rok niewoli}
\paragraph{}
\begin{quote}
,,W dziewiątym roku, w dziesiątym miesiącu, dziesiątego dnia tego miesiąca doszło mnie słowo Pana tej treści:'' \mbox{(Ez 24, 1)}
\end{quote}
\paragraph{}
\begin{quote}
,,W dziewiątym roku jego panowania w dziesiątym miesiącu, dziesiątego dnia tegoż miesiąca, przybył Nebukadnesar, król babiloński, wraz z całym swoim wojskiem pod Jeruzalem i obległ je, i usypali wokoło niego szańce.'' \mbox{(Jr 52, 4)}
\end{quote}
\paragraph{}
\begin{quote}
,,I był tam, gdy zdobyto Jeruzalem. W dziewiątym roku Sedekiasza, króla judzkiego, w dziesiątym miesiącu, przybył Nebukadnesar, król babiloński, z całym swoim wojskiem pod Jeruzalem i oblegali je,'' \mbox{(Jr 39, 1)}
\end{quote}
\paragraph{}
\begin{quote}
,,W dziewiątym roku jego panowania, w dziesiątym miesiącu dziesiątego dnia tegoż miesiąca, nadciągnął Nebukadnesar, król babiloński, wraz z całym swoim wojskiem pod Jeruzalem, obległ je i usypał wokoło niego szańce.'' \mbox{(2 Krl 25, 1)}
\end{quote}
\subsubsection{Dziesiąty rok niewoli}
\paragraph{}
\begin{quote}
,,W dziesiątym roku, w dziewiątym miesiącu, dwunastego dnia tego miesiąca doszło mnie słowo Pana tej treści:'' \mbox{(Ez 29, 1)}
\end{quote}
\subsubsection{Jedenasty rok niewoli}
\paragraph{}
\begin{quote}
,,W jedenastym roku, w pierwszym miesiącu, siódmego dnia miesiąca doszło mnie słowo Pana tej treści:'' \mbox{(Ez 30, 20)}
\end{quote}
\paragraph{}
\begin{quote}
,,W jedenastym roku, w trzecim miesiącu, pierwszego dnia tego miesiąca doszło mnie słowo Pana tej treści:'' \mbox{(Ez 31, 1)}
\end{quote}
\paragraph{}
\begin{quote}
,,W czwartym miesiącu, dziewiątego dnia tego miesiąca, gdy głód się wzmógł w mieście i nie stało już chleba dla prostego ludu,'' \mbox{(Jr 52, 6)}
\end{quote}
\paragraph{}
\begin{quote}
,,A w jedenastym roku panowania Sedekiasza, w czwartym miesiącu, dziewiątego dnia tegoż miesiąca zrobiono wyłom w murze miasta.'' \mbox{(Jr 39, 2)}
\end{quote}
\paragraph{}
\begin{quote}
,,I miasto było oblężone aż do jedenastego roku panowania króla Sedekiasza. Dziewiątego dnia tego miesiąca, gdy wzmógł się głód w mieście i nie stało już chleba dla prostego ludu, Zrobiono wyłom w murze miasta i król oraz wszyscy wojownicy uciekli w nocy bramą między dwoma murami obok królewskiego ogrodu, podczas gdy Chaldejczycy jeszcze otaczali miasto, i skierowali się w stronę puszczy.'' \mbox{(2 Krl 25, 2-4)}
\end{quote}
\paragraph{}
\begin{quote}
,,A w piątym miesiącu siódmego dnia tego miesiąca, był to dziewiętnasty rok króla Nebukadnesara, króla babilońskiego, przybył do Jeruzalemu Nebuzaradan, dowódca straży przybocznej, który należał do bezpośredniego otoczenia króla babilońskiego,'' \mbox{(Jr 52, 12)}
\end{quote}
\paragraph{}
\begin{quote}
,,W piątym miesiącu, siódmego dnia tegoż miesiąca, a był to dziewiętnasty rok panowania króla Nebukadnesara, króla babilońskiego, przybył Nebuzaradan, dowódca gwardii przybocznej, dworzanin króla babilońskiego, do Jeruzalemu. I spalił świątynię Pana, dom królewski i wszystkie domy w Jeruzalemie, wszystkie duże domy spalił ogniem, Wszystkie zaś mury otaczające Jeruzalem zburzyło całe wojsko chaldejskie, które było przy dowódcy gwardii przybocznej,'' \mbox{(2 Krl 25, 8-10)}
\end{quote}
\paragraph{}
\begin{quote}
,,W siódmym miesiącu przybył Ismael, syn Netaniasza, syna Eliszamy, z rodu królewskiego, jeden z dostojników królewskich, a wraz z nim dziesięciu mężów, do Gedaliasza, syna Achikama, do Mispy; a gdy tam w Mispie spożywali razem posiłek,'' \mbox{(Jr 41, 1)}
\end{quote}
\paragraph{}
\begin{quote}
,,W siódmym miesiącu przybył Ismael, syn Netaniasza, syn Eliszamy, z rodu królewskiego, wraz z dziesięcioma mężami i zabili Gedaliasza oraz Judejczyków i Chaldejczyków, którzy byli z nim w Mispie. Wówczas cały lud, od najmniejszego do największego oraz dowódcy wojskowi zerwali się i udali się do Egiptu, gdyż bali się Chaldejczyków.'' \mbox{(2 Krl 25, 25-26)}
\end{quote}
\paragraph{}
\begin{quote}
,,W jedenastym roku, pierwszego dnia jedenastego miesiąca, doszło mnie Słowo Pana tej treści:'' \mbox{(Ez 26, 1)}
\end{quote}
\subsubsection{Dwunasty rok niewoli}
\paragraph{}
\begin{quote}
,,W dwunastym roku, w pierwszym miesiącu, piętnastego dnia tego miesiąca doszło mnie słowo Pana tej treści:'' \mbox{(Ez 32, 17)}
\end{quote}
\paragraph{}
\begin{quote}
,,W dwunastym roku naszego wygnania, piątego dnia dziesiątego miesiąca przybył do mnie uchodźca z Jeruzalemu z wieścią, że miasto zostało zdobyte.'' \mbox{(Ez 33, 21)}
\end{quote}
\paragraph{}
\begin{quote}
,,W dwunastym roku, w dwunastym miesiącu, pierwszego dnia tego miesiąca doszło mnie słowo Pana tej treści:'' \mbox{(Ez 32, 1)}
\end{quote}
\subsubsection{Dwudziesty piąty rok niewoli}
\paragraph{}
\begin{quote}
,,W dwudziestym piątym roku naszego wygnania, w pierwszym miesiącu, dziesiątego dnia tego miesiąca, w czternastym roku po zdobyciu miasta, w tym właśnie dniu spoczęła na mnie ręka Pana i przeniósł mnie tam,'' \mbox{(Ez 40, 1)}
\end{quote}
\subsubsection{Trzydziesty siódmy rok niewoli}
\paragraph{}
\begin{quote}
,,Lecz w trzydziestym siódmym roku od uprowadzenia Jojachina, króla judzkiego, w dwunastym miesiącu, dwudziestego piątego dnia tego miesiąca, Ewil-Merodach, król babiloński, w roku wstąpienia na tron ułaskawił Jojachina, króla judzkiego, i uwolnił go z więzienia.'' \mbox{(Jr 52, 31)}
\end{quote}
\paragraph{}
\begin{quote}
,,W trzydzieści siedem lat po uprowadzeniu do niewoli Jehojachina, króla judzkiego, w dwunastym miesiącu, dwudziestego siódmego dnia tegoż miesiąca, wypuścił Ewil-Merodach, król babiloński, w roku objęcia przezeń władzy królewskiej, Jehojachina, króla judzkiego, z więzienia.'' \mbox{(2 Krl 25, 27)}
\end{quote}
\end{document}