\documentclass[10pt,a4paper,oneside]{article}
\usepackage[utf8]{inputenc}
\usepackage{polski}
\usepackage[polish]{babel}
\usepackage[margin=0.5in,bottom=0.75in]{geometry}
\begin{document}
\centerline{\textbf{\MakeUppercase{Cielesny a duchowy Izraelita}}}
\begin{center}
\textbf{Wersety do studium:} \mbox{(Rz 2, 28-29)}, \mbox{(Am 8, 4-6)}, \mbox{(J 2, 13-16)}, \mbox{(J 13, 6-11)}, \mbox{(Łk 11, 24-26)}, \mbox{(1Kor 15, 44)}, \mbox{(J 4, 31-34)}, \mbox{(Za 3, 1-5)}, \mbox{(Obj 3, 17-18)}, \mbox{(Am 8, 11-12)}, \mbox{(J 1, 33)}, \mbox{(Mt 6, 19-20)}, \mbox{(Ga 5, 16-17)}, \mbox{(J 4, 24)}, \mbox{(Ps 51, 19)}, \mbox{(Mt 12, 46-50)}, \mbox{(Ez 28, 15-19)}, \mbox{(2Kor 2, 17)}
\end{center}
\section{Temat główny}
\subsection{Ziemski Izraelita}
\subsubsection{Cielesna nieczystość}
\subsubsection{Cielesna krew}
\subsubsection{Cielesne ciało}
\subsubsection{Cielesny pokarm}
\subsubsection{Cielesny wzrok i słuch}
\subsubsection{Ziemska szata}
\subsubsection{Obrzezanie na ciele}
\subsubsection{Cielesna nagość}
\subsubsection{Cielesny głód}
\subsubsection{Cielesny sen}
\subsubsection{Cielesny chrzest}
\subsubsection{Cielesna śmierć}
\subsubsection{Ziemski skarb}
\subsubsection{Ziemski dom}
\subsubsection{Cielesna ślepota i głuchota}
\subsubsection{Życie według ciała}
\subsubsection{Ziemska bieda (cielesna)}
\subsubsection{Składanie cielesnych ofiar}
\subsubsection{Ziemska rodzina}
\subsubsection{Cielesny handel}
\paragraph{}
\begin{quote}
,,Słuchajcie tego wy, którzy depczecie ubogiego i tępicie biednych kraju, mówiąc: Kiedyż minie nów, abyśmy mogli kupczyć zbożem, i sabat, abyśmy mogli ziarno wystawić na sprzedaż, pomniejszyć efę, a powiększyć odważniki, przechylić oszukańczo wagę, Nabywać nędzarzy za pieniądze, a ubogich za parę sandałów i sprzedawać poślad zboża?'' \mbox{(Am 8, 4-6)}
\end{quote}
\paragraph{}
\begin{quote}
,,A gdy się zbliżała Pascha żydowska, udał się Jezus do Jerozolimy. I zastał w świątyni sprzedających woły i owce, i gołębie, i siedzących wekslarzy. I skręciwszy bicz z powrózków, wypędził ich wszystkich ze świątyni wraz z owcami i wołami; wekslarzom rozsypał pieniądze i stoły powywracał, A do sprzedawców gołębi rzekł: Zabierzcie to stąd, z domu Ojca mego nie czyńcie targowiska.'' \mbox{(J 2, 13-16)}
\end{quote}
\subsection{Niebiański Izraelita}
\subsubsection{Duchowa nieczystość}
\paragraph{}
\begin{quote}
,,Podszedł też do Szymona Piotra, który mu rzekł: Panie, Ty miałbyś umywać nogi moje? Odpowiedział Jezus i rzekł mu: Co Ja czynię, ty nie wiesz teraz, ale się potem dowiesz. Rzecze mu Piotr: Przenigdy nie będziesz umywał nóg moich! Odpowiedział mu Jezus: Jeśli cię nie umyję, nie będziesz miał działu ze mną. Rzecze mu Szymon Piotr: Panie, nie tylko nogi moje, lecz i ręce, i głowę. Rzecze mu Jezus: Kto jest umyty, nie ma potrzeby myć się, chyba tylko nogi, bo czysty jest cały. I wy czyści jesteście, lecz nie wszyscy. Wiedział bowiem, kto ma go wydać; dlatego rzekł: Nie wszyscy jesteście czyści.'' \mbox{(J 13, 6-11)}
\end{quote}
\paragraph{}
\begin{quote}
,,Gdy duch nieczysty wyjdzie z człowieka, wędruje po miejscach bezwodnych, szukając ukojenia, a gdy nie znajdzie, mówi: Wrócę do domu swego, skąd wyszedłem. I przyszedłszy, zastaje go wymiecionym i przyozdobionym. Wówczas idzie i zabiera z sobą siedem innych duchów, gorszych niż on, i wchodzą, i mieszkają tam. I bywa końcowy stan człowieka tego, gorszy niż pierwotny.'' \mbox{(Łk 11, 24-26)}
\end{quote}
\subsubsection{Duchowa krew}
\subsubsection{Duchowe ciało}
\paragraph{}
\begin{quote}
,,Sieje się ciało cielesne, bywa wzbudzone ciało duchowe. Jeżeli jest ciało cielesne, to jest także ciało duchowe.'' \mbox{(1 Kor 15, 44)}
\end{quote}
\subsubsection{Duchowy pokarm}
\paragraph{}
\begin{quote}
,,Tymczasem jego uczniowie, prosili go, mówiąc: Mistrzu, jedz! Ale On rzekł do nich: Ja mam pokarm do jedzenia, o którym wy nie wiecie. Wtedy uczniowie mówili między sobą: Czy kto przyniósł mu jeść? Jezus rzekł do nich: Moim pokarmem jest pełnić wolę tego, który mnie posłał, i dokonać jego dzieła.'' \mbox{(J 4, 31-34)}
\end{quote}
\subsubsection{Duchowy wzrok i słuch}
\subsubsection{Niebiańska szata}
\paragraph{}
\begin{quote}
,,Potem ukazał mi Jozuego, arcykapłana, stojącego przed aniołem Pana i szatana stojącego po jego prawicy, aby go oskarżać. Wtedy anioł Pana rzekł do szatana; Niech cię zgromi Pan, szatanie, niech cię zgromi Pan, który obrał Jeruzalem! Czyż nie jest ono głownią wyrwaną z ognia? A Jozue był ubrany w szatę brudną i tak stał przed aniołem, A ten tak odezwał się i rzekł do sług, którzy stali przed nim: Zdejmijcie z niego brudną szatę! Do niego zaś rzekł: Oto ja zdjąłem z ciebie twoją winę i każę cię przyoblec w szaty odświętne. Potem rzekł: Włóżcie mu na głowę czysty zawój! I włożyli mu na głowę czysty zawój, i przyoblekli go w szaty. A anioł Pana stał przy tym.'' \mbox{(Za 3, 1-5)}
\end{quote}
\subsubsection{Obrzezanie w duchu}
\subsubsection{Duchowa nagość}
\paragraph{}
\begin{quote}
,,Ponieważ mówisz: Bogaty jestem i wzbogaciłem się, i niczego nie potrzebuję, a nie wiesz, żeś pożałowania godzien nędzarz i biedak, ślepy i goły, Radzę ci, abyś nabył u mnie złota w ogniu wypróbowanego, abyś się wzbogacił i abyś przyodział szaty białe, aby nie wystąpiła na jaw haniebna nagość twoja, oraz maści, by nią namaścić oczy twoje, abyś przejrzał.'' \mbox{(Obj 3, 17-18)}
\end{quote}
\subsubsection{Duchowy głód}
\paragraph{}
\begin{quote}
,,Oto idą dni - mówi Wszechmogący Pan - że ześlę głód na ziemię, nie głód chleba ani pragnienie wody, lecz słuchania słów Pana. I wlec się będą od morza do morza, i tułać się z północy na wschód, szukając słowa Pana, lecz nie znajdą.'' \mbox{(Am 8, 11-12)}
\end{quote}
\subsubsection{Duchowy sen}
\subsubsection{Duchowy chrzest}
\paragraph{}
\begin{quote}
,,I ja go nie znałem; lecz Ten, który mnie posłał, abym chrzcił wodą, rzekł do mnie: Ujrzysz tego, na którego Duch zstępuje i na nim spocznie, Ten chrzci Duchem Świętym.'' \mbox{(J 1, 33)}
\end{quote}
\subsubsection{Duchowa śmierć}
\subsubsection{Niebiański skarb}
\paragraph{}
\begin{quote}
,,Nie gromadźcie sobie skarbów na ziemi, gdzie je mól i rdza niszczą i gdzie złodzieje podkopują i kradną: Ale gromadźcie sobie skarby w niebie, gdzie ani mól, ani rdza nie niszczą i gdzie złodzieje nie podkopują i nie kradną.'' \mbox{(Mt 6, 19-20)}
\end{quote}
\subsubsection{Niebiański dom}
\subsubsection{Duchowa ślepota i głuchota}
\paragraph{}
\begin{quote}
,,Ponieważ mówisz: Bogaty jestem i wzbogaciłem się, i niczego nie potrzebuję, a nie wiesz, żeś pożałowania godzien nędzarz i biedak, ślepy i goły, Radzę ci, abyś nabył u mnie złota w ogniu wypróbowanego, abyś się wzbogacił i abyś przyodział szaty białe, aby nie wystąpiła na jaw haniebna nagość twoja, oraz maści, by nią namaścić oczy twoje, abyś przejrzał.'' \mbox{(Obj 3, 17-18)}
\end{quote}
\subsubsection{Życie według ducha}
\paragraph{}
\begin{quote}
,,Mówię więc: Według Ducha postępujcie, a nie będziecie pobłażali żądzy cielesnej. Gdyż ciało pożąda przeciwko Duchowi, a Duch przeciwko ciału, a te są sobie przeciwne, abyście nie czynili tego, co chcecie.'' \mbox{(Ga 5, 16-17)}
\end{quote}
\subsubsection{Niebiańska bieda (duchowa)}
\paragraph{}
\begin{quote}
,,Ponieważ mówisz: Bogaty jestem i wzbogaciłem się, i niczego nie potrzebuję, a nie wiesz, żeś pożałowania godzien nędzarz i biedak, ślepy i goły, Radzę ci, abyś nabył u mnie złota w ogniu wypróbowanego, abyś się wzbogacił i abyś przyodział szaty białe, aby nie wystąpiła na jaw haniebna nagość twoja, oraz maści, by nią namaścić oczy twoje, abyś przejrzał.'' \mbox{(Obj 3, 17-18)}
\end{quote}
\subsubsection{Składanie duchowych ofiar}
\paragraph{}
\begin{quote}
,,Bóg jest duchem, a ci, którzy mu cześć oddają, winni mu ją oddawać w duchu i w prawdzie.'' \mbox{(J 4, 24)}
\end{quote}
\paragraph{}
\begin{quote}
,,Ofiarą Bogu miłą jest duch skruszony, Sercem skruszonym i zgnębionym nie wzgardzisz, Boże.'' \mbox{(Ps 51, 19)}
\end{quote}
\subsubsection{Niebiańska rodzina}
\paragraph{}
\begin{quote}
,,A gdy On jeszcze mówił do tłumów, oto matka i bracia jego stanęli na dworze, chcąc z nim mówić. I rzekł mu ktoś: Oto matka twoja i bracia twoi stoją na dworze i chcą z tobą mówić. A On, odpowiadając, rzekł temu, co mu to powiedział: Któż jest moją matką? I kto to bracia moi? I wyciągnąwszy rękę ku uczniom swoim, rzekł: Oto matka moja i bracia moi! Albowiem ktokolwiek czyni wolę Ojca mojego, który jest w niebie, ten jest moim bratem i siostrą, i matką.'' \mbox{(Mt 12, 46-50)}
\end{quote}
\subsubsection{Duchowy handel}
\paragraph{}
\begin{quote}
,,Nienagannym byłeś w postępowaniu swoim od dnia, gdy zostałeś stworzony, aż dotąd, gdy odkryto u ciebie niegodziwość. Przy rozległym swoim handlu napełniłeś swoje wnętrze gwałtem i zgrzeszyłeś. Wtedy to wypędziłem cię z góry Bożej, a cherub, który bronił wstępu, wygubił cię spośród kamieni ognistych. Twoje serce było wyniosłe z powodu twojej piękności. Zniweczyłeś swoją mądrość skutkiem swojej świetności. Zrzuciłem cię na ziemię; postawiłem cię przed królami, aby się z ciebie naigrawali. Zbezcześciłeś moją świątynię z powodu mnóstwa swoich win, przy niegodziwym swoim handlu. Dlatego wywiodłem z ciebie ogień i ten cię strawił; obróciłem cię w popiół na ziemi na oczach wszystkich, którzy cię widzieli. Wszyscy, którzy cię znali pośród ludów, zdumiewali się nad tobą; stałeś się odstraszającym przykładem, przepadłeś na wieki.'' \mbox{(Ez 28, 15-19)}
\end{quote}
\paragraph{}
\begin{quote}
,,Bo my nie jesteśmy handlarzami Słowa Bożego, jak wielu innych, lecz mówimy w Chrystusie przed obliczem Boga jako ludzie szczerzy, jako ludzie mówiący z Boga.'' \mbox{(2 Kor 2, 17)}
\end{quote}
\end{document}