\documentclass[10pt,a4paper,oneside]{article}
\usepackage[utf8]{inputenc}
\usepackage{polski}
\usepackage[polish]{babel}
\usepackage[margin=0.5in,bottom=0.75in]{geometry}
\begin{document}
\centerline{\textbf{\MakeUppercase{Grzech}}}
\begin{center}
\textbf{Wersety do studium:} \mbox{(1Moj 4, 6-7)}, \mbox{(1Kor 15, 53-56)}, \mbox{(Rz 5, 12-14)}, \mbox{(1Moj 4, 3-7)}, \mbox{(Rz 6, 3.10-12)}, \mbox{(Rz 6, 16-22)}, \mbox{(Rz 6, 20-21.23)}, \mbox{(Rz 7, 9-11)}, \mbox{(Rz 7, 14-20)}, \mbox{(Rz 8, 6)}, \mbox{(Rz 8, 12-13)}, \mbox{(Jk 1, 12-15)}, \mbox{(Rz 7, 7-11.13)}, \mbox{(1Kor 15, 56)}, \mbox{(Rz 6, 1-14)}, \mbox{(Rz 7, 4-6)}
\end{center}
\section{Temat główny}
\paragraph{}
\begin{quote}
,,Albowiem to, co skażone, musi przyoblec się w to, co nieskażone, a to, co śmiertelne, musi przyoblec się w nieśmiertelność. A gdy to, co skażone, przyoblecze się w to, co nieskażone, i to, co śmiertelne, przyoblecze się w nieśmiertelność, wtedy wypełni się słowo napisane: Pochłonięta jest śmierć w zwycięstwie! Gdzież jest, o śmierci, zwycięstwo twoje? Gdzież jest, o śmierci, żądło twoje? A żądłem śmierci jest grzech, a mocą grzechu jest zakon;'' \mbox{(1 Kor 15, 53-56)}
\end{quote}
\paragraph{}
\begin{quote}
,,Przeto jak przez jednego człowieka grzech wszedł na świat, a przez grzech śmierć, tak i na wszystkich ludzi śmierć przyszła, bo wszyscy zgrzeszyli; Albowiem już przed zakonem grzech był na. świecie, ale grzechu się nie liczy, gdy zakonu nie ma; Lecz śmierć panowała od Adama aż do Mojżesza nawet nad tymi, którzy nie popełnili takiego przestępstwa jak Adam, będący obrazem tego, który miał przyjść.'' \mbox{(Rz 5, 12-14)}
\end{quote}
\subsection{Grzech i jego pożądliwości}
\paragraph{}
\begin{quote}
,,Po niejakim czasie Kain złożył Panu ofiarę z plonów rolnych; Abel także złożył ofiarę z pierworodnych trzody swojej i z tłuszczu ich. A Pan wejrzał na Abla i na jego ofiarę. Ale na Kaina i na jego ofiarę nie wejrzał; wtedy Kain rozgniewał się bardzo i zasępiło się jego oblicze. I rzekł Pan do Kaina: Czemu się gniewasz i czemu zasępiło się twoje oblicze? Wszak byłoby pogodne, gdybyś czynił dobrze, a jeśli nie będziesz czynił dobrze, u drzwi czyha grzech. Kusi cię, lecz ty masz nad nim panować.'' \mbox{(1 Moj 4, 3-7)}
\end{quote}
\paragraph{}
\begin{quote}
,,Czyż nie wiecie, że my wszyscy, ochrzczeni w Chrystusa Jezusa, w śmierć jego zostaliśmy ochrzczeni? (\ldots) Umarłszy bowiem, dla grzechu raz na zawsze umarł, a żyjąc, żyje dla Boga. Podobnie i wy uważajcie siebie za umarłych dla grzechu, a za żyjących dla Boga w Chrystusie Jezusie, Panu naszym. Niechże więc nie panuje grzech w śmiertelnym ciele waszym, abyście nie byli posłuszni pożądliwościom jego,'' \mbox{(Rz 6, 3.10-12)}
\end{quote}
\subsection{Niewola grzechu}
\paragraph{}
\begin{quote}
,,Czyż nie wiecie, że jeśli się oddajecie jako słudzy w posłuszeństwo, stajecie się sługami tego, komu jesteście posłuszni, czy to grzechu ku śmierci, czy też posłuszeństwa ku sprawiedliwości? Lecz Bogu niech będą dzięki, że wy, którzy byliście sługami grzechu, przyjęliście ze szczerego serca zarys tej nauki, której zostaliście przekazani, A uwolnieni od grzechu, staliście się sługami sprawiedliwości, Po ludzku mówię przez wzgląd na słabość waszego ciała. Jak bowiem oddawaliście członki wasze na służbę nieczystości i nieprawości ku popełnianiu nieprawości, tak teraz oddawajcie członki wasze na służbę sprawiedliwości ku poświęceniu. Gdy bowiem byliście sługami grzechu, byliście dalecy od sprawiedliwości. Jakiż więc mieliście wtedy pożytek? Taki, którego się teraz wstydzicie, a końcem tego jest śmierć. Teraz zaś, wyzwoleni od grzechu, a oddani w służbę Bogu, macie pożytek w poświęceniu, a za cel żywot wieczny.'' \mbox{(Rz 6, 16-22)}
\end{quote}
\subsection{Końcem grzechu jest śmierć}
\paragraph{}
\begin{quote}
,,Gdy bowiem byliście sługami grzechu, byliście dalecy od sprawiedliwości. Jakiż więc mieliście wtedy pożytek? Taki, którego się teraz wstydzicie, a końcem tego jest śmierć. (\ldots) Albowiem zapłatą za grzech jest śmierć, lecz darem łaski Bożej jest żywot wieczny w Chrystusie Jezusie, Panu naszym.'' \mbox{(Rz 6, 20-21.23)}
\end{quote}
\paragraph{}
\begin{quote}
,,I ja żyłem niegdyś bez zakonu, lecz gdy przyszło przykazanie, grzech ożył. A ja umarłem i okazało się, że to przykazanie, które miało mi być ku żywotowi, było ku śmierci. Albowiem grzech otrzymawszy podnietę przez przykazanie, zwiódł mnie i przez nie mnie zabił.'' \mbox{(Rz 7, 9-11)}
\end{quote}
\paragraph{}
\begin{quote}
,,Wiemy bowiem, że zakon jest duchowy, ja zaś jestem cielesny, zaprzedany grzechowi. Albowiem nie rozeznaję się w tym, co czynię; gdyż nie to czynię, co chcę, ale czego nienawidzę, to czynię. A jeśli to czynię, czego nie chcę, zgadzam się z tym, że zakon jest dobry. Ale wtedy czynię to już nie ja, lecz grzech, który mieszka we mnie. Wiem tedy, że nie mieszka we mnie, to jest w ciele moim, dobro; mam bowiem zawsze dobrą wolę, ale wykonania tego, co dobre, brak; Albowiem nie czynię dobrego, które chcę, tylko złe, którego nie chcę, to czynię. A jeśli czynię to, czego nie chcę, już nie ja to czynię, ale grzech, który mieszka we mnie.'' \mbox{(Rz 7, 14-20)}
\end{quote}
\paragraph{}
\begin{quote}
,,Albowiem zamysł ciała, to śmierć, a zamysł Ducha, to życie i pokój.'' \mbox{(Rz 8, 6)}
\end{quote}
\paragraph{}
\begin{quote}
,,Tak więc, bracia, jesteśmy dłużnikami nie ciała, aby żyć według ciała. Jeśli bowiem według ciała żyjecie, umrzecie; ale jeśli Duchem sprawy ciała umartwiacie, żyć będziecie.'' \mbox{(Rz 8, 12-13)}
\end{quote}
\paragraph{}
\begin{quote}
,,Błogosławiony mąż, który wytrwa w próbie, bo gdy wytrzyma próbę, weźmie wieniec żywota, obiecany przez Boga tym, którzy go miłują. Niechaj nikt, gdy wystawiony jest na pokusę, nie mówi: Przez Boga jestem kuszony; Bóg bowiem nie jest podatny na pokusy do złego ani sam nikogo nie kusi. Lecz każdy bywa kuszony przez własne pożądliwości, które go pociągają i nęcą; Potem, gdy pożądliwość pocznie, rodzi grzech, a gdy grzech dojrzeje, rodzi śmierć.'' \mbox{(Jk 1, 12-15)}
\end{quote}
\subsection{Mocą grzechu jest Prawo Boże}
\paragraph{}
\begin{quote}
,,Cóż więc powiemy? Że zakon to grzech? Przenigdy! Przecież nie poznałbym grzechu, gdyby nie zakon; wszak i o pożądliwości nie wiedziałbym, gdyby zakon nie mówił: Nie pożądaj! Lecz grzech przez przykazanie otrzymał bodziec i wzbudził we mnie wszelką pożądliwość, bo bez zakonu grzech jest martwy. I ja żyłem niegdyś bez zakonu, lecz gdy przyszło przykazanie, grzech ożył. A ja umarłem i okazało się, że to przykazanie, które miało mi być ku żywotowi, było ku śmierci. Albowiem grzech otrzymawszy podnietę przez przykazanie, zwiódł mnie i przez nie mnie zabił. (\ldots) Czy zatem to, co dobre, stało się dla mnie śmiercią? Przenigdy! To właśnie grzech, żeby się okazać grzechem, posłużył się rzeczą dobrą, by spowodować moją śmierć, aby grzech przez przykazanie okazał ogrom swojej grzeszności.'' \mbox{(Rz 7, 7-11.13)}
\end{quote}
\paragraph{}
\begin{quote}
,,A żądłem śmierci jest grzech, a mocą grzechu jest zakon;'' \mbox{(1 Kor 15, 56)}
\end{quote}
\subsection{Śmierć dla grzechu}
\paragraph{}
\begin{quote}
,,Cóż więc powiemy? Czy mamy pozostać w grzechu, aby łaska obfitsza była? Przenigdy! Jakże my, którzy grzechowi umarliśmy, jeszcze w nim żyć mamy? Czyż nie wiecie, że my wszyscy, ochrzczeni w Chrystusa Jezusa, w śmierć jego zostaliśmy ochrzczeni? Pogrzebani tedy jesteśmy wraz z nim przez chrzest w śmierć, abyśmy jak Chrystus wskrzeszony został z martwych przez chwałę Ojca, tak i my nowe życie prowadzili. Bo jeśli wrośliśmy w podobieństwo jego śmierci, wrośniemy również w podobieństwo jego zmartwychwstania, Wiedząc to, że nasz stary człowiek został wespół z nim ukrzyżowany, aby grzeszne ciało zostało unicestwione, byśmy już nadal nie służyli grzechowi; Kto bowiem umarł, uwolniony jest od grzechu. Jeśli tedy umarliśmy z Chrystusem, wierzymy, że też z nim żyć będziemy, Wiedząc, że zmartwychwzbudzony Chrystus już nie umiera, śmierć nad nim już nie panuje. Umarłszy bowiem, dla grzechu raz na zawsze umarł, a żyjąc, żyje dla Boga. Podobnie i wy uważajcie siebie za umarłych dla grzechu, a za żyjących dla Boga w Chrystusie Jezusie, Panu naszym. Niechże więc nie panuje grzech w śmiertelnym ciele waszym, abyście nie byli posłuszni pożądliwościom jego, I nie oddawajcie członków swoich grzechowi na oręż nieprawości, ale oddawajcie siebie Bogu jako ożywionych z martwych, a członki swoje Bogu na oręż sprawiedliwości. Albowiem grzech nad wami panować nie będzie, bo nie jesteście pod zakonem, lecz pod łaską.'' \mbox{(Rz 6, 1-14)}
\end{quote}
\paragraph{}
\begin{quote}
,,Przeto, bracia moi, i wy umarliście dla zakonu przez ciało Chrystusowe, by należeć do innego, do tego, który został wzbudzony z martwych, abyśmy owoc wydawali dla Boga. Albowiem gdy byliśmy w ciele, grzeszne namiętności rozbudzone przez zakon były czynne w członkach naszych, aby rodzić owoce śmierci; Lecz teraz zostaliśmy uwolnieni od zakonu, gdy umarliśmy temu, przez co byliśmy opanowani, tak iż służymy w nowości ducha, a nie według przestarzałej litery.'' \mbox{(Rz 7, 4-6)}
\end{quote}
\end{document}