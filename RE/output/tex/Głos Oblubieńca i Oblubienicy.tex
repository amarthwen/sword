\documentclass[10pt,a4paper,oneside]{article}
\usepackage[utf8]{inputenc}
\usepackage{polski}
\usepackage[polish]{babel}
\usepackage[margin=0.5in,bottom=0.75in]{geometry}
\begin{document}
\centerline{\textbf{\MakeUppercase{Głos Oblubieńca i Oblubienicy}}}
\begin{center}
\textbf{Wersety do studium:} \mbox{(Obj 18, 23)}, \mbox{(Jr 7, 34)}, \mbox{(Jr 16, 9)}, \mbox{(Jr 25, 9-11)}
\end{center}
\section{Temat główny}
\paragraph{}
\begin{quote}
,,I nie zabłyśnie już w tobie blask świecy, i nie usłyszy się w tobie głosu oblubieńca ani oblubienicy; gdyż kupcy twoi byli wielmożami ziemi, gdyż czarami twymi dały się zwieść wszystkie narody.'' \mbox{(Obj 18, 23)}
\end{quote}
\paragraph{}
\begin{quote}
,,I sprawię, ze ustanie w miastach judzkich i na ulicach Jeruzalemu głos wesela i głos radości, głos oblubieńca i głos oblubienicy, gdyż ziemia stanie się pustynią.'' \mbox{(Jr 7, 34)}
\end{quote}
\paragraph{}
\begin{quote}
,,Gdyż tak mówi Pan Zastępów, Bóg Izraela: Oto Ja sprawię, iż ustanie na tym miejscu na waszych oczach i za waszych dni głos wesela i głos radości, głos oblubieńca i głos oblubienicy.'' \mbox{(Jr 16, 9)}
\end{quote}
\paragraph{}
\begin{quote}
,,Oto Ja poślę i zbiorę wszystkie plemiona z północy - mówi Pan - i poślę po Nebukadnesara, króla babilońskiego, mojego sługę, i sprowadzę ich na tę ziemię i na jej mieszkańców, i na wszystkie narody dokoła. I zniszczę je doszczętnie, i uczynię je przedmiotem zgrozy, pośmiewiskiem i hańbą na zawsze. I sprawię, że zamilknie u nich głos radości i głos wesela, głos oblubieńca i głos oblubienicy, ustanie turkot żaren i blask pochodni. I cała ta ziemia stanie się rumowiskiem i pustkowiem, i narody te będą poddane królowi babilońskiemu, siedemdziesiąt lat.'' \mbox{(Jr 25, 9-11)}
\end{quote}
\end{document}