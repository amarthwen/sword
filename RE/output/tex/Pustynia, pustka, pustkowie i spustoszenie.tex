\documentclass[10pt,a4paper,oneside]{article}
\usepackage[utf8]{inputenc}
\usepackage{polski}
\usepackage[polish]{babel}
\usepackage[margin=0.5in,bottom=0.75in]{geometry}
\begin{document}
\centerline{\textbf{\MakeUppercase{Pustynia, pustka, pustkowie i spustoszenie}}}
\begin{center}
\textbf{Wersety do studium:} \mbox{(Na 2, 9-11)}, \mbox{(Jr 12, 10-12)}, \mbox{(Iz 34, 9-11)}, \mbox{(Iz 1, 7-9)}, \mbox{(So 2, 9.13)}, \mbox{(Jr 6, 8)}, \mbox{(Iz 51, 3)}, \mbox{(Iz 64, 10)}
\end{center}
\section{Temat główny}
\paragraph{}
\begin{quote}
,,Sama Niniwa podobna jest do stawu, którego wody odpływają z hukiem, a choć wołają: Stójcie, stójcie! jednak nikt się nie odwraca. Złupcie srebro, złupcie złoto, skarbiec jest niewyczerpany, klejnotów wielka obfitość! Pustka, pustkowie i spustoszenie, serce zwątpiałe, dygocące kolana, wszystkie biodra drżą, twarze wszystkich rozpalone!'' \mbox{(Na 2, 9-11)}
\end{quote}
\paragraph{}
\begin{quote}
,,Liczni pasterze zniszczyli moją winnicę, podeptali mój dział, mój dział rozkoszny zamienili w głuchą pustynię. Obrócono go w pustynię, w żałosną dla mnie pustynię. Spustoszony cały kraj, nikt nie wziął tego do serca. Na wszystkie gołe wzniesienia w pustyni wtargnęli rabusie, gdyż miecz Pana pożera od jednego krańca ziemi do drugiego krańca ziemi. Żadne ciało nie ma spokoju.'' \mbox{(Jr 12, 10-12)}
\end{quote}
\paragraph{}
\begin{quote}
,,Toteż potoki Edomu zamienią się w smołę, a jego glina w siarkę, a jego ziemia stanie się smołą gorejącą: Ani w nocy, ani w dzień nie zgaśnie, jego dym wznosić się będzie zawsze; z pokolenia w pokolenie będzie pustynią, po wiek wieków nikt nie będzie nią chodził. Pelikan i jeż zagnieżdżą się w niej, a sowa i kruk będą w niej mieszkać; Pan rozciągnie na niej sznur na spustoszenie i ustawi wagę na zniszczenie.'' \mbox{(Iz 34, 9-11)}
\end{quote}
\paragraph{}
\begin{quote}
,,Wasz kraj pustynią, wasze miasta ogniem spalone, plony waszej ziemi obcy na waszych oczach pożerają; spustoszenie jak po zniszczeniu Sodomy. I pozostała córka syjońska jak szałas w winnicy, jak budka w polu ogórkowym, jak miasto oblężone. Gdyby Pan Zastępów nie był nam pozostawił resztki, bylibyśmy jak Sodoma, podobni do Gomory.'' \mbox{(Iz 1, 7-9)}
\end{quote}
\paragraph{}
\begin{quote}
,,Dlatego jako żyję Ja - mówi Pan Zastępów, Bóg Izraela - Moab stanie się jak Sodoma, a synowie Ammonowi jak Gomora, miejscem bujnych pokrzyw i dołem solnym, i pustynią po wszystkie czasy. Resztki mojego ludu złupią ich, a pozostali z mojego narodu posiądą ich. (\ldots) Wyciągnie swoją rękę przeciwko północy i zniszczy Asyrię, Niniwę obróci w pustkowie, w kraj suchy, jak pustynię.'' \mbox{(So 2, 9.13)}
\end{quote}
\paragraph{}
\begin{quote}
,,Pozwól się ostrzec, Jeruzalemie, aby moja dusza nie odwróciła się od ciebie, abym cię nie obrócił w pustynię, w kraj nie zamieszkany.'' \mbox{(Jr 6, 8)}
\end{quote}
\paragraph{}
\begin{quote}
,,Gdyż Pan pocieszy Syjon, pocieszy wszystkie jego rozpadliny. Uczyni z jego pustkowia Eden, a z jego pustyni ogród Pana, radość i wesele zapanują w nim, pieśń dziękczynna i dźwięk pieśni.'' \mbox{(Iz 51, 3)}
\end{quote}
\paragraph{}
\begin{quote}
,,Święte twoje miasta stały się pustynią, Syjon stał się pustynią, Jeruzalem pustkowiem.'' \mbox{(Iz 64, 10)}
\end{quote}
\end{document}