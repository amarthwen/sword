\documentclass[10pt,a4paper,oneside]{article}
\usepackage[utf8]{inputenc}
\usepackage{polski}
\usepackage[polish]{babel}
\usepackage[margin=0.5in,bottom=0.75in]{geometry}
\begin{document}
\centerline{\textbf{\MakeUppercase{Obrazy człowieka w Piśmie Świętym - naczynie gliniane}}}
\begin{center}
\textbf{Wersety do studium:} \mbox{(2Kor 4, 7)}, \mbox{(Jr 18, 1-6)}, \mbox{(Rz 9, 20-24)}, \mbox{(Iz 45, 9)}
\end{center}
\section{Temat główny}
\paragraph{}
\begin{quote}
,,Mamy zaś ten skarb w naczyniach glinianych, aby się okazało, że moc, która wszystko przewyższa, jest z Boga, a nie z nas.'' \mbox{(2 Kor 4, 7)}
\end{quote}
\paragraph{}
\begin{quote}
,,Słowo, które doszło Jeremiasza od Pana tej treści: Wstań i zajdź do domu garncarza, a tam objawię ci moje słowa! I wstąpiłem do domu garncarza, a oto on pracował w swoim warsztacie. A gdy naczynie, które robił ręcznie z gliny, nie udało się - wtedy zaczął z niej robić inne naczynie, jak garncarzowi wydawało się, że powinno być zrobione. I doszło mnie słowo Pana tej treści: Czy nie mogę postąpić z wami, domu Izraela, jak garncarz? - mówi Pan. Oto jak glina w ręku garncarza, tak wy jesteście w moim ręku, domu Izraela!'' \mbox{(Jr 18, 1-6)}
\end{quote}
\paragraph{}
\begin{quote}
,,O człowiecze! Kimże ty jesteś, że wdajesz się w spór z Bogiem? Czy powie twór do twórcy: Czemuś mnie takim uczynił? Albo czy garncarz nie ma władzy nad gliną, żeby z tej samej bryły ulepić jedno naczynie kosztowne, a drugie pospolite? A cóż, jeśli Bóg, chcąc okazać gniew i objawić moc swoją, znosił w wielkiej cierpliwości naczynia gniewu przeznaczone na zagładę, A uczynił tak, aby objawić bogactwo chwały swojej nad naczyniami zmiłowania, które uprzednio przygotował ku chwale, Takimi naczyniami jesteśmy i my, których powołał, nie tylko z Żydów, ale i z pogan,'' \mbox{(Rz 9, 20-24)}
\end{quote}
\paragraph{}
\begin{quote}
,,Biada temu, kto się spiera ze swoim stwórcą, skorupka wśród glinianych skorupek! Czy glina może powiedzieć do tego, kto ją formuje: Co robisz? albo dzieło do swojego mistrza: On nie ma rąk?'' \mbox{(Iz 45, 9)}
\end{quote}
\end{document}