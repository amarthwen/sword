\documentclass[10pt,a4paper,oneside]{article}
\usepackage[utf8]{inputenc}
\usepackage{polski}
\usepackage[polish]{babel}
\usepackage[margin=0.5in,bottom=0.75in]{geometry}
\begin{document}
\centerline{\textbf{\MakeUppercase{Obrazy człowieka w Piśmie Świętym}}}
\begin{center}
\textbf{Wersety do studium:} \mbox{(2Kor 4, 7)}, \mbox{(Jr 13, 12-14)}, \mbox{(Jr 18, 1-6)}, \mbox{(Jr 19, 1-6.10-13)}, \mbox{(Rz 9, 20-24)}, \mbox{(Iz 45, 9)}, \mbox{(Iz 29, 16)}, \mbox{(Iz 30, 13-14)}, \mbox{(Iz 64, 8)}
\end{center}
\section{Temat główny}
\paragraph{}
\begin{quote}
,,Mamy zaś ten skarb w naczyniach glinianych, aby się okazało, że moc, która wszystko przewyższa, jest z Boga, a nie z nas.'' \mbox{(2 Kor 4, 7)}
\end{quote}
\paragraph{}
\begin{quote}
,,Powiedz im więc to słowo: Tak mówi Pan, Bóg Izraela: Każdy dzban napełnia się winem. A gdy ci powiedzą; Wiemy dobrze, że każdy dzban napełnia się winem, Wtedy im powiedz: Tak mówi Pan: Oto Ja napełnię wszystkich mieszkańców tej ziemi i królów, którzy zasiadają na tronie Dawida, i kapłanów, i proroków, i wszystkich mieszkańców Jeruzalemu pijackim odurzeniem I roztrącę jednego o drugiego, i nie będę oszczędzał ani ojców, ani synów - mówi Pan - nie będę żałował, i nie będę się litował, lecz ich zniszczę.'' \mbox{(Jr 13, 12-14)}
\end{quote}
\paragraph{}
\begin{quote}
,,Słowo, które doszło Jeremiasza od Pana tej treści: Wstań i zajdź do domu garncarza, a tam objawię ci moje słowa! I wstąpiłem do domu garncarza, a oto on pracował w swoim warsztacie. A gdy naczynie, które robił ręcznie z gliny, nie udało się - wtedy zaczął z niej robić inne naczynie, jak garncarzowi wydawało się, że powinno być zrobione. I doszło mnie słowo Pana tej treści: Czy nie mogę postąpić z wami, domu Izraela, jak garncarz? - mówi Pan. Oto jak glina w ręku garncarza, tak wy jesteście w moim ręku, domu Izraela!'' \mbox{(Jr 18, 1-6)}
\end{quote}
\paragraph{}
\begin{quote}
,,Tak mówi Pan: Idź i kup u garncarza gliniany dzban, potem weź z sobą kilku ze starszych ludu i ze starszych kapłanów, A udaj się do Doliny Ben-Hinnoma u wejścia do Bramy Skorup i zwiastuj tam słowa, które ci oznajmię, I powiedz: Słuchajcie słowa Pana, królowie judzcy i mieszkańcy Jeruzalemu! Tak mówi Pan Zastępów, Bóg Izraela: Oto ja sprowadzę nieszczęście na to miejsce, tak że każdemu, kto o nim usłyszy, zadzwoni w uszach. Ponieważ opuścili mnie i nie do poznania odmienili to miejsce, składając na nim ofiary cudzym bogom, których nie znali ani oni, ani ich ojcowie, ani królowie judzcy, i napełnili to miejsce krwią niewinnych, I urządzili miejsce ofiarnicze dla Baala, aby swoich synów palić w ogniu całopalenia dla Baala, czego nie nakazałem, o czym nie mówiłem ani też nie myślałem. Dlatego oto idą dni, mówi Pan, że tego miejsca już nie będą nazywali Tofet ani Doliną Ben-Hinnoma, lecz Doliną Mordu. (\ldots) Potem rozbij ten dzban na oczach mężów, którzy pójdą z tobą I powiedz im: Tak mówi Pan Zastępów: Rozbiję ten lud i to miasto tak, jak się rozbija naczynie garncarskie, którego już nie da się naprawić, a będą grzebali w Tofet z powodu braku miejsca na grzebanie. Tak postąpię z tym miejscem - mówi Pan - i z jego mieszkańcami; to miasto uczynię podobnym do Tofet, I będą domy Jeruzalemu i domy królów judzkich nieczyste jak miejsce Tofet, wszystkie domy, na których dachach spalali kadzidła wszystkim zastępom niebieskim i lali ofiary z płynów cudzym bogom.'' \mbox{(Jr 19, 1-6.10-13)}
\end{quote}
\paragraph{}
\begin{quote}
,,O człowiecze! Kimże ty jesteś, że wdajesz się w spór z Bogiem? Czy powie twór do twórcy: Czemuś mnie takim uczynił? Albo czy garncarz nie ma władzy nad gliną, żeby z tej samej bryły ulepić jedno naczynie kosztowne, a drugie pospolite? A cóż, jeśli Bóg, chcąc okazać gniew i objawić moc swoją, znosił w wielkiej cierpliwości naczynia gniewu przeznaczone na zagładę, A uczynił tak, aby objawić bogactwo chwały swojej nad naczyniami zmiłowania, które uprzednio przygotował ku chwale, Takimi naczyniami jesteśmy i my, których powołał, nie tylko z Żydów, ale i z pogan,'' \mbox{(Rz 9, 20-24)}
\end{quote}
\paragraph{}
\begin{quote}
,,Biada temu, kto się spiera ze swoim stwórcą, skorupka wśród glinianych skorupek! Czy glina może powiedzieć do tego, kto ją formuje: Co robisz? albo dzieło do swojego mistrza: On nie ma rąk?'' \mbox{(Iz 45, 9)}
\end{quote}
\paragraph{}
\begin{quote}
,,O, jak przewrotni jesteście! Tak jak gdyby można garncarza stawiać na równi z gliną! Jak gdyby dzieło mogło mówić o swoim twórcy: Nie on mnie stworzył, a garnek mówił o garncarzu: On nic nie umie!'' \mbox{(Iz 29, 16)}
\end{quote}
\paragraph{}
\begin{quote}
,,Dlatego ta wina stanie się dla was jak rysa grożąca zawaleniem, występująca w wysokim murze, który nagle, w okamgnieniu się rozpada, A z jego rozpadem jest tak, jak z rozbitym dzbanem glinianym, stłuczonym bezlitośnie, tak że wśród jego czerepów nie można znaleźć ani skorupki, aby nabrać ognia z ogniska lub zaczerpnąć wody z kałuży.'' \mbox{(Iz 30, 13-14)}
\end{quote}
\paragraph{}
\begin{quote}
,,Lecz teraz, Panie, Ty jesteś naszym Ojcem, my jesteśmy gliną, a Ty naszym Stwórcą i wszyscyśmy dziełem twoich rąk!'' \mbox{(Iz 64, 8)}
\end{quote}
\end{document}