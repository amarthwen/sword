\documentclass[10pt,a4paper,oneside]{article}
\usepackage[utf8]{inputenc}
\usepackage{polski}
\usepackage[polish]{babel}
\usepackage[margin=0.5in,bottom=0.75in]{geometry}
\begin{document}
\centerline{\textbf{\MakeUppercase{Klątwa}}}
\begin{center}
\textbf{Wersety do studium:} 
\mbox{(Iz 24, 5-6)}, \mbox{(Iz 34, 1-5)}, \mbox{(Ml 3, 9)}
\end{center}
\section{Temat główny}
\paragraph{}
\begin{quote}
,,Ziemia jest splugawiona pod swoimi mieszkańcami, gdyż przestąpili prawa, wykroczyli przeciwko przykazaniom, zerwali odwieczne przymierze. Dlatego klątwa pożera ziemię i jej mieszkańcy muszą odpokutować swoje winy; dlatego przerzedzają się mieszkańcy ziemi i niewielu ludzi pozostaje.'' \mbox{(Iz 24, 5-6)}
\end{quote}
\paragraph{}
\begin{quote}
,,Zbliżcie się, narody, aby słuchać, i wy, ludy, słuchajcie uważnie! Niech słucha ziemia i to, co ją napełnia, okrąg ziemi i wszystko, co z niej wyrasta! Gdyż rozgniewał się Pan na wszystkie narody, jest oburzony na całe ich wojsko, obłożył je klątwą, przeznaczył je na rzeź. Ich zabici nie będą pochowani, a z trupów będzie się unosił smród; Góry rozmiękną od ich krwi. I wszystkie pagórki rozpłyną się. Niebiosa zwiną się jak zwój księgi i wszystkie ich zastępy opadną, jak opada liść z krzewu winnego i jak opada zwiędły liść z drzewa figowego. Gdyż mój miecz zwisa z nieba, oto spada na Edom i na lud obłożony przeze mnie klątwą.'' \mbox{(Iz 34, 1-5)}
\end{quote}
\paragraph{}
\begin{quote}
,,Jesteście obłożeni klątwą, ponieważ mnie oszukujecie, wy, cały naród.'' \mbox{(Ml 3, 9)}
\end{quote}
\end{document}