\documentclass[10pt,a4paper,oneside]{article}
\usepackage[utf8]{inputenc}
\usepackage{polski}
\usepackage[polish]{babel}
\usepackage[margin=0.5in,bottom=0.75in]{geometry}
\begin{document}
\centerline{\textbf{\MakeUppercase{Upadłe Anioły}}}
\begin{center}
\textbf{Wersety do studium:} \mbox{(Ef 6, 10-12)}, \mbox{(Dn 11, 33)}, \mbox{(Jr 12, 12)}, \mbox{(Iz 24, 16)}, \mbox{(Ab 1, 5-6)}, \mbox{(Iz 33, 1)}, \mbox{(Iz 1, 21-23)}, \mbox{(Ha 1, 9-11)}, \mbox{(Jr 30, 15-16)}, \mbox{(Ez 7, 20-24)}, \mbox{(Jr 49, 9)}, \mbox{(Jr 18, 22)}, \mbox{(1Krl 22, 18-22)}, \mbox{(Mi 2, 11)}, \mbox{(Oz 4, 12; 5, 4)}, \mbox{(Mk 9, 25-27)}
\end{center}
\section{Temat główny}
\paragraph{}
\begin{quote}
,,W końcu, bracia moi, umacniajcie się w Panu i w potężnej mocy jego. Przywdziejcie całą zbroję Bożą, abyście mogli ostać się przed zasadzkami diabelskimi. Gdyż bój toczymy nie z krwią i z ciałem, lecz z nadziemskimi władzami, ze zwierzchnościami, z władcami tego świata ciemności, ze złymi duchami w okręgach niebieskich.'' \mbox{(Ef 6, 10-12)}
\end{quote}
\subsection{Rabusie/złodzieje}
\paragraph{}
\begin{quote}
,,A roztropni wśród ludu doprowadzą wielu do właściwego poznania; lecz przez pewien czas padać będą od miecza i ognia, od uprowadzenia i rabunku.'' \mbox{(Dn 11, 33)}
\end{quote}
\paragraph{}
\begin{quote}
,,Na wszystkie gołe wzniesienia w pustyni wtargnęli rabusie, gdyż miecz Pana pożera od jednego krańca ziemi do drugiego krańca ziemi. Żadne ciało nie ma spokoju.'' \mbox{(Jr 12, 12)}
\end{quote}
\paragraph{}
\begin{quote}
,,Od krańca ziemi słyszeliśmy pieśni pochwalne: Chwała Sprawiedliwemu! Lecz ja rzekłem: Zginąłem! Zginąłem! Biada mi! Rabusie rabują, drapieżnie rabują rabusie!'' \mbox{(Iz 24, 16)}
\end{quote}
\paragraph{}
\begin{quote}
,,Gdy złodzieje wtargną do ciebie albo nocni rabusie jakże będziesz splądrowany! Czy nie będą kradli do woli? Gdy zbieracze winogron przyjdą do ciebie, czy pozostawią choć jedno grono? Jakże ogołocony jest Ezaw; przetrząśnięte są jego skarby ukryte.'' \mbox{(Ab 1, 5-6)}
\end{quote}
\paragraph{}
\begin{quote}
,,Biada tobie, pustoszycielu, sam jeszcze nie spustoszony, i tobie, rabusiu, sam nie obrabowany! Gdy przestaniesz pustoszyć, będziesz spustoszony; gdy skończysz rabunek, obrabują ciebie.'' \mbox{(Iz 33, 1)}
\end{quote}
\paragraph{}
\begin{quote}
,,Jakąż nierządnicą stało się to miasto wierne, niegdyś pełne praworządności, sprawiedliwość w nim mieszkała, a teraz mordercy! Twoje srebro obróciło się w żużel, twoje wino zmieszane z wodą. Twoi przewodnicy są buntownikami i wspólnikami złodziei, wszyscy lubią łapówki i gonią za darami, nie wymierzają sprawiedliwości sierocie, a sprawa wdów nie dochodzi przed nich.'' \mbox{(Iz 1, 21-23)}
\end{quote}
\paragraph{}
\begin{quote}
,,Każdy z nich przychodzi, aby grabić, ich twarze są zwrócone na wschód; biorą jeńców licznych jak piasek. Szydzą z królów, śmieją się z książąt, śmieją się z każdej twierdzy; sypią dookoła wały i zdobywają ją. Potem przechodzą jak burza i mkną dalej, jak ten, kto swoją siłę uważa za boga!'' \mbox{(Ha 1, 9-11)}
\end{quote}
\paragraph{}
\begin{quote}
,,Czemu krzyczysz z powodu swojej rany, że dotkliwy jest twój ból? Za wielką twoją winę ci to uczyniłem, za to, że liczne są twoje grzechy. Dlatego wszyscy, którzy cię pożerali, będą pożarci, a wszyscy twoi ciemięzcy pójdą do niewoli. Ci, którzy cię grabią, będą ograbieni, a wszystkich twoich łupieżców wydam na łup.'' \mbox{(Jr 30, 15-16)}
\end{quote}
\paragraph{}
\begin{quote}
,,Z kosztownej swojej ozdoby uczynili pożywkę swej pychy i nią powodowani robili swoje obrzydliwe posągi, swoje ohydy. Dlatego przemienię je dla nich w śmiecie. I wydam je w ręce wrogów na łup i jako zdobycz dla bezbożnych w kraju, i ci je zbezczeszczą. Odwrócę od nich swoje oblicze, a wtedy rabusie zbezczeszczą mój skarb, wtargną do niego i zbezczeszczą go. Przygotuj pęta, gdyż ziemia pełna jest krwawych wyroków, a miasto pełne jest gwałtu. Sprowadzę najgorsze z narodów, aby opanowali ich domy i położę kres ich dumnej potędze, i będą zbezczeszczone ich święte przybytki.'' \mbox{(Ez 7, 20-24)}
\end{quote}
\paragraph{}
\begin{quote}
,,Gdy winogrodnicy przyjdą do ciebie, to nic po sobie nie pozostawią, gdy złodzieje w nocy - to wyrządzą szkodę, jaką tylko zechcą.'' \mbox{(Jr 49, 9)}
\end{quote}
\paragraph{}
\begin{quote}
,,Niech z ich domów rozlega się krzyk, gdy znienacka sprowadzisz na nich zgraję rabusiów, gdyż wykopali dół, aby mnie pochwycić, a pułapkę skrycie zastawili na moje nogi!'' \mbox{(Jr 18, 22)}
\end{quote}
\subsection{Mordercy}
\subsection{Grabieżcy}
\subsection{Włamywacze}
\subsection{Różne rodzaje duchów}
\subsubsection{Duch kłamstwa}
\paragraph{}
\begin{quote}
,,Wtedy król izraelski rzekł do Jehoszafata: Czy nie mówiłem ci, że ten nie zwiastuje mi nic dobrego, a tylko złe? On zaś rzekł: Słuchaj przeto słowa Pańskiego: Widziałem Pana siedzącego na swoim tronie, a cały zastęp niebieski stał przy nim, po jego prawicy i po lewicy. A Pan rzekł: Kto zwiedzie Achaba, aby wyruszył i poległ w Ramot Gileadzkim? I jeden mówił to, a drugi owo. Wtedy wystąpił Duch i stanął przed Panem, i rzekł: Ja go zwiodę. A Pan rzekł do niego: W jaki sposób? A on odpowiedział: Wyjdę i stanę się duchem kłamliwym w ustach wszystkich jego proroków. Wtedy On rzekł: Tak, ty go zwiedziesz, ty to potrafisz. Idź więc i uczyń tak!'' \mbox{(1 Krl 22, 18-22)}
\end{quote}
\paragraph{}
\begin{quote}
,,Gdyby ktoś przyszedł i w duchu kłamstwa i fałszu mówił: Będę wam prorokował o winie i trunku - to byłby kaznodzieja dla tego ludu!'' \mbox{(Mi 2, 11)}
\end{quote}
\subsubsection{Duch wszeteczeństwa}
\paragraph{}
\begin{quote}
,,Mój lud radzi się swojego drewna, a jego kij daje mu wyrocznię, gdyż duch wszeteczeństwa ich omamił, a cudzołożąc odstąpili od swojego Boga. (\ldots) Ich uczynki nie pozwalają im zawrócić do swojego Boga, gdyż duch wszeteczeństwa jest w ich sercu, tak że nie znają Pana.'' \mbox{(Oz 4, 12; 5, 4)}
\end{quote}
\subsubsection{Duch głuchy i niemy}
\paragraph{}
\begin{quote}
,,A Jezus, widząc, że tłum się zbiega, zgromił ducha nieczystego i rzekł mu: Duchu niemy i głuchy! Nakazuję ci: Wyjdź z niego i już nigdy do niego nie wracaj. I krzyknął, i szarpnął nim gwałtownie, po czym wyszedł; a chłopiec wyglądał jak martwy, tak iż wielu mówiło, że umarł. A Jezus ujął go za rękę i podniósł go, a on powstał.'' \mbox{(Mk 9, 25-27)}
\end{quote}
\end{document}