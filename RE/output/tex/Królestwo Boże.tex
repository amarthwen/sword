\documentclass[10pt,a4paper,oneside]{article}
\usepackage[utf8]{inputenc}
\usepackage{polski}
\usepackage[polish]{babel}
\usepackage[margin=0.5in,bottom=0.75in]{geometry}
\begin{document}
\centerline{\textbf{\MakeUppercase{Królestwo Boże}}}
\begin{center}
\textbf{Wersety do studium:} 
\mbox{(Mt 22, 2-14)}, \mbox{(Mt 13, 31-32)}, \mbox{(Mt 13, 33)}, \mbox{(Mt 13, 44)}, \mbox{(Mt 13, 45-46)}, \mbox{(Mt 13, 47-50)}, \mbox{(Mt 13, 24-30)}
\end{center}
\section{Temat główny}
\paragraph{}
\begin{quote}
,,Podobne jest Królestwo Niebios do pewnego króla, który sprawił wesele swemu synowi. I posłał swe sługi, aby wezwali zaproszonych na wesele, ale ci nie chcieli przyjść. Znowu posłał inne sługi, mówiąc: Powiedzcie zaproszonym: Oto ucztę moją przygotowałem, woły moje i bydło tuczne pobito, i wszystko jest gotowe, pójdźcie na wesele. Ale oni, nie dbając o to, odeszli, jeden do własnej roli, drugi do swego handlu. A pozostali, pochwyciwszy jego sługi, znieważyli i pozabijali ich. I rozgniewał się król, a wysławszy swe wojska, wytracił owych morderców, i miasto ich spalił. Wtedy rzecze sługom swoim: Wesele wprawdzie jest gotowe, ale zaproszeni nie byli godni. Idźcie przeto na rozstajne drogi, a kogokolwiek spotkacie, zaproście na wesele. Słudzy ci wyszli na drogi, sprowadzili wszystkich, których napotkali: złach i dobrych, i sala weselna zapełniła się gośćmi. A gdy wszedł król, aby przypatrzeć się gościom, ujrzał tam człowieka nie odzianego w szatę weselną. I rzecze do niego: Przyjacielu, jak wszedłeś tutaj, nie mając szaty weselnej? A on oniemiał. Wtedy król rzekł sługom: Zwiążcie mu nogi i ręce i wyrzućcie go do ciemności zewnętrznej; tam będzie płacz i zgrzytanie zębów. Albowiem wielu jest wezwanych, ale mało wybranych.'' \mbox{(Mt 22, 2-14)}
\end{quote}
\paragraph{}
\begin{quote}
,,Inne podobieństwo podał im, mówiąc: Podobne jest Królestwo Niebios do ziarnka gorczycznego, który wziąwszy człowiek, zasiał na roli swojej. Jest ono, co prawda, najmniejsze ze wszystkich nasion, ale kiedy urośnie, jest największe ze wszystkich jarzyn, i staje się drzewem, tak iż przylatują ptaki niebieskie i gnieżdżą się w gałęziach jego.'' \mbox{(Mt 13, 31-32)}
\end{quote}
\paragraph{}
\begin{quote}
,,Inne podobieństwo powiedział im: Podobne jest Królestwo Niebios do kwasu, który wzięła niewiasta i rozczyniła w trzech miarach mąki, aż się wszystko zakwasiło.'' \mbox{(Mt 13, 33)}
\end{quote}
\paragraph{}
\begin{quote}
,,Podobne jest Królestwo Niebios do ukrytego w roli skarbu, który człowiek znalazł, ukrył i uradowany odchodzi, i sprzedaje wszystko, co ma, i kupuje oną rolę.'' \mbox{(Mt 13, 44)}
\end{quote}
\paragraph{}
\begin{quote}
,,Dalej podobne jest Królestwo Niebios do kupca, szukającego pięknych pereł, Który, gdy znalazł jedną perłę drogocenną, poszedł, sprzedał wszystko, co miał, i kupił ją.'' \mbox{(Mt 13, 45-46)}
\end{quote}
\paragraph{}
\begin{quote}
,,Dalej podobne jest Królestwo Niebios do sieci, zapuszczonej w morze i zagarniającej ryby wszelkiego rodzaju, Którą, gdy była pełna, wyciągnęli na brzeg, a usiadłszy dobre wybrali do naczyń, a złe wyrzucili. Tak będzie przy końcu świata; wyjdą aniołowie i wyłączą złych spośród sprawiedliwych, I wrzucą ich w piec ognisty; tam będzie płacz i zgrzytanie zębów.'' \mbox{(Mt 13, 47-50)}
\end{quote}
\paragraph{}
\begin{quote}
,,Inne podobieństwo podał im, mówiąc: Podobne jest Królestwo Niebios do człowieka, który posiał dobre nasienie na swojej roli. A kiedy ludzie spali, przyszedł jego nieprzyjaciel i nasiał kąkolu między pszenicę, i odszedł. A gdy zboże podrosło i wydało owoc, wtedy się pokazał i kąkol. Przyszli więc słudzy gospodarza i powiedzieli mu: Panie, czy nie posiałeś dobrego nasienia na swojej roli? Skąd więc ma ona kąkol? A on im rzekł: To nieprzyjaciel uczynił. A słudzy mówią do niego: Czy chcesz więc, abyśmy poszli i wybrali go? A on odpowiada: Nie! Abyście czasem wybierając kąkol, nie powyrywali wraz z nim i pszenicy. Pozwólcie obydwom róść razem aż do żniwa. A w czasie żniwa powiem żeńcom: Zbierzcie najpierw kąkol i powiążcie go w snopki na spalenie, a pszenicę zwieźcie do mojej stodoły.'' \mbox{(Mt 13, 24-30)}
\end{quote}
\end{document}