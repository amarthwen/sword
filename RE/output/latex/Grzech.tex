\documentclass[10pt,a4paper,oneside]{article}
\usepackage[utf8]{inputenc}
\usepackage{polski}
\usepackage[polish]{babel}
\usepackage[margin=0.5in,bottom=0.75in]{geometry}
\begin{document}
\centerline{\textbf{\MakeUppercase{Grzech}}}
\begin{center}
\textbf{Wersety do studium:} (Rz~7,~7-20), (Ga~5,~16-17), (1~Moj~4,~3-7), (Ef~6,~10-18), (1~Kor~15,~53-56)
\end{center}
\begin{quote}
,,Cóż więc powiemy? Że zakon to grzech? Przenigdy! Przecież nie poznałbym grzechu, gdyby nie zakon; wszak i o pożądliwości nie wiedziałbym, gdyby zakon nie mówił: Nie pożądaj! Lecz grzech przez przykazanie otrzymał bodziec i wzbudził we mnie wszelką pożądliwość, bo bez zakonu grzech jest martwy. I ja żyłem niegdyś bez zakonu, lecz gdy przyszło przykazanie, grzech ożył. A ja umarłem i okazało się, że to przykazanie, które miało mi być ku żywotowi, było ku śmierci. Albowiem grzech otrzymawszy podnietę przez przykazanie, zwiódł mnie i przez nie mnie zabił. Tak więc zakon jest święty i przykazanie jest święte i sprawiedliwe, i dobre. Czy zatem to, co dobre, stało się dla mnie śmiercią? Przenigdy! To właśnie grzech, żeby się okazać grzechem, posłużył się rzeczą dobrą, by spowodować moją śmierć, aby grzech przez przykazanie okazał ogrom swojej grzeszności. Wiemy bowiem, że zakon jest duchowy, ja zaś jestem cielesny, zaprzedany grzechowi. Albowiem nie rozeznaję się w tym, co czynię; gdyż nie to czynię, co chcę, ale czego nienawidzę, to czynię. A jeśli to czynię, czego nie chcę, zgadzam się z tym, że zakon jest dobry. Ale wtedy czynię to już nie ja, lecz grzech, który mieszka we mnie. Wiem tedy, że nie mieszka we mnie, to jest w ciele moim, dobro; mam bowiem zawsze dobrą wolę, ale wykonania tego, co dobre, brak; Albowiem nie czynię dobrego, które chcę, tylko złe, którego nie chcę, to czynię. A jeśli czynię to, czego nie chcę, już nie ja to czynię, ale grzech, który mieszka we mnie.'' (Rz~7,~7-20)
\end{quote}
\begin{quote}
,,Mówię więc: Według Ducha postępujcie, a nie będziecie pobłażali żądzy cielesnej. Gdyż ciało pożąda przeciwko Duchowi, a Duch przeciwko ciału, a te są sobie przeciwne, abyście nie czynili tego, co chcecie.'' (Ga~5,~16-17)
\end{quote}
\begin{quote}
,,Po niejakim czasie Kain złożył Panu ofiarę z plonów rolnych; Abel także złożył ofiarę z pierworodnych trzody swojej i z tłuszczu ich. A Pan wejrzał na Abla i na jego ofiarę. Ale na Kaina i na jego ofiarę nie wejrzał; wtedy Kain rozgniewał się bardzo i zasępiło się jego oblicze. I rzekł Pan do Kaina: Czemu się gniewasz i czemu zasępiło się twoje oblicze? Wszak byłoby pogodne, gdybyś czynił dobrze, a jeśli nie będziesz czynił dobrze, u drzwi czyha grzech. Kusi cię, lecz ty masz nad nim panować.'' (1~Moj~4,~3-7)
\end{quote}
\begin{quote}
,,W końcu, bracia moi, umacniajcie się w Panu i w potężnej mocy jego. Przywdziejcie całą zbroję Bożą, abyście mogli ostać się przed zasadzkami diabelskimi. Gdyż bój toczymy nie z krwią i z ciałem, lecz z nadziemskimi władzami, ze zwierzchnościami, z władcami tego świata ciemności, ze złymi duchami w okręgach niebieskich. Dlatego weźcie całą zbroję Bożą, abyście mogli stawić opór w dniu złym i, dokonawszy wszystkiego, ostać się. Stójcie tedy, opasawszy biodra swoje prawdą, przywdziawszy pancerz sprawiedliwości I obuwszy nogi, by być gotowymi do zwiastowania ewangelii pokoju, A przede wszystkim, weźcie tarczę wiary, którą będziecie mogli zgasić wszystkie ogniste pociski złego; Weźcie też przyłbicę zbawienia i miecz Ducha, którym jest Słowo Boże. W każdej modlitwie i prośbie zanoście o każdym czasie modły w Duchu i tak czuwajcie z całą wytrwałością i błaganiem za wszystkich świętych'' (Ef~6,~10-18)
\end{quote}
\begin{quote}
,,Albowiem to, co skażone, musi przyoblec się w to, co nieskażone, a to, co śmiertelne, musi przyoblec się w nieśmiertelność. A gdy to, co skażone, przyoblecze się w to, co nieskażone, i to, co śmiertelne, przyoblecze się w nieśmiertelność, wtedy wypełni się słowo napisane: Pochłonięta jest śmierć w zwycięstwie! Gdzież jest, o śmierci, zwycięstwo twoje? Gdzież jest, o śmierci, żądło twoje? A żądłem śmierci jest grzech, a mocą grzechu jest zakon;'' (1~Kor~15,~53-56)
\end{quote}
\end{document}