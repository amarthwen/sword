\documentclass[10pt,a4paper,oneside]{article}
\usepackage[utf8]{inputenc}
\usepackage{polski}
\usepackage[polish]{babel}
\usepackage[margin=0.5in,bottom=0.75in]{geometry}
\begin{document}
\centerline{\textbf{\MakeUppercase{Odjęcia stałej ofiary codziennej i postawienie ohydy spustoszenia}}}
\begin{center}
\textbf{Wersety do studium:} \mbox{(Dn 8, 11-14)}, \mbox{(Dn 11, 31)}, \mbox{(Dn 12, 11-12)}, \mbox{(Jl 1, 4-10)}, \mbox{(Obj 9, 1-8)}, \mbox{(Am 4, 9)}, \mbox{(Mt 24, 15-16.21-22.29-30)}, \mbox{(Jl 2, 12-14)}, \mbox{(1 Kor 10, 19-21)}, \mbox{(1 Kor 6, 15-20)}, \mbox{(Jl 1, 1-20)}, \mbox{(Iz 6, 8-13)}, \mbox{(Iz 24, 3)}, \mbox{(Jr 16, 18)}, \mbox{(Iz 44, 9-20)}, \mbox{(Jr 7, 30)}, \mbox{(Dn 9, 27)}, \mbox{(Obj 14, 11-12)}, \mbox{(Jr 15, 14)}, \mbox{(Jr 16, 13)}, \mbox{(Jr 17, 4)}, \mbox{(Ez 28, 18)}, \mbox{(1 Kor 10, 19-20)}
\end{center}
\begin{quote}
,,Wmówił w siebie potęgę, jaką ma książę wojsk, tak że odjęta mu została stała codzienna ofiara i zostało zbezczeszczone miejsce jego świątyni. Na codziennej ofierze dopuszczono się przestępstwa; prawda została powalona na ziemię, a cokolwiek czynił, to mu się udawało. I usłyszałem jednego świętego mówiącego, a inny święty rzekł do tego właśnie świętego, który mówił: Jak długo zachowuje ważność widzenie dotyczące stałej codziennej ofiary, przestępstwa pustoszenia i bezczeszczenia świątyni i deptania prześlicznej ziemi? A ten odpowiedział mu: Aż do dwóch tysięcy trzystu wieczorów i poranków, potem świątynia znowu wróci do swojego prawa.'' \mbox{(Dn 8, 11-14)}
\end{quote}
\begin{quote}
,,A wojska wysłane przez niego wystąpią i zbezczeszczą świątynię i twierdzę, zniosą stałą codzienną ofiarę i postawią obrzydliwość spustoszenia.'' \mbox{(Dn 11, 31)}
\end{quote}
\begin{quote}
,,Od czasu zniesienia stałej ofiary codziennej i postawienia obrzydliwości spustoszenia upłynie tysiąc dwieście dziewięćdziesiąt dni. Błogosławiony ten, kto wytrwa i dożyje do tysiąca trzystu trzydziestu pięciu dni.'' \mbox{(Dn 12, 11-12)}
\end{quote}
\begin{quote}
,,Co zostało po gąsienicy, pożarła szarańcza, a co zostało po szarańczy, pożarł konik polny, co zaś zostało po koniku polnym, pożarła larwa. Ocućcie się, pijani i płaczcie, zawodźcie z powodu moszczu, wy wszyscy, którzy pijecie wino, że odjęty jest od waszych ust. Gdyż moją ziemię naszedł lud, mocny i niezliczony. Jego zęby są jak zęby lwa, jego uzębienie jak uzębienie lwicy. Spustoszył moją winnicę, połamał moje drzewa figowe, odarł je doszczętnie z kory i zostawił tak, że zbielały jego gałęzie. Zawodźcie jak panna ubrana we włosiennicę nad narzeczonym z młodych lat. Dom Pana pozbawiono ofiary z pokarmów i płynów; żałobą okryci są kapłani, słudzy ołtarza. Zniszczone jest pole, żałobą okryta jest rola, gdyż zniszczone jest zboże moszcz wysechł, zniknęła oliwa.'' \mbox{(Jl 1, 4-10)}
\end{quote}
\begin{quote}
,,I zatrąbił piąty anioł; i widziałem gwiazdę, która spadła z nieba na ziemię; i dano jej klucz ud studni otchłani. I otwarła studnię otchłani. I wzbił się ze studni dym jakby dym z wielkiego pieca, a słońce i powietrze zaćmiły się od dymu ze studni. A z tego dymu wyszły na ziemię szarańcze, którym dana została moc jaką jest moc skorpionów na ziemi. I powiedziano im, aby nie wyrządzały szkody trawie, ziemi ani żadnym ziołom, ani żadnemu drzewu, a tylko ludziom, którzy nie mają pieczęci Bożej na czołach. I nakazano im, aby nie zabijały ich, lecz dręczyły przez pięć miesięcy; a ból przez nie wywołany był jak ból od ukłucia skorpiona, gdy ukłuje człowieka. A w owe dni będą ludzie szukać śmierci, lecz jej nie znajdą, i będą chcieli umrzeć, ale śmierć omijać ich będzie Z wyglądu szarańcze te podobne były do koni gotowych do boju, a na głowach ich coś jakby złote korony. a twarze ich jakby twarze ludzkie. A włosy miały jak włosy kobiece, a zęby ich były jak u lwów.'' \mbox{(Obj 9, 1-8)}
\end{quote}
\begin{quote}
,,Smagałem was posuchą i rdzą, spustoszyłem wasze ogrody i wasze winnice, i szarańcza pożarła wasze drzewa figowe i oliwne, a jednak nie nawróciliście się do mnie - mówi Pan.'' \mbox{(Am 4, 9)}
\end{quote}
\begin{quote}
,,Gdy więc ujrzycie na miejscu świętym ohydę spustoszenia, którą przepowiedział prorok Daniel - kto czyta, niech uważa - Wtedy ci, co są w Judei, niech uciekają w góry; (...) Wtedy bowiem nastanie wielki ucisk, jakiego nie było od początku świata aż dotąd, i nie będzie. A gdyby nie były skrócone owe dni, nie ocalałaby żadna istota, lecz ze względu na wybranych będą skrócone owe dni. (...) A zaraz po udręce owych dni słońce się zaćmi i księżyc nie zajaśnieje swoim blaskiem, i gwiazdy spadać będą z nieba, i moce niebieskie będą poruszone. I wtedy ukaże się na niebie znak Syna Człowieczego, przychodzącego na obłokach nieba z wielką mocą i chwałą.'' \mbox{(Mt 24, 15-16.21-22.29-30)}
\end{quote}
\begin{quote}
,,Wszakże jeszcze teraz mówi Pan: Nawróćcie się do mnie całym swym sercem, w poście, płaczu i narzekaniu! Rozdzierajcie swoje serca, a nie swoje szaty, i nawróćcie się do Pana, swojego Boga, gdyż On jest łaskawy i miłosierny, nierychły do gniewu i pełen litości, i żal mu karania! Kto wie, może pożałuje i zlituje się, i pozostawi błogosławieństwo, abyście mogli Składać ofiarę z pokarmów i płynów Panu, waszemu Bogu.'' \mbox{(Jl 2, 12-14)}
\end{quote}
\begin{quote}
,,Cóż tedy chcę powiedzieć? Czy to, że mięso składane w ofierze bałwanom, jest czymś więcej niż mięsem? Albo że bożek jest czymś więcej niż bałwanem? Nie, chcę powiedzieć, że to, co składają w ofierze, ofiarują demonom, a nie Bogu; ja zaś nie chcę, abyście mieli społeczność z demonami. Nie możecie pić kielicha Pańskiego i kielicha demonów; nie możecie być uczestnikami stołu Pańskiego i stołu demonów.'' \mbox{(1 Kor 10, 19-21)}
\end{quote}
\begin{quote}
,,Czy nie wiecie, że ciała wasze są członkami Chrystusowymi? Czy mam tedy wziąć członki Chrystusowe i uczynić je członkami wszetecznicy? Przenigdy! Albo czy nie wiecie, że, kto się łączy z wszetecznicą, jest z nią jednym ciałem? Albowiem, mówi Pismo, ci dwoje będą jednym ciałem. Kto zaś łączy się z Panem, jest z nim jednym duchem. Uciekajcie przed wszeteczeństwem. Wszelki grzech, jakiego człowiek się dopuszcza, jest poza ciałem; ale kto się wszeteczeństwa dopuszcza, ten grzeszy przeciwko własnemu ciału. Albo czy nie wiecie, że ciało wasze jest świątynią Ducha Świętego, który jest w was i którego macie od Boga, i że nie należycie też do siebie samych? Drogoście bowiem kupieni. Wysławiajcie tedy Boga w ciele waszym.'' \mbox{(1 Kor 6, 15-20)}
\end{quote}
\begin{quote}
,,Słowo Pana, które doszło Joela, syna Petuela. Słuchajcie tego wy, starcy, nasłuchujcie wy, wszyscy mieszkańcy ziemi! Czy stało się coś takiego za waszych dni lub też za dni waszych ojców? Opowiadajcie to swoim dzieciom, a wasze dzieci niech opowiadają to swoim dzieciom, a ich dzieci następnemu pokoleniu! Co zostało po gąsienicy, pożarła szarańcza, a co zostało po szarańczy, pożarł konik polny, co zaś zostało po koniku polnym, pożarła larwa. Ocućcie się, pijani i płaczcie, zawodźcie z powodu moszczu, wy wszyscy, którzy pijecie wino, że odjęty jest od waszych ust. Gdyż moją ziemię naszedł lud, mocny i niezliczony. Jego zęby są jak zęby lwa, jego uzębienie jak uzębienie lwicy. Spustoszył moją winnicę, połamał moje drzewa figowe, odarł je doszczętnie z kory i zostawił tak, że zbielały jego gałęzie. Zawodźcie jak panna ubrana we włosiennicę nad narzeczonym z młodych lat. Dom Pana pozbawiono ofiary z pokarmów i płynów; żałobą okryci są kapłani, słudzy ołtarza. Zniszczone jest pole, żałobą okryta jest rola, gdyż zniszczone jest zboże moszcz wysechł, zniknęła oliwa. Trwóżcie się, rolnicy, narzekajcie, winiarze, z powodu pszenicy i jęczmienia, gdyż nie ma zbiorów na polu! Uschła winorośl, a drzewo figowe zwiędło; drzewo granatowe, palma i jabłoń, wszystkie drzewa polne uschły, u synów ludzkich znikła radość. Przepaszcie się i narzekajcie, wy kapłani, zawodźcie, wy słudzy ołtarza! Przychodźcie, noc spędzajcie we włosiennicach, wy słudzy mojego Boga, gdyż dom waszego boga pozbawiony jest ofiary z pokarmów i płynów! Ogłoście święty post, zwołajcie zgromadzenie, zbierzcie starszych, wszystkich mieszkańców ziemi do domu Pana, waszego Boga, i głośno wołajcie do Pana! Ach! Cóż to za dzień! Bo bliski jest dzień Pana, a przychodzi jak zagłada od Wszechmocnego. Czyż nie została zniszczona na naszych oczach żywność, a z domu naszego Boga radość i wesele? Ziarna wyschły pod swymi grudami, spichrze są spustoszone, stodoły rozwalone, gdyż zboże uschło. Jakże porykują zwierzęta! Stada bydła się błąkają, gdyż nie ma dla nich pastwisk; nawet stada owiec giną. Do ciebie wołam, Panie: Ogień pożarł pastwiska na stepie, żar słoneczny spalił wszystkie drzewa polne. Nawet dzikie zwierzęta ryczą do ciebie, gdyż wyschły potoki, a ogień pożarł pastwiska na stepie.'' \mbox{(Jl 1, 1-20)}
\end{quote}
\begin{quote}
,,Potem usłyszałem głos Pana, który rzekł: Kogo poślę? I kto tam pójdzie? Tedy odpowiedziałem: Oto jestem, poślij mnie! A On rzekł: Idź i mów do tego ludu: Słuchajcie bacznie, lecz nie rozumiejcie, i patrzcie uważnie, lecz nie poznawajcie! Znieczul serce tego ludu i dotknij jego uszy głuchotą, a jego oczy ślepotą, aby nie widział swoimi oczyma i nie słyszał swoimi uszyma, i nie rozumiał swoim sercem, żeby się nie nawrócił i nie ozdrowiał! I rzekłem: Dopókiż, Panie? A On odpowiedział: Dopóki nie opustoszeją miasta i nie będą bez mieszkańców, domy bez ludzi, a kraj nie stanie się pustynią. Pan uprowadzi ludzi daleko, i w kraju będzie wielkie spustoszenie. A choćby została w nim jeszcze dziesiąta część, ta ponownie będzie zniszczona jak dąb lub jak terebint, z którego przy ścięciu został tylko pień. Pędem świętym jest jego pień.'' \mbox{(Iz 6, 8-13)}
\end{quote}
\begin{quote}
,,Doszczętnie spustoszona i złupiona będzie ziemia, gdyż Pan wypowiedział to słowo.'' \mbox{(Iz 24, 3)}
\end{quote}
\begin{quote}
,,I odpłacę im najpierw w dwójnasób za ich winę i ich grzech, że splamili moją ziemię ścierwem swoich obrzydliwości i swoimi ohydami napełnili moje dziedzictwo.'' \mbox{(Jr 16, 18)}
\end{quote}
\begin{quote}
,,Wytwórcy bałwanów są w ogóle niczym, a ich ulubione wytwory są bez wartości, ich czciciele są ślepi, a oczywista ich niewiedza naraża ich na hańbę. Któż wytwarza bóstwo i odlewa bałwana, który na nic się nie zda? Oto wszyscy jego towarzysze narażają się na wstyd, a rzemieślnicy wszak to tylko ludzie; niech się zbiorą wszyscy i staną, a przelękną się i wszyscy się zawstydzą. Kowal wytwarza je pracując przy żarze węgla, nadaje mu kształt uderzeniami młota i robi go za pomocą swojego ramienia; gdy jest głodny, traci siłę, gdy nie pije wody, omdlewa. Snycerz rozciąga sznur, kreśli zarysy czerwonym ołówkiem, wycina go dłutem, wymierza go cyrklem i wykonuje go na podobieństwo człowieka, jako piękną postać ludzką, która ma być ustawiona w domu. Narąbie sobie cedrów lub bierze cyprys albo dąb i czeka, aż wyrośnie najsilniejsze pośród drzew leśnych, sadzi jodłę, która rośnie dzięki deszczom. Te służą człowiekowi na opał, bierze je, aby się ogrzać, roznieca także ogień, aby napiec chleba. Nadto robi sobie boga i oddaje mu pokłon, czyni z niego bałwana i pada przed nim na kolana. Połowę jego spala w ogniu, przy drugiej jego połowie spożywa mięso, piecze pieczeń i je do syta, nadto ogrzewa się przy tym i mówi: Ej, rozgrzałem się, poczułem ciepło! A z reszty czyni sobie boga, swojego bałwana, przed którym klęka i któremu oddaje pokłon, i do którego się modli, mówiąc: Ratuj mnie, gdyż jesteś moim bogiem! Nie mają poznania ani rozumu, bo zaślepione są ich oczy, tak że nie widzą, a serca zatwardziałe, tak że nie rozumieją. A nikt tego nie rozważa i nikt nie ma tyle poznania i rozumu, aby rzec: Jedną jego połowę spaliłem w ogniu i na jego węglach napiekłem chleba, upiekłem też mięso i najadłem się, a z reszty zrobię ohydę i będę klękał przed drewnianym klocem? Kto się zadaje z popiołem, tego zwodzi omamione serce, tak że nie uratuje swojej duszy ani też nie powie: Czy to nie złuda, czego się trzymam?'' \mbox{(Iz 44, 9-20)}
\end{quote}
\begin{quote}
,,Bo synowie judzcy czynili to, co złe przed moimi oczyma - mówi Pan - postawili w domu, który jest nazwany moim imieniem, swoje obrzydliwości, aby go bezcześcić.'' \mbox{(Jr 7, 30)}
\end{quote}
\begin{quote}
,,I zawrze ścisłe przymierze z wieloma na jeden tydzień, w połowie tygodnia zniesie ofiary krwawe i z pokarmów. A w świątyni stanie obraz obrzydliwości, który sprawi spustoszenie, dopóki nie nadejdzie wyznaczony kres spustoszenia.'' \mbox{(Dn 9, 27)}
\end{quote}
\begin{quote}
,,A dym ich męki unosi się w górę na wieki wieków i nie mają wytchnienia we dnie i w nocy ci, którzy oddają pokłon zwierzęciu i jego posągowi, ani nikt, kto przyjmuje znamię jego imienia. Tu się okaże wytrwanie świętych, którzy przestrzegają przykazań Bożych i wiary Jezusa.'' \mbox{(Obj 14, 11-12)}
\end{quote}
\begin{quote}
,,I sprawię, że będziesz służył twoim wrogom w ziemi, której nie znasz, gdyż ogień rozpalił się w moim gniewie i będzie płonął nad wami.'' \mbox{(Jr 15, 14)}
\end{quote}
\begin{quote}
,,Dlatego też wyrzucę was z tej ziemi do ziemi, której nie znaliście ani wy, ani wasi ojcowie, i tam będziecie służyć cudzym bogom dniem i nocą, ponieważ nie okażę wam zmiłowania.'' \mbox{(Jr 16, 13)}
\end{quote}
\begin{quote}
,,I będziesz musiał wypuścić z ręki twoje dziedzictwo, które ci dałem; i uczynię cię niewolnikiem twoich wrogów w ziemi, której nie znasz, gdyż w moim gniewie rozpalił się ogień, który na wieki będzie płonął.'' \mbox{(Jr 17, 4)}
\end{quote}
\begin{quote}
,,Zbezcześciłeś moją świątynię z powodu mnóstwa swoich win, przy niegodziwym swoim handlu. Dlatego wywiodłem z ciebie ogień i ten cię strawił; obróciłem cię w popiół na ziemi na oczach wszystkich, którzy cię widzieli.'' \mbox{(Ez 28, 18)}
\end{quote}
\begin{quote}
,,Cóż tedy chcę powiedzieć? Czy to, że mięso składane w ofierze bałwanom, jest czymś więcej niż mięsem? Albo że bożek jest czymś więcej niż bałwanem? Nie, chcę powiedzieć, że to, co składają w ofierze, ofiarują demonom, a nie Bogu; ja zaś nie chcę, abyście mieli społeczność z demonami.'' \mbox{(1 Kor 10, 19-20)}
\end{quote}
\end{document}