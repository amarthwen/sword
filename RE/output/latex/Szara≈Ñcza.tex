\documentclass[10pt,a4paper,oneside]{article}
\usepackage[utf8]{inputenc}
\usepackage{polski}
\usepackage[polish]{babel}
\usepackage[margin=0.5in,bottom=0.75in]{geometry}
\begin{document}
\centerline{\textbf{\MakeUppercase{Szarańcza}}}
\begin{center}
\textbf{Wersety do studium:} \mbox{(Obj 9, 1-11)}, \mbox{(Jl 1, 4)}, \mbox{(Am 4, 9)}, \mbox{(Ml 3, 11)}, \mbox{(Jl 2, 25)}, \mbox{(Am 7, 1-3)}, \mbox{(Na 3, 14-17)}, \mbox{(2 Moj 10, 1-20)}
\end{center}
\begin{quote}
,,I zatrąbił piąty anioł; i widziałem gwiazdę, która spadła z nieba na ziemię; i dano jej klucz ud studni otchłani. I otwarła studnię otchłani. I wzbił się ze studni dym jakby dym z wielkiego pieca, a słońce i powietrze zaćmiły się od dymu ze studni. A z tego dymu wyszły na ziemię szarańcze, którym dana została moc jaką jest moc skorpionów na ziemi. I powiedziano im, aby nie wyrządzały szkody trawie, ziemi ani żadnym ziołom, ani żadnemu drzewu, a tylko ludziom, którzy nie mają pieczęci Bożej na czołach. I nakazano im, aby nie zabijały ich, lecz dręczyły przez pięć miesięcy; a ból przez nie wywołany był jak ból od ukłucia skorpiona, gdy ukłuje człowieka. A w owe dni będą ludzie szukać śmierci, lecz jej nie znajdą, i będą chcieli umrzeć, ale śmierć omijać ich będzie Z wyglądu szarańcze te podobne były do koni gotowych do boju, a na głowach ich coś jakby złote korony. a twarze ich jakby twarze ludzkie. A włosy miały jak włosy kobiece, a zęby ich były jak u lwów. Miały też pancerze niby pancerze żelazne, a szum ich skrzydeł jak turkot wozów wojennych i wielu koni pędzących do boju. Miały też ogony podobne do skorpionowych oraz żądła, a w ogonach ich moc wyrządzania ludziom szkody przez pięć miesięcy. I miały nad sobą jako króla anioła otchłani, którego imię brzmi po hebrajsku Abaddon, po grecku zaś imię jego brzmi Apollyon (Niszczyciel).'' \mbox{(Obj 9, 1-11)}
\end{quote}
\begin{quote}
,,Co zostało po gąsienicy, pożarła szarańcza, a co zostało po szarańczy, pożarł konik polny, co zaś zostało po koniku polnym, pożarła larwa.'' \mbox{(Jl 1, 4)}
\end{quote}
\begin{quote}
,,Smagałem was posuchą i rdzą, spustoszyłem wasze ogrody i wasze winnice, i szarańcza pożarła wasze drzewa figowe i oliwne, a jednak nie nawróciliście się do mnie - mówi Pan.'' \mbox{(Am 4, 9)}
\end{quote}
\begin{quote}
,,I zabronię potem szarańczy pożerać wasze plony rolne, wasz winograd zaś w polu nie będzie bez owocu - mówi Pan Zastępów.'' \mbox{(Ml 3, 11)}
\end{quote}
\begin{quote}
,,I wynagrodzę wam szkody lat, których plony pożarła szarańcza, konik polny, larwa i gąsienica, moje wielkie wojsko, które na was wyprawiłem.'' \mbox{(Jl 2, 25)}
\end{quote}
\begin{quote}
,,Takie widzenie dał mi oglądać Wszechmogący Pan: Oto stworzył szarańcze, gdy zaczął odrastać potraw, a był to potraw po kośbie królewskiej. A gdy już dojadała zieleń ziemi, rzekłem: Wszechmogący Panie, przebacz! Jakże ostoi się Jakub, wszak jest tak maleńki? I żałował tego Pan: Nie stanie się - rzekł Pan.'' \mbox{(Am 7, 1-3)}
\end{quote}
\begin{quote}
,,Naczerp sobie wody na czas oblężenia, wzmocnij swoje obwarowania, wejdź w błoto i depcz glinę, weź do rąk formę do cegieł. I tak strawi cię ogień, wytnie cię miecz, pożre cię jak szarańcza. Rozmnażaj się jak szarańcza, rozmnażaj się jak koniki polne! Namnożyłaś swoich handlarzy więcej niż gwiazd na niebie; jak szarańcza rozciąga skrzydełka i odlatuje, tak odlecą. Twoi urzędnicy są jak szarańcza, a twoi pisarze jak roje koników polnych, które leżą na murach w czas chłodu; lecz gdy słońce wschodzi, odlatują i nie wiadomo, gdzie są.'' \mbox{(Na 3, 14-17)}
\end{quote}
\begin{quote}
,,Potem rzekł Pan do Mojżesza: Idź do faraona, gdyż to Ja sam przywiodłem do zatwardziałości serce jego i serce sług jego, aby czynić te moje znaki wśród nich I abyś ty opowiadał dzieciom swoim i wnukom swoim, jak obszedłem się z Egipcjanami, i jakie znaki czyniłem wśród nich, byście poznali, żem Ja Pan. Poszedł więc Mojżesz z Aaronem do faraona i rzekli do niego: Tak mówi Pan, Bóg Hebrajczyków: Jak długo wzbraniać się będziesz, by się przede mną upokorzyć? Wypuść lud mój, aby mi służył! Bo jeżeli będziesz się wzbraniał wypuścić lud mój, to Ja jutro sprowadzę szarańczę na twój kraj. Pokryje ona całą ziemię, tak że nie będzie można zobaczyć ziemi, i pożre resztę tego, co ocalało, co pozostało po gradzie, i obgryzie wszystkie drzewa, które wam rosną na polu. I napełni domy twoje i domy wszystkich sług twoich, i domy wszystkich Egipcjan, czego nie widzieli ojcowie twoi i ojcowie ojców twoich, odkąd żyją na ziemi aż do dnia dzisiejszego. Potem odwrócił się i wyszedł od faraona. A słudzy faraona rzekli do niego: Jak długo będzie nam ten człowiek przynosił nieszczęście? Wypuść tych ludzi, aby służyli Panu, Bogu swemu. Czy jeszcze nie rozumiesz, że Egipt ginie? Sprowadzono więc z powrotem Mojżesza i Aarona do faraona, a on rzekł do nich: Idźcie, służcie Panu, Bogu waszemu. Lecz którzy to mają iść? Mojżesz odpowiedział: Pójdziemy z naszą młodzieżą i z naszymi starcami; pójdziemy z naszymi synami i naszymi córkami, z naszymi trzodami i z naszym bydłem, gdyż mamy obchodzić święto Pana. Wtedy rzekł do nich: Pan niech będzie z wami, jeżeli ja kiedykolwiek wypuszczę was i dzieci wasze. Patrzcie, jak złe macie zamiary. Nie tak! Idźcie wy, mężczyźni, i służcie Panu, skoro tak tego żądacie! I wypędzono ich od faraona. I rzekł Pan do Mojżesza: Wyciągnij rękę swoją nad ziemią egipską, by przyszła szarańcza. Niech spadnie na ziemię egipską i pożre wszelką roślinność ziemi, wszystko, co pozostawił grad. I wyciągnął Mojżesz laskę swoją nad ziemią egipską. A Pan sprowadził wiatr wschodni na kraj i wiał przez cały dzień i całą noc. A gdy nastał poranek, wiatr wschodni przyniósł szarańczę. I szarańcza przyleciała nad całą ziemię egipską, i osiadła w bardzo wielkiej ilości na całym obszarze Egiptu. Nie było przedtem takiej ilości szarańczy ani już nie będzie. Pokryła ona całą powierzchnię ziemi, tak że ziemia pociemniała. I pożarła całą roślinność ziemi i wszelki owoc drzew, który pozostawił grad. Nie pozostała żadna zieleń na drzewach ani żadna roślinność w całej ziemi egipskiej. Wtedy faraon śpiesznie wezwał Mojżesza i Aarona i rzekł: Zgrzeszyłem przeciwko Panu, Bogu waszemu, i przeciwko wam. Przeto przebaczcie mi jeszcze tym razem mój grzech i wstawcie się do Pana, Boga waszego, by przynajmniej tę zagładę oddalił ode mnie. A gdy wyszedł od faraona, wstawił się do Pana. Wtedy odwrócił Pan wiatr, który zaczął dąć bardzo mocno od zachodu. A ten uniósł szarańczę i wrzucił ją do Morza Czerwonego, tak że na całym obszarze Egiptu nie pozostała ani jedna szarańcza. Lecz Pan przywiódł do zatwardziałości serce faraona, tak iż nie wypuścił synów izraelskich.'' \mbox{(2 Moj 10, 1-20)}
\end{quote}
\end{document}