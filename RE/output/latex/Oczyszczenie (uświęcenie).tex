\documentclass[10pt,a4paper,oneside]{article}
\usepackage[utf8]{inputenc}
\usepackage{polski}
\usepackage[polish]{babel}
\usepackage[margin=0.5in,bottom=0.75in]{geometry}
\begin{document}
\centerline{\textbf{\MakeUppercase{Oczyszczenie (uświęcenie)}}}
\begin{center}
\textbf{Wersety do studium:} (1~J~1,~6-10), (5~Moj~32,~43), (Iz~1,~25-26), (Iz~1,~16-18), (Jr~13,~26-27), (Jr~6,~27-30), (Ml~3,~1-4), (Za~13,~8-9), (Ez~22,~17-22)
\end{center}
\begin{quote}
,,Jeśli mówimy, że z nim społeczność mamy, a chodzimy w ciemności, kłamiemy i nie trzymamy się prawdy. Jeśli zaś chodzimy w światłości, jak On sam jest w światłości, społeczność mamy z sobą, i krew Jezusa Chrystusa, Syna jego, oczyszcza nas od wszelkiego grzechu. Jeśli mówimy, że grzechu nie mamy, sami siebie zwodzimy, i prawdy w nas nie ma. Jeśli wyznajemy grzechy swoje, wierny jest Bóg i sprawiedliwy i odpuści nam grzechy, i oczyści nas od wszelkiej nieprawości. Jeśli mówimy, że nie zgrzeszyliśmy, kłamcę z niego robimy i nie ma w nas Słowa jego.'' (1~J~1,~6-10)
\end{quote}
\begin{quote}
,,Wysławiajcie, narody, lud jego, Gdyż pomści krew sług swoich I zemstę wywrze na wrogach swoich, Ale ziemię swoją i lud oczyści z grzechu.'' (5~Moj~32,~43)
\end{quote}
\begin{quote}
,,I zwrócę swoją rękę przeciwko tobie, i wytopię w tyglu twój żużel, i usunę wszystkie twoje przymieszki. I przywrócę ci twoich sędziów jak niegdyś, i twoich radców jak na początku. Potem nazywać cię będą grodem sprawiedliwości, miastem wiernym.'' (Iz~1,~25-26)
\end{quote}
\begin{quote}
,,Obmyjcie się, oczyśćcie się, usuńcie wasze złe uczynki sprzed moich oczu, przestańcie źle czynić! Uczcie się dobrze czynić, przestrzegajcie prawa, brońcie pokrzywdzonego, wymierzajcie sprawiedliwość sierocie, wstawiajcie się za wdową! Chodźcie więc, a będziemy się prawować - mówi Pan! Choć wasze grzechy będą czerwone jak szkarłat, jak śnieg zbieleją; choć będą czerwone jak purpura, staną się białe jak wełna.'' (Iz~1,~16-18)
\end{quote}
\begin{quote}
,,Tak więc Ja sam podniosę poły twojej szaty aż na twoją twarz, aby była widoczna twoja sromota, Twoje cudzołóstwa i lubieżne twoje okrzyki, haniebne twoje wszeteczeństwo! Na wzgórzach i na polu widziałem twoje obrzydliwości. Biada ci, Jeruzalemie, że nie chcesz się oczyścić. Jak długo jeszcze będziesz zwlekać?'' (Jr~13,~26-27)
\end{quote}
\begin{quote}
,,Ustanowiłem cię badaczem wśród mojego ludu, abyś poznał i badał ich postępowanie. Arcybuntownikami są wszyscy, chodzą wokoło i oczerniają; twardzi jak miedź i żelazo, wszyscy są zepsuci. Miech sapie, aby ołów się roztopił w ogniu; lecz na próżno się roztapia i roztapia, gdyż złych nie daje się oddzielić. Nazwie ich srebrem odrzuconym, gdyż Pan ich odrzucił.'' (Jr~6,~27-30)
\end{quote}
\begin{quote}
,,Oto Ja posyłam mojego anioła, aby mi przygotował drogę przede mną. Potem nagle przyjdzie do swej świątyni Pan, którego oczekujecie, to jest anioł przymierza, którego pragniecie. Zaiste, on przyjdzie - mówi Pan Zastępów. Lecz kto będzie mógł znieść dzień jego przyjścia i kto się ostoi, gdy się ukaże? Gdyż jest on jak ogień odlewacza, jak ług foluszników. Usiądzie, aby wytapiać i czyścić srebro. Będzie czyścił synów Lewiego i będzie ich płukał jak złoto i srebro. Potem będą mogli składać Panu ofiary w sprawiedliwości. I miła będzie Panu ofiara Judy i Jeruzalemu jak za dni dawnych, jak w latach minionych.'' (Ml~3,~1-4)
\end{quote}
\begin{quote}
,,I stanie się w całym kraju - mówi Pan: Dwie trzecie zginą i pomrą, a tylko trzecia część pozostanie z nim. Tę trzecią część wrzucę w ogień i będę ją wytapiał, jak się wytapia srebro, będę ją próbował, jak się próbuje złoto. Będzie wzywać mojego imienia i wysłucham ją. Ja powiem: Moim jest ludem, a ona odpowie: Pan jest moim Bogiem.'' (Za~13,~8-9)
\end{quote}
\begin{quote}
,,I doszło mnie słowo Pana tej treści: Synu człowieczy, żużlem stał się dla mnie dom izraelski; wszyscy oni, to tylko brąz i cyna, żelazo i ołów w tyglu, a stali się żużlem. Dlatego mówi Wszechmocny Pan: Ponieważ wy wszyscy staliście się żużlem, oto Ja zgromadzę was w Jeruzalemie. Jak wrzuca się razem srebro i brąz, żelazo, ołów i cynę do tygla i rozpala pod nim ogień, aby je stopić, tak Ja zgromadzę was w moim gniewie i w mojej zapalczywości i wrzucę was razem, i stopię. Zgromadzę was i rozdmucham ogień swojej popędliwości przeciwko wam, abyście byli w nim stopieni. Jak topi się srebro w tyglu, tak wy będziecie w nim roztopieni i poznacie, że Ja wylałem na was swoją zapalczywość.'' (Ez~22,~17-22)
\end{quote}
\end{document}