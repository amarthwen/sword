\documentclass[10pt,a4paper,oneside]{article}
\usepackage[utf8]{inputenc}
\usepackage{polski}
\usepackage[polish]{babel}
\usepackage[margin=0.5in,bottom=0.75in]{geometry}
\begin{document}
\centerline{\textbf{\MakeUppercase{Wieczne przymierze}}}
\begin{center}
\textbf{Wersety do studium:} \mbox{(Ez 21, 28)}, \mbox{(J 14, 16-17)}, \mbox{(Iz 24, 4-5)}
\end{center}
\begin{quote}
,,Lecz jest to w ich oczach wyrocznia złudna, chociaż składali uroczyste przysięgi; ale on przypomni ich winę, aby zostali pojmani.'' \mbox{(Ez 21, 28)}
\end{quote}
\begin{quote}
,,Ja prosić będę Ojca i da wam innego Pocieszyciela, aby był z wami na wieki - Ducha prawdy, którego świat przyjąć nie może, bo go nie widzi i nie zna; wy go znacie, bo przebywa wśród was i w was będzie.'' \mbox{(J 14, 16-17)}
\end{quote}
\begin{quote}
,,Więdnie i obumiera ziemia, marnieje, obumiera świat, marnieją dostojnicy ludu ziemi. Ziemia jest splugawiona pod swoimi mieszkańcami, gdyż przestąpili prawa, wykroczyli przeciwko przykazaniom, zerwali odwieczne przymierze.'' \mbox{(Iz 24, 4-5)}
\end{quote}
\end{document}