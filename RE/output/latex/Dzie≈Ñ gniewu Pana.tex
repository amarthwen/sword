\documentclass[10pt,a4paper,oneside]{article}
\usepackage[utf8]{inputenc}
\usepackage{polski}
\usepackage[polish]{babel}
\usepackage[margin=0.5in,bottom=0.75in]{geometry}
\begin{document}
\centerline{\textbf{\MakeUppercase{Dzień gniewu Pana}}}
\begin{center}
\textbf{Wersety do studium:} \mbox{(So 1, 14-18)}, \mbox{(Iz 13, 2-13)}, \mbox{(Iz 34, 1-10)}, \mbox{(Ha 3, 1-19)}, \mbox{(Jl 2, 1-11)}, \mbox{(So 1, 8-13)}, \mbox{(Iz 24, 1-23)}, \mbox{(Jl 3, 16-26)}, \mbox{(Am 5, 18.20)}, \mbox{(Am 8, 9-10)}, \mbox{(Za 14, 6-7)}, \mbox{(Ml 3, 19-21)}
\end{center}
\begin{quote}
,,Bliski jest wielki dzień Pana, bliski i bardzo szybko nadchodzi. Słuchaj! Dzień Pana jest gorzki! Wtedy nawet i bohater będzie krzyczał. Dzień ów jest dniem gniewu, dniem ucisku i utrapienia, dniem huku i hałasu, dniem ciemności i mroku, dniem obłoków i gęstych chmur, Dniem trąby i okrzyku wojennego przeciwko miastom obronnym i przeciwko basztom wysokim. Wtedy ześlę strach na ludzi, tak iż chodzić będą jak ślepi, gdyż zgrzeszyli przeciwko Panu. Ich krew będzie rozbryzgana niby proch, a ich wnętrzności rozrzucone niby błoto. Ani ich srebro, ani ich złoto nie będzie mogło ich wyratować w dniu gniewu Pana, bo ogień gniewu Pana pochłonie całą ziemię. Doprawdy, koniec straszną zagładę zgotuje wszystkim mieszkańcom ziemi.'' \mbox{(So 1, 14-18)}
\end{quote}
\begin{quote}
,,Na łysej górze zatknijcie sztandar, podnieście na nich głos, machajcie ręką, aby wkroczyli do bram panów! Ja wezwałem moich poświęconych i powołałem moich bohaterów, dumnych z mojej chwały, aby wykonali wyrok mojego sądu. Słuchaj! Na górach wrzawa jakby licznego ludu. Słuchaj! Zgiełk królestw, zgromadzonych narodów. Pan Zastępów dokonuje przeglądu zastępów bojowych. Ciągną z dalekiej ziemi, od krańców nieba, Pan i narzędzia jego grozy, aby zniszczyć całą ziemię. Biadajcie! Bo bliski jest dzień Pana, który nadchodzi jako zagłada od Wszechmocnego. Dlatego opadają wszystkie ręce i truchleje każde serce ludzkie. I są przerażeni; ogarniają ich kurcze i bóle, wiją się z bólu jak rodząca, jeden na drugiego spogląda osłupiały, ich oblicza są rozpłomienione. Oto nadchodzi dzień Pana, okrutny, pełen srogości i płonącego gniewu, aby obrócić ziemię w pustynię, a grzeszników z niej wytępić. Gdyż gwiazdy niebios i ich planety nie dadzą swojego światła, słońce zaćmi się zaraz, gdy wzejdzie, a księżyc nie błyśnie swym światłem. I nawiedzę okręg ziemski za jego złość, a bezbożnych za ich winę; i ukrócę pychę zuchwalców, a wyniosłość tyranów poniżę. Sprawię, że śmiertelnik będzie rzadszy niż szczere złoto, a człowiek niż złoto z Ofiru. Dlatego sprawię, że niebiosa zadrżą, a ziemia ruszy się ze swojego miejsca z powodu srogości Pana Zastępów w dniu płomiennego jego gniewu,'' \mbox{(Iz 13, 2-13)}
\end{quote}
\begin{quote}
,,Zbliżcie się, narody, aby słuchać, i wy, ludy, słuchajcie uważnie! Niech słucha ziemia i to, co ją napełnia, okrąg ziemi i wszystko, co z niej wyrasta! Gdyż rozgniewał się Pan na wszystkie narody, jest oburzony na całe ich wojsko, obłożył je klątwą, przeznaczył je na rzeź. Ich zabici nie będą pochowani, a z trupów będzie się unosił smród; Góry rozmiękną od ich krwi. I wszystkie pagórki rozpłyną się. Niebiosa zwiną się jak zwój księgi i wszystkie ich zastępy opadną, jak opada liść z krzewu winnego i jak opada zwiędły liść z drzewa figowego. Gdyż mój miecz zwisa z nieba, oto spada na Edom i na lud obłożony przeze mnie klątwą. Miecz Pana ocieka krwią, pokryty jest tłuszczem, krwią jagniąt i kozłów, tłuszczem nerek baranów, gdyż krwawą ofiarę urządza Pan w Bosra, wielką rzeź w ziemi edomskiej. I padną wraz z nimi bawoły i cielce z tucznymi wołami; ich ziemia będzie przesiąknięta krwią, a ich proch przesycony tłuszczem. Bo jest to dzień pomsty Pana, rok odwetu za spór z Syjonem. Toteż potoki Edomu zamienią się w smołę, a jego glina w siarkę, a jego ziemia stanie się smołą gorejącą: Ani w nocy, ani w dzień nie zgaśnie, jego dym wznosić się będzie zawsze; z pokolenia w pokolenie będzie pustynią, po wiek wieków nikt nie będzie nią chodził.'' \mbox{(Iz 34, 1-10)}
\end{quote}
\begin{quote}
,,Modlitwa błagalna proroka Habakuka na melodię trenów. Panie! Słyszałem od ciebie wieść, widziałem, Panie twoje dzieło. W najbliższych latach tchnij w nie życie, w najbliższych latach objaw je! W gniewie pomnij na miłosierdzie! Bóg przychodzi z Temanu, Święty z góry Paran. Sela. Jego wspaniałość okrywa niebiosa, a ziemia jest pełna jego chwały. Pod nim jest blask jak światłość, promienie wychodzą z jego rąk i tam jest ukryta jego moc. Przed nim idzie zaraza, a za nim podąża mór. Trzęsie się ziemia, gdy powstaje, gdy patrzy, drżą narody Pękają odwieczne góry, zapadają się prastare pagórki jego drogi są wieczne. Widziałem namioty Kuszanu zagrożone zagładą, rozerwane zasłony ziemi Midianitów. Czy twoja popędliwość, Panie, zwraca się przeciw rzekom, czy gniewasz się na strumienie, czy oburzyłeś się na morze, że siadasz na swoje rumaki, na swoje wozy zwycięskie? Wydobyłeś swój łuk, na swoją cięciwę położyłeś strzałę. Sela. Strumieniami rozdzieliłeś ziemię. Na twój widok drżą góry obłoki spuszczają ulewne deszcze; otchłań morska wydaje swój głos, wysoko podnosi swe ramiona. Słońce i księżyc wstrzymały swoją jasność w świetle twych szybkich strzał, w blasku twojej lśniącej włóczni. W zawziętości kroczysz po ziemi, w gniewie depczesz narody. Wyruszasz na pomoc swojemu ludowi, aby ratować swojego pomazańca. Rozbijasz dom bezbożnego, obnażasz fundamenty aż do skały. Sela. Przebiłeś wodza jego wojska jego własną strzałą, jego przywódcy są rozproszeni jak plewa przez burzę, gdy rozwarli swoje szczęki, aby pochłonąć w skrytości ubogiego. Jego rumaki wpędziłeś w morze, w potoki wielkich wód. Gdy to usłyszałem, struchlało moje ciało na wieść o tym drgnęły moje wargi; lęk śmiertelny przeszył moje członki, a kolana zachwiały się pode mną. Zadrżałem przed dniem utrapienia, który nadchodzi na lud, na lud, który mnie uciska. Zaiste, drzewo figowe nie wydaje owocu, a na winoroślach nie ma gron. Zawodzi drzewo oliwne, a rola nie dostarcza pożywienia. W ogrodzeniu nie ma owiec, a w oborach nie ma bydła. Lecz ja będę radował się w Panu, weselił się w Bogu mojego zbawienia. Wszechmogący Pan jest moją mocą. Sprawia, że moje nogi są chyże jak nogi łań, i pozwala mi kroczyć po wyżynach. Przewodnikowi chóru przy wtórze instrumentów strunowych.'' \mbox{(Ha 3, 1-19)}
\end{quote}
\begin{quote}
,,Zatrąbcie na rogu na Syjonie! Krzyczcie na mojej świętej górze! Niech zadrżą wszyscy mieszkańcy ziemi, gdyż nadchodzi dzień Pana, gdyż jest bliski! Dzień ciemności i mroku, dzień pochmurny i mglisty. Jak zorza poranna kładzie się na góry, tak nadciąga lud wielki i potężny, któremu równego nie było od wieków i po nim już nie będzie aż do lat najdalszych pokoleń. Przed nim ogień płonący, a po nim płomień gorejący. Przed nim kraj jest jak ogród Eden, a po nim jak step pusty. Nikt też przed nim nie ujdzie. Wyglądają jak konie, a biegną jak rumaki. Podskakują po wierzchołkach gór z turkotem wozów wojennych, z trzaskiem płomienia ognia, który pożera ściernisko, podobnie jak potężny lud, gotowy do bitwy. Przed nim ze strachu drżą ludy, wszystkie twarze bledną. Biegną naprzód jak bohaterowie, wdzierają się na mury jak wojownicy; każdy idzie prosto swoją drogą i nie zbacza ze swojej ścieżki Nie wypiera jeden drugiego, każdy idzie swoim torem, prą naprzód wśród pocisków, w ich szeregach nie ma przerw. Uderzają na miasto, szturmem zdobywają mur, wpadają do domów i wdzierają się oknami jak złodzieje. Drży przed nimi ziemia, trzęsie się niebo, słońce i księżyc są zaćmione, a gwiazdy tracą swój blask. I Pan wydaje swój donośny głos przed swoim wojskiem, gdyż bardzo liczne są jego zastępy i potężny jest wykonawca jego rozkazu. Tak! Wielki jest dzień Pana i pełen grozy, któż go przetrwa?'' \mbox{(Jl 2, 1-11)}
\end{quote}
\begin{quote}
,,A w dzień ofiary Pana tak się stanie: Ukarzę książęta i synów królewskich, i wszystkich, którzy ubierają się w strój cudzoziemski, Ukarzę wszystkich, którzy pewnie przekraczają próg pałacu, którzy napełniają dom swojego Pana przemocą i oszustwem. W owym dniu - mówi Pan - słychać będzie od Bramy Rybnej krzyk, od nowego miasta narzekanie, z pagórków wielki trzask. Narzekajcie, mieszkańcy Dzielnicy Dolnej, bo zniszczony będzie cały lud kramarzy, wytępieni wszyscy, którzy odważają srebro! W owym czasie będę przeszukiwał Jeruzalem w świetle pochodni i będę karał mężów, którzy siedzą zdrętwieli nad mętnymi resztkami wina i mówią w swoim sercu: Nie uczyni Pan nic dobrego ani też nic złego. Ich mienie będzie rozgrabione, a ich domy będą spustoszone; gdy odbudują domy, nie będą w nich mieszkać, a gdy nasadzą winnice, nie będą z nich pić wina.'' \mbox{(So 1, 8-13)}
\end{quote}
\begin{quote}
,,Oto Pan spustoszy ziemię i zniszczy ją, sprawi zamieszanie na jej powierzchni i rozproszy jej mieszkańców. Ten sam los co lud spotka i kapłana, sługę i jego pana, służącą i jej panią, kupującego i sprzedającego, pożyczającego i biorącego pożyczkę, lichwiarza i jego dłużnika. Doszczętnie spustoszona i złupiona będzie ziemia, gdyż Pan wypowiedział to słowo. Więdnie i obumiera ziemia, marnieje, obumiera świat, marnieją dostojnicy ludu ziemi. Ziemia jest splugawiona pod swoimi mieszkańcami, gdyż przestąpili prawa, wykroczyli przeciwko przykazaniom, zerwali odwieczne przymierze. Dlatego klątwa pożera ziemię i jej mieszkańcy muszą odpokutować swoje winy; dlatego przerzedzają się mieszkańcy ziemi i niewielu ludzi pozostaje. Moszcz jest w żałobie, latorośl usycha, wzdychają wszyscy wesołkowie. Ustało wesołe bicie w bębny, skończyły się krzyki weselących się, ustały wesołe dźwięki lutni. Nie piją wina przy pieśni, gorzko smakuje mocny trunek tym, którzy go piją. Zburzone jest puste miasto, zamknięty każdy dom, że wejść nie można. Narzekanie na ulicach na brak wina, zniknęła wszelka radość, wesele jest wygnane z ziemi. W mieście pozostało spustoszenie, a brama w gruzy rozbita. Gdyż tak będzie na ziemi wśród ludów, jak jest przy otrząsaniu oliwek, jak przy zbieraniu resztek po winobraniu. Tamci podnoszą swój głos, weselą się nad wyniosłością Pana; Wykrzykujcie radośnie od zachodu, Aż do krańców wschodu oddajcie chwałę Panu, na wyspach morskich imieniu Pana, Boga Izraelskiego! Od krańca ziemi słyszeliśmy pieśni pochwalne: Chwała Sprawiedliwemu! Lecz ja rzekłem: Zginąłem! Zginąłem! Biada mi! Rabusie rabują, drapieżnie rabują rabusie! Groza i przepaść, i pułapka na ciebie, mieszkańcu ziemi! I będzie tak, że kto będzie uciekał przed okrzykiem grozy, wpadnie w przepaść, a kto wyjdzie z przepaści, uwikła się w pułapkę, gdyż upusty w górze się otworzą, a ziemia zadrży w posadach. W bryły rozleci się ziemia, w kawałki rozpadnie się ziemia, zatrzęsie i zachwieje się ziemia. Wkoło zatacza się ziemia jak pijany i kołysze się jak budka nocna. Zaciążyło na niej jej przestępstwo, tak że upadnie i już nie powstanie. I stanie się w owym dniu, że Pan nawiedzi wojsko górne w górze, a królów ziemi na ziemi. I będą zebrani razem jak więźniowie w lochu, i będą zamknięci w więzieniu, i po wielu dniach będą ukarani. I zarumieni się księżyc, a słońce się zawstydzi, gdyż Pan Zastępów będzie królem na górze Syjon i w Jeruzalemie, i przed ich starszymi będzie chwała.'' \mbox{(Iz 24, 1-23)}
\end{quote}
\begin{quote}
,,Przychodźcie śpiesznie, wy, wszystkie okoliczne narody! Zbierzcie się! Panie, sprowadź tam swoich bohaterów! Niech poruszą się ludy i przyjdą do Doliny Józafata, bo tam zasiądę, aby sądzić wszystkie okoliczne narody. Zapuśćcie sierp, gdyż nadeszło żniwo! Przyjdźcie, zstąpcie, bo pełna jest prasa; pełne są kadzie, gdyż złość ich jest wielka! Tłumy i tłumy zebrane w Dolinie Sądu, bo bliski jest dzień Pana w Dolinie Sądu. Słońce i księżyc będą zaćmione, a gwiazdy stracą swój blask. Pan zagrzmi z Syjonu i wyda swój donośny głos z Jeruzalemu, tak że zadrżą niebiosa i ziemia. Lecz dla swojego ludu Pan jest ucieczką i twierdzą dla synów Izraela. I poznacie, że Ja, Pan, jestem waszym Bogiem, który mieszkam na Syjonie, świętej mojej górze. Wówczas Jeruzalem będzie święte, a obcy już nie będą przez nie przechodzili. W owym dniu góry będą ociekać moszczem, a pagórki opływać mlekiem, i wszystkie potoki judzkie będą pełne wody, a z przybytku Pana wypływać będzie źródło i nawodni Dolinę Akacji. Egipt zamieni się w pustynię, a Edom w goły step z powodu gwałtu zadanego synom Judy, gdyż w ich ziemi przelewali krew niewinną, Lecz Juda będzie zamieszkana po wszystkie czasy, a Jeruzalem po wszystkie pokolenia. I pomszczę ich krew, której dotąd nie pomściłem. A Pan zamieszka na Syjonie.'' \mbox{(Jl 3, 16-26)}
\end{quote}
\begin{quote}
,,Biada wam, którzy z utęsknieniem oczekujecie dnia Pana! Na cóż wam ten dzień Pana? Wszak jest on ciemnością, a nie światłością. (...) Zaiste, dzień Pana jest ciemnością, a nie światłością, mrokiem, a nie jasnością.'' \mbox{(Am 5, 18.20)}
\end{quote}
\begin{quote}
,,W owym dniu, mówi Wszechmogący Pan, sprawię, że słońce zajdzie w południe, i mrokiem okryję ziemię w biały dzień, I zamienię wasze święta w żałobę i wszystkie wasze pieśni w pieśń żałobną; na wszystkie biodra włożę włosiennicę, a na każdą głowę łysinę. Przekształcę to w żałobę po jedynaku, a całą przyszłość w dzień goryczy.'' \mbox{(Am 8, 9-10)}
\end{quote}
\begin{quote}
,,W owym dniu stanie się tak: Nie będzie ani upału, ani zimna, ani mrozu. I będzie tylko jeden ciągły dzień, zna go Pan, nie dzień i nie noc, a pod wieczór będzie światło.'' \mbox{(Za 14, 6-7)}
\end{quote}
\begin{quote}
,,Bo oto nadchodzi dzień, który pali jak piec. Wtedy wszyscy zuchwali i wszyscy, którzy czynili zło, staną się cierniem. I spali ich ten nadchodzący dzień, - mówi Pan Zastępów - tak, że im nie pozostawi ani korzenia, ani gałązki. Ale dla was, którzy boicie się mojego imienia, wzejdzie słońce sprawiedliwości z uzdrowieniem na swoich skrzydłach. I będziecie wychodzić z podskakiwaniem, jak cielęta wychodzące z obory. I podepczecie bezbożnych tak, że będą prochem pod waszymi stopami w dniu, który Ja przygotowuję - mówi Pan Zastępów.'' \mbox{(Ml 3, 19-21)}
\end{quote}
\end{document}