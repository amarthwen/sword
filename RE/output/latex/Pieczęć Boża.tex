\documentclass[10pt,a4paper,oneside]{article}
\usepackage[utf8]{inputenc}
\usepackage{polski}
\usepackage[polish]{babel}
\usepackage[margin=0.5in,bottom=0.75in]{geometry}
\begin{document}
\centerline{\textbf{\MakeUppercase{Pieczęć Boża}}}
\begin{center}
\textbf{Wersety do studium:} \mbox{(2 Kor 1, 21-22)}, \mbox{(Rz 8, 9)}, \mbox{(Ef 1, 13)}, \mbox{(Rz 8, 14)}, \mbox{(Ef 4, 30)}
\end{center}
\begin{quote}
,,Tym zaś, który nas utwierdza wraz z wami w Chrystusie, który nas namaścił, jest Bóg, Który też wycisnął na nas pieczęć i dał zadatek Ducha do serc naszych.'' \mbox{(2 Kor 1, 21-22)}
\end{quote}
\begin{quote}
,,Ale wy nie jesteście w ciele, lecz w Duchu, jeśli tylko Duch Boży mieszka w was. Jeśli zaś kto nie ma Ducha Chrystusowego, ten nie jest jego.'' \mbox{(Rz 8, 9)}
\end{quote}
\begin{quote}
,,W nim i wy, którzy usłyszeliście słowo prawdy, ewangelię zbawienia waszego, i uwierzyliście w niego, zostaliście zapieczętowani obiecanym Duchem Świętym,'' \mbox{(Ef 1, 13)}
\end{quote}
\begin{quote}
,,Bo ci, których Duch Boży prowadzi, są dziećmi Bożymi.'' \mbox{(Rz 8, 14)}
\end{quote}
\begin{quote}
,,A nie zasmucajcie Bożego Ducha Świętego, którym jesteście zapieczętowani na dzień odkupienia.'' \mbox{(Ef 4, 30)}
\end{quote}
\end{document}