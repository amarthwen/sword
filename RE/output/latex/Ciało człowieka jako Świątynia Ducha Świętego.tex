\documentclass[10pt,a4paper,oneside]{article}
\usepackage[utf8]{inputenc}
\usepackage{polski}
\usepackage[polish]{babel}
\usepackage[margin=0.5in,bottom=0.75in]{geometry}
\begin{document}
\centerline{\textbf{\MakeUppercase{Ciało człowieka jako Świątynia Ducha Świętego}}}
\begin{center}
\textbf{Wersety do studium:} (Hbr~9,~1-10), (Ef~2,~19-22), (1~P~2,~4-9), (Obj~3,~12), (1~Kor~3,~10-15), (Mt~7,~24-27), (Obj~3,~20), (Łk~11,~24-26), (1~Kor~3,~16-17), (Rz~8,~8-11), (2~Kor~6,~16), (1~Kor~6,~19-20), (J~2,~13-21), (1~Moj~4,~6-7), (Jk~4,~5), (Obj~3,~20), (J~14,~22-24), (Mt~7,~24-27), (1~Kor~3,~10-13), (1~Kor~3,~14-17)
\end{center}
Świątynia Boża w kontekście nowego przymierza:
\begin{quote}
,,Wprawdzie i pierwsze przymierze miało przepisy o służbie Bożej i ziemską świątynię. Wystawiony bowiem został przybytek, którego część przednia nazywa się miejscem świętym, a w niej znajdowały się świecznik i stół, i chleby pokładne; Za drugą zaś zasłoną był przybytek, zwany miejscem najświętszym, Mieszczący złotą kadzielnicę i Skrzynię Przymierza, pokrytą zewsząd złotem, w której był złoty dzban z manną i laska Aarona, która zakwitła i tablice przymierza; Nad nią zaś cherubini chwały, zacieniający wieko skrzyni, o czym teraz nie ma potrzeby szczegółowo mówić. A skoro tak te rzeczy zostały urządzone, kapłani sprawujący służbę Bożą wchodzą stale do pierwszej części przybytku, Do drugiej zaś raz w roku sam tylko arcykapłan, i to nie bez krwi, którą ofiaruje za siebie samego i za uchybienia ludu. Przez to Duch Święty wskazuje wyraźnie, że droga do świątyni nie została jeszcze objawiona, dopóki stoi pierwszy przybytek; Ma to znaczenie obrazowe, odnoszące się do teraźniejszego czasu, kiedy to składane bywają dary i ofiary, które nie mogą doprowadzić do wewnętrznej doskonałości tego, kto pełni służbę Bożą; Są to tylko przepisy zewnętrzne, dotyczące pokarmów i napojów, i różnych obmywań, nałożone do czasu zaprowadzenia nowego porządku.'' (Hbr~9,~1-10)
\end{quote}
\begin{quote}
,,Tak więc już nie jesteście obcymi i przychodniami, lecz współobywatelami świętych i domownikami Boga, Zbudowani na fundamencie apostołów i proroków, którego kamieniem węgielnym jest sam Chrystus Jezus, Na którym cała budowa mocno spojona rośnie w przybytek święty w Panu, Na którym i wy się wespół budujecie na mieszkanie Boże w Duchu.'' (Ef~2,~19-22)
\end{quote}
\begin{quote}
,,Przystąpcie do niego, do kamienia żywego, przez ludzi wprawdzie odrzuconego, lecz przez Boga wybranego jako kosztowny. I wy sami jako kamienie żywe budujcie się w dom duchowy, w kapłaństwo święte, aby składać duchowe ofiary przyjemne Bogu przez Jezusa Chrystusa. Dlatego to powiedziane jest w Piśmie: Oto kładę na Syjonie kamień węgielny, wybrany, kosztowny, A kto weń wierzy, nie zawiedzie się. Dla was, którzy wierzycie, jest on rzeczą cenną; dla niewierzących zaś kamień ten, którym wzgardzili budowniczowie, pozostał kamieniem węgielnym, Ale też kamieniem, o który się potkną, i skałą zgorszenia; ci, którzy nie wierzą Słowu, potykają się oń, na co zresztą są przeznaczeni. Ale wy jesteście rodem wybranym, królewskim kapłaństwem, narodem świętym, ludem nabytym, abyście rozgłaszali cnoty tego, który was powołał z ciemności do cudownej swojej światłości;'' (1~P~2,~4-9)
\end{quote}
\begin{quote}
,,Zwycięzcę uczynię filarem w świątyni Boga mojego i już z niej nie wyjdzie, i wypiszę na nim imię Boga mojego, i nazwę miasta Boga mojego, nowego Jeruzalem, które zstępuje z nieba od Boga mojego, i moje nowe imię.'' (Obj~3,~12)
\end{quote}
Ciało człowieka jako Świątynia Ducha Świętego:
\begin{quote}
,,Według łaski Bożej, która mi jest dana, jako mądry budowniczy założyłem fundament, a inny na nim buduje. Każdy zaś niechaj baczy, jak na nim buduje. Albowiem fundamentu innego nikt nie może założyć oprócz tego, który jest założony, a którym jest Jezus Chrystus. A czy ktoś na tym fundamencie wznosi budowę ze złota, srebra, drogich kamieni, z drzewa, siana, słomy, To wyjdzie na jaw w jego dziele; dzień sądny bowiem to pokaże, gdyż w ogniu się objawi, a jakie jest dzieło każdego, wypróbuje ogień. Jeśli czyjeś dzieło, zbudowane na tym fundamencie, się ostoi, ten zapłatę odbierze; Jeśli czyjeś dzieło spłonie, ten szkodę poniesie, lecz on sam zbawiony będzie, tak jednak, jak przez ogień.'' (1~Kor~3,~10-15)
\end{quote}
\begin{quote}
,,Każdy więc, kto słucha tych słów moich i wykonuje je, będzie przyrównany do męża mądrego, który zbudował dom swój na opoce. I spadł deszcz ulewny, i wezbrały rzeki, i powiały wiatry, i uderzyły na ów dom, ale on nie runął, gdyż był zbudowany na opoce. A każdy, kto słucha tych słów moich, lecz nie wykonuje ich, przyrównany będzie do męża głupiego, który zbudował swój dom na piasku. I spadł ulewny deszcz, i wezbrały rzeki, i powiały wiatry, i uderzyły na ów dom, i runął, a upadek jego był wielki.'' (Mt~7,~24-27)
\end{quote}
\begin{quote}
,,Oto stoję u drzwi i kołaczę; jeśli ktoś usłyszy głos mój i otworzy drzwi, wstąpię do niego i będę z nim wieczerzał, a on ze mną.'' (Obj~3,~20)
\end{quote}
\begin{quote}
,,Gdy duch nieczysty wyjdzie z człowieka, wędruje po miejscach bezwodnych, szukając ukojenia, a gdy nie znajdzie, mówi: Wrócę do domu swego, skąd wyszedłem. I przyszedłszy, zastaje go wymiecionym i przyozdobionym. Wówczas idzie i zabiera z sobą siedem innych duchów, gorszych niż on, i wchodzą, i mieszkają tam. I bywa końcowy stan człowieka tego, gorszy niż pierwotny.'' (Łk~11,~24-26)
\end{quote}
\begin{quote}
,,Czy nie wiecie, że świątynią Bożą jesteście i że Duch Boży mieszka w was? Jeśli ktoś niszczy świątynię Bożą, tego zniszczy Bóg, albowiem świątynia Boża jest święta, a wy nią jesteście.'' (1~Kor~3,~16-17)
\end{quote}
\begin{quote}
,,Ci zaś, którzy są w ciele, Bogu podobać się nie mogą. Ale wy nie jesteście w ciele, lecz w Duchu, jeśli tylko Duch Boży mieszka w was. Jeśli zaś kto nie ma Ducha Chrystusowego, ten nie jest jego. Jeśli jednak Chrystus jest w was, to chociaż ciało jest martwe z powodu grzechu, jednak duch jest żywy przez usprawiedliwienie. A jeśli Duch tego, który Jezusa wzbudził z martwych, mieszka w was, tedy Ten, który Jezusa Chrystusa z martwych wzbudził, ożywi i wasze śmiertelne ciała przez Ducha swego, który mieszka w was.'' (Rz~8,~8-11)
\end{quote}
\begin{quote}
,,Jakiż układ między świątynią Bożą a bałwanami? Myśmy bowiem świątynią Boga żywego, jak powiedział Bóg: Zamieszkam w nich i będę się przechadzał pośród nich, i będę Bogiem ich, a oni będą ludem moim.'' (2~Kor~6,~16)
\end{quote}
\begin{quote}
,,Albo czy nie wiecie, że ciało wasze jest świątynią Ducha Świętego, który jest w was i którego macie od Boga, i że nie należycie też do siebie samych? Drogoście bowiem kupieni. Wysławiajcie tedy Boga w ciele waszym.'' (1~Kor~6,~19-20)
\end{quote}
\begin{quote}
,,A gdy się zbliżała Pascha żydowska, udał się Jezus do Jerozolimy. I zastał w świątyni sprzedających woły i owce, i gołębie, i siedzących wekslarzy. I skręciwszy bicz z powrózków, wypędził ich wszystkich ze świątyni wraz z owcami i wołami; wekslarzom rozsypał pieniądze i stoły powywracał, A do sprzedawców gołębi rzekł: Zabierzcie to stąd, z domu Ojca mego nie czyńcie targowiska. Wtedy uczniowie jego przypomnieli sobie, że napisano: Żarliwość o dom twój pożera mnie. Wtedy odezwali się Żydzi, mówiąc do niego: Jaki znak pokażesz nam na dowód, że ci to wolno czynić? Jezus, odpowiadając, rzekł im: Zburzcie tę świątynię, a Ja w trzy dni ją odbuduję. na to rzekli Żydzi: Czterdzieści sześć lat budowano tę świątynię, a Ty w trzy dni chcesz ją odbudować? Ale On mówił o świątyni ciała swego.'' (J~2,~13-21)
\end{quote}
\begin{quote}
,,I rzekł Pan do Kaina: Czemu się gniewasz i czemu zasępiło się twoje oblicze? Wszak byłoby pogodne, gdybyś czynił dobrze, a jeśli nie będziesz czynił dobrze, u drzwi czyha grzech. Kusi cię, lecz ty masz nad nim panować.'' (1~Moj~4,~6-7)
\end{quote}
\begin{quote}
,,Albo czy sądzicie, że na próżno Pismo mówi: Zazdrośnie chce On mieć tylko dla siebie ducha, któremu dał w nas mieszkanie?'' (Jk~4,~5)
\end{quote}
Ciało człowieka jako Świątynia Ducha Świętego - wywód:
\begin{quote}
,,Oto stoję u drzwi i kołaczę; jeśli ktoś usłyszy głos mój i otworzy drzwi, wstąpię do niego i będę z nim wieczerzał, a on ze mną.'' (Obj~3,~20)
\end{quote}
\begin{quote}
,,Rzekł mu Judasz, nie Iskariota: Panie, cóż się stało, że masz się nam objawić, a nie światu? Odpowiedział Jezus i rzekł mu: Jeśli kto mnie miłuje, słowa mojego przestrzegać będzie, i Ojciec mój umiłuje go, i do niego przyjdziemy i u niego zamieszkamy. Kto mnie nie miłuje, ten słów moich nie przestrzega, a przecież słowo, które słyszycie, nie jest moim słowem, lecz Ojca, który mnie posłał.'' (J~14,~22-24)
\end{quote}
\begin{quote}
,,Każdy więc, kto słucha tych słów moich i wykonuje je, będzie przyrównany do męża mądrego, który zbudował dom swój na opoce. I spadł deszcz ulewny, i wezbrały rzeki, i powiały wiatry, i uderzyły na ów dom, ale on nie runął, gdyż był zbudowany na opoce. A każdy, kto słucha tych słów moich, lecz nie wykonuje ich, przyrównany będzie do męża głupiego, który zbudował swój dom na piasku. I spadł ulewny deszcz, i wezbrały rzeki, i powiały wiatry, i uderzyły na ów dom, i runął, a upadek jego był wielki.'' (Mt~7,~24-27)
\end{quote}
\begin{quote}
,,Według łaski Bożej, która mi jest dana, jako mądry budowniczy założyłem fundament, a inny na nim buduje. Każdy zaś niechaj baczy, jak na nim buduje. Albowiem fundamentu innego nikt nie może założyć oprócz tego, który jest założony, a którym jest Jezus Chrystus. A czy ktoś na tym fundamencie wznosi budowę ze złota, srebra, drogich kamieni, z drzewa, siana, słomy, To wyjdzie na jaw w jego dziele; dzień sądny bowiem to pokaże, gdyż w ogniu się objawi, a jakie jest dzieło każdego, wypróbuje ogień.'' (1~Kor~3,~10-13)
\end{quote}
\begin{quote}
,,Jeśli czyjeś dzieło, zbudowane na tym fundamencie, się ostoi, ten zapłatę odbierze; Jeśli czyjeś dzieło spłonie, ten szkodę poniesie, lecz on sam zbawiony będzie, tak jednak, jak przez ogień. Czy nie wiecie, że świątynią Bożą jesteście i że Duch Boży mieszka w was? Jeśli ktoś niszczy świątynię Bożą, tego zniszczy Bóg, albowiem świątynia Boża jest święta, a wy nią jesteście.'' (1~Kor~3,~14-17)
\end{quote}
\end{document}