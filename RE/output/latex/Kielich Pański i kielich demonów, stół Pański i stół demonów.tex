\documentclass[10pt,a4paper,oneside]{article}
\usepackage[utf8]{inputenc}
\usepackage{polski}
\usepackage[polish]{babel}
\usepackage[margin=0.5in,bottom=0.75in]{geometry}
\begin{document}
\centerline{\textbf{\MakeUppercase{Kielich Pański i kielich demonów, stół Pański i stół demonów}}}
\begin{center}
\textbf{Wersety do studium:} \mbox{(1 Kor 10, 19-21)}, \mbox{(Mt 20, 20-23)}, \mbox{(Łk 22, 39-42)}
\end{center}
\begin{quote}
,,Cóż tedy chcę powiedzieć? Czy to, że mięso składane w ofierze bałwanom, jest czymś więcej niż mięsem? Albo że bożek jest czymś więcej niż bałwanem? Nie, chcę powiedzieć, że to, co składają w ofierze, ofiarują demonom, a nie Bogu; ja zaś nie chcę, abyście mieli społeczność z demonami. Nie możecie pić kielicha Pańskiego i kielicha demonów; nie możecie być uczestnikami stołu Pańskiego i stołu demonów.'' \mbox{(1 Kor 10, 19-21)}
\end{quote}
\begin{quote}
,,Wtedy przystąpiła do niego matka synów Zebedeuszowych z synami swoimi, złożyła mu pokłon i prosiła go o coś. A On jej rzekł: Czego chcesz? Rzecze mu: Powiedz, aby ci dwaj synowie moi zasiedli jeden po prawicy, a drugi po lewicy twojej w Królestwie twoim. A Jezus, odpowiadając, rzekł: Nie wiecie, o co prosicie. Czy możecie pić kielich, który ja pić będę? Mówią mu: Możemy. Mówi im; Kielich mój pić będziecie, ale zasiąść po prawicy mojej czy po lewicy - nie moja to rzecz, lecz Ojca mego, który da to tym, którym zostało przez niego przygotowane.'' \mbox{(Mt 20, 20-23)}
\end{quote}
\begin{quote}
,,I wyszedłszy, udał się według zwyczaju na Górę Oliwną; poszli też z nim uczniowie. A gdy przyszedł na miejsce, rzekł do nich: Módlcie się, aby nie popaść w pokuszenie. A sam oddalił się od nich, jakby na rzut kamienia, i padłszy na kolana, modlił się. Mówiąc: Ojcze, jeśli chcesz, oddal ten kielich ode mnie; wszakże nie moja, lecz twoja wola niech się stanie.'' \mbox{(Łk 22, 39-42)}
\end{quote}
\end{document}