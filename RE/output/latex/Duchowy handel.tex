\documentclass[10pt,a4paper,oneside]{article}
\usepackage[utf8]{inputenc}
\usepackage{polski}
\usepackage[polish]{babel}
\usepackage[margin=0.5in,bottom=0.75in]{geometry}
\begin{document}
\centerline{\textbf{\MakeUppercase{Duchowy handel}}}
\begin{center}
\textbf{Wersety do studium:} \mbox{(Mt 21, 12-13)}, \mbox{(J 2, 13-21)}, \mbox{(Ez 28, 12-19)}, \mbox{(2 Kor 2, 17)}, \mbox{(2 Kor 4, 1-2)}, \mbox{(Oz 7, 13)}, \mbox{(So 3, 11-13)}, \mbox{(Za 14, 20-21)}
\end{center}
\begin{quote}
,,I wszedł Jezus do świątyni, i wyrzucił wszystkich, którzy sprzedawali i kupowali w świątyni, a stoły wekslarzy i stragany handlarzy gołębiami powywracał. I rzekł im: Napisano: Dom mój będzie nazwany domem modlitwy, a wy uczyniliście z niego jaskinię zbójców.'' \mbox{(Mt 21, 12-13)}
\end{quote}
\begin{quote}
,,A gdy się zbliżała Pascha żydowska, udał się Jezus do Jerozolimy. I zastał w świątyni sprzedających woły i owce, i gołębie, i siedzących wekslarzy. I skręciwszy bicz z powrózków, wypędził ich wszystkich ze świątyni wraz z owcami i wołami; wekslarzom rozsypał pieniądze i stoły powywracał, A do sprzedawców gołębi rzekł: Zabierzcie to stąd, z domu Ojca mego nie czyńcie targowiska. Wtedy uczniowie jego przypomnieli sobie, że napisano: Żarliwość o dom twój pożera mnie. Wtedy odezwali się Żydzi, mówiąc do niego: Jaki znak pokażesz nam na dowód, że ci to wolno czynić? Jezus, odpowiadając, rzekł im: Zburzcie tę świątynię, a Ja w trzy dni ją odbuduję. na to rzekli Żydzi: Czterdzieści sześć lat budowano tę świątynię, a Ty w trzy dni chcesz ją odbudować? Ale On mówił o świątyni ciała swego.'' \mbox{(J 2, 13-21)}
\end{quote}
\begin{quote}
,,Synu człowieczy, zanuć pieśń żałobną nad królem Tyru i powiedz mu: Tak mówi Wszechmocny Pan: Ty, który byłeś odbiciem doskonałości, pełnym mądrości i skończonego piękna, Byłeś w Edenie, ogrodzie Bożym; okryciem twoim były wszelakie drogie kamienie: karneol, topaz i jaspis, chryzolit, beryl i onyks, szafir, rubin i szmaragd; ze złota zrobione były twoje bębenki, a twoje ozdoby zrobiono w dniu, gdy zostałeś stworzony. Obok cheruba, który bronił wstępu, postawiłem cię; byłeś na świętej górze Bożej, przechadzałeś się pośród kamieni ognistych. Nienagannym byłeś w postępowaniu swoim od dnia, gdy zostałeś stworzony, aż dotąd, gdy odkryto u ciebie niegodziwość. Przy rozległym swoim handlu napełniłeś swoje wnętrze gwałtem i zgrzeszyłeś. Wtedy to wypędziłem cię z góry Bożej, a cherub, który bronił wstępu, wygubił cię spośród kamieni ognistych. Twoje serce było wyniosłe z powodu twojej piękności. Zniweczyłeś swoją mądrość skutkiem swojej świetności. Zrzuciłem cię na ziemię; postawiłem cię przed królami, aby się z ciebie naigrawali. Zbezcześciłeś moją świątynię z powodu mnóstwa swoich win, przy niegodziwym swoim handlu. Dlatego wywiodłem z ciebie ogień i ten cię strawił; obróciłem cię w popiół na ziemi na oczach wszystkich, którzy cię widzieli. Wszyscy, którzy cię znali pośród ludów, zdumiewali się nad tobą; stałeś się odstraszającym przykładem, przepadłeś na wieki.'' \mbox{(Ez 28, 12-19)}
\end{quote}
\begin{quote}
,,Bo my nie jesteśmy handlarzami Słowa Bożego, jak wielu innych, lecz mówimy w Chrystusie przed obliczem Boga jako ludzie szczerzy, jako ludzie mówiący z Boga.'' \mbox{(2 Kor 2, 17)}
\end{quote}
\begin{quote}
,,Dlatego, mając tę służbę, która nam została poruczona z miłosierdzia, nie upadamy na duchu, Lecz wyrzekliśmy się tego, co ludzie wstydliwie ukrywają, i nie postępujemy przebiegle ani nie fałszujemy Słowa Bożego, ale przez składanie dowodu prawdy polecamy siebie samych sumieniu wszystkich ludzi przed Bogiem.'' \mbox{(2 Kor 4, 1-2)}
\end{quote}
\begin{quote}
,,Biada im, że uciekli ode mnie! Grozi im zguba, gdyż odstąpili ode mnie. A przecież to Ja ich wyzwoliłem; lecz oni rozsiewali o mnie kłamstwa.'' \mbox{(Oz 7, 13)}
\end{quote}
\begin{quote}
,,W owym dniu nie doznasz już hańby z powodu wszystkich swych czynów, którymi grzeszyłeś przeciwko mnie. Gdyż wtedy usunę spośród ciebie twoich pyszałków i nie będziesz się już chełpić na mojej świętej górze. I pozostawię pośród ciebie lud pokorny i ubogi: to oni ufać będą imieniu Pana. Resztka Izraela nie będzie się dopuszczała bezprawia, nie będzie mówiła kłamstwa i w ich ustach nie znajdzie się język zdradliwy. Zaiste, będą się paść i odpoczywać, a nikt nie będzie ich straszył.'' \mbox{(So 3, 11-13)}
\end{quote}
\begin{quote}
,,W owym dniu będzie na dzwoneczkach koni napis: Poświęcony Panu. A garnków domu Pana będzie jak czasz ofiarnych przed ołtarzem. Każdy garnek w Jeruzalemie i w Judzie będzie poświęcony Panu Zastępów, tak że wszyscy, którzy przyjdą składać ofiary, będą je brali i będą w nich gotowali mięso ofiarne. W dniu owym już nie będzie handlarza w domu Pana Zastępów.'' \mbox{(Za 14, 20-21)}
\end{quote}
\end{document}