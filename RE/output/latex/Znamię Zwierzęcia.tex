\documentclass[10pt,a4paper,oneside]{article}
\usepackage[utf8]{inputenc}
\usepackage{polski}
\usepackage[polish]{babel}
\usepackage[margin=0.5in,bottom=0.75in]{geometry}
\begin{document}
\centerline{\textbf{\MakeUppercase{Znamię Zwierzęcia}}}
\begin{center}
\textbf{Wersety do studium:} \mbox{(Obj 13, 11-18)}, \mbox{(Obj 14, 9-12)}, \mbox{(Obj 15, 1-2)}, \mbox{(Obj 16, 2.11)}, \mbox{(Obj 19, 20)}, \mbox{(Obj 20, 4)}
\end{center}
\begin{quote}
,,I widziałem inne zwierzę, wychodzące z ziemi, które miało dwa rogi podobne do baranich, i mówiło jak smok. A wykonuje ono wszelką władzę pierwszego zwierzęcia na jego oczach. Ono to sprawia, że ziemia i jej mieszkańcy oddają pokłon pierwszemu zwierzęciu, którego śmiertelna rana była wygojona. I czyni wielkie cuda, tak że i ogień z nieba spuszcza na ziemię na oczach ludzi. I zwodzi mieszkańców ziemi przez cuda, jakie dano mu czynić na oczach zwierzęcia, namawiając mieszkańców ziemi, by postawili posąg zwierzęciu, które ma ranę od miecza, a jednak zostało przy życiu. I dano mu tchnąć ducha w posąg zwierzęcia, aby posąg zwierzęcia przemówił i sprawił, że wszyscy, którzy nie oddali pokłonu posągowi zwierzęcia, zostaną zabici. On też sprawia, że wszyscy, mali i wielcy, bogaci i ubodzy, wolni i niewolnicy otrzymują znamię na swojej prawej ręce albo na swoim czole, I że nikt nie może kupować ani sprzedawać, jeżeli nie ma znamienia, to jest imienia zwierzęcia lub liczby jego imienia. Tu potrzebna jest mądrość. Kto ma rozum, niech obliczy liczbę zwierzęcia; jest to bowiem liczba człowieka. A liczba jego jest sześćset sześćdziesiąt sześć.'' \mbox{(Obj 13, 11-18)}
\end{quote}
\begin{quote}
,,A trzeci anioł szedł za nimi, mówiąc donośnym głosem: Jeżeli ktoś odda pokłon zwierzęciu i jego posągowi i przyjmie znamię na swoje czoło lub na swoją rękę, To i on pić będzie samo czyste wino gniewu Bożego z kielicha jego gniewu i będzie męczony w ogniu i w siarce wobec świętych aniołów i wobec Baranka. A dym ich męki unosi się w górę na wieki wieków i nie mają wytchnienia we dnie i w nocy ci, którzy oddają pokłon zwierzęciu i jego posągowi, ani nikt, kto przyjmuje znamię jego imienia. Tu się okaże wytrwanie świętych, którzy przestrzegają przykazań Bożych i wiary Jezusa.'' \mbox{(Obj 14, 9-12)}
\end{quote}
\begin{quote}
,,I widziałem inny znak na niebie wielki i dziwny: siedmiu aniołów z siedmiu ostatnimi plagami, gdyż na nich zakończył się gniew Boży. I widziałem jakby morze szkliste zmieszane z ogniem, i tych, którzy odnieśli zwycięstwo nad zwierzęciem i jego posągiem, i nad liczbą imienia jego; ci stali nad morzem szklistym, trzymając harfy Boże.'' \mbox{(Obj 15, 1-2)}
\end{quote}
\begin{quote}
,,I wyszedł pierwszy, i wylał czaszę swoją na ziemię; i pojawiły się złośliwe i odrażające wrzody na ludziach, mających znamię zwierzęcia i oddających pokłon jego posągowi. (\ldots) I bluźnili Bogu niebieskiemu z powodu swoich bólów i z powodu swoich wrzodów, i nie upamiętali się w swoich uczynkach.'' \mbox{(Obj 16, 2.11)}
\end{quote}
\begin{quote}
,,I pojmane zostało zwierzę, a wraz z nim fałszywy prorok, który przed nim czynił cuda, jakimi zwiódł tych, którzy przyjęli znamię zwierzęcia i oddawali pokłon posągowi jego. Zostali oni obaj wrzuceni żywcem do jeziora ognistego, gorejącego siarką.'' \mbox{(Obj 19, 20)}
\end{quote}
\begin{quote}
,,I widziałem trony, i usiedli na nich ci, którym dano prawo sądu; widziałem też dusze tych, którzy zostali ścięci za to, że składali świadectwo o Jezusie i głosili Słowo Boże, oraz tych, którzy nie oddali pokłonu zwierzęciu ani posągowi jego i nie przyjęli znamienia na czoło i na rękę swoją. Ci ożyli i panowali z Chrystusem przez tysiąc lat.'' \mbox{(Obj 20, 4)}
\end{quote}

\end{document}