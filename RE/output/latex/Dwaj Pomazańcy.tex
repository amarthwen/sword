\documentclass[10pt,a4paper,oneside]{article}
\usepackage[utf8]{inputenc}
\usepackage{polski}
\usepackage[polish]{babel}
\usepackage[margin=0.5in,bottom=0.75in]{geometry}
\begin{document}
\centerline{\textbf{\MakeUppercase{Dwaj Pomazańcy}}}
\begin{center}
\textbf{Wersety do studium:} \mbox{(Obj 11, 3)}, \mbox{(Obj 11, 4)}, \mbox{(Za 4, 11)}, \mbox{(Obj 11, 4)}, \mbox{(Obj 11, 10)}, \mbox{(Za 4, 11-14)}, \mbox{(Ps 105, 6-15)}, \mbox{(Ps 10, 1-10)}
\end{center}
Dwaj Świadkowie:
\begin{quote}
,,I dam dwom moim świadkom moc, i będą, odziani w wory, prorokowali przez tysiąc dwieście sześćdziesiąt dni.'' \mbox{(Obj 11, 3)}
\end{quote}
Dwa Drzewa Oliwne:
\begin{quote}
,,Oni to są dwoma drzewami oliwnymi i dwoma świecznikami, które stoją przed Panem ziemi.'' \mbox{(Obj 11, 4)}
\end{quote}
\begin{quote}
,,Wtedy odezwałem się i zapytałem go: Co oznaczają te dwa drzewa oliwne po prawej i lewej stronie świecznika?'' \mbox{(Za 4, 11)}
\end{quote}
Dwa Świeczniki:
\begin{quote}
,,Oni to są dwoma drzewami oliwnymi i dwoma świecznikami, które stoją przed Panem ziemi.'' \mbox{(Obj 11, 4)}
\end{quote}
Dwaj Prorocy:
\begin{quote}
,,A mieszkańcy ziemi radować się będą nimi i weselić się, i podarunki sobie nawzajem posyłać, dlatego że ci dwaj prorocy udręczyli mieszkańców ziemi.'' \mbox{(Obj 11, 10)}
\end{quote}
Dwaj Pomazańcy:
\begin{quote}
,,Wtedy odezwałem się i zapytałem go: Co oznaczają te dwa drzewa oliwne po prawej i lewej stronie świecznika? I powtórnie odezwałem się, i zapytałem go: Co oznaczają te dwie gałązki drzew oliwnych, które dwiema złotymi rurkami wypuszczają z siebie oliwę do złotych lamp? Wtedy on odpowiedział mi: Czy nie wiesz, co one oznaczają? I odpowiedziałem: Nie, mój panie! A on odrzekł: To są dwaj pomazańcy, którzy stoją przed Panem całej ziemi.'' \mbox{(Za 4, 11-14)}
\end{quote}
\begin{quote}
,,Wy, potomkowie Abrahama, sługi jego, Synowie Jakuba, wybrańcy jego! On jest Panem, Bogiem waszym, Prawa jego na całej ziemi. Pamięta wiecznie o przymierzu swoim, O słowie, które dał tysiącznym pokoleniom, O przymierzu, które zawarł z Abrahamem, I o przysiędze swej dla Izaaka. Ustanowił je dla Jakuba jako prawo, Dla Izraela jako przymierze wieczne, Mówiąc: Tobie dam ziemię Kanaan, W dziedziczne wasze posiadanie. Gdy było ich jeszcze niewielu, Nieliczni i obcy w niej, Wędrowali wtedy od narodu do narodu, Z jednego królestwa do innego ludu. Nikomu nie dozwolił ich krzywdzić I z powodu nich karał nawet królów: Nie tykajcie pomazańców moich I nie czyńcie nic złego prorokom moim!'' \mbox{(Ps 105, 6-15)}
\end{quote}
Pomazaniec:
\begin{quote}
,,Czemu o Panie, stoisz z daleka, Ukrywasz się w czasach niedoli? Z powodu pychy bezbożnego trapi się ubogi. Niech uwikłają się w knowaniach, które obmyślili! Bo pyszni się bezbożny zachcianką swoją, A chciwiec bluźni, znieważa Pana. Bezbożny myśli w pysze swojej: Nie będzie dochodził... Nie ma Boga. Oto całe rozumowanie jego. Zabiegi jego w każdym czasie udają się, Sądy twoje nie obchodzą go, Wszystkimi przeciwnikami gardzi. Mówi w sercu swoim: Nie zachwieję się, Nigdy nie spotka mnie nieszczęście. Przekleństwa pełne są usta jego, także fałszu i obłudy, Pod językiem jego jest krzywda i nieprawość. Czatuje za węgłem zagród, Skrycie zabija niewinnego; Oczy jego wypatrują nieszczęśnika. Czyha w kryjówce jak lew w gęstwinie, Czyha, aby porwać ubogiego. Porywa ubogiego, zarzucając sieć swoją. Schyla się, przyczaja, I wpadają w szpony jego nieszczęśliwi.'' \mbox{(Ps 10, 1-10)}
\end{quote}
\end{document}