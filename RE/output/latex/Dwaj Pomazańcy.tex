\documentclass[10pt,a4paper,oneside]{article}
\usepackage[utf8]{inputenc}
\usepackage{polski}
\usepackage[polish]{babel}
\usepackage[margin=0.5in,bottom=0.75in]{geometry}
\begin{document}
\centerline{\textbf{\MakeUppercase{Dwaj Pomazańcy}}}
\begin{center}
\textbf{Wersety do studium:} (Obj~11,~3), (Obj~11,~4), (Za~4,~11), (Obj~11,~4), (Obj~11,~10), (Za~4,~11-14), (Ps~105,~6-15)
\end{center}
Dwaj Świadkowie:
\begin{quote}
,,I dam dwom moim świadkom moc, i będą, odziani w wory, prorokowali przez tysiąc dwieście sześćdziesiąt dni.'' (Obj~11,~3)
\end{quote}
Dwa Drzewa Oliwne:
\begin{quote}
,,Oni to są dwoma drzewami oliwnymi i dwoma świecznikami, które stoją przed Panem ziemi.'' (Obj~11,~4)
\end{quote}
\begin{quote}
,,Wtedy odezwałem się i zapytałem go: Co oznaczają te dwa drzewa oliwne po prawej i lewej stronie świecznika?'' (Za~4,~11)
\end{quote}
Dwa Świeczniki:
\begin{quote}
,,Oni to są dwoma drzewami oliwnymi i dwoma świecznikami, które stoją przed Panem ziemi.'' (Obj~11,~4)
\end{quote}
Dwaj Prorocy:
\begin{quote}
,,A mieszkańcy ziemi radować się będą nimi i weselić się, i podarunki sobie nawzajem posyłać, dlatego że ci dwaj prorocy udręczyli mieszkańców ziemi.'' (Obj~11,~10)
\end{quote}
Dwaj Pomazańcy:
\begin{quote}
,,Wtedy odezwałem się i zapytałem go: Co oznaczają te dwa drzewa oliwne po prawej i lewej stronie świecznika? I powtórnie odezwałem się, i zapytałem go: Co oznaczają te dwie gałązki drzew oliwnych, które dwiema złotymi rurkami wypuszczają z siebie oliwę do złotych lamp? Wtedy on odpowiedział mi: Czy nie wiesz, co one oznaczają? I odpowiedziałem: Nie, mój panie! A on odrzekł: To są dwaj pomazańcy, którzy stoją przed Panem całej ziemi.'' (Za~4,~11-14)
\end{quote}
\begin{quote}
,,Wy, potomkowie Abrahama, sługi jego, Synowie Jakuba, wybrańcy jego! On jest Panem, Bogiem waszym, Prawa jego na całej ziemi. Pamięta wiecznie o przymierzu swoim, O słowie, które dał tysiącznym pokoleniom, O przymierzu, które zawarł z Abrahamem, I o przysiędze swej dla Izaaka. Ustanowił je dla Jakuba jako prawo, Dla Izraela jako przymierze wieczne, Mówiąc: Tobie dam ziemię Kanaan, W dziedziczne wasze posiadanie. Gdy było ich jeszcze niewielu, Nieliczni i obcy w niej, Wędrowali wtedy od narodu do narodu, Z jednego królestwa do innego ludu. Nikomu nie dozwolił ich krzywdzić I z powodu nich karał nawet królów: Nie tykajcie pomazańców moich I nie czyńcie nic złego prorokom moim!'' (Ps~105,~6-15)
\end{quote}
\end{document}