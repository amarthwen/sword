\documentclass[10pt,a4paper,oneside]{article}
\usepackage[utf8]{inputenc}
\usepackage{polski}
\usepackage[polish]{babel}
\usepackage[margin=0.5in,bottom=0.75in]{geometry}
\begin{document}
\centerline{\textbf{\MakeUppercase{Objawione tajemnice Boże}}}
\begin{center}
\textbf{Wersety do studium:} (Ef~3,~1-7), (1~Kor~15,~51-57), (Kol~1,~24-28), (Rz~11,~25-32)
\end{center}
\begin{quote}
,,Dlatego ja, Paweł, jestem więźniem Chrystusa Jezusa za was pogan - Bo zapewne słyszeliście o darze łaski Bożej, która mi została dana dla waszego dobra, Że przez objawienie została mi odsłonięta tajemnica, jak to powyżej krótko opisałem. Czytając to, możecie zrozumieć moje pojmowanie tajemnicy Chrystusowej, Która nie była znana synom ludzkim w dawnych pokoleniach, a teraz została przez Ducha objawiona jego świętym apostołom i prorokom, Mianowicie, że poganie są współdziedzicami i członkami jednego ciała i współuczestnikami obietnicy w Chrystusie Jezusie przez ewangelię, Której sługą zostałem według daru łaski Bożej, okazanej mi przez jego wszechmocne działanie.'' (Ef~3,~1-7)
\end{quote}
\begin{quote}
,,Oto tajemnicę wam objawiam: Nie wszyscy zaśniemy, ale wszyscy będziemy przemienieni W jednej chwili, w oka mgnieniu, na odgłos trąby ostatecznej; bo trąba zabrzmi i umarli wzbudzeni zostaną jako nie skażeni, a my zostaniemy przemienieni. Albowiem to, co skażone, musi przyoblec się w to, co nieskażone, a to, co śmiertelne, musi przyoblec się w nieśmiertelność. A gdy to, co skażone, przyoblecze się w to, co nieskażone, i to, co śmiertelne, przyoblecze się w nieśmiertelność, wtedy wypełni się słowo napisane: Pochłonięta jest śmierć w zwycięstwie! Gdzież jest, o śmierci, zwycięstwo twoje? Gdzież jest, o śmierci, żądło twoje? A żądłem śmierci jest grzech, a mocą grzechu jest zakon; Ale Bogu niech będą dzięki, który nam daje zwycięstwo przez Pana naszego, Jezusa Chrystusa.'' (1~Kor~15,~51-57)
\end{quote}
\begin{quote}
,,Teraz raduję się z cierpień, które za was znoszę i dopełniam na ciele moim niedostatku udręk Chrystusowych za ciało jego, którym jest Kościół. Tego Kościoła sługą zostałem zgodnie z postanowieniem Boga, powziętym Co do mnie ze względu na was, abym w pełni rozgłosił Słowo Boże, Tajemnicę, zakrytą od wieków i od pokoleń, a teraz objawioną świętym jego. Im to chciał Bóg dać poznać, jak wielkie jest między poganami bogactwo chwały tej tajemnicy, którą jest Chrystus w was, nadzieja chwały. Jego to zwiastujemy, napominając i nauczając każdego człowieka we wszelkiej mądrości, aby stawić go doskonałym w Chrystusie Jezusie;'' (Kol~1,~24-28)
\end{quote}
\begin{quote}
,,A żebyście nie mieli zbyt wysokiego o sobie mniemania, chcę wam, bracia, odsłonić tę tajemnicę: zatwardziałość przyszła na część Izraela aż do czasu, gdy poganie w pełni wejdą, I w ten sposób będzie zbawiony cały Izrael, jak napisano: Przyjdzie z Syjonu wybawiciel I odwróci bezbożność od Jakuba. A to będzie przymierze moje z nimi, Gdy zgładzę grzechy ich. Co do ewangelii, są oni nieprzyjaciółmi Bożymi dla waszego dobra, lecz co do wybrania, są umiłowanymi ze względu na praojców. Nieodwołalne są bowiem dary i powołanie Boże. Bo jak i wy byliście niegdyś nieposłuszni Bogu, a teraz dostąpiliście miłosierdzia z powodu ich nieposłuszeństwa Tak i oni teraz, gdy wy dostępujecie miłosierdzia, stali się nieposłuszni, ażeby i oni teraz miłosierdzia dostąpili. Albowiem Bóg poddał wszystkich w niewolę nieposłuszeństwa, aby się nad wszystkimi zmiłować.'' (Rz~11,~25-32)
\end{quote}
\end{document}