\documentclass[10pt,a4paper,oneside]{article}
\usepackage[utf8]{inputenc}
\usepackage{polski}
\usepackage[polish]{babel}
\usepackage[margin=0.5in,bottom=0.75in]{geometry}
\begin{document}
\centerline{\textbf{\MakeUppercase{Dary Ducha Świętego, talenty, dary łaski}}}
\begin{center}
\textbf{Wersety do studium:} \mbox{(Mt 25, 14-15)}, \mbox{(Rz 12, 6-8)}, \mbox{(Ef 4, 7-13)}, \mbox{(1 Kor 12, 4-11)}, \mbox{(1 Kor 14, 12)}
\end{center}
\begin{quote}
,,Będzie bowiem tak jak z człowiekiem, który odjeżdżając, przywołał swoje sługi i przekazał im swój majątek, I dał jednemu pięć talentów, a drugiemu dwa, a trzeciemu jeden, każdemu według zdolności i odjechał.'' \mbox{(Mt 25, 14-15)}
\end{quote}
\begin{quote}
,,A mamy różne dary według udzielonej nam łaski; jeśli dar prorokowania, to niech będzie używany stosownie do wiary; Jeśli posługiwanie, to w usługiwaniu ; jeśli kto naucza, to w nauczaniu; Jeśli kto napomina, to w napominaniu; jeśli kto obdarowuje, to w szczerości; kto jest przełożony, niech okaże gorliwość; kto okazuje miłosierdzie, niech to czyni z radością.'' \mbox{(Rz 12, 6-8)}
\end{quote}
\begin{quote}
,,A każdemu z nas dana została łaska według miary daru Chrystusowego. Dlatego powiedziano: Wstąpiwszy na wysokość, Powiódł za sobą jeńców I ludzi darami obdarzył. A to, że wstąpił, cóż innego oznacza, aniżeli to, że wpierw zstąpił do podziemi? Ten, który zstąpił, to ten sam, co i wstąpił wysoko ponad wszystkie niebiosa, aby napełnić wszystko. I On ustanowił jednych apostołami, drugich prorokami, innych ewangelistami, a innych pasterzami i nauczycielami, Aby przygotować świętych do dzieła posługiwania, do budowania ciała Chrystusowego, Aż dojdziemy wszyscy do jedności wiary i poznania Syna Bożego, do męskiej doskonałości, i dorośniemy do wymiarów pełni Chrystusowej,'' \mbox{(Ef 4, 7-13)}
\end{quote}
\begin{quote}
,,A różne są dary łaski, lecz Duch ten sam. I różne są posługi, lecz Pan ten sam. I różne są sposoby działania, lecz ten sam Bóg, który sprawia wszystko we wszystkich. A w każdym różnie przejawia się Duch ku wspólnemu pożytkowi. Jeden bowiem otrzymuje przez Ducha mowę mądrości, drugi przez tego samego Ducha mowę wiedzy, Inny wiarę w tym samym Duchu, inny dar uzdrawiania w tym samym Duchu. Jeszcze inny dar czynienia cudów, inny dar proroctwa, inny dar rozróżniania duchów, inny różne rodzaje języków, inny wreszcie dar wykładania języków. Wszystko to zaś sprawia jeden i ten sam Duch, rozdzielając każdemu poszczególnie, jak chce.'' \mbox{(1 Kor 12, 4-11)}
\end{quote}
\begin{quote}
,,Tak i wy, ponieważ usilnie zabiegacie o dary Ducha, starajcie się obfitować w te, które służą ku zbudowaniu zboru.'' \mbox{(1 Kor 14, 12)}
\end{quote}
\end{document}