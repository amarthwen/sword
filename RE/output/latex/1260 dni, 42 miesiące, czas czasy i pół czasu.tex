\documentclass[10pt,a4paper,oneside]{article}
\usepackage[utf8]{inputenc}
\usepackage{polski}
\usepackage[polish]{babel}
\usepackage[margin=0.5in,bottom=0.75in]{geometry}
\begin{document}
\centerline{\textbf{\MakeUppercase{1260 dni, 42 miesiące, czas czasy i pół czasu}}}
\begin{center}
\textbf{Wersety do studium:} (Obj~11,~3), (Obj~12,~6), (Obj~11,~2), (Obj~13,~5), (Obj~12,~14), (Dn~7,~25), (Dn~12,~7)
\end{center}
Tysiąc dwieście sześćdziesiąt dni:
\begin{quote}
,,I dam dwom moim świadkom moc, i będą, odziani w wory, prorokowali przez tysiąc dwieście sześćdziesiąt dni.'' (Obj~11,~3)
\end{quote}
\begin{quote}
,,I uciekła niewiasta na pustynię, gdzie ma miejsce przygotowane przez Boga, aby ją tam żywiono przez tysiąc dwieście sześćdziesiąt dni.'' (Obj~12,~6)
\end{quote}
Czterdzieści dwa miesiące:
\begin{quote}
,,Lecz zewnętrzny przedsionek świątyni wyłącz i nie mierz go, gdyż oddany został poganom, którzy tratować będą święte miasto przez czterdzieści dwa miesiące.'' (Obj~11,~2)
\end{quote}
\begin{quote}
,,I dano mu paszczę mówiącą rzeczy wyniosłe i bluźniercze, dano mu też moc działania przez czterdzieści i dwa miesiące.'' (Obj~13,~5)
\end{quote}
Czas, czasy i pół czasu:
\begin{quote}
,,I dano niewieście dwa skrzydła wielkiego orła, aby poleciała na pustynię na miejsce swoje, gdzie ją żywią przez czas i czasy, i pól czasu, z dala od węża.'' (Obj~12,~14)
\end{quote}
\begin{quote}
,,I będzie mówił zuchwałe słowa przeciwko Najwyższemu, będzie męczył świętych Najwyższego, będzie zamyślał odmienić czasy i zakon; i będą wydani w jego moc aż do czasu i dwóch czasów i pół czasu.'' (Dn~7,~25)
\end{quote}
\begin{quote}
,,Wtedy usłyszałem, jak mąż obleczony w szatę lnianą, który stał nad wodami rzeki, podniósł prawicę i lewicę ku niebu i przysięgał na tego, który żyje wiecznie: Będzie to trwało czas wyznaczony, dwa czasy i pół czasu, a gdy doszczętnie będzie zniszczona, wtedy się to wszystko spełni.'' (Dn~12,~7)
\end{quote}
\end{document}