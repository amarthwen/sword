\documentclass[10pt,a4paper,oneside]{article}
\usepackage[utf8]{inputenc}
\usepackage{polski}
\usepackage[polish]{babel}
\usepackage[margin=0.5in,bottom=0.75in]{geometry}
\begin{document}
\centerline{\textbf{\MakeUppercase{Ród Królewski, Kapłani Boga Żywego, Kamienie Żywe, Kamienie Ogniste}}}
\begin{center}
\textbf{Wersety do studium:} (1~P~2,~9), (Obj~1,~6), (Obj~5,~10), (Ez~28,~13-16), (1~P~2,~4-8), (Oz~6,~5), (1~Krl~6,~7)
\end{center}
\begin{quote}
,,Ale wy jesteście rodem wybranym, królewskim kapłaństwem, narodem świętym, ludem nabytym, abyście rozgłaszali cnoty tego, który was powołał z ciemności do cudownej swojej światłości;'' (1~P~2,~9)
\end{quote}
\begin{quote}
,,I uczynił nas rodem królewskim, kapłanami Boga i Ojca swojego, niech będzie chwała i moc na wieki wieków. Amen.'' (Obj~1,~6)
\end{quote}
\begin{quote}
,,I uczyniłeś z nich dla Boga naszego ród królewski i kapłanów, i będą królować na ziemi.'' (Obj~5,~10)
\end{quote}
\begin{quote}
,,Byłeś w Edenie, ogrodzie Bożym; okryciem twoim były wszelakie drogie kamienie: karneol, topaz i jaspis, chryzolit, beryl i onyks, szafir, rubin i szmaragd; ze złota zrobione były twoje bębenki, a twoje ozdoby zrobiono w dniu, gdy zostałeś stworzony. Obok cheruba, który bronił wstępu, postawiłem cię; byłeś na świętej górze Bożej, przechadzałeś się pośród kamieni ognistych. Nienagannym byłeś w postępowaniu swoim od dnia, gdy zostałeś stworzony, aż dotąd, gdy odkryto u ciebie niegodziwość. Przy rozległym swoim handlu napełniłeś swoje wnętrze gwałtem i zgrzeszyłeś. Wtedy to wypędziłem cię z góry Bożej, a cherub, który bronił wstępu, wygubił cię spośród kamieni ognistych.'' (Ez~28,~13-16)
\end{quote}
\begin{quote}
,,Przystąpcie do niego, do kamienia żywego, przez ludzi wprawdzie odrzuconego, lecz przez Boga wybranego jako kosztowny. I wy sami jako kamienie żywe budujcie się w dom duchowy, w kapłaństwo święte, aby składać duchowe ofiary przyjemne Bogu przez Jezusa Chrystusa. Dlatego to powiedziane jest w Piśmie: Oto kładę na Syjonie kamień węgielny, wybrany, kosztowny, A kto weń wierzy, nie zawiedzie się. Dla was, którzy wierzycie, jest on rzeczą cenną; dla niewierzących zaś kamień ten, którym wzgardzili budowniczowie, pozostał kamieniem węgielnym, Ale też kamieniem, o który się potkną, i skałą zgorszenia; ci, którzy nie wierzą Słowu, potykają się oń, na co zresztą są przeznaczeni.'' (1~P~2,~4-8)
\end{quote}
\begin{quote}
,,Dlatego ociosywałem ich przez proroków, zabijałem ich słowami moich ust, i moje prawo wzeszło jak światłość.'' (Oz~6,~5)
\end{quote}
\begin{quote}
,,Świątynię zaś budowano z kamieni gotowych, przyciosanych już w kamieniołomach, tak iż w czasie budowy w świątyni nie było słychać w niej młotów czy siekier, w ogóle żadnego narzędzia żelaznego.'' (1~Krl~6,~7)
\end{quote}
\end{document}