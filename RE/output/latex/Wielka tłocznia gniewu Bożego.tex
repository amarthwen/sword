\documentclass[10pt,a4paper,oneside]{article}
\usepackage[utf8]{inputenc}
\usepackage{polski}
\usepackage[polish]{babel}
\usepackage[margin=0.5in,bottom=0.75in]{geometry}
\begin{document}
\centerline{\textbf{\MakeUppercase{Wielka tłocznia gniewu Bożego}}}
\begin{center}
\textbf{Wersety do studium:} (Ab~1,~5-6), (Obj~14,~17-20), (Obj~19,~11-15), (Iz~63,~1-6), (Am~9,~13), (Jl~3,~17-19), (Obj~16,~19), (Iz~32,~10), (Obj~14,~9-10), (Jr~25,~15-17.27-29)
\end{center}
\begin{quote}
,,Gdy złodzieje wtargną do ciebie albo nocni rabusie jakże będziesz splądrowany! Czy nie będą kradli do woli? Gdy zbieracze winogron przyjdą do ciebie, czy pozostawią choć jedno grono? Jakże ogołocony jest Ezaw; przetrząśnięte są jego skarby ukryte.'' (Ab~1,~5-6)
\end{quote}
\begin{quote}
,,I wyszedł inny anioł ze świątyni, która jest w niebie, mając również ostry sierp. I jeszcze inny anioł wyszedł z ołtarza, a ten miał władzę nad ogniem; i zawołał donośnie na tego, który miał ostry sierp, mówiąc: Zapuść swój ostry sierp i obetnij kiście winogron z winorośli ziemi, gdyż dojrzały jej grona. I zapuścił anioł sierp swój na ziemi, i poobcinał grona winne na ziemi, i wrzucił je do wielkiej tłoczni gniewu Bożego. I deptano tłocznię poza miastem, i popłynęła z tłoczni krew, aż dosięgła wędzideł końskich na przestrzeni tysiąca sześciuset stadiów.'' (Obj~14,~17-20)
\end{quote}
\begin{quote}
,,I widziałem niebo otwarte, a oto biały koń, a Ten, który na nim siedział nazywa się Wierny i Prawdziwy, gdyż sprawiedliwie sądzi i sprawiedliwie walczy. Oczy zaś jego jak płomień ognia, a na głowie jego liczne diademy. Imię swoje miał wypisane, lecz nie znał go nikt, tylko on sam. A przyodziany był w szatę zmoczoną we krwi, imię zaś jego brzmi: Słowo Boże. I szły za nim wojska niebieskie na białych koniach, przyobleczone w czysty, biały bisior. A z ust jego wychodzi ostry miecz, którym miał pobić narody, i będzie nimi rządził laską żelazną, On sam też tłoczy kadź wina zapalczywego gniewu Boga, Wszechmogącego.'' (Obj~19,~11-15)
\end{quote}
\begin{quote}
,,Któż to przychodzi z Edomu, z Bosry w czerwonych szatach? Wspaniały On w swoim odzieniu, dumnie kroczy w pełni swojej siły. To Ja, który wyrokuję sprawiedliwie, mam moc wybawić. Skąd ta czerwień twojej szaty? A twoje odzienie jak u tego, który wytłacza wino w tłoczni? Ja sam tłoczyłem do kadzi, bo spośród ludów żadnego nie było ze mną; a tłoczyłem ich w gniewie i deptałem ich w zapalczywości tak, że ich sok pryskał na moją szatę i całe moje odzienie zbryzgałem. Ustanowiłem bowiem dzień pomsty i nadszedł rok mojego odkupienia. Rozglądałem się, lecz nie było pomocnika, zdziwiłem się, że nie było nikogo, kto by mię podparł. Lecz wtedy dopomogło mi moje ramię i wsparła mnie moja zapalczywość. Podeptałem więc ludy w moim gniewie i zdruzgotałem je w mojej zapalczywości, i sprawiłem, że ich sok spłynął na ziemię.'' (Iz~63,~1-6)
\end{quote}
\begin{quote}
,,Oto idą dni, mówi Pan, w których oracz będzie przynaglał żniwiarza, a tłoczący winogrona siewcę ziarna; i góry będą ociekały moszczem, a wszystkie pagórki nim opływały.'' (Am~9,~13)
\end{quote}
\begin{quote}
,,Niech poruszą się ludy i przyjdą do Doliny Józafata, bo tam zasiądę, aby sądzić wszystkie okoliczne narody. Zapuśćcie sierp, gdyż nadeszło żniwo! Przyjdźcie, zstąpcie, bo pełna jest prasa; pełne są kadzie, gdyż złość ich jest wielka! Tłumy i tłumy zebrane w Dolinie Sądu, bo bliski jest dzień Pana w Dolinie Sądu.'' (Jl~3,~17-19)
\end{quote}
\begin{quote}
,,I rozpadło się wielkie miasto na trzy części, i legły w gruzach miasta pogan. I wspomniano przed Bogiem o wielkim Babilonie, że należy mu dać kielich wina zapalczywego gniewu Bożego.'' (Obj~16,~19)
\end{quote}
\begin{quote}
,,Nie dłużej niż za rok będziecie drżeć, wy pewne siebie! Gdyż winobranie się skończyło, a owocobrania już nie będzie.'' (Iz~32,~10)
\end{quote}
\begin{quote}
,,A trzeci anioł szedł za nimi, mówiąc donośnym głosem: Jeżeli ktoś odda pokłon zwierzęciu i jego posągowi i przyjmie znamię na swoje czoło lub na swoją rękę, To i on pić będzie samo czyste wino gniewu Bożego z kielicha jego gniewu i będzie męczony w ogniu i w siarce wobec świętych aniołów i wobec Baranka.'' (Obj~14,~9-10)
\end{quote}
\begin{quote}
,,Tak bowiem rzekł do mnie Pan, Bóg Izraela: Weź z mojej ręki ten kielich wina gniewu i daj z niego pić wszystkim narodom, do których cię wysyłam, Aby piły i zataczały się, i szalały przed mieczem, który między nie posyłam. Wziąłem więc kielich z ręki Pana i napoiłem wszystkie narody, do których mnie posłał Pan; (...) I powiedz im: Tak mówi Pan Zastępów, Bóg Izraela: Pijcie, upijcie się i wymiotujcie, i padnijcie, aby już nie powstać przed mieczem, który ja między was posyłam! Lecz jeśliby nie chcieli wziąć z twojej ręki tego kielicha, aby pić, wtedy powiedz im: Tak mówi Pan Zastępów: Musicie pić! Bo oto od miasta, które jest nazwane moim imieniem, zaczynam zsyłać nieszczęście, a wy mielibyście ujść bezkarnie? Nie ujdziecie bezkarnie, gdyż oto ja przyzywam miecz na wszystkich mieszkańców ziemi - mówi Pan Zastępów.'' (Jr~25,~15-17.27-29)
\end{quote}
\end{document}