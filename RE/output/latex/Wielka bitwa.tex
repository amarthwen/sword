\documentclass[10pt,a4paper,oneside]{article}
\usepackage[utf8]{inputenc}
\usepackage{polski}
\usepackage[polish]{babel}
\usepackage[margin=0.5in,bottom=0.75in]{geometry}
\begin{document}
\centerline{\textbf{\MakeUppercase{Wielka bitwa}}}
\begin{center}
\textbf{Wersety do studium:} \mbox{(Za 14, 1-3.6-7.12-15)}
\end{center}
\begin{quote}
,,Oto nadchodzi dzień Pana, gdy będę dzielił w twoim gronie łupy odebrane tobie. I zbiorę wszystkie narody do walki z Jeruzalemem. Miasto będzie zajęte, domy splądrowane, kobiety zhańbione. Połowa miasta pójdzie w niewolę, lecz resztka ludności nie będzie usunięta z miasta. Potem wyruszy Pan i będzie walczył z tymi narodami, jak zwykł walczyć w dniu bitwy. (\ldots) W owym dniu stanie się tak: Nie będzie ani upału, ani zimna, ani mrozu. I będzie tylko jeden ciągły dzień, zna go Pan, nie dzień i nie noc, a pod wieczór będzie światło. (\ldots) A taka będzie plaga, jaką Pan dotknie wszystkie ludy, które wystąpiły zbrojnie przeciwko Jeruzalemowi: Ciało każdego, kto stoi jeszcze na nogach, będzie gnić, jego oczy będą gnić w oczodołach, a język zgnije w jego ustach. W owym dniu padnie na nich za sprawą Pana wielka trwoga, tak że jeden chwyci drugiego za rękę, a inny podniesie rękę przeciwko ręce drugiego. Nawet Juda będzie walczył w Jeruzalemie. I zostanie zebrane mienie wszystkich narodów wokoło, złoto i srebro oraz szaty w wielkiej ilości. I taka sama plaga jak tamta dotknie konie, muły, wielbłądy i osły, i wszelkie bydło, które będzie w tych obozach.'' \mbox{(Za 14, 1-3.6-7.12-15)}
\end{quote}
\end{document}