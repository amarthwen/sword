\documentclass[10pt,a4paper,oneside]{article}
\usepackage[utf8]{inputenc}
\usepackage{polski}
\usepackage[polish]{babel}
\usepackage[margin=0.5in,bottom=0.75in]{geometry}
\begin{document}
\centerline{\textbf{\MakeUppercase{Serce}}}
\begin{center}
\textbf{Wersety do studium:} \mbox{(Ef 3, 14-19)}, \mbox{(Ga 4, 6)}, \mbox{(Rz 8, 15)}, \mbox{(Jr 31, 31-34)}, \mbox{(2 Kor 1, 21-22)}, \mbox{(2 Kor 3, 3)}, \mbox{(1 Kor 2, 9-10)}, \mbox{(Jr 17, 9-10)}, \mbox{(Oz 5, 4)}, \mbox{(Łk 6, 45)}, \mbox{(Jk 3, 7-12)}, \mbox{(Jk 4, 8)}, \mbox{(Ps 51, 12)}, \mbox{(Mt 6, 19-21)}, \mbox{(Rz 1, 18-25)}, \mbox{(1 Moj 6, 5-6)}
\end{center}
\begin{quote}
,,Dlatego zginam kolana moje przed Ojcem, Od którego wszelkie ojcostwo na niebie i na ziemi bierze swoje imię, By sprawił według bogactwa chwały swojej, żebyście byli przez Ducha jego mocą utwierdzeni w wewnętrznym człowieku, Żeby Chrystus przez wiarę zamieszkał w sercach waszych, a wy, wkorzenieni i ugruntowani w miłości, Zdołali pojąć ze wszystkimi świętymi, jaka jest szerokość i długość, i wysokość, i głębokość, I mogli poznać miłość Chrystusową, która przewyższa wszelkie poznanie, abyście zostali wypełnieni całkowicie pełnią Bożą.'' \mbox{(Ef 3, 14-19)}
\end{quote}
\begin{quote}
,,A ponieważ jesteście synami, przeto Bóg zesłał Ducha Syna swego do serc waszych, wołającego: Abba, Ojcze!'' \mbox{(Ga 4, 6)}
\end{quote}
\begin{quote}
,,Wszak nie wzięliście ducha niewoli, by znowu ulegać bojaźni, lecz wzięliście ducha synostwa, w którym wołamy: Abba, Ojcze!'' \mbox{(Rz 8, 15)}
\end{quote}
\begin{quote}
,,Oto idą dni - mówi Pan - że zawrę z domem izraelskim i z domem judzkim nowe przymierze. Nie takie przymierze, jakie zawarłem z ich ojcami w dniu, gdy ich ująłem za rękę, aby ich wyprowadzić z ziemi egipskiej, które to przymierze oni zerwali, chociaż Ja byłem ich Panem - mówi Pan - Lecz takie przymierze zawrę z domem izraelskim po tych dniach, mówi Pan: Złożę mój zakon w ich wnętrzu i wypiszę go na ich sercu. Ja będę ich Bogiem, a oni będą moim ludem. I już nie będą siebie nawzajem pouczać, mówiąc: Poznajcie Pana! Gdyż wszyscy oni znać mnie będą, od najmłodszego do najstarszego z nich - mówi Pan - Odpuszczę bowiem ich winę, a ich grzechu nigdy nie wspomnę.'' \mbox{(Jr 31, 31-34)}
\end{quote}
\begin{quote}
,,Tym zaś, który nas utwierdza wraz z wami w Chrystusie, który nas namaścił, jest Bóg, Który też wycisnął na nas pieczęć i dał zadatek Ducha do serc naszych.'' \mbox{(2 Kor 1, 21-22)}
\end{quote}
\begin{quote}
,,Wiadomo przecież, że jesteście listem Chrystusowym sporządzonym przez nasze usługiwanie, napisanym nie atramentem, ale Duchem Boga żywego, nie na tablicach kamiennych, lecz na tablicach serc ludzkich.'' \mbox{(2 Kor 3, 3)}
\end{quote}
\begin{quote}
,,Głosimy tedy, jak napisano: Czego oko nie widziało i ucho nie słyszało, i co do serca ludzkiego nie wstąpiło, to przygotował Bóg tym, którzy go miłują. Albowiem nam objawił to Bóg przez Ducha; gdyż Duch bada wszystko, nawet głębokości Boże.'' \mbox{(1 Kor 2, 9-10)}
\end{quote}
\begin{quote}
,,Podstępne jest serce, bardziej niż wszystko inne, i zepsute, któż może je poznać? Ja, Pan, zgłębiam serce, wystawiam na próbę nerki, aby oddać każdemu według jego postępowania, według owocu jego uczynków.'' \mbox{(Jr 17, 9-10)}
\end{quote}
\begin{quote}
,,Ich uczynki nie pozwalają im zawrócić do swojego Boga, gdyż duch wszeteczeństwa jest w ich sercu, tak że nie znają Pana.'' \mbox{(Oz 5, 4)}
\end{quote}
\begin{quote}
,,Człowiek dobry z dobrego skarbca serca wydobywa dobro, a zły ze złego wydobywa zło; albowiem z obfitości serca mówią usta jego.'' \mbox{(Łk 6, 45)}
\end{quote}
\begin{quote}
,,Bo wszelki rodzaj dzikich zwierząt i ptaków, płazów i stworzeń morskich może być ujarzmiony i został ujarzmiony przez rodzaj ludzki. Natomiast nikt z ludzi nie może ujarzmić języka, tego krnąbrnego zła, pełnego śmiercionośnego jadu. Nim wysławiamy Pana i Ojca i nim przeklinamy ludzi, stworzonych na podobieństwo Z tych samych ust wychodzi błogosławieństwo i przekleństwo. Tak, bracia moi, być nie powinno. Czy źródło wydaje z tego samego otworu wodę słodką i gorzką? Czy drzewo figowe, bracia moi, może rodzić oliwki albo winna latorośl figi? Tak też słony zdrój nie może wydać słodkiej wody.'' \mbox{(Jk 3, 7-12)}
\end{quote}
\begin{quote}
,,Zbliżcie się do Boga, a zbliży się do was. Obmyjcie ręce, grzesznicy, i oczyśćcie serca, ludzie o rozdwojonej duszy.'' \mbox{(Jk 4, 8)}
\end{quote}
\begin{quote}
,,Serce czyste stwórz we mnie, o Boże, A ducha prawego odnów we mnie!'' \mbox{(Ps 51, 12)}
\end{quote}
\begin{quote}
,,Nie gromadźcie sobie skarbów na ziemi, gdzie je mól i rdza niszczą i gdzie złodzieje podkopują i kradną: Ale gromadźcie sobie skarby w niebie, gdzie ani mól, ani rdza nie niszczą i gdzie złodzieje nie podkopują i nie kradną. Albowiem gdzie jest skarb twój - tam będzie i serce twoje'' \mbox{(Mt 6, 19-21)}
\end{quote}
\begin{quote}
,,Albowiem gniew Boży z nieba objawia się przeciwko wszelkiej bezbożności i nieprawości ludzi, którzy przez nieprawość tłumią prawdę. Ponieważ to, co o Bogu wiedzieć można, jest dla nich jawne, gdyż Bóg im to objawił. Bo niewidzialna jego istota, to jest wiekuista jego moc i bóstwo, mogą być od stworzenia świata oglądane w dziełach i poznane umysłem, tak iż nic nie mają na swoją obronę, Dlatego że poznawszy Boga, nie uwielbili go jako Boga i nie złożyli mu dziękczynienia, lecz znikczemnieli w myślach swoich, a ich nierozumne serce pogrążyło się w ciemności. Mienili się mądrymi, a stali się głupi. I zamienili chwałę nieśmiertelnego Boga na obrazy przedstawiające śmiertelnego człowieka, a nawet ptaki, czworonożne zwierzęta i płazy; Dlatego też wydał ich Bóg na łup pożądliwości ich serc ku nieczystości, aby bezcześcili ciała swoje między sobą, Ponieważ zamienili Boga prawdziwego na fałszywego i oddawali cześć, i służyli stworzeniu zamiast Stwórcy, który jest błogosławiony na wieki. Amen.'' \mbox{(Rz 1, 18-25)}
\end{quote}
\begin{quote}
,,A gdy Pan widział, że wielka jest złość człowieka na ziemi i że wszelkie jego myśli oraz dążenia jego serca są ustawicznie złe, Żałował Pan, że uczynił człowieka na ziemi i bolał nad tym w sercu swoim.'' \mbox{(1 Moj 6, 5-6)}
\end{quote}
\end{document}