\documentclass[10pt,a4paper,oneside]{article}
\usepackage[utf8]{inputenc}
\usepackage{polski}
\usepackage[polish]{babel}
\usepackage[margin=0.5in,bottom=0.75in]{geometry}
\begin{document}
\centerline{\textbf{\MakeUppercase{Duchowa niewola}}}
\begin{center}
\textbf{Wersety do studium:} (2~Tm~2,~24-26), (1~Moj~4,~6-7), (Rz~7,~14-25), (Rz~8,~14-15), (Ga~4,~1-7), (Iz~30,~1)
\end{center}
\begin{quote}
,,A sługa Pański nie powinien wdawać się w spory, lecz powinien być uprzejmy dla wszystkich, zdolny do nauczania, cierpliwie znoszący przeciwności, Napominający z łagodnością krnąbrnych, w nadziei, że Bóg przywiedzie ich kiedyś do upamiętania i do poznania prawdy I że wyzwolą się z sideł diabła, który ich zmusza do pełnienia swojej woli.'' (2~Tm~2,~24-26)
\end{quote}
\begin{quote}
,,I rzekł Pan do Kaina: Czemu się gniewasz i czemu zasępiło się twoje oblicze? Wszak byłoby pogodne, gdybyś czynił dobrze, a jeśli nie będziesz czynił dobrze, u drzwi czyha grzech. Kusi cię, lecz ty masz nad nim panować.'' (1~Moj~4,~6-7)
\end{quote}
\begin{quote}
,,Wiemy bowiem, że zakon jest duchowy, ja zaś jestem cielesny, zaprzedany grzechowi. Albowiem nie rozeznaję się w tym, co czynię; gdyż nie to czynię, co chcę, ale czego nienawidzę, to czynię. A jeśli to czynię, czego nie chcę, zgadzam się z tym, że zakon jest dobry. Ale wtedy czynię to już nie ja, lecz grzech, który mieszka we mnie. Wiem tedy, że nie mieszka we mnie, to jest w ciele moim, dobro; mam bowiem zawsze dobrą wolę, ale wykonania tego, co dobre, brak; Albowiem nie czynię dobrego, które chcę, tylko złe, którego nie chcę, to czynię. A jeśli czynię to, czego nie chcę, już nie ja to czynię, ale grzech, który mieszka we mnie. Znajduję tedy w sobie zakon, że gdy chcę czynić dobrze, trzyma się mnie złe; Bo według człowieka wewnętrznego mam upodobanie w zakonie Bożym. A w członkach swoich dostrzegam inny zakon, który walczy przeciwko zakonowi, uznanemu przez mój rozum i bierze mnie w niewolę zakonu grzechu, który jest w członkach moich. Nędzny ja człowiek! Któż mnie wybawi z tego ciała śmierci? Bogu niech będą dzięki przez Jezusa Chrystusa, Pana naszego! Tak więc ja sam służę umysłem zakonowi Bożemu, ciałem zaś zakonowi grzechu'' (Rz~7,~14-25)
\end{quote}
\begin{quote}
,,Bo ci, których Duch Boży prowadzi, są dziećmi Bożymi. Wszak nie wzięliście ducha niewoli, by znowu ulegać bojaźni, lecz wzięliście ducha synostwa, w którym wołamy: Abba, Ojcze!'' (Rz~8,~14-15)
\end{quote}
\begin{quote}
,,A mówię: Dopóki dziedzic jest dziecięciem, niczym się nie różni od niewolnika, chociaż jest panem wszystkiego, Ale jest pod nadzorem opiekunów i rządców aż do czasu wyznaczonego przez ojca. Podobnie i my, gdy byliśmy dziećmi, byliśmy poddani w niewolę żywiołów tego świata; Lecz gdy nadeszło wypełnienie czasu, zesłał Bóg Syna swego, który się narodził z niewiasty i podlegał zakonowi, Aby wykupił tych, którzy byli pod zakonem, abyśmy usynowienia dostąpili. A ponieważ jesteście synami, przeto Bóg zesłał Ducha Syna swego do serc waszych, wołającego: Abba, Ojcze! Tak więc już nie jesteś niewolnikiem, lecz synem a jeśli synem, to i dziedzicem przez Boga.'' (Ga~4,~1-7)
\end{quote}
\begin{quote}
,,Biada przekornym synom, mówi Pan, którzy wykonują plan, lecz nie mój, i zawierają przymierze, lecz nie w moim duchu, aby dodawać grzech do grzechu.'' (Iz~30,~1)
\end{quote}
\end{document}